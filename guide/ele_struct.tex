\chapter{The Ele\_struct}
\section{overview}

\elestruct\ is the structure that holds all the information about an
individual element. The definition of the \elestruct\ is shown in 
figure~\ref{f:ele_struct}. Substructure definitions are shown in 
figure~\ref{f:subele_struct}. The subelements of the \elestruct\ are as
follows:
\begin{description}
\item[\%name] This is the element name from the input lattice file.
\item[\%type] This is the element type name from the input lattice file.
\item[\%alias] This is the element alias name from the input lattice file.
\item[\%attribute\_name] This is used by overlays. See below.
\item[\%x, \%y, \%z] Twiss parameters for the $a$, $b$ and $z$ modes (yes it is 
known that the labeling is misleading).
\item[\%value(:)] this array holds the attribute values for the element. 
For example, the value of the  k1 attribute for a quadrupole element is stored
in \%value(k1\$) where k1\$ is an integer parameter. In general to get the correct
index for the \%value(:) for a given attribute just use a "\$" as a suffix.
\item[\%gen0] Constant part of a \genfield.
\end{description} 


\section{Initializing}

  Dependent and independent variables

  How to locate attributes

  Allocation/deallocation

\section{Element Control}

    * Overlays
    * Superimpose
    * Groups

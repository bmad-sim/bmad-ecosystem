\chapter{Tracking Methods}

%----------------------------------------------------------------------
\section{Tracking Methods}

\begin{description}
\item[Adaptive\_Boris]
Second order Boris integration with adaptive step size control.
This should be nearly symplectic but slow.

\item[Boris]
Second order Boris Integration. Like Runge-Kutta Boris does tracking by
integrating the equation of motion. The difference is that Boris integration
is symplectic.

\item[Bmad\_Standard]
This is meant to be quick and dirty (ie if it is not symplectic
who cares). Appropriate when you are only interested in single turn stuff
(that is, not long term tracking). It
does an exact calculation through sector bends using thin quads at either end
for non-sector focusing. Sextupoles and octupoles are tracked using a single
kick-drift-kick integration.

\item[Custom]
This method will call a routine \vn{track1_custom} which can be supplied by
the user. The default \vn{track1_custom} supplied with the BMAD release does
Runge Kutta tracking. See the Programmer's Guide for more details.

\item[Linear]
Linear just uses the 0th order vector with the 1st order 6x6 transfer
matrix for an element. Very simple.  Depending upon how the transfer
matrix was generated this might or might not be symplectic.

\item[Runge\_Kutta]
This uses a 4th order Runge Kutta integration algorithm with adaptive
step size control.  This is essentially ODEINT adopted from Numerical
Recipes. This may be slow but it should be accurate. This method is
non-symplectic.

\item[Symp\_Lie\_BMAD]
Symplectic tracking using a Hamiltonian with Lie operation techniques.
This is similar to \vn{Symp_Lie_PTC} (see below) except this uses a
BMAD routine.  The difference between this and \vn{Symp_Lie_PTC} is
that PTC tries do do things correctly while BMAD goes for speed by
making approximations like the small angle approximation, etc. Right
now only implemented for Wigglers.

\item[Symp\_Lie\_PTC]
Symplectic tracking using a Hamiltonian with Lie operation techniques.
This uses Etienne's PTC software for the calculation. This method is
symplectic but can be slow. See below for additional switches that affect this
calculation. This method can only be used on elements that have a Hamiltonian.
Quadrupoles, Solenoids, and most other element types have Hamiltonians. A
Hybrid element is an  example where there is no Hamiltonian.

\item[Symp\_Map]
This uses an implicit (partially inverted) Taylor map.
Since the map is implicit a Newton search method must be used. This will slow
things down from the Taylor method but this is guaranteed symplectic.

\item[Taylor]
This uses a Taylor map generated from Etienne's PTC package. Generating
the map may take time but once you have it it should be very fast. One
possible problem with using a Taylor series is that you have to worry about
the accuracy if you do tracking at points that are far from the point about
which the series was made. This method is non-symplectic. See below for
additional switches that affect this calculation.

\item[Wiedemann]
This is Wiedemann's hard edge model of a wiggler.

\end{description}

\chapter{Coordinates.tex}

\section{Reference Orbit}
\label{sec:ref}

The ``reference orbit'' is the path of a ``reference particle''.  The
reference orbit is used to define a coordinate system for actual
particles (particles whose orbits are simulated in \bmad) as shown in
figure~\ref{f:local_coords}. At a given time $t$ the reference
particle is a distance $s = c \, t$ along the reference orbit from the
reference orbit zero position. The local $(x, y, z)$ coordinate system
is such that the origin is at the reference particle with the
$z$--axis tangent to the local reference orbit and pointing in the
direction of the reference particle motion. The $x$ and $y$--axes are
perpendicular to the reference orbit. If there are no vertical bends,
the $y$--axis is in the vertical direction and the $x$--axis is in the
horizontal plane.

\begin{figure}[tb]
\centering
\includegraphics{local_coords.ps}
\caption{The Local Reference System}
\label{f:local_coords}
\end{figure}

In \bmad\ a lattice is comprised of a sequence of elements:
Quadrupoles, bends, etc. Each element has an entrence point and an
exit point and a reference curve between them. For a bend the
reference curve is circular and for all other elements the reference
curve is a straight line. The elements are assembled together so that
the exit point of one element coinsides with the entrence point of the
next with the reference curves forming a smooth arc with no kinks. The
reference orbit is then just the sum of the reference curves. If not
specified otherwise, the $s = 0$ position is just the enterence point
of the first element.

Notice that in a wiggler the reference orbit, which is a straight
line, does {\em not} correspond to the orbit that any actual particle
could travel. Typically the physical entity of an element is centered
about the reference curve but in general this may not be so. For
example, by specifying offsets and pitches, the centerline of a
quadrupole magnet may be arbitrarily oriented with respect to its
reference curve (which is a line segment). Since the reference curve
of an element is fixed to the reference curves of the preceding and
following elements it is the physical magnet that moves and not the
reference curve. Shifting a physical magnet with respect to its
reference curve means that the reference curve does {\em not}
correspond to the orbit that any actual particle could travel.


\section{Global Reference System}

The frame of reference that describes reference orbit in terms of where it
is in the laboratory is called the global reference system.
\bmad, following the \mad\ convention, uses a Cartesian coordinate system
$(X, Y, Z)$ for the global reference system along with three angles
$\theta, \phi, \psi$ used to define the reference orbit's orientation
as shown in figure~\ref{f:global_coords}. $Y$ is the vertical
coordinate and $(X, Z)$ are the ``floor'' coordinates.  The definition
of the three angles is:
\begin{description}
\item[$\theta$ Azimuth angle:] Angle in the $(X, Z)$ plane 
between the $Z$--axis and the projection of the $z$--axis onto the
$(X, Z)$ plane. A positive angle of $\theta = \pi/2$ corresponds to the
projected $z$ axis pointing in the positive $X$ direction.
\item[$\phi$ Pitch (elevation) angle:] Angle between the $z$--axis 
and the $Y$--axis. A positive angle of $\phi = \pi/2$ corresponds to
the $s$--axis pointing in the positive $Y$ direction.
\item[$\psi$ Roll angle:] Angle of the $x$--axis with respect 
to the intersection of the $(X, Z)$ plane with the $(x, y)$ plane. A
positive $\psi$ forms a right--handed scew with the $s$--axis.
\end{description}

\begin{figure}
\centering
\includegraphics{global_coords.ps}
\caption{The Global Reference System}
\label{f:global_coords}
\end{figure}

By default, at the reference orbit's $s = 0$ point coincides with the
$(X, Y, Z)$ origin and at $s = 0$ the $x$, $y$, and $z$ axes
correspond to the $X$, $Y$, and $Z$ axes respectively. $\theta$
decreases as one folows the reference orbit when going through a
horizontal bend with a positive bending angle. This corresponds to $x$
pointing radially outward. Without any vertical bends, the $Y$ and $y$
axes will coinside and $\phi$ and $\psi$ will both be zero.

\section{Phase Space Coordinate System}

The canonical phase space coordinates that \bmad\ uses for tracking a
particle is $(x, p_x, y, p_y, z, p_z)$. $x$, $y$, and $z$ are the
coordinates with respect to the reference particle as explained in
\ref{sec:ref}. $p_x$ and $p_y$ are the normalized momentum
\begin{align}
  p_x = &\frac{P_x}{P_0} \\
  p_y = &\frac{P_y}{P_0}
\end{align}
where $P_x$ and $P_y$ are the momentum components along the $x$ and
$y$ axes respectively and $P_0$ is the reference (sometimes called the
design) momentum. The longitudinal canonical momentum $p_z$ is given by
\begin{equation}
  p_z = \frac{\Delta E}{E_0}
\end{equation}
Where $E_0$ is the reference energy and $\Delta E = E - E_0$ is the
deviation of the particle's energy from the reference energy. \mad\ uses
a slightly different coordinate system. With \mad\ $(z, p_z)$ is
replaced by $(-c\Delta t, p_t)$. where $\Delta t$ is the time
difference for a particle to pass a point relative to the reference
particle and $p_t \equiv \Delta E / P_0 c$. For highly relativistic
particles the two coordinate systems are the same. For
non-relativistic particles \bmad\ is not to be trusted in any
case. \bmad\ generally uses the small angle (paraxial) approximation
where it is assumed that $p_x, p_y \ll 1$. With this approximation the
relationship between the cononical momenta and the slopes $x' \equiv dx/ds$
and $y' \equiv dy/ds$ is
\begin{align}
  x' &\approx \frac{p_x}{1 + p_z} (1 + g x) \\
  y' &\approx \frac{p_y}{1 + p_z} (1 + g x) 
\end{align}
where $g = 1/\rho$ is the curvature function with $\rho$ being the radius
of curvature of the reference orbit. $g = 0$ in a straight section.

For those programmers using the PTC software package directly (ignore
this if you don't know what I'm talking about) \'Etienne Forest uses a still
different coordinate system. with PTC $(z, p_z)$ is replaced by
$(\Delta E/P_0 c, c \Delta t)$
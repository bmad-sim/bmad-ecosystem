\chapter{Twiss Parameter and Other Calculations}


%-----------------------------------------------------------------------------
\section{Twiss Parameter Calculations}

The Twiss parameters for an element are stored in the
\vn{ele_struct}. See Chapter~\ref{c:ele_struct} for more details.

Before any Twiss parameters can be calculated the transfer matrices
stored in the lattice elements must be computed. \vn{bmad_parser} does
this automatically about the zero orbit. If, to see nonlinear effects,
a different orbit needs to be used for the reference, The routine
\vn{ring_make_mat6} can be used. For example
\begin{example}
  type (ring_struct) ring
  type (coord_struct), allocatable :: orbit(:)
  call bmad_parser ('my_lattice', ring)
  call closed_orbit (ring, orbit, 4)
  call ring_make_mat6 (ring, -1, orbit)
\end{example}
This example reads in a lattice, finds the closed orbit which may be
non--zero due to, say, kicks due to a separator, and then remakes the
transfer matrices (which are stored in \vn{ring%ele_(i)%mat6} around
the closed orbit.

Once the transfer matrices are calculated the Twiss parameters at the
start of the lattice need to be defined. The Twiss parameters at the
start are in \vn{ring%ele_(0)}. If the lattice is open then generally
the Twiss parameters are set in the lattice file or may easily be set
in a program. For example
\begin{example}
  ring%ele_(0)%x%beta = 1.2
  ring%ele_(0)%x%alpha = 0.1
  ring%ele_(0)%x%gamma = (1 + ring%ele_(0)%x%alpha**2) / ring%ele_(0)%beta
  ring%ele_(0)%x%eta  = 0
\end{example}
Note that \vn{%beta}, \vn{%alpha}, and \vn{%gamma} all must be specified.

If the lattice is closed then \vn{twiss_at_start} may be used to
calculate the self--consistant starting Twiss parameters. Once the
starting Twiss parameters are set \vn{twiss_propagate_all} can be used
to propagate the Twiss parameters to the rest of the elements. Example
\begin{example}
  type (ring_struct) ring
  call bmad_parser ('my_lattice', ring)
  call twiss_at_start (ring)
  call twiss_propagate_all (ring)
\end{example}

%-----------------------------------------------------------------------------
\section{Optimizers}

Optimzers attempt to solve nonlinar optimization problems. This is good
for designing lattices and such. 




\chapter{Tracking}

\bmad\ can do two types of tracking. One type uses a single particle 
and tracks its coordinates throughout the lattice. The other type
takes a beam distribution and tracks the beam RMS sigmas. This latter
tracking is associated with wakefields and is discussed in another
chapter. The starting point of any tracking is the \tn{coord\_struct}
shown in~\ref{f:coord_struct}

\begin{figure}[tb]
\centering
\small
\begin{verbatim}
  type coord_struct
    real(rp) vec(6)   ! (x, p_x, y, p_y, z, p_z)
  end type
\end{verbatim}
\caption{Definition of the \tn{coord\_struct}.}
\label{f:ele_struct}
\end{figure}

To get an orbit you will need an array of \tn{coord_struct}'s. Since
the number of elements in the lattice is not known in advance the
array must be decleared via a pointer
\begin{verbatim}
  type (coord_struct), pointer :: coords(:) = null()
\end{verbatim}
You cannot, in general, use an allocatable array since the BMAD
subroutines expect a pointer and Fortran does not allow you to mix and
match. Since you have a variable that alloocates memory from the heap
you are responsable for making sure you clean up after yourself if you
want to prevent a memory leak. If you are writing a subroutine where
the \tn{coord\_struct} array is local and if the number of elements in 
the \tn{ring\_struct} is constant then the fastest way to allocate is:
\begin{berbatim}
  type (coord_struct), pointer, save :: orb_(:) => null()�
  ...
  call allocate_coord (orb_, ring%n_ele_ring)
\end{verbatim}


Advantages: Minimal reallocation so fast.
  Can always handle case when size of ring%ele_(:) changes.
Disadvantages: Slightly more complicated, You always have the allocated array
  filling memory.

3) The "clean" way of doing things is:

      type (coord-struct), allocatable :: orb_(:)
      ...
      allocate (orb_(0:ring%n_ele_ring)
      ...
      deallocate (orb_)

Advantages: Clean
Disadvantages: Can be slow. coord_struct gets allocated and deallocated every
  time the routine is called.

-------------------------------------------------------------------

For other types of variables:

1) In each ele_struct there is for general use:
      ele%r(:)
      ele%ix_pointer
      ele%logic
These subelements are not used by BMAD.


2) For allocated arrays there are the following routines:
      reallocate_real
      reallocate_integer
      reallocate_logical
      reallocate_string

3) For pointer arrays Numerical Recipes has the overloaded reallocate function.
For example:
      integer, pointer :: int(:)
      ...
      int => reallocate (int, 500)





When using radiation excitation the random number generator is based
upon the fortran90 intrinsic random\_number. If random\_seed is not called
then the fluctuations from run to run will be exactly the same.


How to allocate coord\_struct arrays. 

  Switches to switch tracking

  Runge Kutta

  Boris

  Custom
    * how to setup


\SECTION{Overview}

\bmad\ (Otherwise known as ``Baby MAD" or ``Better MAD" or just plain
``Be MAD!")  is a software subroutine library used for relativistic
charged--particle dynamics simulations in high energy accelerators and
storage rings. \bmad\ has been developed at Cornell University's
Laboratory for Elementary Particle Physics and has been in use since the
1990's. \bmad\ was developed because there was a need at Cornell to do
calculations that couldn't be done with existing programs in general use
in the accelerator physics community. What had been hapening up to then
was that individual physicists were writing programs almost from scratch 
to do individual calculations. This lead to much duplication of effert.
As a response the \bmad\ subroutines, written in
Fortran90, were developed to:
\begin{Itemize}
\item Cut down on the time needed to develop programs,
\item Reduce computation times to a minimum,
\item Cut down on programming errors, 
\item Provide a standard input format for specifying lattices, and
\item Provide a standardized means to share lattice information between 
programs.
\end{Itemize}

\bmad\ can be used to study single and multi--particle beam dynamics. Its 
features include the ability to track particles, calculate transfer matrices,
emittances, Twiss parameters, 
dispersion, coupling, etc. The elements that \bmad\ knows about include
quadrupoles, RF cavities (both storage ring and LINAC accelerating types),
solenoids, dipole bends, etc. 

\vfill
\break

\documentclass{book}

\usepackage{graphicx}
\usepackage{moreverb}
\usepackage{amsmath}
\usepackage{alltt}
\usepackage{rotating}
\usepackage{subfigure}
\usepackage{toc}
\usepackage{xspace}
\usepackage{makeidx}

\newcommand{\sref}[1]{\S\ref{#1}}
\newcommand{\Sref}[1]{Sec.~\sref{#1}}

\newcommand{\vn}{\begingroup\catcode`\_=11 \catcode`\%=11 \dottcmd}
\newcommand\dottcmd[1]{{\usefont{T1}{lmss}{bx}{n} #1}\endgroup}

\newenvironment{example}
  {\vspace{-3.0ex} \begin{alltt}}
  {\end{alltt} \vspace{-2.5ex}}


\definecolor{light-gray}{gray}{0.95}
\lstset{backgroundcolor=\color{light-gray}}
\lstset{xleftmargin=0cm}
\lstset{framexleftmargin=0.3em}

\lstnewenvironment{Xcode}{}{}

\definecolor{lightcyan}{rgb}{0.88, 1.0, 1.0}
\newcounter{main}
\setcounter{main}{1}
\lstnewenvironment{code}[1][firstnumber=\themain,name=main]
  {\lstset{ %language=haskell,
           %columns=fullflexible,
           columns=fixed,
           basicstyle=\small\ttfamily,
           %numbers=left,
           numberstyle=\tiny\color{gray},
           backgroundcolor=\color{lightcyan},
           #1
          }
}
{\setcounter{main}{\value{lstnumber}}}



\setlength{\textwidth}{6.25in}
\setlength{\hoffset}{0.0in}
\setlength{\oddsidemargin}{0.25in}
\setlength{\evensidemargin}{0.0in}
\setlength{\textheight}{8.5in}
\setlength{\topmargin}{0in}

\makeindex

\begin{document}

\setlength{\parskip}{\dPar}
\setlength{\parindent}{0ex}

%-----------------------------------------------------------------
\section{Bmad\_Standard Matrix calculations and Tracking}
\label{s:bmad_standard}


\vn{Bmad_standard} transfer matrix calculations (\sref{s:tkm})
tracking (\sref{s:xfer}) are meant to be quick. To this end the
paraxial approximation (\Eq{xpa1p}) is always used.


%-----------------------------------------
\index{Drift} 
\subsection{Drift}
The Hamiltonian for a drift is
\Begineq
  H = \frac{p_x^2 + p_y^2}{2 (1 + p_z)} 
\Endeq
This gives the map
\begin{align}
  x   &\rightarrow x + \frac{l \, p_x}{1 + p_z} \CRNO
  p_x &\rightarrow p_x  \CRNO
  y   &\rightarrow y + \frac{l \, p_y}{1 + p_z} \CRNO
  p_y &\rightarrow p_y  \\
  z   &\rightarrow z - \frac{l \, (p_x^2 + p_y^2)}{2 (1 + p_z)^2} \CRNO
  p_z &\rightarrow p_z \nonumber
\end{align}
The Jacobian is 
\Begineq
  \begin{pmatrix}
    x \\ p_x \\ z \\ p_z
  \end{pmatrix}
  = 
  \begin{pmatrix}
    1 & \frac{l}{1 + p_z}             & 0 & \frac{-l \, p_x}{(1 + p_z)^2} \\
    0 & 1                             & 0 & 0 \\
    0 & -\frac{l \, p_x}{(1 + p_z)^2} & 1 & 
                              \frac{l \, (p_x^2 + p_y^2)}{(1 + p_z)^3} \\
    0 & 0                             & 0 & 1
  \end{pmatrix}
\Endeq


%-----------------------------------------
\index{LCavity}
\subsection{LCavity}

The traversal time across a cavity is
\Begineq
  t_L = \int_0^L \frac{ds}{\beta \, c}
\Endeq
Writing
\Begineq
  \frac{1}{\beta} = \frac{E}{cP} = \frac{E}{\sqrt{E^2 - m^2c^4}}
\Endeq
and assuming a constant gradient $G$, so that $E$ increases linearly
with length, the transit time is easily computed to be
\Begineq
  t_L = \frac{P_2 - P_1}{G}
\Endeq
Using this in \Eq{zbbzb} gives the change in $z$
\Begineq
  z_2 = \frac{\beta_2}{\beta_1} \, z_1 + 
  \beta_2 \, c \left[ \frac{P_{02} - P_{01}}{G_0} - \frac{P_2 - P_1}{G} \right]
\Endeq
where $P_{01}$ and $P_{02}$ are the reference momentum at the entrance
and exits of the \vn{LCavity} with $G_0$ the reference gradient. 

The derivatives are straight forword if tedious
\begin{align}
  m(5,5) &= \frac{dz_2}{dz_1} = 
    \frac{\beta_2}{\beta_1} + 
    \frac{z_2 \, m(6,5)}{\beta_2} \frac{d\beta_2}{dp_{z2}} - 
    \frac{\beta_2}{G} \frac{dcP_2}{dz_1} +
    \frac{\beta_2 \, c \, (P_2 - P_1) \, c P_2}{L \, G^2 \, E_2} 
      \frac{dcP_2}{dz_1} \CRNO
  m(5,6) &= \frac{dz_2}{dp_{z1}} = 
    \frac{-\beta_2 \, z_1}{\beta_1^2} \frac{d\beta_1}{dp_{z1}} + 
    \frac{z_2 \, m(6,6)}{\beta_2} \frac{d\beta_2}{dp_{z2}} -
    \frac{\beta_2 ( c P_{02} \, m(6,6) - c P_{01})}{G} \CRNO
  m(6,5) &= \frac{dp_{z2}}{dz1} =
    \frac{E_2}{cP_2 \, cP_{02}} \frac{cP_1 \, cP_{01}}{E_1}  \\
  m(6,6) &= \frac{dp_{z2}}{dp_{z1}} = 
    \frac{E_2}{cP_2} \frac{G \, L}{c P_{02}} \frac{2 \, \pi \, f \, \sin\phi}{c}
    \nonumber
\end{align}
where
\begin{align}
  \frac{d\beta_1}{dp_{z1}}  &= \frac{(mc^2)^2}{E_1^3} \, cP_{01} \CRNO
  \frac{d\beta_2}{dp_{z2}}  &= \frac{(mc^2)^2}{E_2^3} \, cP_{02} \\
  \frac{dcP_2}{dz_1}        &= m(6,5) \, cP_{02}  \nonumber
\end{align}
%-----------------------------------------------------------------

\end{document}

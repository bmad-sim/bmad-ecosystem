\chapter{Transfer Matrix Calculation Methods}

\begin{description}

\item[\vn{Bmad\_Standard}]
This is meant to be quick and dirty. It tries to be symplectic but this
is not guaranteed. Sextupoles and octupoles are done using a simple
kick-drift-kick model.

\item[\vn{Custom}]
This method calls a routine \rn{make_mat6_custom} which may be supplied
by the user (which must have been linked into whatever program is
running).  The default \rn{make_mat6_custom} supplied by the \bmad\ release
will use Runge Kutta tracking. 

\item[\vn{Runge\_Kutta}]
This uses a Runge Kutta tracking algorithm adopted from Numerical Recipes.
This method tracks 6 particles around the central orbit. This method is
susceptible to inaccuracies caused by nonlinearities. Furthermore this method
is almost surely slow. While non--symplectic, the advantage of this method is
that it relies only on the field map for the element.

\item[\vn{Symp\_Lie\_Bmad}]
Symplectic integration using a Hamiltonian and Lie operators. The
difference between this and \vn{Symp_lie_PTC} is that PTC tries to do things
correctly while \bmad\ goes for speed by making approximations like the small
angle approximation, etc. Right now only implemented for Wigglers.

\item[\vn{Symp\_Lie\_PTC}]
Symplectic integration using a Hamiltonian and Lie operators.
This uses Etienne's PTC software for the calculation.
This method is symplectic but can be slow. See below for additional switches
that affect this calculation. This method can only be used on elements that
have a Hamiltonian. Quadrupoles, Solenoids, and most other element types have
Hamiltonians. A Hybrid element is an example where there is no Hamiltonian.

\item[\vn{Taylor}]
This uses a Taylor map generated from Etienne's PTC package. Generating
the map may take time but once you have it it should be very fast. One
possible problem with using a Taylor series is that you have to worry about
the accuracy of the series if you calculate the Jacobian about a point that is
far from the point about which the series was made. This method is
non-symplectic. See below for additional switches that affect this
calculation.

\item[\vn{Tracking}]
This uses the tracking method set by \vn{tracking_method} to track 6
particles around the central orbit. This method is susceptible to inaccuracies
caused by nonlinearities. While non--symplectic, the advantage of this method
is that it is directly related to any tracking results.

\end{description}

Symplectic integration is like ordinary integration of a function f(x)
but what is integrated here is the Hamiltonian H(y) where y here could
be a 6-dimensional vector (for tracking) or be a taylor series (for
the mat6 calculation). The order at which a Taylor series is truncated
at is set by \vn{taylor_order} (this is a global variable). Like
ordinary integration there are various formulas that one can use to do
symplectic integration. In \bmad\ (or more precisely Etienne's PTC)
you can use one of 3 methods. This is set by \vn{integration_order}. 
\vn{integration_order} = n (n = 2, 4, or 6)
means that the error scales as $dz^n$ where $dz$ is the integration step
size. The step size dz is set by the length of the element and the
value of \vn{num_steps}. Remember, as in ordinary integration, higher
order does not necessarily imply higher accuracy.

\vn{PTC_kind} sets how the symplectic integrator divides up the hamiltonian.
The details are in the PTC manual. The default \vn{PTC_kind} = 0
means \bmad\ will choose what it thinks as best (see the routine
\rn{Bmad_ele_to_fibre} for more details).

\section*{Introduction}

The strength of \bmad is that as a subroutine library it provides a flexible
framework from which sophisticated simulation programs may easily be developed.
The weakness of \bmad comes from its strength: Someone must put the pieces 
together into a program. As a consequence this manual serves two masters:

The \bmad lattice input standard was developed using the MAD lattice
input standard as a starting point. MAD (Methodical Accelerator
Design) is a widely used stand--alone program developed at CERN by
Christoph Iselin for charged--particle optics calculations. The
limitations of the MAD program was the impetus for writing
\bmad. Since it can be convenient to do simulations with both MAD and
\bmad, differences and similarities between the two input formats are
noted in this guide.

Errors and omissions are a fact of life for any reference work and
comments from you, dear reader, are therefore most welcome. Please
send any missives (or chocolate, etc.) to:
\begin{verbatim}
  David Sagan <dcs16@cornell.edu>
\end{verbatim}

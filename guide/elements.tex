\chapter{Elements}

A lattice for a storage ring or linac is made up of a collection of
elements --- Quadrupoles, Bends, etc. This chapter discusses the
various types of elements available in \bmad.

\section{Bmad Elements}

Most element types available in \mad\ are provided in \bmad.
Additionally, \bmad\ provides a number of element types that are not
available in \mad.  A word of caution: In some cases where both \mad\
and \bmad\ provide the same element type, there will be an overlap of 
the attributes available but the two sets of attributes will not be the same.
The list of element types known to \bmad\ is shown in Table~\ref{tab:elements}.
In
\begin{table}[h]
{\centering
{\tt
\begin{tabular}{|l|l||l|l|} \hline
  {\it Element} & {\it Section}     & {\it Element} & {\it Section}    \\ \hline
  ab\_multipole & \ref{s:ab_m}      &  multipole    & \ref{s:mult}     \\ \hline
  accel\_sol    & \ref{s:accel_sol} &  octupole     & \ref{s:oct}      \\ \hline
  beambeam      & \ref{s:bbi}       &  overlay      & \ref{s:over}     \\ \hline
  custom        & \ref{s:custom}    &  patch        & \ref{s:patch}    \\ \hline
  drift         & \ref{s:drift}     &  quadrupole   & \ref{s:quad}     \\ \hline
  ecollimator   & \ref{s:col}       &  rbend        & \ref{s:bend}     \\ \hline
  elseparator   & \ref{s:elsep}     &  rcollimator  & \ref{s:col}      \\ \hline
  group         & \ref{s:group}     &  rfcavity     & \ref{s:rfcav}    \\ \hline
  hkicker       & \ref{s:hvkicker}  &  sbend        & \ref{s:bend}     \\ \hline
  hybrid        & \ref{s:hybrid}    &  sextupole    & \ref{s:sex}      \\ \hline
  instument     & \ref{s:monitor}   &  solenoid     & \ref{s:sol}      \\ \hline
  kicker        & \ref{s:kicker}    &  sol\_quad    & \ref{s:sq}       \\ \hline
  lcavity       & \ref{s:lcav}      &  taylor       & \ref{s:tay}      \\ \hline
  marker        & \ref{s:mark}      &  vkicker      & \ref{s:hvkicker} \\ \hline
  monitor       & \ref{s:monitor}   &  wiggler      & \ref{s:wig}      \\ \hline
\end{tabular}
}}
\caption{\bmad\ elements.}
\label{tab:elements}\center
\end{table}

\vfil
\break

%-----------------------------------------------------------------
\section{AB\_Multipole}
\label{s:ab_m}

An \vn{AB_Multipole} is a thin multipole lens up to 20th order. The only
difference between this and a \vn{Multipole} is the input format. See the 
Magnetic fields section \ref{s:fields} for more details.

\begin{table}[h]
\centering 
{\tt
\begin{tabular}{|l|l||l|l||l|l|} \hline
  {\sl Attribute} & {\sl Sec}  & {\sl Attribute} & {\sl Sec} & {\sl Attribute} & {\sl Sec} \\ \hline
  a$n$, b$n$ = Real  &  \ref{s:fields} &  type = String                & \ref{s:string} & x\_limit = Real  & \ref{s:limit} \\ \hline
  tilt       = Real  &  \ref{s:offset} &  alias = String               & \ref{s:string} & y\_limit = Real  & \ref{s:limit} \\ \hline
  x\_offset  = Real  &  \ref{s:offset} &  descrip = String             & \ref{s:string} & aperture = Real  & \ref{s:limit} \\ \hline
  y\_offset  = Real  &  \ref{s:offset} &  mat6\_calc\_method = Switch  & \ref{s:track}  & is\_on = Logical & \ref{s:is_on} \\ \hline
  s\_offset  = Real  &  \ref{s:offset} &  tracking\_method = Switch    & \ref{s:track}  &                  &               \\ \hline
\end{tabular}
}
\end{table}

\noindent
Possible \vn{mat6_calc_method} and \vn{tracking_method} values are:
\vskip 0.01in
\begin{example}
   bmad\_standard  (default) 
\end{example}

\vskip0.2in \noindent
Example:
\begin{example}
  abc: ab_multipole, a2 = 0.034e-2, b3 = 5.7, a11 = 5.6e6/2
\end{example}

\vskip0.1in \noindent
Dependent attributes:
\begin{example}
  beam\_energy  ! See section \ref{s:energy}
\end{example}

%-----------------------------------------------------------------
\section{Accel\_Sol}
\label{s:accel_sol}

An \vn{Accel_Sol} element is a combination LINAC RF accelerating
section with a solenoid on top of it. For historical reasons this
element is not currently available but could be revived if there is
any demand for it.

%-----------------------------------------------------------------
\section{BeamBeam}
\label{s:bbi}

A \vn{BeamBeam} element simulates an interaction with an opposing
(``strong'') beam traveling in the opposite direction (a
``weak--strong beam--beam Interaction''). The strong beam is assumed
to be Gaussian in shape.

\begin{table}[h]
\centering {
\begin{tabular}{|l|l||l|l||l|l|} \hline
  {\sl Attribute} & {\sl Section}  & {\sl Attribute} & {\sl Section} & {\sl Attribute} & {\sl Section} \\ \hline
  sig\_x   = Real       &                 &  type = String                & \ref{s:string} & tilt = Real        &  \ref{s:offset}  \\ \hline
  sig\_y   = Real       &                 &  alias = String               & \ref{s:string} & x\_offset  = Real  &  \ref{s:offset}  \\ \hline
  sig\_z   = Real       &                 &  descrip = String             & \ref{s:string} & y\_offset  = Real  &  \ref{s:offset}  \\ \hline
  charge   = Real       &                 &  mat6\_calc\_method = Switch  & \ref{s:method} & s\_offset  = Real  &  \ref{s:offset}  \\ \hline
  n\_slice = Integer    &                 &  tracking\_method = Switch    & \ref{s:method} & x\_pitch = Real    &  \ref{s:offset}  \\ \hline
  symplectify = Logical & \ref{s:symp}    &  x\_limit = Real              & \ref{s:limit}  & y\_pitch = Real    &  \ref{s:offset}  \\ \hline
  is\_on = Logical      & \ref{s:is_on}   &  y\_limit = Real              & \ref{s:limit}  &                    &                  \\ \hline
                        &                 &  aperture = Real              & \ref{s:limit}  &                    &                  \\ \hline
\end{tabular}
}
\end{table}

The magnitude of the strong beam's charge is set by the \vn{beam}
command (see \ref{s:beam}).  The sign of the strong beam's charge is
set by the \vn{charge} attribute.  Thus if \vn{charge} = -1 then the
strong beam has the opposite charge. This is the default.

\vn{sig_x}, \vn{sig_y}, \vn{sig_z} are the strong beam sigmas. 
In \bmad, \vn{x_offset} and \vn{y_offset} are used to offset the
\vn{BeamBeam} element instead of the \mad\ standard attributes
\vn{xma} and \vn{yma}.

\vn{x_pitch} and \vn{y_pitch} gives the beam--beam interaction a
crossing angle. This is the full crossing angle, not the half-angle.

The strong beam is divided up into \vn{n_slice} equal charge (not
equal thickness) slices. The default for \vn{n_slice} is 1.
Propagation through the strong beam involves a kick at the charge
center of each slice with drifts inbetween the kicks. The kicks are
calculated using the standard Bassetti--Erskine formula.  Even though
the strong beam can have a finite \vn{sig_z} the length of the element
is always considered to be zero. The longitudinal $s$--position of the
\vn{BeamBeam} element is at the spot where the reference particle's
position coinsides with the center of the strong bunch. For example,
with \vn{n_slice} = 2 the calculation would proceed as follows:
\begin{example}
  0) Start with the reference particle at the center of the strong bunch.
  1) Propagate (drift) backwards to the center of the first slice.
  2) Apply the beam--beam kick due to the first slice.
  3) Propagate (drift) forwards to the center of the second slice.
  4) Apply the beam--beam kick due to the second slice.
  5) Propagate (drift) backwards to end up with the reference particle
     at the center of the strong bunch.
\end{example}

\vskip0.2in \noindent
Possible \vn{mat6_calc_method} and \vn{tracking_method} values are:
\vskip 0.01in
\begin{example}
   bmad\_standard  (default) 
\end{example}

\vskip0.2in \noindent
Example:
\begin{example}
  bbi: beambeam, sig\_x = 3e-3, sig\_y = 3e-4, x\_offset = 0.05
\end{example}

\vskip0.2in \noindent
Dependent attributes:
\begin{example}
  beam\_energy  ! See section \ref{s:energy}
  bbi\_constant 
\end{example}
\vn{bbi_constant} = $N \, m_e \, r_e / (2 \, \pi \, \gamma \, (\sigma_x + \sigma_y))$ 
is a measure of the beam--beam interaction strungth. For example,
in the linear region near $x = y = 0$ the horizontal component of the
beam--beam kick is approximately 
$k_x = -4\, \pi \, x \, \mbox{bbi\_constant} / \sigma_x$ and the
horizontal beam--beam tune shift is 
$dQ_x = \mbox{bbi\_constant} \, \beta_x / \sigma_x$.

%-----------------------------------------------------------------
\section{Custom}
\label{s:custom}

A \vn{Custom} element is an element whose properites are defined
outside of the standard \bmad\ subroutine library. That is, to use a
custom element some programmer must write the appropriate custom
routines which are then linked with the \bmad\ subroutines into a
program. \bmad\ will call the custom routines at the appropriate time
to do tracking and transfer matrix calculations. See the programmer
who wrote the custom routines for more details!

\begin{table}[h]
\centering {
\begin{tabular}{|l|l||l|l||l|l|} \hline
  {\sl Attribute} & {\sl Section}  & {\sl Attribute} & {\sl Section} & {\sl Attribute} & {\sl Section} \\ \hline
  l        = Real       & \ref{s:l}       &  type = String                & \ref{s:string} & x\_offset  = Real  &  \ref{s:offset}  \\ \hline
  val$n$, $n$ = 1 - 12 = Real &           &  alias = String               & \ref{s:string} & y\_offset  = Real  &  \ref{s:offset}  \\ \hline
  rel\_tol = Real       & \ref{s:tol}     &  descrip = String             & \ref{s:string} & s\_offset  = Real  &  \ref{s:offset}  \\ \hline
  abs\_tol = Real       & \ref{s:tol}     &  mat6\_calc\_method = Switch  & \ref{s:method} & x\_pitch = Real    &  \ref{s:offset}  \\ \hline
  num\_steps = Integer  & \ref{s:tol}     &  tracking\_method = Switch    & \ref{s:method} & y\_pitch = Real    &  \ref{s:offset}  \\ \hline
  symplectify = Logical & \ref{s:symp}    &  x\_limit = Real              & \ref{s:limit}  & tilt = Real        &  \ref{s:tilt}    \\ \hline
  is\_on = Logical      & \ref{s:is_on}   &  y\_limit = Real              & \ref{s:limit}  & integration\_ord   &  \ref{s:ord}     \\ \hline
                        &                 &  aperture = Real              & \ref{s:limit}  &                    &                  \\ \hline
\end{tabular}
}
\end{table}

\vskip0.2in \noindent
Possible \vn{mat6_calc_method} and \vn{tracking_method} values are:
\vskip 0.01in
\begin{example}
  custom  (default)
  runge\_kutta
  boris
\end{example}

\vskip0.2in \noindent
Example:
\begin{example}
  c1: custom, l = 3, v4 = 5.6, v12 = 0.9, num_steps = 12, tracking_method = boris
\end{example}

\vskip0.2in \noindent
Dependent attributes:
\begin{example}
  beam\_energy  ! See section \ref{s:energy}
\end{example}


%-----------------------------------------------------------------
\section{Drift}
\label{s:drift}

A \vn{Drift} element is just a space free and clear.

\begin{table}[h]
\centering {
\begin{tabular}{|l|l||l|l||l|l|} \hline
  {\sl Attribute} & {\sl Section}  & {\sl Attribute} & {\sl Section} & {\sl Attribute} & {\sl Section} \\ \hline
  l        = Real       & \ref{s:l}       &  type = String                & \ref{s:string} & x\_limit = Real              & \ref{s:limit}  & 
                        &                 &  alias = String               & \ref{s:string} & y\_limit = Real              & \ref{s:limit}  & 
  rel\_tol = Real       & \ref{s:tol}     &  descrip = String             & \ref{s:string} & aperture = Real              & \ref{s:limit}  & 
  abs\_tol = Real       & \ref{s:tol}     &  mat6\_calc\_method = Switch  & \ref{s:method} & symplectify = Logical & \ref{s:symp}    &  
  num\_steps = Integer  & \ref{s:tol}     &  tracking\_method = Switch    & \ref{s:method} & integration\_ord = Integer & \ref{s:int}&  
                        &                 &  
\end{tabular}
}
\end{table}

\vskip0.2in \noindent
Possible \vn{mat6_calc_method} and \vn{tracking_method} values are:
\vskip 0.01in
\begin{example}
  bmad\_standard  (default)
  symp\_lie\_ptc
  taylor
\end{example}

\vskip0.2in \noindent
Example:
\begin{example}
  d21: drift, l = 4.5
\end{example}

\vskip0.2in \noindent
Dependent attributes:
\begin{example}
  beam\_energy  ! See section \ref{s:energy}
\end{example}


%-----------------------------------------------------------------
\section{Ecollimator and Rcollimator}
\label{s:col}

An \vn{Ecollimator} is a drift with elliptic collimation.
A \vn{Rcollimator} is a drift with rectangular collimation.
The aperture is considered to be at the end edge of the element.

\begin{table}[h]
\centering {
\begin{tabular}{|l|l||l|l||l|l|} \hline
  {\sl Attribute} & {\sl Section}  & {\sl Attribute} & {\sl Section} & {\sl Attribute} & {\sl Section} \\ \hline
  l        = Real       & \ref{s:l}       &  type = String                & \ref{s:string} & x\_offset  = Real  & \ref{s:offset}  \\ \hline
                        &                 &  alias = String               & \ref{s:string} & y\_offset  = Real  & \ref{s:offset}  \\ \hline
  rel\_tol = Real       & \ref{s:tol}     &  descrip = String             & \ref{s:string} & s\_offset  = Real  & \ref{s:offset}  \\ \hline
  abs\_tol = Real       & \ref{s:tol}     &  mat6\_calc\_method = Switch  & \ref{s:method} & tilt = Real        & \ref{s:tilt}  \\ \hline
  num\_steps = Integer  & \ref{s:tol}     &  tracking\_method = Switch    & \ref{s:method} & aperture = Real    & \ref{s:limit}   \\ \hline
  symplectify = Logical & \ref{s:symp}    &  x\_limit = Real              & \ref{s:limit}  &                    &                 \\ \hline
  integration\_ord = Integer & \ref{s:int}&  y\_limit = Real              & \ref{s:limit}  &                    &                 \\ \hline

\end{tabular}
}
\end{table}
\vskip0.2in \noindent
Possible \vn{mat6_calc_method} and \vn{tracking_method} values are:
\vskip 0.01in
\begin{example}
  bmad\_standard  (default)
  symp\_lie\_ptc
  taylor
\end{example}

\vskip0.2in \noindent
Example:
\begin{example}
  d21: ecollimator, l = 4.5, x_limit = 0.09/2, y_limit = 0.05/2
\end{example}

\vskip0.2in \noindent
Dependent attributes:
\begin{example}
  beam\_energy  ! See section \ref{s:energy}
\end{example}


%-----------------------------------------------------------------
\section{Elseperator}
\label{s:elsep}

A \vn{ElSeperator} is an electrostatic separator.

\begin{table}[h]
\centering {
\begin{tabular}{|l|l||l|l||l|l|} \hline
  {\sl Attribute} & {\sl Section}  & {\sl Attribute} & {\sl Section} & {\sl Attribute} & {\sl Section} \\ \hline
  l        = Real       & \ref{s:l}       & type = String                & \ref{s:string} & x\_offset  = Real  & \ref{s:offset}  \\ \hline
  hkick    = Real       & \ref{s:kick}    & alias = String               & \ref{s:string} & y\_offset  = Real  & \ref{s:offset}  \\ \hline
  vkick    = Real       & \ref{s:kick}    & descrip = String             & \ref{s:string} & s\_offset  = Real  & \ref{s:offset}  \\ \hline
  gap      = Real       &                 & mat6\_calc\_method = Switch  & \ref{s:method} & x\_pitch = Real    & \ref{s:offset}  \\ \hline
  tilt     = Real       & \ref{s:tilt}    & tracking\_method = Switch    & \ref{s:method} & y\_pitch = Real    & \ref{s:offset}  \\ \hline
  rel\_tol = Real       & \ref{s:tol}     & x\_limit = Real              & \ref{s:limit}  &                    &                 \\ \hline
  abs\_tol = Real       & \ref{s:tol}     & y\_limit = Real              & \ref{s:limit}  & is\_on             & \ref{s:is_on}   \\ \hline
  num\_steps = Integer  & \ref{s:tol}     & aperture = Real              & \ref{s:limit}  & a$n$, b$n$         & \ref{s:ab}      \\ \hline
  integration\_ord = Integer & \ref{s:int}& symplectify = Logical        & \ref{s:symp}   & radius             & \ref{s:ab}      \\ \hline
\end{tabular}
}
\end{table}

For an \vn{Elseparator}, the kick is determined by \vn{hkick} and
\vn{vkick}. The \vn{gap} for an \vn{Elseparator} is used to compute
the electric field for a given kick. The voltage is a dependent
attribute determined by:
\begin{example}
  e\_field (V/m) = sqrt(hkick$^2$ + vkick$^2$) * beam\_energy / L
  volt (V) = e\_field * gap  
\end{example}


\vskip0.2in \noindent
Possible \vn{mat6_calc_method} and \vn{tracking_method} values are:
\vskip 0.01in
\begin{example}
  bmad\_standard  (default)
  symp\_lie\_ptc
  taylor
\end{example}

\vskip0.2in \noindent
Example:
\begin{example}
  h_sep: elsep, l = 4.5, hkick = 0.003, gap = 0.11
\end{example}

\vskip0.2in \noindent
Dependent attributes:
\begin{example}
  beam\_energy  ! See section \ref{s:energy}
  e_field
  volt
\end{example}

%-----------------------------------------------------------------
\section{Hkicker and Vkicker}
\label{s:hvkicker}

A \vn{Hkicker} is a horizontal bend and a vn{Vkicker} is a vertical
bend.  Note that \vn{Hkicker} and \vn{Vkicker} elements use the
\vn{kick} attribute while a \vn{kicker} uses the \vn{hkick} and \vn{vkick} 
attributes.

\begin{table}[h]
\centering {
\begin{tabular}{|l|l||l|l||l|l|} \hline
  {\sl Attribute} & {\sl Section} & {\sl Attribute} & {\sl Section} &  {\sl Attribute} & {\sl Section} \\ \hline
  l        = Real       & \ref{s:l}       & type = String                & \ref{s:string} & x\_limit = Real       & \ref{s:limit}  \\ \hline
  kick     = Real       & \ref{s:kick}    & alias = String               & \ref{s:string} & y\_limit = Real       & \ref{s:limit}  \\ \hline
  tilt     = Real       & \ref{s:tilt}    & descrip = String             & \ref{s:string} & aperture = Real       & \ref{s:limit}  \\ \hline
  rel\_tol = Real       & \ref{s:tol}     & mat6\_calc\_method = Switch  & \ref{s:method} & is\_on                & \ref{s:is_on}  \\ \hline 
  abs\_tol = Real       & \ref{s:tol}     & tracking\_method = Switch    & \ref{s:method} & symplectify = Logical & \ref{s:symp}   \\ \hline
  num\_steps = Integer  & \ref{s:tol}     & integration\_ord = Integer   & \ref{s:int}    &                       &                \\ \hline
\end{tabular}
}
\end{table}

\vskip0.2in \noindent
Example:
\begin{example}
  h_kick: hkicker, l = 4.5, kick = 0.003
\end{example}

\vskip0.2in \noindent
Dependent attributes:
\begin{example}
  beam\_energy  ! See section \ref{s:energy}
\end{example}

%-----------------------------------------------------------------
\section{Hybrid}
\label{s:hybrid}

A \vn{Hybrid} element is an element that is formed by concatinating
other element together. \vn{Hybrid} elements are not part of the input
lattice file but are created by a program, usually for speed purposes.

%-----------------------------------------------------------------
\section{Instrument and Monitor}
\label{s:monitor}

\bmad\ treats \vn{Instrument} and \vn{Monitor} elements exactly like
a drift.

\begin{table}[h]
\centering {
\begin{tabular}{|l|l||l|l||l|l|} \hline
  {\sl Attribute} & {\sl Section}  & {\sl Attribute} & {\sl Section} & {\sl Attribute} & {\sl Section} \\ \hline
  l        = Real       & \ref{s:l}       &  type = String                & \ref{s:string} & x\_limit = Real              & \ref{s:limit}  \\ \hline 
                        &                 &  alias = String               & \ref{s:string} & y\_limit = Real              & \ref{s:limit}  \\ \hline 
  rel\_tol = Real       & \ref{s:tol}     &  descrip = String             & \ref{s:string} & aperture = Real              & \ref{s:limit}  \\ \hline 
  abs\_tol = Real       & \ref{s:tol}     &  mat6\_calc\_method = Switch  & \ref{s:method} & symplectify = Logical        & \ref{s:symp}   \\ \hline  
  num\_steps = Integer  & \ref{s:tol}     &  tracking\_method = Switch    & \ref{s:method} & integration\_ord = Integer   & \ref{s:int}    \\ \hline  
\end{tabular}
}
\end{table}

\vskip0.2in \noindent
Possible \vn{mat6_calc_method} and \vn{tracking_method} values are:
\vskip 0.01in
\begin{example}
  bmad\_standard  (default)
  symp\_lie\_ptc
  taylor
\end{example}

\vskip0.2in \noindent
Example:
\begin{example}
  d21: instr, l = 4.5
\end{example}

\vskip0.2in \noindent
Dependent attributes:
\begin{example}
  beam\_energy  ! See section \ref{s:energy}
\end{example}

%-----------------------------------------------------------------
\section{kicker}
\label{s:kicker}

A \vn{Kicker} can deflect a beam in both planes. Note that a
\vn{Kicker} uses the \vn{hkick} and \vn{vkick} attributes while
\vn{Hkicker} and \vn{Vkicker} elements use the \vn{kick} attrubte. 
In addition a \vn{Kicker} can apply a displacement using the
\vn{h_displace} and \vn{v_displace} attributes.

\begin{table}[h]
\centering {
\begin{tabular}{|l|l||l|l||l|l|} \hline
  {\sl Attribute} & {\sl Section} & {\sl Attribute} & {\sl Section} &  {\sl Attribute} & {\sl Section} \\ \hline
  l        = Real       & \ref{s:l}       & type = String                & \ref{s:string} & x\_limit = Real       & \ref{s:limit}  \\ \hline
  hkick    = Real       & \ref{s:kick}    & alias = String               & \ref{s:string} & y\_limit = Real       & \ref{s:limit}  \\ \hline
  vkick    = Real       & \ref{s:kick}    & descrip = String             & \ref{s:string} & aperture = Real       & \ref{s:limit}  \\ \hline
  tilt     = Real       & \ref{s:tilt}    & mat6\_calc\_method = Switch  & \ref{s:method} & rel\_tol = Real       & \ref{s:tol}    \\ \hline 
  h\_displace = Real    & \ref{s:disp}    & tracking\_method = Switch    & \ref{s:method} & abs\_tol = Real       & \ref{s:tol}    \\ \hline
  v\_displace = Real    & \ref{s:disp}    & integration\_ord = Integer   & \ref{s:int}    & num\_steps = Integer  & \ref{s:tol}    \\ \hline
  symplectify = Logical & \ref{s:symp}    & is\_on                       & \ref{s:is_on}  & a$n$, b$n$            & \ref{s:ab}     \\ \hline
                        &                 &                              &                & radius                & \ref{s:ab}     \\ \hline
\end{tabular}
}
\end{table}

\vskip0.2in \noindent
Possible \vn{mat6_calc_method} and \vn{tracking_method} values are:
\vskip 0.01in
\begin{example}
  bmad\_standard (default)
  symp\_lie\_ptc
  taylor
\end{example}

\vskip0.2in \noindent
Example:
\begin{example}
  a_kick: kicker, l = 4.5, hkick = 0.003
\end{example}

\vskip0.2in \noindent
Dependent attributes:
\begin{example}
  beam\_energy  ! See section \ref{s:energy}
\end{example}

%-----------------------------------------------------------------
\section{Lcavity}
\label{s:lcav}

An \vn{Lcavity} is a LINAC accelerating cavity.

\begin{table}[h]
\centering {
\begin{tabular}{|l|l||l|l||l|l|} \hline
  {\sl Attribute} & {\sl Section}  & {\sl Attribute} & {\sl Section} & {\sl Attribute} & {\sl Section} \\ \hline
  l        = Real       & \ref{s:l}       & type = String                & \ref{s:string} & x\_offset  = Real  & \ref{s:offset}  \\ \hline
  rf\_frequency = Real  &                 & alias = String               & \ref{s:string} & y\_offset  = Real  & \ref{s:offset}  \\ \hline
  gradiant      = Real  &                 & descrip = String             & \ref{s:string} & s\_offset  = Real  & \ref{s:offset}  \\ \hline
  phi0          = Real  &                 & mat6\_calc\_method = Switch  & \ref{s:method} & x\_pitch = Real    & \ref{s:offset}  \\ \hline
  e_loss        = Real  &                 & tracking\_method = Switch    & \ref{s:method} & y\_pitch = Real    & \ref{s:offset}  \\ \hline
  hkick         = Real  & \ref{s:kick}    & x\_limit = Real              & \ref{s:limit}  & is\_on             & \ref{s:is_on}   \\ \hline
  vkick         = Real  & \ref{s:kick}    & y\_limit = Real              & \ref{s:limit}  & sr\_wake\_file = String &            \\ \hline
  rel\_tol = Real       & \ref{s:tol}     & aperture = Real              & \ref{s:limit}  & lr\_wake\_file = String &            \\ \hline
  abs\_tol = Real       & \ref{s:tol}     & symplectify = Logical        & \ref{s:symp}   &                         &            \\ \hline
  num\_steps = Integer  & \ref{s:tol}     & integration\_ord = Integer   & \ref{s:int}    &                         &            \\ \hline

\end{tabular}
}
\end{table}


\vskip0.2in \noindent
Possible \vn{mat6_calc_method} and \vn{tracking_method} values are:
\vskip 0.01in
\begin{example}
  bmad\_standard (default)
\end{example}

\vskip0.2in \noindent
Example:
\begin{example}
  rf1: lcav, l = 4.5, gradiant = 1.2e6, sr\_wake\_file = "sr1.dat"
\end{example}

\vskip0.2in \noindent
Dependent attributes:
\begin{example}
  energy\_start
  beam\_energy  ! See section \ref{s:energy}
  e\_loss       ! If sr_\wake\_file is present
  delta\_e
\end{example}

%-----------------------------------------------------------------
\section{Marker}
\label{s:mark}

A \vn{Marker} is a zero length element meant to mark a position

\begin{table}[h]
\centering {
\begin{tabular}{|l|l||l|l||l|l|} \hline
  {\sl Attribute} & {\sl Section}  & {\sl Attribute} & {\sl Section} & {\sl Attribute} & {\sl Section} \\ \hline
                               & \ref{s:l}       &  type = String      & \ref{s:string} & x\_limit = Real  & \ref{s:limit} \\ \hline 
  mat6\_calc\_method = Switch  & \ref{s:method}  &  alias = String     & \ref{s:string} & y\_limit = Real  & \ref{s:limit} \\ \hline 
  tracking\_method = Switch    & \ref{s:method}  &  descrip = String   & \ref{s:string} & aperture = Real  & \ref{s:limit} \\ \hline 
\end{tabular}
}
\end{table}

\vskip0.2in \noindent
Possible \vn{mat6_calc_method} and \vn{tracking_method} values are:
\vskip 0.01in
\begin{example}
  bmad\_standard (default)
\end{example}

\vskip0.2in \noindent
Example:
\begin{example}
  mm: mark, type = "BPM"
\end{example}

\vskip0.2in \noindent
Dependent attributes:
\begin{example}
  beam\_energy  ! See section \ref{s:energy}
\end{example}

%-----------------------------------------------------------------
\section{Multipole}
\label{s:mult}

A \vn{Multipole} is a thin multipole lens up to 20th order. The only
difference between this and an \vn{AB_Multipole} is the input format. See the 
Magnetic fields section \ref{s:fields} for more details.

\begin{table}[h]
\centering 
{\tt
\begin{tabular}{|l|l||l|l||l|l|} \hline
  {\sl Attribute} & {\sl Sec}  & {\sl Attribute} & {\sl Sec} & {\sl Attribute} & {\sl Sec} \\ \hline
  K$n$L, T$n$ = Real &  \ref{s:fields} &  type = String                & \ref{s:string} & x\_limit = Real  & \ref{s:limit} \\ \hline
  tilt       = Real  &  \ref{s:offset} &  alias = String               & \ref{s:string} & y\_limit = Real  & \ref{s:limit} \\ \hline
  x\_offset  = Real  &  \ref{s:offset} &  descrip = String             & \ref{s:string} & aperture = Real  & \ref{s:limit} \\ \hline
  y\_offset  = Real  &  \ref{s:offset} &  mat6\_calc\_method = Switch  & \ref{s:track}  & is\_on = Logical & \ref{s:is_on} \\ \hline
  s\_offset  = Real  &  \ref{s:offset} &  tracking\_method = Switch    & \ref{s:track}  &                  &               \\ \hline
\end{tabular}
}
\end{table}

\noindent
Possible \vn{mat6_calc_method} and \vn{tracking_method} values are:
\vskip 0.01in
\begin{example}
   bmad\_standard  (default) 
\end{example}

\vskip0.2in \noindent
Example:
\begin{example}
  m1: multipole, k1l = 0.034e-2, t1
\end{example}

\vskip0.1in \noindent
Dependent attributes:
\begin{example}
  beam\_energy  ! See section \ref{s:energy}
\end{example}


%-----------------------------------------------------------------
\section{Octupole}
\label{s:oct}

An \vn{Octupole} has a cubic field dependence.

\begin{table}[h]
\centering {
\begin{tabular}{|l|l||l|l||l|l|} \hline
  {\sl Attribute} & {\sl Section}  & {\sl Attribute} & {\sl Section} & {\sl Attribute} & {\sl Section} \\ \hline
  l        = Real       & \ref{s:l}       & type = String                & \ref{s:string} & x\_offset  = Real  & \ref{s:offset}      \\ \hline
  k3       = Real       &                 & alias = String               & \ref{s:string} & y\_offset  = Real  & \ref{s:offset}      \\ \hline
  b\_gradiant = Real    &                 & descrip = String             & \ref{s:string} & s\_offset  = Real  & \ref{s:offset}      \\ \hline
  tilt     = Real       & \ref{s:tilt}    & mat6\_calc\_method = Switch  & \ref{s:method} & x\_pitch = Real    & \ref{s:offset}      \\ \hline
  hkick    = Real       & \ref{s:kick}    & tracking\_method = Switch    & \ref{s:method} & y\_pitch = Real    & \ref{s:offset}      \\ \hline
  vkick    = Real       & \ref{s:kick}    & x\_limit = Real              & \ref{s:limit}  & integration\_ord = Integer & \ref{s:int} \\ \hline
  rel\_tol = Real       & \ref{s:tol}     & y\_limit = Real              & \ref{s:limit}  & is\_on             & \ref{s:is_on}       \\ \hline
  abs\_tol = Real       & \ref{s:tol}     & aperture = Real              & \ref{s:limit}  & a$n$, b$n$         & \ref{s:ab}          \\ \hline
  num\_steps = Integer  & \ref{s:tol}     & symplectify = Logical        & \ref{s:symp}   & radius             & \ref{s:ab}          \\ \hline
\end{tabular}
}
\end{table}

\vskip0.2in \noindent
Possible \vn{mat6_calc_method} and \vn{tracking_method} values are:
\vskip 0.01in
\begin{example}
  bmad\_standard  (default)
  symp\_lie\_ptc
  taylor
\end{example}

\vskip0.2in \noindent
Example:
\begin{example}
  oct1: octopole, l = 4.5, k3 = 0.003
\end{example}

\vskip0.2in \noindent
Dependent attributes:
\begin{example}
  beam\_energy  ! See section \ref{s:energy}
  b\_gradiant or k3
\end{example}


%-----------------------------------------------------------------
\section{Patch}
\label{s:patch}

A \vn{Patch} offsets the reference orbit.

\begin{tabular}{|l|l||l|l||l|l|} \hline
  {\sl Attribute} & {\sl Sec}  & {\sl Attribute} & {\sl Sec} & {\sl Attribute} & {\sl Sec} \\ \hline
  x\_offset  = Real  & \ref{s:offset} &  type = String                & \ref{s:string} & x\_limit = Real  & \ref{s:limit} \\ \hline
  y\_offset  = Real  & \ref{s:offset} &  alias = String               & \ref{s:string} & y\_limit = Real  & \ref{s:limit} \\ \hline
  z\_offset  = Real  & \ref{s:offset} &  descrip = String             & \ref{s:string} & aperture = Real  & \ref{s:limit} \\ \hline
  x\_pitch   = Real  &                &  mat6\_calc\_method = Switch  & \ref{s:track}  & is\_on = Logical & \ref{s:is_on} \\ \hline
  y\_pitch   = Real  &                &  tracking\_method = Switch    & \ref{s:track}  &                  &               \\ \hline
  de\_offset = Real  &                &  tilt = Real                  &                &                  &               \\ \hline            
\end{tabular}
}
\end{table}

\noindent
Possible \vn{mat6_calc_method} and \vn{tracking_method} values are:
\vskip 0.01in
\begin{example}
   bmad\_standard  (default) 
\end{example}

\vskip0.2in \noindent
Example:
\begin{example}
  pt: patch, x\_offset = 3.2
\end{example}

\vskip0.1in \noindent
Dependent attributes:
\begin{example}
  beam\_energy  ! See section \ref{s:energy}
\end{example}




%-----------------------------------------------------------------
\section{Quadrupole}
\label{s:quad}

An \vn{Octupole} has a cubic field dependence.

\begin{table}[h]
\centering {
\begin{tabular}{|l|l||l|l||l|l|} \hline
  {\sl Attribute} & {\sl Section}  & {\sl Attribute} & {\sl Section} & {\sl Attribute} & {\sl Section} \\ \hline
  l        = Real       & \ref{s:l}       & type = String                & \ref{s:string} & x\_offset  = Real  & \ref{s:offset}      \\ \hline
  k3       = Real       &                 & alias = String               & \ref{s:string} & y\_offset  = Real  & \ref{s:offset}      \\ \hline
  b\_gradiant = Real    &                 & descrip = String             & \ref{s:string} & s\_offset  = Real  & \ref{s:offset}      \\ \hline
  tilt     = Real       & \ref{s:tilt}    & mat6\_calc\_method = Switch  & \ref{s:method} & x\_pitch = Real    & \ref{s:offset}      \\ \hline
  hkick    = Real       & \ref{s:kick}    & tracking\_method = Switch    & \ref{s:method} & y\_pitch = Real    & \ref{s:offset}      \\ \hline
  vkick    = Real       & \ref{s:kick}    & x\_limit = Real              & \ref{s:limit}  & integration\_ord = Integer & \ref{s:int} \\ \hline
  rel\_tol = Real       & \ref{s:tol}     & y\_limit = Real              & \ref{s:limit}  & is\_on             & \ref{s:is_on}       \\ \hline
  abs\_tol = Real       & \ref{s:tol}     & aperture = Real              & \ref{s:limit}  & a$n$, b$n$         & \ref{s:ab}          \\ \hline
  num\_steps = Integer  & \ref{s:tol}     & symplectify = Logical        & \ref{s:symp}   & radius             & \ref{s:ab}          \\ \hline
\end{tabular}
}
\end{table}

\vskip0.2in \noindent
Possible \vn{mat6_calc_method} and \vn{tracking_method} values are:
\vskip 0.01in
\begin{example}
  bmad\_standard  (default)
  symp\_lie\_ptc
  taylor
\end{example}

\vskip0.2in \noindent
Example:
\begin{example}
  q03w: quad, l = 0.6, k1 = 0.003, tilt
\end{example}

\vskip0.2in \noindent
Dependent attributes:
\begin{example}
  beam\_energy  ! See section \ref{s:energy}
  b\_gradiant or k3
\end{example}

%-----------------------------------------------------------------
\section{Rbend and Sbend}
\label{s:bend}



%-----------------------------------------------------------------
\section{Rfcavity}
\label{s:rfcav}


%-----------------------------------------------------------------
\section{Sextupole}
\label{s:sex}
A \vn{Sextupole} has a quadratic field dependence.

\begin{table}[h]
\centering {
\begin{tabular}{|l|l||l|l||l|l|} \hline
  {\sl Attribute} & {\sl Section}  & {\sl Attribute} & {\sl Section} & {\sl Attribute} & {\sl Section} \\ \hline
  l        = Real       & \ref{s:l}       & type = String                & \ref{s:string} & x\_offset  = Real  & \ref{s:offset}      \\ \hline
  k2       = Real       &                 & alias = String               & \ref{s:string} & y\_offset  = Real  & \ref{s:offset}      \\ \hline
  b\_gradiant = Real    &                 & descrip = String             & \ref{s:string} & s\_offset  = Real  & \ref{s:offset}      \\ \hline
  tilt     = Real       & \ref{s:tilt}    & mat6\_calc\_method = Switch  & \ref{s:method} & x\_pitch = Real    & \ref{s:offset}      \\ \hline
  hkick    = Real       & \ref{s:kick}    & tracking\_method = Switch    & \ref{s:method} & y\_pitch = Real    & \ref{s:offset}      \\ \hline
  vkick    = Real       & \ref{s:kick}    & x\_limit = Real              & \ref{s:limit}  & integration\_ord = Integer & \ref{s:int} \\ \hline
  rel\_tol = Real       & \ref{s:tol}     & y\_limit = Real              & \ref{s:limit}  & is\_on             & \ref{s:is_on}       \\ \hline
  abs\_tol = Real       & \ref{s:tol}     & aperture = Real              & \ref{s:limit}  & a$n$, b$n$         & \ref{s:ab}          \\ \hline
  num\_steps = Integer  & \ref{s:tol}     & symplectify = Logical        & \ref{s:symp}   & radius             & \ref{s:ab}          \\ \hline
\end{tabular}
}
\end{table}

\vskip0.2in \noindent
Possible \vn{mat6_calc_method} and \vn{tracking_method} values are:
\vskip 0.01in
\begin{example}
  bmad\_standard  (default)
  symp\_lie\_ptc
  taylor
\end{example}

\vskip0.2in \noindent
Example:
\begin{example}
  q03w: quad, l = 0.6, k1 = 0.003, tilt
\end{example}

\vskip0.2in \noindent
Dependent attributes:
\begin{example}
  beam\_energy  ! See section \ref{s:energy}
  b\_gradiant or k3
\end{example}


%-----------------------------------------------------------------
\section{Solenoid}
\label{s:sol}

\begin{table}[h]
\centering {
\begin{tabular}{|l|l||l|l||l|l|} \hline
  {\sl Attribute} & {\sl Section}  & {\sl Attribute} & {\sl Section} & {\sl Attribute} & {\sl Section} \\ \hline
  l        = Real       & \ref{s:l}       & type = String                & \ref{s:string} & x\_offset  = Real  & \ref{s:offset}      \\ \hline
  ks       = Real       &                 & alias = String               & \ref{s:string} & y\_offset  = Real  & \ref{s:offset}      \\ \hline
  b\_field = Real       &                 & descrip = String             & \ref{s:string} & s\_offset  = Real  & \ref{s:offset}      \\ \hline
                        &                 & mat6\_calc\_method = Switch  & \ref{s:method} & x\_pitch = Real    & \ref{s:offset}      \\ \hline
  hkick    = Real       & \ref{s:kick}    & tracking\_method = Switch    & \ref{s:method} & y\_pitch = Real    & \ref{s:offset}      \\ \hline
  vkick    = Real       & \ref{s:kick}    & x\_limit = Real              & \ref{s:limit}  & integration\_ord = Integer & \ref{s:int} \\ \hline
  rel\_tol = Real       & \ref{s:tol}     & y\_limit = Real              & \ref{s:limit}  & is\_on             & \ref{s:is_on}       \\ \hline
  abs\_tol = Real       & \ref{s:tol}     & aperture = Real              & \ref{s:limit}  & a$n$, b$n$         & \ref{s:ab}          \\ \hline
  num\_steps = Integer  & \ref{s:tol}     & symplectify = Logical        & \ref{s:symp}   & radius             & \ref{s:ab}          \\ \hline
\end{tabular}
}
\end{table}

\vskip0.2in \noindent
Possible \vn{mat6_calc_method} and \vn{tracking_method} values are:
\vskip 0.01in
\begin{example}
  bmad\_standard  (default)
  symp\_lie\_ptc
  taylor
\end{example}

\vskip0.2in \noindent
Example:
\begin{example}
  cleo_sol: solenoid, l = 2.6, ks = 1.5*beam[energy]
\end{example}

\vskip0.2in \noindent
Dependent attributes:
\begin{example}
  beam\_energy  ! See section \ref{s:energy}
  b\_field or ks
\end{example}

%-----------------------------------------------------------------
\section{Sol\_Quad}
\label{s:sq}

\begin{table}[h]
\centering {
\begin{tabular}{|l|l||l|l||l|l|} \hline
  {\sl Attribute} & {\sl Section}  & {\sl Attribute} & {\sl Section} & {\sl Attribute} & {\sl Section} \\ \hline
  l        = Real       & \ref{s:l}       & type = String                & \ref{s:string} & x\_offset  = Real  & \ref{s:offset}      \\ \hline
  ks       = Real       &                 & alias = String               & \ref{s:string} & y\_offset  = Real  & \ref{s:offset}      \\ \hline
  k1       = Real       &                 & descrip = String             & \ref{s:string} & s\_offset  = Real  & \ref{s:offset}      \\ \hline
  tilt     = Real       &                 & mat6\_calc\_method = Switch  & \ref{s:method} & x\_pitch = Real    & \ref{s:offset}      \\ \hline
  hkick    = Real       & \ref{s:kick}    & tracking\_method = Switch    & \ref{s:method} & y\_pitch = Real    & \ref{s:offset}      \\ \hline
  vkick    = Real       & \ref{s:kick}    & x\_limit = Real              & \ref{s:limit}  & integration\_ord = Integer & \ref{s:int} \\ \hline
  rel\_tol = Real       & \ref{s:tol}     & y\_limit = Real              & \ref{s:limit}  & is\_on             & \ref{s:is_on}       \\ \hline
  abs\_tol = Real       & \ref{s:tol}     & aperture = Real              & \ref{s:limit}  & a$n$, b$n$         & \ref{s:ab}          \\ \hline
  num\_steps = Integer  & \ref{s:tol}     & symplectify = Logical        & \ref{s:symp}   & radius             & \ref{s:ab}          \\ \hline
\end{tabular}
}
\end{table}

\vskip0.2in \noindent
Possible \vn{mat6_calc_method} and \vn{tracking_method} values are:
\vskip 0.01in
\begin{example}
  bmad\_standard  (default)
  symp\_lie\_ptc
  taylor
\end{example}

\vskip0.2in \noindent
Example:
\begin{example}
  sq02: sol_quad, l = 2.6, k1 = 0.632, ks = 1.5*beam[energy]
\end{example}

\vskip0.2in \noindent
Dependent attributes:
\begin{example}
  beam\_energy  ! See section \ref{s:energy}
  b\_field or ks
\end{example}


%-----------------------------------------------------------------
\section{Taylor}
\label{s:tay}

A \vn{Taylor} is a Taylor map. This can be used in place of the \mad\ 
\vn{matrix} element.

%-----------------------------------------------------------------
\section{Wiggler} 
\label{s:wig}

A \vn{Wiggler} is a periodic array of alternating bends.




A \vn{Group} does not represent a physical element. Rather a
\vn{Group} element's purpose is to control the attributes of other elements.
This is akin to a knob in the control room.
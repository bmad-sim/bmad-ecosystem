\documentclass{book}
\usepackage{graphicx}
\usepackage{amsmath}
\newcommand{\sref}[1]{\S\ref{#1}}
\newcommand{\Sref}[1]{Sec.~\sref{#1}}

\newcommand{\vn}{\begingroup\catcode`\_=11 \catcode`\%=11 \dottcmd}
\newcommand\dottcmd[1]{{\usefont{T1}{lmss}{bx}{n} #1}\endgroup}

\newenvironment{example}
  {\vspace{-3.0ex} \begin{alltt}}
  {\end{alltt} \vspace{-2.5ex}}


\definecolor{light-gray}{gray}{0.95}
\lstset{backgroundcolor=\color{light-gray}}
\lstset{xleftmargin=0cm}
\lstset{framexleftmargin=0.3em}

\lstnewenvironment{Xcode}{}{}

\definecolor{lightcyan}{rgb}{0.88, 1.0, 1.0}
\newcounter{main}
\setcounter{main}{1}
\lstnewenvironment{code}[1][firstnumber=\themain,name=main]
  {\lstset{ %language=haskell,
           %columns=fullflexible,
           columns=fixed,
           basicstyle=\small\ttfamily,
           %numbers=left,
           numberstyle=\tiny\color{gray},
           backgroundcolor=\color{lightcyan},
           #1
          }
}
{\setcounter{main}{\value{lstnumber}}}



\setlength{\textwidth}{6in}
\setlength{\oddsidemargin}{0in}
\setlength{\evensidemargin}{0.54in}
\setlength{\textheight}{8.5in}
\setlength{\topmargin}{0in}

%----------------------------------------------------------------

\begin{document}

\title{The \bmad\ Reference Manual}

\date{4 July, 2003}
\maketitle

%----------------------------------------------------------------
\chapter{Overview}

\tao is an open source general purpose program for charged particle and X-ray simulations in
accelerators and storage rings. It is built on top of the \bmad toolkit (software library) which
provides the needed computational routines needed to do simulations. Essentially you can think of
\tao as a car and \bmad as the engine that powers the car. In fact \bmad powers a number of other
simulation programs but that is getting outside of the scope of this manual. 

Documention for \bmad and \tao, as well as information for downloading the code if needed is given
on the \bmad web page
\hfill\break
\hspace*{0.3in} \url{https://www.classe.cornell.edu/bmad}

\tao by itself 



\chapter{Introduction}
\label{c:introduction}

%----------------------------------------------------------------
\section{Obtaining Tao}
\index{tao!Obtaining}
\label{s:obtaining}

A \vn{Distribution} is a set of files, including \bmad and \tao source files, which are used to
build the \bmad, the \tao program, and various other simulation programs. A \vn{Release} is like a
\vn{Distribution} except that it is created on the Linux computer system at CLASSE (Cornell's
Laboratory for Accelerator-based Sciences and Education). More information can be obtained from the
\bmad web site. 

If there is no local \bmad Guru to guide you, download and setup instructions for downloading a
Distribution, environment variable setup, and building \tao is contained on the \bmad web
site and will not be covered here.

%----------------------------------------------------------------
\section{Starting and Initializing Tao}
\index{initializing!files}
\label{s:initializing}

The syntax for starting \tao is given in \Sref{s:command.line}.

Initialization occurs when \tao is started. Initialization information is stored in one or more
files as discussed in Chapter \sref{c:init}.

%%----------------------------------------------------------------
\section{Running Tao with OpenMP}
\index{openMP}
\label{s:openmp}

\vn{OpenMP} is a standard that enables programs to run calculations with multiple threads which will
reduce computation time. Certain calculations done by \tao, including beam tracking and dynamic
aperture calculations, can be run multithreaded via OpenMP if the \tao executable file has been
properly compiled.  Interested users should consult their local \bmad Guru for guidance. Note:
\vn{OpenMP} multithreading involves using multiple cores of a single machine (unlike \vn{Open MPI}
which involves multiple machines). Therefore, it is not necessary to have a cluster of machines to
use \vn{OpenMP}.

To set the number of threads when running a program compiled with \vn{OpenMP}, set the environment variable
\vn{OMP_NUM_THREADS}. Example:
\begin{example}
  export OMP_NUM_THREADS=8
\end{example}

This may also be set during Tao runtime as the global parameter \vn{n_threads}. For example:
\begin{example}
  set global n_threads = 1  ! Use only a single thread
  set global n_threads = 4  ! Use four threads
\end{example}

See \sref{s:set.global} for more information.

To the local \bmad Guru: Compiling and linking of \tao with \vn{OpenMP} is documented on the \bmad
web site. By default, \vn{OpenMP} is not enabled. Essentially, OpenMP is enabled by modifying
the \vn{dist_prefs} file before compiling and linking.

%----------------------------------------------------------------
\section{Command Line Mode and Single Mode}
\label{s:modes}

After \tao is initialized, \tao interacts with the user though the command line. \tao has two modes
for this. In \vn{command line} mode, which is the default mode, \tao waits until the the \vn{return}
key is depressed to execute a command. Command line mode is described in Chapter~\sref{c:command}. 

In \vn{single} mode, single keystrokes are interpreted as commands. \tao can be set up so that in
\vn{single mode} the pressing of certain keys increase or decrease variables. While the same effect
can be achieved in the standard \vn{line mode}, \vn{single mode} allows for quick adjustments of
variables. See Chapter~\sref{c:single} for more details.

%-----------------------------------------------------------------
\section{Lattice Calculations}
\index{lattice calculaitons}
\label{s:lat.calc.overview} 

By default \tao recalculates lattice parameters and does tracking of particles after each command.
The exception is for commands that do not change any parameter that would affect such calculations
such as the \vn{show} command. See \sref{s:lat.calc} for more details. If the recalculation takes a
significant amount of time, the recalculation may be suppressed using the \vn{set global
lattice_calc_on} command (\sref{s:set.global}) or the \vn{set universe} command
(\sref{s:set.universe}).

%-----------------------------------------------------------------
\section{Command Files and Aliases}
\index{command files}
\label{s:command.files} 

Typing repetitive commands in command line mode can become tedious. \tao has two constructs to
mitigate this: Aliases and Command Files. 

Aliases are just like aliases in Unix. See Section~\sref{s:alias} for more details.

Command files are like Unix shell scripts. A series of commands are
put in a file and then that file can be called using the \vn{call}
command (\sref{s:call}).

\tao will call a command file at startup. The default name of this startup file is \vn{tao.startup}
but this name can be changed (\sref{s:format}).

Do loops (\sref{s:do}) are allowed with the following syntax:
\begin{example}
  do <var> = <begin>, <end> \{, <step>\} 
    ...
    tao command [[<var>]]
    ...
  enddo
\end{example}
The \vn{<var>} can be used as a variable in the loop body but must be
bracketed ``[[<var>]]''.  The step size can be any integer positive or
negative but not zero.  Nested loops are allowed and command files can
be called within do loops.

\begin{example}
  do i = 1, 100
    call set_quad_misalignment [[i]] ! command file to misalign quadrupoles
    zero_quad 1e-5*2^([[i]]-1) ! Some user supplied command to zero quad number [[i]]
  enddo
\end{example}

To reduce unnecessary calculations, the logicals \vn{global%lattice_calc_on}
and \vn{global%plot_on} can be toggled from within the command file. Also 
setting \vn{global%quiet} can turn off verbose output to the terminal. Example
\begin{example}
  set global quiet = all          ! Turn off verbose output to the terminal.
  set global lattice_calc_on = F  ! Turn off lattice calculations
  set global plot_on = F          ! Turn off plot calculations
  ... do some stuff ...
  set global plot_on = T          ! Turn back on 
  set global lattice_calc_on = T  ! Turn back on
  set global quiet = off         
\end{example}
See \sref{s:globals} for more details.

A \vn{end_file} command (\sref{s:end.file}) can be used to signal the
end of the command file.

The \vn{pause} command (\sref{s:pause}) can be used to temporarily
pause the command file.


%----------------------------------------------------------------
\tableofcontents
\listoffigures
\listoftables

%----------------------------------------------------------------
\part{Physics and Conventions}
%----------------------------------------------------------------
\chapter{Coordinates}

%-----------------------------------------------------------------------------
\section{Reference Orbit}
\label{s:ref}

The ``reference orbit'' is the path of a ``reference particle'' and is
used to define a coordinate system for actual particles (whose orbits
are simulated in \bmad) as shown in Figure~\ref{f:local_coords}. At a
given time $t$ the reference particle is a distance $s = c \, t$ along
the reference orbit from the zero position of the reference orbit. The
origin of the local $(x, y, z)$ coordinate system at time $t$ is at the
reference particle with the $z$--axis tangent to the reference
orbit and pointing in the direction of the reference particle
motion. The $x$ and $y$--axes are perpendicular to the reference
orbit. If there are no vertical bends, the $y$--axis is in the
vertical direction and the $x$--axis is in the horizontal plane.

\begin{figure}[tb]
\centering
\includegraphics{local_coords.psfig}
\caption{The Local Reference System}
\label{f:local_coords}
\end{figure}

In \bmad, a lattice is comprised of a sequence of elements such as
quadrupoles, bends, rfcavities, etc. Each element has an entrance
point, an exit point, and a reference curve between them. For a bend,
the reference curve is a segment of a circular arc. For all other
elements, the reference curve is a straight line segment.  The
reference orbit is constructed by arranging the elements so that the
exit point of one element coincides with the entrance point of the
next with the reference curves forming an arc with no kinks.
The reference orbit is then the sum of the reference curves. If
not specified otherwise, the $s = 0$ position is the entrance
point of the first element.

Notice that, in a wiggler, the reference orbit, which is a straight
line, does {\em not} correspond to the orbit that any actual particle
could travel. Typically the physical entity of an element is centered
about the reference curve, However, by specifying offsets and pitches
(See Section~\ref{s:offset}), the center line of an element may be
arbitrarily oriented with respect to its reference curve. Since the
reference curve of an element is fixed to the reference curves of the
neighboring elements, setting a nonzero offset or pitch
for an element moves the physical magnet and does not affect the
reference curve. Shifting a physical magnet with respect to its
reference curve generally means that the reference curve does {\em
not} correspond to the orbit that any actual particle could travel.

%-----------------------------------------------------------------------------
\section{Global Reference System}
\label{s:global}

The global reference system describes the orientation of the reference
orbit with respect to the laboratory coordinate system.  \bmad,
following the \mad\ convention, uses a Cartesian coordinate system
$(X, Y, Z)$ for the global reference system, along with three angles
$\theta, \phi, \psi$ used to define the reference orbit's orientation
as shown in Figure~\ref{f:global_coords}. Conventionally, $Y$ is the
vertical coordinate and $(X, Z)$ are the ``floor'' coordinates.  The
three angles are defined as follows:
\begin{description}
\item[$\theta$ Azimuth angle:] Angle in the $(X, Z)$ plane 
between the $Z$--axis and the projection of the $z$--axis onto the
$(X, Z)$ plane. A positive angle of $\theta = \pi/2$ corresponds to the
projected $z$--axis pointing in the positive $X$ direction.
\item[$\phi$ Pitch (elevation) angle:] Angle between the $z$--axis 
and the $Y$--axis. A positive angle of $\phi = \pi/2$ corresponds to
the $z$--axis pointing in the positive $Y$ direction.
\item[$\psi$ Roll angle:] Angle of the $x$--axis with respect 
to the line formed by the
intersection of the $(X, Z)$ plane with the $(x, y)$ plane. A
positive $\psi$ forms a right--handed screw with the $z$--axis.
\end{description}

\begin{figure}
\centering
\includegraphics{global_coords.psfig}
\caption{The Global Reference System}
\label{f:global_coords}
\end{figure}

By default, the reference orbit's origin at $s = 0$
coincides with the
$(X, Y, Z)$ origin and at $s = 0$ the $x$, $y$, and $z$ axes
correspond to the $X$, $Y$, and $Z$ axes respectively. $\theta$
decreases as one follows the reference orbit when going through a
horizontal bend with a positive bending angle. This corresponds to $x$
pointing radially outward. Without any vertical bends, the $Y$ and $y$
axes will coincide, and $\phi$ and $\psi$ will both be zero.

\vfill

%-----------------------------------------------------------------------------
\section{Phase Space Coordinate System}
\label{s:phase_space_coords}

\bmad uses the canonical phase space coordinates 
$(x, p_x, y, p_y, z, p_z)$. $x$, $y$, and $z$ are the
coordinates with respect to the reference particle as explained in
Section~\ref{s:ref}. $p_x$ and $p_y$ are the normalized momenta
\begin{align}
  p_x = &\frac{P_x}{P_0} \\
  p_y = &\frac{P_y}{P_0}
\end{align}
where $P_x$ and $P_y$ are, respectively, the momentum components along the $x$ and
$y$ axes, and $P_0$ is the reference (sometimes called the
design) momentum. The longitudinal canonical momentum $p_z$ is given by
\begin{equation}
  p_z = \frac{\Delta E}{E_0}
\end{equation}
where $E_0$ is the reference energy (energy here always refers to the 
total energy) and $\Delta E = E - E_0$ is the
deviation of the particle's energy from the reference energy. \mad\ uses
a slightly different coordinate system where $(z, p_z)$ is
replaced by $(-c\Delta t, p_t)$. $\Delta t$ is the time
difference for a particle to pass a point relative to the reference
particle and $p_t \equiv \Delta E / P_0 c$. For highly relativistic
particles the two coordinate systems are identical. For
non-relativistic particles, \bmad\ is not to be trusted in any
case. \bmad\ generally uses the small angle (paraxial) approximation
where it is assumed that $p_x, p_y \ll 1$. With this approximation, the
relationship, in a magnetic field free region, 
between the canonical momenta and the slopes $x' \equiv dx/ds$
and $y' \equiv dy/ds$ is
\begin{align}
  x' &\approx \frac{p_x}{1 + p_z} (1 + g x) \\
  y' &\approx \frac{p_y}{1 + p_z} (1 + g x) 
\end{align}
where $g = 1/\rho$ is the curvature function with $\rho$ being the radius
of curvature of the reference orbit and it has been assumed that the 
bending is in the $x$--$z$ plane. $g = 0$ in a straight section.

For those programmers using the PTC software package directly (ignore
this if you don't know what I'm talking about) \'Etienne Forest uses a still
different coordinate system where $(z, p_z)$ is replaced by
$(\Delta E/P_0 c, c \Delta t)$
\chapter{Physics}

%-----------------------------------------------------------------
\section{Units}
\label{s:units}
\index{Units|textbf}

\index{MAD!units}
\index{Units!with MAD}
\bmad uses SI (Syst\'eme International) units as shown in
Table~\ref{t:units}.  Note that \mad uses different units. For example,
\mad's unit of Particle Energy is GeV not eV.
\begin{table}[ht]
\centering
\begin{tabular}{|l|l|} \hline
  {\em Quantity}     & {\em Units}       \\ \hline
  Angles             &    radians        \\ 
  Charge             &    Coulombs       \\
  Current            &    Amps           \\ 
  Frequency          &    Hz             \\ 
  Kick               &    radians        \\ 
  Length             &    meters         \\ 
  Magnetic Field     &    Tesla          \\ 
  Particle Energy    &    eV             \\ 
  Phase Angles (RF)  &    radians/2$\pi$ \\ 
  Voltage            &    Volts          \\ \hline
\end{tabular}
\caption{Physical units used by \bmad.}
\label{t:units}
\end{table}


%-----------------------------------------------------------------
\section{Constants}
\label{s:constants}
\index{Constants|textbf}

\index{MAD}
\bmad defines commonly used physical and mathematical constants
shown in Table~\ref{t:constants}.  All symbols use straight SI units
except for \vn{e_mass} and \vn{p_mass} which are provided for
compatibility with \mad.

\begin{table}
\centering
\begin{tabular}{|l|l|l|l|} \hline
  {\em Symbol}   & {\em Value}       & {\em Units} &  {\em Name}     \\ \hline
  pi             & 3.14159265359          &        &                   \\
  twopi          & 2 * pi                 &        &                   \\
  fourpi         & 4 * pi                 &        &                   \\
  sqrt\_2        & 1.4142135623731        &        &                   \\
  m\_electron    & $0.51099906 \pow{6}$   & eV     & Electron mass     \\
  m\_proton      & $0.938271998 \pow{9}$  & eV     & Proton mass       \\
  c\_light       & $2.99792458 \pow{8}$   & m/s    & Speed of light    \\
  r\_e           & $2.8179380 \pow{-15}$  & m      & Electron radius   \\
  r\_p           & $1.5346980 \pow{-18}$  & m      & Proton radius     \\
  e\_charge      & $1.6021892 \pow{-19}$  & C      & Electron charge   \\
  h\_planck      & $6.626196 \pow{-34}$   & J/Hz   & Planck's constant \\
  h\_bar\_planck & $1.054591 \pow{-34}$   & J s    & Planck / $2\pi$   \\
  e\_mass        & $0.51099906 \pow{-3}$  & GeV    & Electron mass     \\
  p\_mass        & $0.938271998$          & GeV    & Proton mass     \\ \hline
\end{tabular}
\caption{Physical and mathematical constants recognized by \bmad.}
\label{t:constants}
\end{table}


%-----------------------------------------------------------------
\section{Magnetic Fields}
\label{s:fields}
\index{Magnetic fields|textbf}

Start with the assumption that the local magnetic field has no
longitudinal component (obviously this assumption does not work with,
say, a solenoid).  Following \mad, the vertical magnetic field along
the $y = 0$ axis is expanded in a Taylor series
\Begineq
  B_y(x, 0) = \sum_n B_n \, \frac{x^n}{n!}
  \label{byx0b}
\Endeq
This is not the most
general form for the magnetic field. Essentially all of the skew
components have been ignored here. Assuming that the
reference orbit is locally straight (there are correction terms if the
Reference Orbit is locally curved), the field up to $3^{rd}$ order is
\begin{alignat}{5}
  B_x &=           &&B_1 y \plus         &&B_2 \, xy       && \plus && \frac{1}{6} B_3 (3x^2 y - y^3) \plus \ldots \\
  B_y &= B_0 \plus &&B_1 x + \frac{1}{2} &&B_2 (x^2 - y^2) && \plus && \frac{1}{6} B_3 (x^3 - 3x y^2) \plus \ldots
\end{alignat}
The normalized integrated multipole $K_nL$ is used when specifying magnetic
multipole components
\index{Multipole!KnL, Tn|textbf}
\Begineq
  K_nL \equiv \frac{q \, L \, B_n}{P_0}
\Endeq
$L \, B_n$ is the integrated multipole component over a length $L$,
and $P_0$ is the reference momentum. Note that $P_0/q$ is sometimes
written as $B\rho$. This is just an old notation where $\rho$ is the
bending radius of a particle with the reference energy in a field of
strength $B$. The kicks $\Delta p_x$ and $\Delta p_y$ that a
particle experiences going through a multipole field is
\begin{alignat}{5}
  \Delta p_x & = \frac{-q \, L \, B_y}{P_0} \label{pqlbp1} \\
             & = -K_0 L \;-\; 
             && K_1 L \, x \plus 
             \frac{1}{2} && K_2 L (y^2 - x^2) && \plus 
             && \frac{1}{6} K_3 L (3x y^2 - x^3) \plus \ldots 
             \nonumber \\
  \Delta p_y & = \frac{q \, L \, B_x}{P_0} \label{pqlbp2} \\
             & =     
             && K_1 L \, y \plus 
             && K_2 L \, xy && \plus 
             && \frac{1}{6} K_3L (3x^2 y - y^3) \plus \ldots \nonumber 
\end{alignat}
A positive $K_1L$ quadrupole component gives
horizontal focusing and vertical defocusing. The general form is
\begin{align}
  \Delta p_x &= \sum_{n = 0}^{\infty} \frac{K_n L}{n!} 
             \sum_{m = 0}^{\lfloor \frac{n}{2} \rfloor}
             \begin{pmatrix} n \cr 2m \end{pmatrix} \,
             (-1)^{m+1} \, x^{n-2m} \, y^{2m} \\
  \Delta p_y &= \sum_{n = 0}^{\infty} \frac{K_n L}{n!} 
             \sum_{m = 0}^{\lfloor \frac{n-1}{2} \rfloor}
             \begin{pmatrix} n \cr 2m+1 \end{pmatrix} \,
             (-1)^{m} \, x^{n-2m-1} \, y^{2m+1}
\end{align}

\index{Multipole!KnL, Tn|textbf}
So far only the normal components of the field have been
considered. If the fields associated with a particular $B_n$ multipole
component are rotated in the $(x, y)$ plane by an angle $\theta_n$, the
magnetic field at a point $(x,y)$ can be expressed in complex notation
as
\Begineq
  B_y(x,y) + i B_x(x,y) = 
    \frac{1}{n!} B_n e^{-i(n+1)\theta_n} \, e^{i n \theta} \, r^n 
  \label{bib1nb}
\Endeq
where $(r, \theta)$ are the polar coordinates of the point $(x, y)$.

\index{Multipole!an, bn|textbf}
Another representation of the magnetic field used by \bmad divides
the fields into normal $b_n$ and skew $a_n$ components. In terms of
these components the magnetic field for the $n$\Th\ order multipole is
\Begineq
  \frac{q \, L}{P_0} \, (B_y + i B_x) = (b_n + i a_n) \, (x + i y)^n
\Endeq
The conversion between $(a_n, b_n)$ and $(K_nL, \theta_n)$ is
\Begineq
  b_n + i a_n = \frac{1}{n!} \, K_nL \, e^{-i(n+1)\theta_n}
\Endeq
or
\begin{align}
  K_n L &= n! \, \sqrt{a_n^2 + b_n^2} \\
  \tan[(n+1) \theta_n] &= \frac{-a_n}{b_n}
\end{align}
To convert a normal magnet (a magnet with no skew component) into a skew
magnet (a magnet with no normal component) the magnet should be rotated
about its longitudinal axis with a rotation angle of
\Begineq
  (n+1) \theta_n = \frac{\pi}{2}
\Endeq
For example, a normal quadrupole rotated by $45^\circ$ becomes a
skew quadrupole.

\index{AB_Multipole}
\index{Radius}
When the $a_n$ and $b_n$ are associated with a physical element (as
opposed to the $a_n$ and $b_n$ associated with an \vn{AB_Multipole} element),
a measurement radius $r_0$ and a scale factor $F$ are used to scale
the $a_n$ and $b_n$ according to the formula
\Begineq
  \bigl[ a_n (\text{actual}), b_n (\text{actual}) \bigr] =
  \bigl[ a_n (\text{input}), b_n (\text{input}) \bigr] 
  \cdot F \cdot \frac{r_0^{n_\text{ref}}}{r_0^n} 
  \label{ababf}
\Endeq
$a_n(\text{input})$ and $b_n(\text{input})$ are the multipole values as given in the
lattice file. $a_n(\text{actual})$ and $b_n(\text{actual})$ are the multipole values
that are used in any simulation calculations. $r_0$ is set by the
\vn{radius} attribute of an element. $F$ and $n_\text{ref}$ are set
automatically depending upon the type of element as shown in
Table~\ref{t:ab}.

\index{AB_Multipole}
\index{Multipole}
Note that the $n = 0$ component of an \vn{AB_Multipole} or \vn{Multipole}
element rotates the reference orbit essentially acting as a zero length bend.
This is not true for multipoles that are associated with 
non-multipole elements.

\index{Kicker}
\index{Hkicker}
\index{Vkicker}
\index{Rbend}
\index{Sbend}
\index{Elseparator}
\index{Quadrupole}
\index{Solenoid}
\index{Sol_Quad}
\index{Sextupole}
\index{Octupole}
\begin{table}[ht]
\centering
\begin{tabular}{|l|l|l|} \hline
\tt
  {\em Element} & $F$                              & $n_\text{ref}$ \\ \hline
  \vn{Kicker}      & $\sqrt{{\tt Hkick}^2 + {\tt Vkick}^2}$ & 0 \\
  \vn{Hkicker}     & Kick                                   & 0 \\
  \vn{Vkicker}     & Kick                                   & 0 \\
  \vn{Rbend}       & G * L                                  & 0 \\
  \vn{Sbend}       & G * L                                  & 0 \\
  \vn{Elseparator} & $\sqrt{{\tt Hkick}^2 + {\tt Vkick}^2}$ & 0 \\
  \vn{Quadrupole}  & K1 * L                                 & 1 \\
  \vn{Solenoid}    & KS * L                                 & 1 \\
  \vn{Sol_Quad}    & K1 * L                                 & 1 \\
  \vn{Sextupole}   & K2 * L                                 & 2 \\
  \vn{Octupole}    & K3 * L                                 & 3 \\ \hline
\end{tabular}
\caption{$F$ and $n_\text{ref}$ for various elements.}
\label{t:ab}
\end{table}

%-----------------------------------------------------------------
\section{Taylor Maps}
\label{s:taylor_phys}
\index{Taylor map|textbf}

A transport map ${\cal M}: {\cal R}^6 \rightarrow {\cal R}^6$ through
an element or a section of a lattice is a function that maps the
starting phase space coordinates $\Bf r(\In)$ to the ending
coordinates $\Bf r(\Out)$
\begin{equation}
  \Bf r(\Out) = {\cal M} \, \Bf r(\In)
\end{equation}
${\cal M}$ is made up of six functions ${\cal M}_i: {\cal R}^6
 \rightarrow {\cal R}$. Each of these functions maps to one of the $r(\Out)$
coordinates. These functions can be expanded in a Taylor
series and truncated at some order. Each Taylor series is in the form
\Begineq
  r_i(\Out) = \sum_{j = 1}^N \, C_{ij} \, \prod_{k = 1}^6 \, r_k^{e_{ijk}}(\In)
  \label{rcr}
\Endeq
Where the $C_{ij}$ are coefficients and the $e_{ijk}$ are integer exponents.
The order of the map is
\Begineq
  \mbox{order} = \max_{i,j} \left( \sum_{k = 1}^6 e_{ijk} \right)
\Endeq

The standard \bmad routine for printing a Taylor map might produce something 
like this: 
\begin{example}
   Taylor Terms:
    Out     Coef              Exponents           Order        Reference
   ---------------------------------------------------
      1:     -0.600000000000  0  0  0  0  0  0        0       0.200000000
      1:      1.000000000000  1  0  0  0  0  0        1
      1:      0.145000000000  2  0  0  0  0  0        2
   ---------------------------------------------------
      2:     -0.185000000000  0  0  0  0  0  0        0       0.000000000
      2:      1.300000000000  0  1  0  0  0  0        1
      2:      3.800000000000  2  0  0  0  0  1        3
   ---------------------------------------------------
      3:      1.000000000000  0  0  1  0  0  0        1       0.100000000
      3:      1.600000000000  0  0  0  1  0  0        1
      3:    -11.138187077310  1  0  1  0  0  0        2
   ---------------------------------------------------
      4:      1.000000000000  0  0  0  1  0  0        1       0.000000000
   ---------------------------------------------------
      5:      0.000000000000  0  0  0  0  0  0        0       0.000000000
      5:      0.000001480008  0  1  0  0  0  0        1
      5:      1.000000000000  0  0  0  0  1  0        1
      5:      0.000000000003  0  0  0  0  0  1        1
      5:      0.000000000003  2  0  0  0  0  0        2
   ---------------------------------------------------
      6:      1.000000000000  0  0  0  0  0  1        1       0.000000000
\end{example}
Each line in the example represents a single taylor term. The Taylor
terms are grouped into 6 Taylor series, one each output phase space
coordinate.  The first column in the example, labeled ``out'',
(corresponding to the $i$ index in \Eq{rcr}) indicates the Taylor
series: $1 = x(out)$, $2 = p_x(out)$, etc. The 6 exponent columns give
the $e_{ijk}$ of \Eq{rcr}. In this example, the second Taylor series
(\vn{out} = 2), when expressed as a formula, would read:
\Begineq
  p_x(out) = -0.185 + 1.3 \, p_x(in) + 3.8 \, x^2(in) \, p_z(in)
\Endeq

\index{Taylor map!reference coordinates}
The reference column in the above example shows the input coordinates around
which the Taylor map is calculated. In this case, the reference
coordinates where 
\Begineq
  (x, p_x, y, p_y, z, p_z)_{ref} = (0.2, 0, 0.1, 0, 0, 0, 0)
\Endeq
The choice of the reference point will affect the values of the
coefficients of the Taylor map. For example, suppose that the exact
map through an element looks like
\Begineq
  x(out) = A \, \sin(k \, x(in))
\Endeq
Then a Taylor map to 1\St order is
\Begineq
  x(out) = c_0 + c_1 \, x(in)
\Endeq
where
\begin{align}
  c_1 &= A \, k \, \cos(k \, x_{\mbox{ref}}) \\
  c_0 &= A \, \sin(k \, x_{\mbox{ref}}) - c_1 \, x_{\mbox{ref}} \nonumber
\end{align}
Notice that once the coefficient values are determined the reference
point does not play any role when the Taylor map is evaluated to
determine the output coordinates as a function of the input
coordinates.

\index{Taylor map!feed-down}
Of importance in working with Taylor maps is the concept of
\vn{feed-down}.  This is best explained with an example. To keep the
example simple, the discussion is limited to one phase space
dimension so that the Taylor maps are a single Taylor series. Take the
map $M_1$ from point 0 to point 1 to be
\Begineq
  M_1: x_1 = x_0 + 2
  \label{xx2}
\Endeq
and the map $M_2$ from point 1 to point 2 to be
\Begineq
  M_2: x_2 = x_1^2 + 3 \, x_1
  \label{xx3x}
\Endeq
Then concatenating the maps to form the map $M_3$ from point 0 to point 2
gives
\Begineq
  M_3: x_2 = (x_0 + 2)^2 + 3 (x_0 + 2) = x_0^2 + 7 \, x_0 + 10
  \label{xx23x2}
\Endeq
However if we are evaluating our maps to only 1\St order the map $M_2$
becomes
\Begineq
  M_2: x_2 = 3 \, x_1
\Endeq
and concatenating the maps now gives
\Begineq
  M_3: x_2 = 3 (x_0 + 2) = 3 \, x_0 + 6
  \label{x3x23}
\Endeq
Comparing this to \Eq{xx23x2} shows that by neglecting the 2\Nd order
term in \Eq{xx3x} leads to 0\Th and 1\St order errors in
\Eq{x3x23}. These errors can be traced to the finite 0\Th order term in
\Eq{xx2}. This is the principal of feed--down: Given $M_3$ which is a map
produced from the concatenation of two other maps, $M_1$, and $M_2$
\Begineq
  M_3 = M_2(M_1)
\Endeq
Then if $M_1$ and $M_2$ are correct to n\Th order, $M_3$ will also be
correct to n\Th order as long as $M_1$ has no constant (0\Th order)
term. [Notice that a constant term in $M_2$ does not affect the
argument.]  What happens if we know there are constant terms in our
maps? One possibility is to go to a coordinate system where the
constant terms vanish. In the above example that would mean using the
coordinate $\widetilde x_0$ at point 0 given by
\Begineq
  \widetilde x_0 = x_0 + 2
\Endeq
\index{Symplectic integration}
The other possibility is to use symplectic integration. By its nature,
symplectic integration never has problems with feed--down.

The subject of symplectic integration is too large to be covered in
this guide. The reader is referred to the book ``Beam Dynamics: A New
Attitude and Framework'' by Etienne Forest\cite{b:forest}. A brief
synopsis: Symplectic integration uses as input 1) The Hamiltonian that
defines the equations of motion, and 2) a Taylor map $M_1$ from point 0 to
point 1. Symplectic integration from point 1 to point 2 produces a
Taylor map $M_3$ from point 0 to point 2. Symplectic integration can
produce maps to arbitrary order. In any practical application the
order $n$ of the final map is specified and in the integration
procedure all terms of order higher than $n$ are ignored. If one is
just interested in knowing the final coordinates of a particle at
point 2 given the initial coordinates at point 1 then $M_1$ is just
the constant map
\Begineq
  M_1: x_1 = c_i
\Endeq
where $c_i$ is the initial starting point. The order of the
integration is set to 0 so that all non--constant terms are
ignored. The final map is also just a constant map
\Begineq
  M_3: x_2 = c_f
\Endeq
If the map from point 1 to point 2 is desired then the map $M_1$ is
just set to the identity map
\Begineq
  M_1: x_1 = x_0
\Endeq
In general it is impossible to exactly integrate any non--linear
system. In practice, the symplectic integration is achieved by slicing
the interval between point 1 and point 2 into a number of (generally
equally spaced) slices. The integration is performed, slice step by
slice step. This is analogous to integrating a function by evaluating
the function at a number of points. Using more slices gives better
results but slows down the calculation. The speed and accuracy of the
calculation is determined by the number of slices and the \vn{order}
of the integrator. The concept of integrator order can best be
understood by analogy by considering the trapezoidal rule for
integrating a function of one variable:
\Begineq
  \int_{y_a}^{y_b} f(y) \, dy = 
  h \left[ \frac{1}{2} f(y_a) + \frac{1}{2} f(y_b) \right] +
  o(h^3 \, f^{(2)})
\Endeq
In the formula $h = y_b - y_a$ is the slice width. $0(h^3 \, f^{(2)})$
means that the error of the trapezoidal rule scales as the second
derivative of $f$. Since the error scales as $f^{(2)}$ this is an
example of a second order integrator. To integrate a function between
points $y_1$ and $y_N$ we slice the interval at points $y_2 \ldots y_{N-1}$
and apply the trapezoidal rule to each interval. Examples of higher
order integrators can be found, for example, in Numerical
Recipes\cite{b:nr}. The concept of integrator order in symplectic
integration is analogous. 

The optimum number of slices is determined by the smallest number that
gives an acceptable error. Integrators of higher order will generally
need a smaller number of slices to achieve a given accuracy. However,
since integrators of higher order take more time per slice step, and
since it is computation time and not number of slices which is
important, only a measurement of error and calculation time as a
function of slice number and integrator order will unambiguously give
the optimum integrator order and slice width.  In doing a timing test
it must be remembered that since the magnitude of any non-linearities
will depend upon position, the integration error will be dependent
upon the starting map $M_1$. For example, \bmad has integrators of
order 2, 4, and 6. To test them, timing tests were performed for some
wiggler elements (which have strong nonlinearities) and it was found
that in this case the 2\Nd order integrator gave the fastest computation 
time for a given accuracy.

%-----------------------------------------------------------------
\section{Symplectification}
\label{s:symp_method}
\index{Symplectic!symplectification}

If the evolution of a system can be described using a Hamiltonian then
it can be shown that the linear part of any transport map (the Jacobian)
must obey the symplectic condition. If a matrix $\Bf M$ is not symplectic,
Healy\cite{b:healy} has provided an elegant method for finding a symplectic 
matrix that is ``close'' to $\Bf M$. The procedure is as follows:
From $\Bf M$ a matrix $\Bf V$ is formed via
\begin{equation}
  \Bf V = \Bf S (\Bf I - \Bf M)(\Bf I + \Bf M)^{-1} 
  \label{e:vsimi}
\end{equation}
where $\Bf S$ is the matrix
\Begineq
  \Bf S = 
  \begin{pmatrix} 
      0 &  1 &  0 &  0 &  0 &  0 \cr
     -1 &  0 &  0 &  0 &  0 &  0 \cr
      0 &  0 &  0 &  1 &  0 &  0 \cr
      0 &  0 & -1 &  0 &  0 &  0 \cr
      0 &  0 &  0 &  0 &  0 & -1 \cr
      0 &  0 &  0 &  0 & -1 &  0 \cr
  \end{pmatrix}
  \label{s0100}
\Endeq
$\Bf V$ is symmetric if and only if $\Bf M$ is symplectic. In any case,
a symmetric matrix $\Bf W$ near $\Bf V$ can be
formed via
\begin{equation}
  \Bf W = \frac{\Bf V + \Bf V^t}{2}
\end{equation}
A symplectic matrix $\Bf F$ is now obtained by inverting \eq{e:vsimi}
\Begineq
  \Bf F = (\Bf I + \Bf S \Bf W) (\Bf I - \Bf S \Bf W)^{-1}
\Endeq

%-----------------------------------------------------------------
\section{Wigglers}
\label{s:wiggler_phys}

\index{Wiggler}
As discussed in \sref{s:wiggler}, \bmad \vn{Wiggler} elements are
split into two classes: map type and periodic type. The map type
\vn{Wigglers} are modeled using the method of Sagan, Crittenden, and
Rubin\cite{b:wiggler}. In this model the magnetic field is written as
a sum of terms $B_i$
\Begineq
  B(x,y,z) = \sum_i B_i(x, y, z; C, k_x, k_y, k_z, \phi_z)
\Endeq 
Each term $B_i$ is specified using five numbers: 
$(C, k_x, k_y, k_z, \phi_z)$. A term can take one of three forms: The first
form is
\begin{alignat}{4}
  B_x &= -&C &\dfrac{k_x}{k_y} & \sin(\kxx) \sinh(\kyy) \cos(\ksss) \CRNEG
  B_y &=  &C &                 & \cos(\kxx) \cosh(\kyy) \cos(\ksss) \CRNEG
  B_s &= -&C &\dfrac{k_s}{k_y} & \cos(\kxx) \sinh(\kyy) \sin(\ksss) \CRneg
  & \makebox[1pt][l]{with $k_y^2 = k_x^2 + k_s^2$ .} &&&  \label{f1}
\end{alignat}
The second form is
\begin{alignat}{4}
  B_x &=  &C &\dfrac{k_x}{k_y} & \sinh(\kxx) \sinh(\kyy) \cos(\ksss) \CRNEG
  B_y &=  &C &                 & \cosh(\kxx) \cosh(\kyy) \cos(\ksss) \CRNEG
  B_s &= -&C &\dfrac{k_s}{k_y} & \cosh(\kxx) \sinh(\kyy) \sin(\ksss) \CRneg
  & \makebox[1pt][l]{with $k_y^2 = k_s^2 - k_x^2$ ,} &&&  \label{f2}
\end{alignat}
The third form is
\begin{alignat}{4}
  B_x &=  &C &\dfrac{k_x}{k_y} & \sinh(\kxx) \sin(\kyy) \cos(\ksss) \CRNEG
  B_y &=  &C &                 & \cosh(\kxx) \cos(\kyy) \cos(\ksss) \CRNEG
  B_s &= -&C &\dfrac{k_s}{k_y} & \cosh(\kxx) \sin(\kyy) \sin(\ksss) \CRneg
  & \makebox[1pt][l]{with $k_y^2 = k_x^2 - k_s^2$ .} &&& \label{f3}
\end{alignat}
The relationship between $k_x$, $k_y$, and $k_z$ ensures that
Maxwell's equations are satisfied. Since the field is given by
analytic equations, Lie algebraic techniques can be use to construct
Taylor maps to arbitrary order.

``Periodic type'' wigglers use a simplified model where the magnetic
field components are
\begin{alignat}{1}
  B_y &= \hphantom{-} B_{\max} \, \cosh(k_z \, y) \, \cos(k_z \, z) \CRNO
  B_z &= -B_{\max} \, \sinh(k_z \, y) \, \sin(k_z \, z) 
\end{alignat}
where $B_{\max}$ is the maximum field on the centerline and $k$ is
given in terms of the pole length (\vn{l_pole}) by
\Begineq
  k_z = \frac{\pi}{l_{\mbox{pole}}}
\Endeq
This type of wiggler has infinitely wide poles. With
\vn{bmad_standard} tracking and transfer matrix calculations the
vertical focusing is assumed small so averaged over a period the
horizontal motion looks like a drift and the vertical motion is
modeled as a combination focusing quadrupole and focusing octupole
giving a kick\cite{b:corbett}
\Begineq
  \frac{dp_y}{ds} = k_1 \left( y + \frac{2}{3} \, k_z^2 \, y^3 \right)
\Endeq
where
\begin{alignat}{1}
  k_1 &= \frac{-1}{2} \, \left( \frac{c \, B_{\max}}{P_0 \, (1 + p_z)} \right)^2 
\end{alignat}
with $k_1$ being the linear focusing constant. For radiation
calculations the true horizontal trajectory with $y = 0$ is needed
\Begineq
  x = \frac{\sqrt{2 \, |k_1|}}{k_z^2} \, \cos (k_z \, z)
\Endeq

%-----------------------------------------------------------------
\section{Synchrotron Radiation Damping and Excitation}
\label{s:radiation}
\index{Synchrotron radiation!damping and excitation|textbf}

Emission of synchrotron radiation by a particle can be decomposed into
two parts. The deterministic average radiation emitted produces damping
while the stochastic fluctuating part produces excitation\cite{b:jowett}.

The treatment of radiation damping by \bmad essentially follows \mad.
The average change in energy $\Delta E$ of a particle going through a
section of magnet due to synchrotron radiation is
\Begineq
  \frac{\Delta E}{E_0} = -k_d \, (1 + p_z)
\Endeq
where
\Begineq
  k_d \equiv \frac{2 \, r_e}{3} \, \gamma_0^3 \, \ave{g_0^2} \, L_p \,  
  (1 + p_z)
  \label{k2r3g}
\Endeq
$r_e$ is the classical electron radius, $L_p$ is the actual path
length, $\gamma_0$ is the energy factor of an on-energy particle, $1/g_0$
is the bending radius of an on--energy particle, and $\ave{g_0^2}$ is an
average of $g_0^2$ over the actual path.

The energy lost is given by
\Begineq
  \frac{\Delta E}{E_0} = -k_f \, (1 + p_z)
\Endeq
where
\Begineq
  k_f \equiv \left( \frac{55 \, r_e \, \hbar \, c}{24 \, \sqrt{3} \, m_e} \, 
  L_p \, \gamma_0^5 \ave{g_0^3} \right)^{1/2} \, (1 + p_z) \, \xi
  \label{k55rh}
\Endeq
$\xi$ is a Gaussian distributed random number with unit sigma and zero mean.

Using \Eqs{k2r3g} and \eq{k55rh} The total change in $p_z$ can be written as
\Begineq
  \Delta p_z = \frac{\Delta E}{E_0} = -k_E \, (1 + p_z)
\Endeq
where
\Begineq
  k_E = k_d + k_f
\Endeq
Since the radiation is emitted in the forward direction the angles
$x'$ and $y'$ are invariant which leads to the following equations for
the changes in $p_x$ and $p_y$
\begin{align}
  \Delta p_x &= -k_E \, p_x \CRNO
  \Delta p_y &= -k_E \, p_y 
\end{align}

The above formalism does not take into account the fact that radiation is 
emitted with a $1/\gamma$ angular distribution. This means that the calculated
vertical emittance for a lattice with
bends only in the horizontal plane and without any coupling elements such as
skew quadrupoles will be zero. Typically, in practice, the vertical emittance
will be dominated by coupling so this approximation is generally a good one.

%-----------------------------------------------------------------
\section{Coupling and Normal Modes}
\label{s:coupling}
\index{Normal Mode!Coupling}

The coupling formalism used by \bmad is taken from the paper of Sagan
and Rubin\cite{b:coupling}. The main equations are reproduced here.  A
one--turn map $\bfT(s)$ for the transverse two--dimensional phase space
$\bfx = (x, x', y, y')$ starting and ending at some point $s$ can be
written as
  \Begineq
    \bfT = \bfV \, \bfU \, \bfV\inv 
    , \label{tvuv}
  \Endeq 
where $\bfV$ is symplectic, and $\bfU$ is of the form
  \Begineq
    \bfU = 
    \begin{pmatrix}
      \bfA & \Bf0 \cr 
      \Bf0 & \bfB \cr
    \end{pmatrix}
    . \label{ua00b}
  \Endeq
\index{Normal Mode!a--mode}
\index{Normal Mode!b--mode}
Since $\bfU$ is uncoupled the standard Twiss analysis can be
performed on the matrices $\bfA$ and $\bfB$. The normal modes
are labeled $a$ and $b$ and if the one--turn matrix $\bfT$ is
uncoupled then $a$ corresponds to the horizontal mode and $b$
corresponds to the vertical mode. 

$\bfV$ is written in the form
  \Begineq
    \bfV = 
    \begin{pmatrix}
        \gamma \bfI & \bfC \cr 
        -\bfC^+     & \gamma \bfI \cr
    \end{pmatrix}
    , \label{vgicc1}
  \Endeq
where the symplectic conjugate is 
\index{Symplectic!conjugate}
  \Begineq
    \bfC^+ = 
    \begin{pmatrix}
       C_{22} & -C_{12} \cr 
      -C_{21} & C_{11} \cr
    \end{pmatrix}
    . \label{ccccc}
  \Endeq
Since we demand that $\bfV$ be symplectic we have the condition
  \Begineq               
    \gamma^2 + \, ||\bfC|| = 1
    , \label{gc1}
  \Endeq
and $\bfV\inv$ is given by
  \Begineq
    \bfV\inv = 
    \begin{pmatrix}
      \gamma \bfI & -\bfC \cr 
      \bfC^+ & \gamma \bfI \cr
    \end{pmatrix}
    . \label{vgicc2}
  \Endeq 
$\bfC$ is a measure of the coupling. 
$\bfT$ is uncoupled if and only if $\bfC = \Bf 0$. 

It is useful to normalize out the $\beta(s)$ variation in the the above
analysis. Normalized quantities being denoted by a bar above them. The
normalized normal mode matrix $\BAR\bfU$ is defined by
  \Begineq
    \BAR\bfU = \bfG \, \bfU \, \bfG\inv
    , \label{ugug}
  \Endeq
Where $\bfG$ is given by 
  \Begineq
    \bfG \equiv 
    \begin{pmatrix}
      \bfG_a & \Bf0 \cr 
      \Bf0 & \bfG_b
    \end{pmatrix}
    , \label{gg00g}
  \Endeq  
with 
  \Begineq
    \Bf G_a = 
    \begin{pmatrix}
      \frac{\tstyle 1}{\tstyle \sqrt{\beta_a}} & 0 \cr
      \frac{\tstyle \alpha_a}{\tstyle \sqrt{\beta_a}} & \sqrt{\beta_a}
    \end{pmatrix}
    , \label{g1b0a} 
  \Endeq
with a similar equation for $\Bf G_b$. With this definition, the corresponding
$\BAR\bfA$ and $\BAR\bfB$ (cf.~\Eq{ua00b}) are just rotation matrices.
The relationship between $\bfT$ and $\BAR\bfU$ is 
  \Begineq
    \bfT = \bfG\inv \, \BAR\bfV \, \BAR\bfU \, \BAR\bfV\inv \, \bfG
    , \label{tgvuv}
  \Endeq
where
  \Begineq
    \BAR\bfV = \bfG \, \bfV \, \bfG\inv
    . \label{vgvg}
  \Endeq
Using \Eq{gg00g}, $\BAR\bfV$ can be written in the form
  \Begineq
    \BAR\bfV = 
    \begin{pmatrix}
      \gamma \bfI & \BAR\bfC \cr -\BAR\bfC^+ & \gamma \bfI
    \end{pmatrix}
    , \label{vgicc3}
  \Endeq
with the normalized matrix $\BAR\bfC$ given by
  \Begineq
    \BAR\bfC = \bfG_a \, \bfC \, \bfG_b\inv
    . \label{cgcg}
  \Endeq

The normal mode coordinates ${\bf a} = (a, a', b, b')$ are related to
the laboratory frame via
  \Begineq
    {\bf a} = \bfV\inv \, {\bf x}
    . \label{avx}
  \Endeq 
\index{Dispersion}
In particular the normal mode dispersion $\bfeta_a = (\eta_a,
\eta'_a, \eta_b, \eta'_b)$ is related to the laboratory frame
dispersion $\bfeta_x = (\eta_x, \eta'_x, \eta_y, \eta'_y)$ via
  \Begineq
    {\bfeta_a} = \bfV\inv \, {\bfeta_x}
    . \label{etaavx}
  \Endeq 
The dispersion itself is defined by
\begin{align}
  \eta_x &= \frac{dx}{dp_z} \CRNO
  \eta'_x &= \frac{d}{ds} \, \eta_x = \frac{dx'}{dp_z} \\
  \eta_y &= \frac{dy}{dp_z} \CRNO
  \eta'_y &= \frac{d}{ds} \, \eta_y = \frac{dy'}{dp_z} \nonumber
\end{align}
Notice that $\eta'_x$ and $eta'_y$ are essentially defined in terms of
derivatives of $x'$ and $y'$ and not the canonical coordinates $p_x$
and $p_y$. This fact makes complicates the calculation of the
dispersion when the closed orbit does not coincide with the reference
trajectory. Given the first order transport map between points 1 and
2:
\Begineq
  \Bf x_2 = \Bf M \, \Bf x_1 + \Bf k
\Endeq
where $\Bf x_1$ and $\Bf x_2$ are expressed in canonical coordinates,
The transformation of the dispersion between points 1 and 2 is given
by:
\begin{align}
  \eta_{x2} &= 
    M_{11} \, \eta_{x1} + M_{12} \, \eta'_{x1} + 
    M_{13} \, \eta_{y1} + M_{14} \, \eta'_{y1} +
    M_{16} + M_{12} \, p_{x1} + M_{14} \, p_{y1} \CRNO
  \eta'_{x2} &=     
    M_{21} \, \eta_{x1} + M_{22} \, \eta'_{x1} + 
    M_{23} \, \eta_{y1} + M_{24} \, \eta'_{y1} +
    M_{26} - M_{21} \, x_1 - M_{23} \, y_1 - k_2 \\
  \eta_{y2} &= 
    M_{31} \, \eta_{x1} + M_{32} \, \eta'_{x1} + 
    M_{33} \, \eta_{y1} + M_{34} \, \eta'_{y1} +
    M_{36} + M_{32} \, p_{x1} + M_{34} \, p_{y1} \CRNO
  \eta'_{y2} &=     
    M_{41} \, \eta_{x1} + M_{42} \, \eta'_{x1} + 
    M_{43} \, \eta_{y1} + M_{44} \, \eta'_{y1} +
    M_{46} - M_{41} \, x_1 - M_{43} \, y_1 - k_4 \CRNO
\end{align}
where $(x_1, p_{x1}, y_1, p_{y1}$ are the closed orbit coordinates at
point 1 and it has been assumed that the closed orbit $p_z$ is zero.

%-----------------------------------------------------------------
\section{Macroparticles}
\label{s:macro}
\index{Macroparticles|textbf}

A macroparticle\cite{b:transport_appendix} is a bundle of particles
whose distribution is assumed to be Gaussian in shape. A macroparticle
is represented by a centroid position $\bfrbar$ and a $6 \times 6$
$\bfsig$ matrix which defines the shape of the macroparticle in
phase space. $\sigma_i = \sqrt{\bfsig(i,i)}$ is the RMS sigma for the $i$\Th
phase space coordinate. For example $\sigma_z = \sqrt{\bfsig(5,5)}$.

$\bfsig$ is a real, non-negative symmetric matrix. The equation that
defines the ellipsoid at a distance of $n$--sigma from the centroid is
\Begineq
  (\bfr - \bfrbar)^t \bfsig\inv (\bfr - \bfrbar) = n
\Endeq
where the $t$ superscript denotes the transpose. Given the sigma matrix
at some point $s = s_1$, the sigma matrix at a different point $s_2$ is
\Begineq
  \bfsig_2 = \Bf M_{21} \, \bfsig_1 \, \Bf M_{21}^t
\Endeq
where $\Bf M_{21}$ is the Jacobian of the transport map between points
$s_1$ and $s_2$.

\index{dispersion}
The Twiss parameters can be calculated from the sigma matrix. The
dispersion is given by
\begin{align}
  \bfsig(1,6) &= \eta_x \, \bfsig(6,6) \CRNO
  \bfsig(2,6) &= \eta'_x \, \bfsig(6,6) \\
  \bfsig(3,6) &= \eta_y \, \bfsig(6,6) \CRNO
  \bfsig(4,6) &= \eta'_y \, \bfsig(6,6) \nonumber
\end{align}
Ignoring coupling for now the betatron part of the sigma matrix can be
obtained from the linear equations of motion. For example, using
\Begineq
  x = \sqrt{2 \, \beta_x \, \epsilon_x} \cos \phi_x + \eta_x \, p_z
\Endeq
Solving for the first term on the RHS, squaring and averaging over all
particles gives
\Begineq
  \beta_x \, \epsilon_x = \bfsig(1,1) - \frac{\bfsig^2(1,6)}{\bfsig(6,6)}
\Endeq
It is thus convenient to define the betatron part of the sigma matrix
\Begineq
  \bfsigb(i,j) \equiv \bfsig(i,j) - \frac{\bfsig(i,6) \, \bfsig(j,6)}{\bfsig(6,6)}
\Endeq
and in terms of the betatron part the emittance is
\Begineq
  \epsilon_x^2 = \bfsigb(1,1) \, \bfsigb(2,2) - \bfsigb^2(1,2)
\Endeq
and the Twiss parameters are
\Begineq
  \epsilon_x 
  \begin{pmatrix}
    \beta_x   & -\alpha_x \\
    -\alpha_x & \gamma_x
  \end{pmatrix} = 
  \begin{pmatrix}
    \bfsigb(1,1) & \bfsigb(1,2) \\
    \bfsigb(1,2) & \bfsigb(2,2) 
  \end{pmatrix}
\Endeq

If there is coupling the transformation between the $4\times 4$
transverse normal mode sigma matrix $\bfsig_a$ and the $4\times 4$
laboratory matrix $\bfsig_x$ is
\Begineq
  \bfsig_x = \bfV \, \bfsig_a \bfV^t
\Endeq

For tracking a relativistic charged particle bunch is modeled by dividing it
longitudinally into a series of slices. Each slice is made up of a
number of macroparticles. It is assumed that the range of the
longitudinal positions of the macroparticles of a given slice do not
overlap the ranges of the other slices.

\bmad has a routine to initialize macroparticles to mirror the initial
distribution as set--up by the LIAR program\cite{b:liar}. The
initialization is as follows: The center of a
slice will be at a longitudinal position $z_j$ given by
\Begineq
  z_j = z_b + \frac{(n_s - 2 \, j + 1) \, \sigma_z \, N_{\sigma z}}{n_s}
\Endeq
where $z_b$ the overall offset of the bunch, 
$n_s$ is the number of slices, $j = 1, \ldots,
n_s$ is the index of the slice, $\sigma_z$ is the RMS bunch length,
and $N_{\sigma z}$ is the number of $\sigma_z$ the slices go out
to. The width $dz_s$ of each slice is
\Begineq
    dz_s = \frac{2 \, \sigma_z \, N_{\sigma z}}{n_s}
\Endeq
The charge associated with each slice is, within a constant factor,
the charge contained within the region between $z_j - dz_s/2$ and $z_j
+ dz_s/2$ assuming a Gaussian distribution.  The charge of all the
slices are multiplied by a factor so that the total charge of the
slices is equal to the input bunch charge.

The initialization of the macroparticles within a slice gives all the
macroparticles the same centroid position except with differing
centroid energies. 
The sigma matrix is the same for all macroparticles and is
determined by the input Twiss parameters:
\begin{align}
  \bfsig(1,1) &= \epsilon_x \, \beta_x \CRNEG
  \bfsig(1,2) &= -\epsilon_x \alpha_x  \CRNEG
  \bfsig(2,2) &= \epsilon_x \, \gamma_x = 
      \epsilon_x \, (1 + \alpha_x^2) / \beta_x \CRNEG
  \bfsig(3,3) &= \epsilon_y \, \beta_y \\
  \bfsig(3,4) &= -\epsilon_y \alpha_b \CRNEG
  \bfsig(3,4) &= \epsilon_y \, \gamma_y = 
      \epsilon_y \, (1 + \alpha_b^2) / \beta_y \CRNEG
  \bfsig(i,j) &= 0 \quad \mbox{otherwise} \nonumber
\end{align}
The centroid energy of the $k$\Th macroparticle is
\Begineq
  E_k = E_b + \frac{(n_{mp} - 2 \, k + 1) \, \sigma_E \, N_{\sigma E}}{n_{mp}}
\Endeq
where $E_b$ is the central energy of the bunch, $n_{mp}$ is the number
of macroparticles per slice, $\sigma_E$ is the energy sigma, and
$N_{\sigma E}$ is the number of sigmas in energy that the range of
macroparticle energies cover. The charge of each macroparticle is,
within a constant factor, the charge contained within the energy
region $E_k - dE_{mp}/2$ to $E_k + dE_{mp}/2$ assuming a Gaussian
distribution where the energy width $dE_{mp}$ is
\Begineq
  dE_{mp} = \frac{2 \, \sigma_E \, N_{\sigma E}}{n_{mp}}
\Endeq
The charge of all the macroparticles is adjusted by a constant factor
so that the total charge of the macroparticles within a slice is equal
to the charge associated with the slice.

%-----------------------------------------------------------------
\section{Wakefields}
\label{s:wakefields}
\index{Wakefields}

%-----------------------------
\subsection{Short--Range Wakes}

Wakefield effects are divided into short--range (within a bunch) and
long--range (between bunches).

Only the transverse dipole and longitudinal monopole components of the
short--range wakefield are modeled. The longitudinal monopole energy
kick $dE$ for the $i$\Th macroparticle is computed from the equation
\Begineq
  \Delta p_z(i) = \frac{-e \, L}{v \, P_0} \, \left(
        \frac{1}{2}\WlS(0) \,  |q_i| +
        \sum_{j \ne i} \tWl(dz_{ij}) \, |q_j| \right)
  \label{delvp}
\Endeq
where $v$ is the particle velocity, $e$ is the charge on an electron,
$q$ is the macroparticle charge, $L$ is the cavity length, $dz_{ij}$
is the longitudinal distance between the $i$\Th and $j$\Th
macroparticles, $\WlS$ is the short--range longitudinal wakefield
function, and $\tWl$ is a modified wakefield function
\Begineq
  \tWl(dz) = 
  \begin{cases}
    \WlS(0) \cdot \frac{dz + \sigma_{zij}}{2\sigma_{zij}} & 
                                    -\sigma_{zij} < dz < \sigma_{zij} \\
    \WlS(dz)                                            & \text{otherwise}
  \end{cases}
\Endeq
$\tWl$ is used instead of $\WlS$ to avoid simulation artifacts 
is due to the discontinuity of $\WlS$ at $dz = 0$. 
$\sigma_{zij}$ is the combined macroparticle length
\Begineq
  \sigma_{zij} = \sigma_{zi} + \sigma_{zj} + \sigma_{z0}
\Endeq
where $\sigma_{zi}$ and $\sigma_{zj}$ are the longitudinal sizes for
the $i$\Th and $j$\Th particles respectively, and $\sigma_{z0}$ is a
small fudge factor needed when $\sigma_{zi} = \sigma_{zj} = 0$.

The transverse kick $\Delta p_x(i)$ for the $i$\Th macroparticle due to the 
dipole short--range transverse wakefield is modeled with the equation
\Begineq
  \Delta p_x(i) = \frac{-e \, L \, \sum_j |q_j| \, x_j \, \WtS(dz_{ij})}
                 {v \, P_0}
  \label{pelqxw}
\Endeq
There is a similar equation for $\Delta p_y(i)$. $\WtS$ is the
transverse short--range wake function. Note that the LIAR
program\cite{b:liar} uses an opposite sign here.

The wakefield functions $\WlS$ and $\WlS$ can be specified in a \bmad
lattice file using a table of wake vs $z$ or by using ``pseudo'' modes
where
\Begineq
  W(z) = \sum_i A_i \, e^{d_i z} \, \sin (k_i \, z + \phi_i)
  \label{wadzk}
\Endeq
The set of mode parameters $(A_i, d_i, k_i, \phi_i)$ are chosen to fit
the calculated wake potential. The advantage of the mode approach is that
the calculation time scales as the number of particles $N$ while the
calculation based upon a table scales as $N^2$. The disadvantage is that
initially there is an extra step in that a fit to the wake potential must
be performed to generate the mode parameter values. Furthermore, since the
transverse wake typically has a form that looks something like $\sqrt{-z}$ 
for small $z$ (small here means small in magnitude, $z$ is always negative)
a set of pseudo modes may not fit the wake function well at small $z$. 
\bmad implements a hybrid scheme where for small $z$ below some cutoff
$z_{\rm{cut}}$ the wake vs $z$ table is used and for larger $z$ the pseudo
modes are used.


%-----------------------------
\subsection{Long--Range Wakes}

Following Chao\cite{b:chao} Eq.~2.88, The long--range wakefields are
characterized by a set of cavity modes. The wake function $W$ for a
mode is given by
\Begineq
  W(z) = c \, \left( \frac{R}{Q} \right) \,\,
  e^{k z/ 2 Q} \, \sin (k \, z)
\Endeq
where the wake number $k$ is related to the mode frequency $\omega$ by
\Begineq
  k = \omega / c
\Endeq
Here $z$ is the distance behind the particle generating the wake. $z$
is negative for all particles affected by the wake so $W(z)$ is a
decaying exponential as expected.

Assuming that the macroparticle generating the wake is offset a
distance $r_w$ along the $x$--axis, A trailing macroparticle will see a kick
\begin{align}
  \Delta {\Bf p}_\perp &= 
    -C \, I_m \, W_m(z) \, m \, r^{m-1} \, \left( 
    {\bf\hat r} \cos m\theta - {\bf\hat\theta} \sin m\theta \right) \\
  &= -C \, I_m \, W_m(z) \, m \, r^{m-1} \, \left( 
    {\bf\hat x} \cos [(m-1) \theta] - 
    {\bf\hat y} \sin [(m-1)\theta] \right) \CRNO
  \Delta p_z &= -C \, I_m \, W'_m(z) \, r^m \, \cos m\theta
\end{align}
where $m$ is the order of the mode,
\Begineq
  C = \frac{e}{v \, P_0}
\Endeq
 and
\Begineq
  I_m = q_w \, r_w^m
\Endeq
with $q_w$ being the charge on the particle. Generalizing the above, a
macroparticle at $(r_w, \theta_w)$ will generate a wake
\begin{align}
  -\Delta p_x + i\Delta p_y &= C \, I_m \, 
    m \, r^{m-1} \, e^{-i m \theta_w} \, e^{i (m-1) \theta} 
    \label{ppcimr} \\
  \Delta p_z &= C \, I_m \, W'(z) \, r^m \, \cos [m(\theta - \theta_w)]
    \label{pciwr}
\end{align}
Comparing \Eq{ppcimr} to \eq{bib1nb} and using the relationship between
kick and field as given by \eq{pqlbp1} and \eq{pqlbp2} shows that
the form of the wakefield transverse kick is the same as for a
multipole of order $n = m - 1$. 

The wakefield felt by a particle is due to the wakefields generated by
all the particles ahead of it. If the wakefield kicks are computed by
summing over all leading particle, trailing particle pairs then the
computation will scale as $N^2$ where $N$ is the number of
particles. This quickly becomes computationally exorbitant. A better
solution is to keep track of the wakes in a cavity. When a particle
comes through, the wake it generates is simply added to the existing
wake. This computation scales as $N$ and makes simulations with large
number of particles practical. 

To add wakes together, the wakes must be decomposed into its
components.  Spatially there are normal and skew components and
temporally there are sin and cosine components. This gives 4
components which will be labeled $a_{\cos}$, $a_{\sin}$, $b_{\cos}$,
and $b_{\sin}$. For a mode of order $m$, a particle passing through at
a distance $z_w$ with respect to the reference particle will produce
wake components
\begin{alignat}{2}
  a_{\cos} &=  &c \, \left( \frac{R}{Q} \right) \,
    e^{-k z_w/ 2 Q} \, \cos (k \, z_w) \, I_m \, \sin(m \theta_w) \CRNO
  a_{\sin} &= -&c \, \left( \frac{R}{Q} \right) \,
    e^{-k z_w/ 2 Q} \, \sin (k \, z_w) \, I_m \, \sin(m \theta_w) 
    \label{ac2rq} \\
  b_{\cos} &= -&c \, \left( \frac{R}{Q} \right) \,
    e^{-k z_w/ 2 Q} \, \cos (k \, z_w) \, I_m \, \cos(m \theta_w) \CRNO
  b_{\sin} &=  &c \, \left( \frac{R}{Q} \right) \,
    e^{-k z_w/ 2 Q} \, \sin (k \, z_w) \, I_m \, \cos(m \theta_w) \nonumber
\end{alignat}
These are added to the existing wake components. To calculate the kick
due to wake the normal and skew components are added together
\begin{align}
  a &= e^{k z/ 2 Q} \, \left( 
    a_{\cos} \, \cos (k \, z) + a_{\sin} \, \sin (k \, z) \right) 
    \label{akz2q} \\
  b &= e^{k z/ 2 Q} \, \left(
    b_{\cos} \, \cos (k \, z) + b_{\sin} \, \sin (k \, z) \right) \nonumber 
\end{align}
Here $z$ is the distance of the particle with respect the
reference particle. In analogy to \Eq{ppcimr} and \eq{pciwr} the kick
is
\begin{align}
  -\Delta p_x + i\Delta p_y &= C \, 
    m \, (b + i a) \, r^{m-1} \, e^{i (m-1) \theta} 
    \label{ppcmbar} \\
  \Delta p_z &= C \, r^m \, \left( 
    (b' + i a') e^{i m\theta} + (b' - i a') e^{-i m\theta} \right)
\end{align}
where $a' \equiv da/dz$ and $b' \equiv db/dz$.

The above equations were developed assuming cylindrical symmetry. With
cylindrical symmetry the cavity modes are actually a pair of
degenerate modes. When symmetry is broken the modes no longer have the
same frequency. In this case one has to consider a mode's polarization
angle $\phi$. Equations \eq{akz2q} and \eq{ppcmbar} are unchanged and
in place of \Eq{ac2rq}, the contribution of a particle to a mode is
\begin{alignat}{2}
  a_{\cos} &=  &c \, \left( \frac{R}{Q} \right) \,
    e^{-k z_w/ 2 Q} \, \cos (k \, z_w) \, I_m \, \left[ 
    \sin(m \theta_w) \, \sin^2(m \phi) + 
    \cos(m \theta_w) \, \sin(m \phi) \, \cos(m\phi) \right]
    \CRNO
  a_{\sin} &= -&c \, \left( \frac{R}{Q} \right) \,
    e^{-k z_w/ 2 Q} \, \sin (k \, z_w) \, I_m \, \left[
    \sin(m \theta_w) \, \sin^2(m \phi) + 
    \cos(m \theta_w) \, \sin(m \phi) \, \cos(m\phi) \right]
    \\
  b_{\cos} &= -&c \, \left( \frac{R}{Q} \right) \,
    e^{-k z_w/ 2 Q} \, \cos (k \, z_w) \, I_m \, \left[
    \cos(m \theta_w) \, \cos^2(m \phi) + 
    \sin(m \theta_w) \, \sin(m \phi) \, \cos(m\phi) \right]
    \CRNO
  b_{\sin} &=  &c \, \left( \frac{R}{Q} \right) \,
    e^{-k z_w/ 2 Q} \, \sin (k \, z_w) \, I_m \, \left[
    \cos(m \theta_w) \, \cos^2(m \phi) + 
    \sin(m \theta_w) \, \sin(m \phi) \, \cos(m\phi) \right]
    \nonumber
\end{alignat}

Each mode is characterized by an $R/Q$, $Q$, $\omega$, and $m$. Notice
that $R/Q$ is defined so that it includes the cavity length. Thus the
long--range wake equations, as opposed to the short--range ones, do
not have any explicit dependence on $L$. 

To make life more interesting different people define $R/Q$
differently. A common practice is to define an $R/Q$ ``at the beam
pipe radius''. In this case the above equations must be modified to
include factors of the beam pipe radius.




%-----------------------------------------------------------------
\section{Synchrotron Integrals}
\label{s:synch_ints}
\index{Synchrotron radiation!integrals}

The standard formulas for the synchrotron integrals used to compute
emittances, the energy spread, etc., for a lattice assume no
coupling between the horizontal and vertical
planes\cite{b:helm,b:jowett}.  With coupling the equations need to be
generalized.

\index{dispersion}
In the general case the curvature function $\Bf G = (G_x, G_y)$, which
points away from the center of curvature of the particle's orbit (see
Figure~\ref{f:local_coords}), does not lie in the horizontal
plane. $\Bf G$ has a magnitude $G = 1/\rho$ and a positive $G_y$
indicates a downward bend in the negative $y$ direction.  Similarly
the dispersion, $\bfeta\two = (\eta_x, \eta_y)$ will not lie in the
horizontal plane. With this notation the synchrotron Integrals for the
$a$ and $b$ normal modes are:
  \Begineqs
    I_1 &=& \oint ds \, \Bf G \cdot \bfeta 
         \equiv \oint ds \, (G_x \, \eta_x + G_y \, \eta_y) \\
    I_2 &=& \oint ds \, G^2 \\
    I_3 &=& \oint ds \, G^3 \\
    I_{4a} &=& \oint ds \, \left[ G^2 \, \Bf G \cdot \bfeta\two_a + 
         \nabla G^2 \cdot \bfeta\two_a \right] \\
    I_{4b} &=& \oint ds \, \left[ G^2 \, \Bf G \cdot \bfeta\two_b + 
         \nabla G^2 \cdot \bfeta\two_b \right] \\
    I_{4z} &=& \oint ds \, \left[ G^2 \, \Bf G \cdot \bfeta\two + 
         \nabla G^2 \cdot \bfeta\two \right] \\
    I_{5a} &=& \oint ds \, G^3 \, \calh_a \\
    I_{5b} &=& \oint ds \, G^3 \, \calh_b
  \Endeqs
where $\calh_a$ is 
  \Begineq
    \calh_a = \gamma_a \, \eta_a^2 + 2 \, \alpha_a \, \eta_a \, \eta_a' + 
      \beta_a \eta_a'^2 
  \Endeq
\index{dispersion}
with a similar equation for $\calh_b$. Here $\bfeta\two_a =
(\eta_{ax}, \eta_{ay})$, and $\bfeta\two_b = (\eta_{bx}, \eta_{by})$
are the dispersion vectors for the $a$ and $b$ modes respectively in
$x$--$y$ space (these 2--vectors are not to be confused with the
dispersion 4--vectors used in the previous section). The position
dependence of the curvature function is:
  \Begineqs
    G_x(x,y) = G_{x} + x \, k_1 + y \, s_1 \CRNO
    G_y(x,y) = G_{y} + x \, s_1 - y \, k_1 
  \Endeqs
where $k_1$ is the quadrupole moment and $s_1$ is the skew--quadrupole moment.
Using this gives on--axis ($x = y = 0$)
  \Begineq
    \nabla G^2 = 2 \left( G_x k_1 + G_y s_1, \, G_x s_1 - G_y k_1 \right)
    \label{g2gkg}
  \Endeq

In a dipole a non--zero $e_1$ or $e_2$ gives a contribution to $I_4$
via the $\nabla G^2 \cdot \bfeta$ term. The edge field is modeled as a
thin quadrupole of length $\delta$ and strength $k = -\tan(e) /
\delta$. It is assumed that $\Bf G$ rises linearly within the edge field
from zero on the outside edge of the edge field to its full value on the inside 
edge of the edge field. 
Using this in \Eq{g2gkg} and integrating over the edge field gives the contribution
to $I_4$ from a non--zero $e_1$ as
  \Begineq
    I_{4z} = -\tan(e_1) \, G^2
    \left( \cos(\theta) \, \eta_x + \sin(\theta) \, \eta_y \right)
    \label{iegct}
  \Endeq
With an analogous equation for a finite $e_2$. The extension to
$I_{4a}$ and $I_{4b}$ involves using $\bfeta\two_a$ and $\bfeta\two_b$
in place of $\bfeta\two$.  In \Eq{iegct} $\theta$ is the \vn{tilt}
angle which is non--zero if the bend is not in the horizontal plane.

The above integrals are invariant under rotation of the $(x,y)$ coordinate
system and reduce to the standard equations when $G_y = 0$ as they should.

There are various parameters that can be expressed in terms of these
integrals.  The $I_1$ integral can be related to the momentum
compaction $\alpha_p$ via
  \Begineq
    I_1 = \alpha_p \, L
  \Endeq
where $L$ is the storage ring circumference. The energy loss per turn is
  \Begineq
    U_0 = \frac{2 \, r_e E_0^4}{3 \, mc^2} I_2
  \Endeq
where $E_0$ is the nominal energy and $r_e$ is the classical electron
radius (electrons are assumed here but the formulas are easily
generalized).

The damping partition numbers are
  \Begineq
    J_a = 1 - \frac{I_{4a}}{I_2} \comma \quad
    J_b = 1 - \frac{I_{4b}}{I_2} \comma \, \mbox{and} \quad \label{j1ii}
    J_z = 2 + \frac{I_{4z}}{I_2} \period
  \Endeq
Since 
  \Begineq          
    \bfeta\two_{a} + \bfeta\two_{b} = \bfeta\two
    \comma \label{eee}
  \Endeq
Robinson's theorem, $J_a + J_b + J_z = 4$, is satisfied.
Alternatively, the exponential damping coefficients per turn are
  \Begineq
    \alpha_a = \frac{U_0 \, J_a}{2 E_0} \comma \quad
    \alpha_b = \frac{U_0 \, J_b}{2 E_0} \comma \, \mbox{and} \quad
    \alpha_z = \frac{U_0 \, J_z}{2 E_0} \period
  \Endeq
The energy spread is given by
  \Begineq
    \sigma_{pz}^2 = \left( \frac{\sigma_E}{E_0} \right)^2 = 
    C_q \gamma_0^2 \frac{I_3}{2I_2 + I_{4z}}
  \Endeq
where $\gamma_0$ is the usual energy factor and 
  \Begineq
    C_q = \frac{55}{32 \, \sqrt{3}} \, \frac{\hbar}{mc} = 
    3.84 \times 10^{-13} \, \mbox{meter for electrons}
  \Endeq
If the synchrotron frequency is not too large the bunch length is given by
  \Begineq
    \sigma_z = \frac{I_1}{M(6,5)} \, \sigma_{pz}
  \Endeq
where $M(6,5)$ is the $(6,5)$ element for the 1--turn transfer matrix
of the storage ring. Finally the emittances are given by
  \Begineqs
    \epsilon_a \AND= C_q \, \gamma_0^2 \frac{I_{5a}}{I_2 - I_{4a}} \CRNO
    \epsilon_b \AND= C_q \, \gamma_0^2 \frac{I_{5b}}{I_2 - I_{4b}}
  \Endeqs

For a linac radiation integrals are still of interest if there are
bends but in this case the appropriate energy factors must be included
to take account the changing energy in the line. $I_1$ is not altered and
the $I_4$ integrals are not relevant. The other integrals become
  \Begineqs
    L_2 &=& \int ds \, G^2 \, \gamma_0^4 \\
    L_3 &=& \int ds \, G^3 \, \gamma_0^7 \\
    L_{5a} &=& \int ds \, G^3 \, \calh_a \, \gamma_0^6 \\
    L_{5b} &=& \int ds \, G^3 \, \calh_b \, \gamma_0^6
  \Endeqs
In terms of these integrals the energy loss through the linac is
  \Begineq
    U_0 = \frac{2 \, r_e \, mc^2}{3} L_2
  \Endeq
The energy spread assuming $\sigma_E$ is zero at the start and neglecting
any damping is
  \Begineq
    \sigma_E^2 = \frac{4}{3} \, C_q \, r_e \, \left( m c^2 \right)^2 \, L_3
  \Endeq
and, again neglecting any initial beam width, the transverse beam size
at the end of the linac is
  \Begineqs
    \epsilon_a \AND= \frac{2}{3} \, C_q \, r_e \, 
    \frac{L_{5a}}{\gamma_f} \CRNO
    \epsilon_b \AND= \frac{2}{3} \, C_q \, r_e \, 
    \frac{L_{5b}}{\gamma_f} 
  \Endeqs
Where $\gamma_f$ is the final gamma

%-----------------------------------------------------------------   
 \section{Spin Dynamics}   
 \label{s:spin_dyn}   
 \index{Spin tracking!Spin dynamics}   
    
 The matrix representation of the observable corresponding to the measurement of   
 spin along the unit vector $\Bf u$ in the ${ |+> = \left(\begin{matrix}1 \\ 0   
 \end{matrix} \right), |-> =  \left(\begin{matrix} 0 \\ 1 \end{matrix} \right) }$ 
basis is   
   \Begineqs   
     S_{\Bf u} &=& \mathbf{S \cdot u} \\   
               &=& \frac{\hbar}{2} \sigma_{x} \sin \theta \cos \phi +   
                   \frac{\hbar}{2} \sigma_{y} \sin \theta \sin \phi +   
                   \frac{\hbar}{2} \sigma_{z} \cos \theta \\   
               &=& \frac{\hbar}{2} \left(   
                    \begin{matrix}   
                      \cos \theta            & \sin \theta e^{- i \phi} \\   
                      \sin \theta e^{i \phi} & - \cos \theta   
                    \end{matrix} \right)   
   \Endeqs   
 where $\theta$ and $\phi$ are the polar angles characterizing the vector $\Bf u$   
 and $\sigma_{x,y,z}$ are the three pauli matrices.   
 The expectation value of this operator representing the spin of a particle   
 satisfies the equation of motion of a classical spin vector in the particle's   
 instantaneous rest frame. The classical spin vector $\Bf s$ is described for   
 time $t$ in the   
 laboratory frame by the Thomas-Bargmann-Michel-Telegdi (T-BMT) equation,   
   \Begineqs   
     \frac{\mathrm{d}}{\mathrm{d}t} \Bf s &=& \Bf \Omega_{BMT} (\mathbf{r,p,t})   
     \mathbf{ \times  s}, \\   
             \Omega_{BMT}(\mathbf{r,p,t}) &=& - \frac{q}{m \gamma} \left[\left(1   
             + G \gamma \right) \Bf B - \frac{G \Bf p \cdot \Bf B}{\left( \gamma   
             + 1\right) m^{2} c^{2}} \Bf p - \frac{1}{m c^{2}}\left(G +   
             \frac{1}{1 + \gamma}\right)\mathbf{ p \times  E}\right],   
   \Endeqs   
 where $\Bf E (\Bf r ,t)$ and $\Bf B (\Bf r ,t)$ are the electric and magnetic   
 fields in the laboratory frame, $\Bf p$ and $\gamma$ are the particle's momentum   
 and relativistic gamma factor in the laboratory frame, $q$ and $m$ are the   
 particle's charge and mass, and $G = \frac{\left(g-2\right)}{2}$ is the   
 particle's anaomalous gyro-magnetic g-factor which is $1.793$ for protons and   
 $0.00116$ for electrons and positrons.   
    
 For a distribution of spins the polarization $P$ along the unit vector $\Bf u$ is
 defined as the absolute   
 value of the average expectation value of the spin over all N particles times   
 $\frac{2}{\hbar}$,   
   \Begineqs   
     P &=& \frac{2}{\hbar} \| \frac{1}{N} \sum_{j=1}^{N}<+|S_{\Bf u}|+> \| \\   
       &=& \| \frac{1}{N} \sum_{j=1}^{N} \cos \theta \|.   
   \Endeqs   
    
 It is more efficient to use the SU(2) representation rather than SO(3) when   
 describing rotations of spin. In the SU(2) representation, a spin $\Bf s$ is   
 written as a spinor $\Psi = \left( \psi_{1}, \psi_{2} \right)^{T}$ where   
 $\psi_{1,2}$ are complex numbers and   
   \Begineqs   
                     \Bf s &=& \Psi^{\dagger} \Bf \sigma \Psi \\   
     \leftrightarrow \Psi  &=& \frac{1}{\sqrt{2 \left(s_{3}+1\right)}}   
              \left( \begin{matrix} 1+s_{3} \\ s_{1}+i s_{2} \end{matrix}   
              \right),   
   \Endeqs   
 or in polar coordinates,   
   \Begineqs   
     \Psi &=& \left( \begin{matrix} \psi_{1} \\ \psi_{2} \end{matrix} \right)   
          = e^{i \xi} \left( \begin{matrix} \cos \frac{\phi}{2}\\   
                      \sin \frac{\phi}{2} e^{i \phi}   
                      \end{matrix} \right) \\   
          \leftrightarrow   
     \Bf s &=& \left( \begin{matrix} \sin \theta \cos \phi \\   
                                     \sin \theta \sin \phi \\   
                                     \cos \theta \end{matrix} \right).   
   \Endeqs   
 The $\xi$ being an extra phase factor. Due to the unitarity of the spin vector,   
 $|\psi_{1}|^{2} + |\psi_{2}|^{2} = 1$.   
    
 In spinor notation the T-BMT equation can be written   
   \Begineq   
     \frac{\mathrm{d}}{\mathrm{d} t} \Psi = - \frac{i}{2} \left( \Bf \sigma \cdot   
     \Bf \Omega \right) \Psi.   
   \Endeq   
 The solution leads to a rotation of the spin vector by an angle   
 $\alpha$ around a unit vector $\Bf e$ represented as   
   \Begineqs   
     \Psi &=& e^{-i \frac{\alpha}{2} \Bf e \cdot \Bf \sigma} \Psi_{i} \\   
          &=& \left( a_{0} \Bf 1_{2} - i \boldsymbol{a} \cdot \Bf \sigma \right) \Psi_{i} \\   
          &=& \Bf A \Psi_{i}.   
   \Endeqs   
 The SU(2) matrix $\Bf A$ is called a \textit{quaternion} and has the   
 normalization condition $a_{0}^{2} + \boldsymbol{a}^{2} = 1$. The three components   
 $\left(a_{1}, a_{2}, a_{3}\right)$, along with the normalization condition,   
 describes the transport of spin between any two points in an accelerator. 
 
% LocalWords: rcr


%----------------------------------------------------------------
\part{Language Reference}
%----------------------------------------------------------------
\chapter{Syntax}
\label{c:syntax}

%------------------------------------------------------------------------
\section{Command Line Syntax}
\label{s:com.syntax}

In ``\vn{line mode}'' (\sref{c:command}), commands are case sensitive. Multiple commands may be
entered on one line using the semicolon ``;'' character as a separator.  [However, a semicolon used
as as part of an \vn{alias} (\sref{s:alias}) definition is part of that definition.]  An exclamation
mark ``\vn{!}''  denotes the beginning of a comment and the exclamation mark and everything after it
to the end of the line is ignored.  Example:
\begin{example}
  set default uni = 2; show global  ! Two commands and a comment
\end{example}

The length of a command on a single line is currently limited to 1000 characters. Multiple lines may
be used for a single command by putting a ``\&'' character at the end of a line to be
continued. Example:
\begin{example}
  set default &    ! Continue command to next line
  uni = 2
\end{example}

Note that, for historical reasons, Bmad itself is case insensitive. Thus things like lattice element 
names within \tao commands will similarly be case insensitive.

%------------------------------------------------------------------------
\section{Specifying a Single Lattice Element}
\label{s:ele.name}

A full description of how to specify a lattice element is given in section \extref{B-s:ele.match}
``\vn{Matching to Lattice Element Names}'' in the \bmad manual. Generally, elements are specified
using either their names or by their index number. Additionally, in \tao, the universe in which
the element exists may be specified by prepending the element name by the universe number followed by
the ``\vn{@}'' sign.
Examples:
\begin{example}
  Q3##2      ! 2nd instance of element named Q3 in branch 0 of the default universe.
  134        ! Element with index 134 in branch 0 of the default universe.
  1>>13      ! Element with index 13 in branch 1 of the default universe.
  2@1>>TZ    ! Element named TZ in branch 1 of universe 2.
  B37        ! Element named B37 of the default universe.
  0@B37      ! Same as the previous example.
\end{example}
Note: element names are {\em not} case sensitive.

%------------------------------------------------------------------------
\section{Lattice Element List Format}
\label{s:ele.list.format}

The syntax for specifying a set of lattice elements is called \vn{element list} format. 
A element list is a list of items separated by a comma.\footnote
  {
 A blank space may be acceptable in some circumstances but a comma is always safe.
  }
Each item of the list is one of:
\begin{center}
\begin{tabular}{ll}
  {\it Item Type} & {\it Example} \\ \hline     
  A single element (\sref{s:ele.name})                & "1>>Q10W"            \\
  A name with wild card characters                    & "5@q*"               \\
  A range of elements in the form <ele1>:<ele2>       & "b23w:67"            \\
  A class::name specification                         & "sbend::b*"          \\
\end{tabular}
\break
\end{center}

Example element list:
\begin{example}
  23, 45:74, quad::q*
\end{example}

The wild card characters ``*'' and/or ``\%'' can be used. The ``*'' wildcard matches any number of
characters, The ``\%'' wildcard matches a single character. For example, ``q\%1*'' matches any
element whose name begins with ``q'' and whose third character is ``1''.  If there are multiple
elements in the lattice that match a given name, all such elements are included. Thus ``d12'' will
match to all elements of that name. Examples
\begin{example}
  "134"        ! Element with index 134 in branch 0 of the default universe.
  "1>>13"      ! Element with index 13 in branch 1 of the default universe.
  "5@q*"       ! All elements whose name begins with "q" of universe 5.
  "2@3>>q1##4" ! The fourth element named "q1" in branch 3 of universe 2.
  "*@sex10w"   ! Element "sex10w" of all universes.
  "b37"        ! Element "b37" of the default universe.
  "0@b37"      ! Same as the previous example.
\end{example}
Note: element names are {\em not} case sensitive.

An element index item is simply the index of the number in the lattice list of elements. A prefix
followed by the string ">>" can be used to specify a branch. As with element names, a universe
prefix can be given. Example
\begin{example}
  2@3>>183   ! Element number 183 of branch \# 3 of universe 2.
\end{example}

A range of elements is specified using the format:
\begin{example}
  \{<class>::\}<ele1>:<ele2>
\end{example}
\vn{<ele1>} is the element at the beginning of the range and \vn{<ele2>} is the element at the end
of the range. Either an element name or index can be used to specify \vn{<ele1>} and
\vn{<ele2>}. Both \vn{<ele1>} and \vn{<ele2>} are part of the range. The optional \vn{<class>}
prefix can be used to select only those elements in the range that match the class.  Example:
\begin{example}
  quad::sex10w:sex20w
\end{example}
This will select all quadrupoles between elements \vn{sex10w} and \vn{sex20w}.

\index{class::name}
A \vn{class::name} item
selects elements based upon their class (Eg: \vn{quadrupole},
\vn{marker}, etc.), and their name. The syntax is:
\begin{example}
  <element class>::<element name>
\end{example}
where \vn{<element class>} is an element class and \vn{<element name>} is the element name that can
(and generally does) contain the wild card characters ``\%'' and ``*''. Essentially this is an
extension of the \vn{element name} format. As with element names, a universe prefix can be
given. Example:
\begin{example}
  "4@quad::q*"   ! All quadrupole whose name starts with "q" of universe 4.
\end{example}

%------------------------------------------------------------------------
\section{Arithmetic Expressions}
\index{arithmetic Expressions}
\label{s:arithmetic.exp}

\tao is able to handle arithmetic expressions within commands (\sref{c:command}) and in strings in a
\tao initialization file.  Arithmetic expressions can be used in a place where a real value or an
array of real values are required.  The standard binary operators are defined: \hfil\break \hspace*{0.15in}
\begin{tabular}{ll}
  $a + b$           & Addition        \\
  $a - b$           & Subtraction     \\
  $a \, \ast \, b$  & Multiplication  \\
  $a \; / \; b$     & Division        \\
  $a \, \land \, b$ & Exponentiation  \\
\end{tabular} \newline
The following intrinsic functions are also recognized (this is the same list as the \bmad parser):
\hfil\break
\index{intrinsic functions}
\hspace*{0.15in}
\begin{tabular}{ll}
  \vn{sqrt}(x)                & Square Root                      \\
  \vn{log}(x)                 & Logarithm                        \\
  \vn{exp}(x)                 & Exponential                      \\
  \vn{sin}(x), \vn{cos}(x)    & Sine and Cosine                  \\
  \vn{tan}(x), \vn{cot}(x)    & Tangent and Cotangent            \\
  \vn{asin}(x), \vn{acos}(x)  & Arc sine and Arc Cosine          \\
  \vn{atan}(y)                & Arc Tangent                      \\
  \vn{atan2}(y, x)            & Arc Tangent                      \\
  \vn{abs}(x)                 & Absolute Value                   \\
  \vn{factorial(x)}           & Factorial                        \\
  \vn{ran}()                  & Random number between 0 and 1    \\
  \vn{ran_gauss}()            & Gaussian distributed random number with unit RMS \\
  \vn{ran_gauss}(sig_cut)     & Gaussian distributed random number truncated at sig_cut. \\
  \vn{int(x)}                 & Nearest integer with magnitude less then x \\
  \vn{nint(x)}                & Nearest integer to x             \\
  \vn{floor(x)}               & Nearest integer less than x      \\
  \vn{ceiling(x)}             & Nearest integer greater than x   \\
  \vn{modulo(a, p)}           & a - floor(a/p) * p. Will be in range [0, p]. \\
  \vn{average(arr)}           & Average value of an array        \\
  \vn{rms(arr)}               & RMS value of an array            \\
  \vn{sum(arr)}               & Sum of array values.             \\
  \vn{min(arr)}               & Minimum of array values.         \\
  \vn{max(arr)}               & Maximum of array values.         \\
  \vn{mass_of}(A)               & Mass of particle A             \\
  \vn{charge_of}(A)             & Charge, in units of the elementary charge, of particle A \\
  \vn{anomalous_moment_of}(A)   & Anomalous magnetic moment of particle A        \\
  \vn{species}(A)               & Integer ID associated with species A
\end{tabular} \newline
Both \vn{ran} and \vn{ran_gauss} use a seeded random number generator.  Setting the seed is
described in Section~\sref{s:globals}.

Expressions may involve arrays of values. For example:
\begin{example}
  lat::orbit.x[5:8]     ! X-orbit at lattice elements 5 through 8.
  [1, 2, 3]             ! A vector of size three.
\end{example}
When using vectors with binary operators or intrinsic functions, the standard rules apply. For example:
\begin{example}
  s * [a, b, c]         = [s*a, s*b, s*c]
  [a, b, c] - [x, y, z] = [a-x, b-y, c-z]
  tan([a, b, c])        = [tan(a), tan(b), tan(c)]
  sum([a, b, c])        = a+b+c
  min(a, b, c)          ! Error: Correct is min([a, b, c])
\end{example}
Note that \tao does not make a distinction between a scalar and a vector of length one.

See the following sections for the syntax for using data, variable, and lattice parameters in an
expression. Use the \vn{show value} command (\sref{s:show.value}) to show the results of expressions.

%------------------------------------------------------------------------
\section{Specifying Data Parameters in Expressions}
\label{s:data.token}

A data (\sref{s:data.org}) parameter ``\vn{token}'' is a string that specifies a scalar or an array
of data parameters.  The general form for data tokens in expressions (\sref{s:arithmetic.exp}) is:
\begin{example}
  \{[<universe(s)>]@\}data::<d2.d1_name>[<index_list>]|<component>
\end{example}
where:
\begin{example}
  <universe(s)>       Optional universe specification (\sref{s:universe})
  <d2.d1_name>        D2.D1 data name
  <index_list>        List of indexes.
  <component>         Component. 
\end{example}
examples:
\begin{example}
  [2:4,7]@data::orbit.x      ! The \vn{orbit.x} data in universes 2, 3, 4 and 7.
  [2]@data::orbit.x          ! The \vn{orbit.x} data in universe 2. 
  2@data::orbit.x[4]         ! Fourth \vn{orbit.x} datum in universe 2.
  data::orbit.x[4,7:9]|meas  ! Default uni measured values of datums 4, 7, 8, and 9.
  *@data::orbit.x            ! orbit.x data in all the universes.
  *@data::*                  ! All the data in all the universes.
\end{example}

It is important to keep in mind that data must be defined at startup in the appropriate
initialization file as discussed in \Sref{s:init.data} before reference is made to data in an
expression. The \vn{<d2.d1_name>} data names that have been defined at initialization time may be
viewed using the \vn{show data} command. Note that these names are user defined and do not have to
correspond to the data types given in \Sref{s:data.types}. See \Sref{s:lat.token} for how to use
``lattice parameters'' that correspond to the data types given in \Sref{s:data.types} and that do
not need to be defined at initialization.

See \Sref{s:data.anatomy} for a list of datum \vn{<component>}s (when running \tao, view a
particular datum with the \vn{show data} command to see the list).

\vn{<index_list>} is a list of indexes. \vn{<index_list>} will determine how many elements are in
the array. For example, \vn{orbit.x[10:21,44]} represents an array of 13 elements.

Depending upon the context, some parts of a token may be omitted. For example, with the \vn{set
data} command the ``\vn{data::}'' part of the token may be omitted.  Example:
\begin{example}
  set data 2@orbit.x|meas = var::quad_k1[5]|model - orbit.y[3]|ref
\end{example}
Here \tao will default to evaluating a token as data. In general, what may be omitted
should be clear in context.

Data components that are computed by \tao may be used on the right hand side of an equal sign but
may not be set. For example, the \vn{model} value of a datum is computed by \tao but the \vn{ref}
value is not.

If multiple tokens are used in an expression, all tokens must evaluate to arrays of the same size.

%------------------------------------------------------------------------
\section{Specifying Variable Parameters in Expressions}
\label{s:var.token}

A variable (\sref{c:var}) parameter ``\vn{token}'' is a string that specifies a scalar or an array
of variable parameters. The general form for variable tokens in expressions
(\sref{s:arithmetic.exp}) is:
\begin{example}
  var::<v1_name>[<index_list>]<component>
\end{example}
where:
\begin{example}
  <universe(s)>       Optional universe specification (\sref{s:universe})
  <v1_name>           V1 variable name.
  <index_list>        List of indexes.
  <component>         component. 
\end{example}
Examples:
\begin{example}
  var::*                     ! All the variables
  var::quad_k1[*]|design     ! All design values of quad_k1.
  var::quad_k1[]|model       ! No values. That is, the empty set.
  var::quad_k1|model         ! Same as quad_k1[*]|model
\end{example}

It is important to keep in mind that variables must be defined at startup in the appropriate
initialization file as discussed in \Sref{s:init.var} before reference is made to them in an
expression.  The defined \vn{<v1_name>} variable names can be viewed using the \vn{show variable}
command. Since these names are user defined they will change if different initialization files are
used.

See \Sref{c:var} for a list of \vn{<components>} of a variable.

\vn{<index_list>} is a list of indexes. \vn{<index_list>} will determine how many elements are in
the array. For example, \vn{k_quad[10:21,44]} represents an array of 13 elements.

Depending upon the context, some parts of a token may be omitted. For example, with the \vn{set
variable} command the ``\vn{var::}'' part of the token may be omitted.  Example:
\begin{example}
  set var quad_k1[5]|meas = data::2@orbit.x|meas
\end{example}
Here \tao will default to evaluating a token as a variable component. In general, what may be
omitted should be clear in context.

Variable components that are computed by \tao may be used on the right hand side of an equal sign
but may not be set. For example, the \vn{design} value of a variable is computed by \tao but the
\vn{meas} value is not.

If multiple tokens are used in an expression, all tokens must evaluate to arrays of the same size.

%------------------------------------------------------------------------
\section{Specifying Lattice Parameters in Expressions}
\label{s:lat.token}

``Lattice parameters'' are like \vn{data} parameters (\sref{s:data.token}) except lattice parameters
are calculated from the lattice and do not have to be defined at initialization time.  A lattice
parameter ``\vn{token}'' is a string that specifies a scalar or an array of lattice parameters. The
general form for data tokens in expressions (\sref{s:arithmetic.exp}) is:
\begin{example}
  \{[<universe(s)>]@\}lat::<eval_param>[\{<ref_ele>&\}<element_list>\{-><s_offset>\}]\{|<component>\}
\end{example}
where:
\begin{example}
  <universe(s)>       Optional universe specification (\sref{s:universe})
  <eval_param>        Name of the parameter to evaluate. 
                        Possible data types listed in \Sref{s:data.types}. 
  <ref_ele>           Optional reference element.
  <element_list>      Evaluation point or points.
  <s_offset>          Longitudinal offset to evaluate at.
  <component>         Optional component. 
\end{example}
The \vn{<s_offset>} string can be an expression. Any parameter in this expression, if not qualified,
will be interpreted as a parameter of the element containing the evaluation point. For example
\begin{example}
  3@lat::orbit.x[q10w->-L/2]|model
\end{example}
Here ``\vn{L}'' in the \vn{<s_offset>} string ``\vn{-L/2}'' is interpreted as the length of the element
\vn{q10w}. Other examples:
\begin{example}
  3@lat::orbit.x[34:37]            ! Array of orbits at element 34 through 37 in universe 3.
  3@lat::orbit.x[q10w]|model       ! Orbit.x model value at exit end of element q10w
  3@lat::orbit.x[q10w->0.1]|model  ! Same as above except eval point is shifted 0.1 meters.
  lat::sigma.12[q10w]              ! Beam sigma matrix component at element q10w computed 
                                   !  from lattice parameters.
\end{example}

The list of possible lattice \vn{<eval_param>} names is given in \Sref{s:data.types}. The table
\ref{t:data.types} shows which data names are associated with the lattice. Lattice parameters are
independent of \vn{data} parameters. For example, \vn{lat::orbit.x} refers to the horizontal orbit
while \vn{data::orbit.x} refers to user defined data whose name corresponds to \vn{orbit.x} and in
fact there is nothing to prevent a user from assigning the name \vn{orbit.x} to data that is derived
from, say, the Twiss beta function.

Also notice the difference between, say, ``\vn{lat::orbit.x[10]}'' and ``\vn{data::orbit.x[10]}''.
With the ``\vn{lat::}'' source, the element index, in this case \vn{10}, refers to the 10th lattice
element. With the ``\vn{data::}'' source, ``\vn{10}'' refers to the 10\Th element in the
\vn{orbit.x} data array which may or may not correspond to the 10\Th lattice element.

The optional \vn{<ref_ele>} specifies a reference element for the evaluation. For example
\begin{example}
  lat::r.56[q0\&qa:qb]
\end{example}  
is an array of the $r(5,6)$ matrix element of the transport map between element \vn{q0} and each
element in the range from element \vn{qa} and \vn{qb}.

The optional \vn{s_offset>} specifies a longitudinal offset for the evaluation point. This may be
an expression.

%------------------------------------------------------------------------
\section{Specifying Beam Parameters in Expressions}
\label{s:beam.token}

Beam parameters are like lattice parameters (\sref{s:lat.token}) except beam parameters are derived
from tracking a beam of particles and may only be used in an expression if beam tracking is turned
on.  A beam parameter ``\vn{token}'' is a string that specifies a scalar or an array of beam
parameters. The general form for data tokens in expressions (\sref{s:arithmetic.exp}) is:
\begin{example}
  \{[<universe(s)>]@\}beam::<eval_param>[\{<ref_ele>&\}<element_list>]\{|<component>\}
\end{example}
where:
\begin{example}
  <universe(s)>       Optional universe specification (\sref{s:universe})
  <eval_param>        Name of the parameter
  <ref_ele>           Optional reference element.
  <element_list>      Evaluation point or points.
  <component>         Component. 
\end{example}
Examples:
\begin{example}
  2@beam::sigma.x[q10w]           Beam sigma at element q10w.
  beam::n_particle_loss[2&56]     Particle loss between elements 2 and 56.
\end{example}

The list of possible beam \vn{<eval_param>} names is given in \Sref{s:data.types}. The table
\ref{t:data.types} shows which data names are associated with beam tracking.

%------------------------------------------------------------------------
\section{Specifying Element Parameters in Expressions}
\label{s:ele.token}

``Element parameters'' are parameters associated with lattice elements like the quadrupole strength
associated with an element. Element parameters also include derived quantities like the computed
Twiss parameters and the beam orbit. An element parameter ``\vn{token}'' is a string that specifies
a scalar or an array of element parameters. The general form for element tokens in expressions is:
\begin{example}
  \{<universe(s)>@\}ele::<element_list>[<parameter>]\{|<component>\}
  \{<universe(s)>@\}ele_mid::<element_list>[<parameter>]\{|<component>\}
\end{example}
where:
\begin{example}
  <universe(s)>       Optional universe specification (\sref{s:universe})
  <element_list>      List of element names or indexes.
  <parameter>         Name of the element parameter
  <component>         Component. 
\end{example}
Examples:
\begin{example}
  3@ele_mid::34[orbit_x]     Orbit at middle of element with index 34 in universe 3.
  ele::sex01w[k2]            Sextupole component of element \vn{sex01w}
  ele::Q01W[is_on]|model     The on/off status of element \vn{Q01W}.
\end{example}

There is some overlap between element parameters and lattice parameters (\sref{s:lat.token}).  For
historical reasons, the \vn{element} parameter syntax roughly follows a convention developed for
\bmad lattice files which is somewhat different from the convention developed for \tao data. For
example, the $a$-mode beta is named \vn{beta.a} in \tao while \bmad uses the name \vn{beta_a}. See
the \bmad manual for more information on the \bmad lattice file syntax. The following table lists
the parameters that have both \tao datum and \bmad element parameter names
\begin{table}[ht] 
\centering 
{\tt
\begin{tabular}{lll} \toprule
  \vn{\tao Datum}                   & \vn{\bmad Element Parameter}        \\ \midrule
  alpha.a, alpha.b                  & alpha_a, alpha_b                    \\
  beta.a, beta.b                    & beta_a, beta_b                      \\
  cmat.11, etc.                     & cmat_11, etc.                       \\
  e_tot                             & e_tot                               \\
  eta.a, eta.b                      & eta_a, eta_b                        \\
  eta.x, eta.y                      & eta_x, eta_y                        \\
  etap.a, etap.b                    & etap_a, etap_b                      \\
  etap.x, etap.y                    & etap_x, etap_y                      \\
  floor.x, floor.y, floor.z         & x_position, y_position, z_position  \\
  floor.theta, floor.phi, floor.psi & theta_position, phi_position, psi_position \\
  gamma.a, gamma.b                  & gamma_a, gamma_b                    \\
  phase.a, phase.b                  & phi_a, phi_b                        \\
\bottomrule
\end{tabular}
} 
\caption{\tao datums that have equivalent \bmad element parameters.}  
\label{t:bmad.equiv1}
\end{table}

The following table lists the parameters that have both \tao datum and \bmad particle orbit names
\begin{table}[ht] 
\centering 
{\tt
\begin{tabular}{lll} \toprule
  \vn{\tao Datum}               & \vn{\bmad Orbit Parameter}         \\ \midrule
  orbit.x, orbit.y, orbit.z     & x, y, z                            \\
  orbit.px, orbit.py, orbit.pz  & px, py, pz                         \\
  spin.x, spin.y, spin.z        & spin_x, spin_y, spin_z             \\
  spin.amp spin.theta, spin.phi & spinor_polarization, spinor_theta, spinor_phi \\
\bottomrule
\end{tabular}
} 
\caption{\tao datums that have equivalent \bmad orbital parameters.}  
\label{t:bmad.equiv2}
\end{table}

For parameters that are varying throughout the element, like the Twiss parameters, \vn{ele::} will
evaluate the parameter at the exit end of the element and \vn{ele_mid::} will evaluate the parameter
at the middle of the element. For parameters that do not vary, like the quadrupole strength, use the
\vn{ele::} syntax.

Element list format (\sref{s:ele.list.format}) is used for the \vn{<element_list>} so an array of
elements can be defined.

For element parameter that evaluate to a logical, if they are used on the right hand side of an
expression where the result is a real number, a \vn{True} value will be converted to a value of
\vn{1} and a \vn{False} value is converted to a value of \vn{0}.

%------------------------------------------------------------------------
\section{Format Descriptors}
\label{s:edit.descrip}

Some \tao commands like \vn{show lattice} (\sref{s:show.lattice}) have optional arguments where the
format output of various quantities can be specified. \tao follows Fortran format descriptor
notation.  Since complete information is available on the Web (do a search for ``fortran edit
descriptor''), only a brief introduction will be given here.

Format descriptors are case insensitive. The commonly used descriptors with \tao are:
\begin{example}
  Form     Output
  ----     ----------------------------
  Aw       String
  Fw.d     Real numbers. Fixed point (no exponent).
  nPFw.d   Real numbers. Fixed point with the decimal point shifted \vn{n} places.
  ESw.d    Real numbers. Floating point (with exponent).
  Lw       Logicals.
  Iw       Integers.
  Iw.r     Integers padded with zeros to width \vn{r}.
  wX       White space.
  Tc       Tab to column \vn{c}.
\end{example}
In the above, ``\vn{w}'' is the width of the field (number of charactgers in the printed string) and 
``\vn{d}'' is the number of digits to the right of the decimal place, 

Examples:
\begin{example}
           Internal
  Format   Quantity   Output String   Comment
  ----     --------   -------------   -----------
  F7.2     76.1234    "  76.12"       Right justified.
  1PF7.2   76.1234    " 761.23"       Shifted decimal place 1 digit.
  F0.2     76.1234    "76.12"         0 Field width => Output width exactly fits.
  F3.2     76.1234    "***"           Number overflows field width.
  ES9.2    76.1234    " 7.61E+01"     Right justified.
  L3       True       "  T"           Right justified.
  I0       34         "34"            0 Field width => Actual width = number of digits.
  I4       34         "  34"          Right justified.
  I4.3     34         " 034"          Number padded with a zero to three digits.
  A3       "abcdef"   "abc"           String truncated.
  A3       "ab    "   "ab "           String truncated but looks left justified.
  A        "abcdef"   "abcdef"        Output width exactly fits string.
  A8       "abcdef"   "  abcdef"      Right justified.
  4x                  "   "           Four spaces.
  T45                                 Next output string starts at column 45
\end{example}

Note: When a format descriptor is being used to construct a table (EG \vn{show lattice} command),
using a ``\vn{0}'' for the field width is ill-advised since columns will not be properly aligned.

A comma delimited list is used for outputting multiple quantities. For example, the format ``\vn{I4,
A}'' is used to output an integer followed by a string.

If multiple quantities with the same format are to be outputted a \vn{multiplier} prefix number can
be used. For example, ``\vn{3A}'' is equivalent to ``\vn{A, A, A}''. If the format has a \vn{P}
prefix then parentheses can be used to separate the multiplier from the \vn{P} prefix. Example:
``\vn{2(3PF7.2)}'' is equivalent to ``\vn{3PF7.2, 3PF7.2}''.

Note to programmers: In a code file, a format string must always be enclosed in parentheses.

\include{arithmetic_expressions}
\chapter{Parameters}
\chapter{Elements}

A lattice for a storage ring or linac is made up of a collection of
elements --- Quadrupoles, Bends, etc. This chapter discusses the
various types of elements available in \bmad.

\section{Bmad Elements}

Most element types available in \mad\ are provided in \bmad.
Additionally, \bmad\ provides a number of element types that are not
available in \mad.  A word of caution: In some cases where both \mad\
and \bmad\ provide the same element type, there will be an overlap of 
the attributes available but the two sets of attributes will not be the same.
The list of element types known to \bmad\ is shown in Table~\ref{tab:elements}.
In
\begin{table}[h]
{\centering
{\tt
\begin{tabular}{|l|l||l|l|} \hline
  {\it Element} & {\it Section}     & {\it Element} & {\it Section} \\ \hline
  ab\_multipole & \ref{s:ab_m}      &  octupole     & \ref{s:oct}   \\ \hline
  accel\_sol    & \ref{s:accel_sol} &  overlay      & \ref{s:over}  \\ \hline
  beambeam      & \ref{s:bbi}       &  patch        & \ref{s:patch} \\ \hline
  custom        & \ref{s:custom}    &  quadrupole   & \ref{s:quad}  \\ \hline
  drift         & \ref{s:drift}     &  rbend        & \ref{s:rbend} \\ \hline
  ecollimator   & \ref{s:col}       &  rcollimator  & \ref{s:col}   \\ \hline
  elseparator   & \ref{s:elsep}     &  rfcavity     & \ref{s:rfcav} \\ \hline
  group         & \ref{s:group}     &  sbend        & \ref{s:sbend} \\ \hline
  hkicker       & \ref{s:kicker}    &  sextupole    & \ref{s:sex}   \\ \hline
  instument     & \ref{s:mon}       &  solenoid     & \ref{s:sol}   \\ \hline
  kicker        & \ref{s:kicker}    &  sol\_quad    & \ref{s:sq}    \\ \hline
  lcavity       & \ref{s:lcav}      &  taylor       & \ref{s:tay}   \\ \hline
  marker        & \ref{s:mark}      &  vkicker      & \ref{s:kicker}\\ \hline
  monitor       & \ref{s:mon}       &  wiggler      & \ref{s:wig}   \\ \hline
  multipole     & \ref{s:m}         &               &               \\ \hline  
\end{tabular}
}}
\caption{\bmad\ elements.}
\label{tab:elements}\center
\end{table}

\vfil
\break

%-----------------------------------------------------------------
\section{AB\_Multipole}
\label{s:ab_m}

An \vn{AB_Multipole} is a thin multipole lens up to 20th order. The only
difference between this and a \vn{Multipole} is the input format. See the 
Magnetic fields section \ref{s:fields} for more details.

\begin{table}[h]
{\centering 
{\tt
\begin{tabular}{|l|l||l|l||l|l|} \hline
  {\sl Attribute} & {\sl Sec}  & {\sl Attribute} & {\sl Sec} & {\sl Attribute} & {\sl Sec} \\ \hline
  a$n$, b$n$ = Real  &  \ref{s:ab}     &  type = String                & \ref{s:string} & x\_limit = Real  & \ref{s:limit} \\ \hline
  tilt       = Real  &  \ref{s:offset} &  alias = String               & \ref{s:string} & y\_limit = Real  & \ref{s:limit} \\ \hline
  x\_offset  = Real  &  \ref{s:offset} &  descrip = String             & \ref{s:string} & aperture = Real  & \ref{s:limit} \\ \hline
  y\_offset  = Real  &  \ref{s:offset} &  mat6\_calc\_method = Switch  & \ref{s:track}  & is\_on = Logical & \ref{s:is_on} \\ \hline
  s\_offset  = Real  &  \ref{s:offset} &  tracking\_method = Switch    & \ref{s:track}  &                  &               \\ \hline
\end{tabular}
}}
\end{table}

\noindent
Possible \vn{mat6_calc_method} and \vn{tracking_method} values are:
\vskip 0.01in
\begin{example}
   bmad\_standard  (default) 
\end{example}

\vskip0.2in \noindent
Example:
\begin{example}
  abc: ab_multipole, a2 = 0.034e-2, b3 = 5.7e-4
\end{example}

\vskip0.1in \noindent
Dependent attributes:
\begin{example}
  beam\_energy  ! See section \ref{s:energy}
\end{example}

%-----------------------------------------------------------------
\section{Accel\_Sol}
\label{s:accel_sol}

An \vn{Accel_Sol} element is a combination LINAC RF accelerating
section with a solenoid on top of it. For historical reasons this
element is not currently available but could be revived if there is
any demand for it.

%-----------------------------------------------------------------
\section{BeamBeam}
\label{s:bbi}

A \vn{BeamBeam} element simulates an interaction with an opposing
(``strong'') beam traveling in the opposite direction (a
``weak--strong beam--beam Interaction''). The strong beam is assumed
to be Gaussian in shape.

\begin{table}[h]
{\centering {
\begin{tabular}{|l|l||l|l||l|l|} \hline
  {\sl Attribute} & {\sl Section}  & {\sl Attribute} & {\sl Section} & {\sl Attribute} & {\sl Section} \\ \hline
  sig\_x   = Real       &                 &  type = String                & \ref{s:string} & tilt = Real        &  \ref{s:offset}  \\ \hline
  sig\_y   = Real       &                 &  alias = String               & \ref{s:string} & x\_offset  = Real  &  \ref{s:offset}  \\ \hline
  sig\_z   = Real       &                 &  descrip = String             & \ref{s:string} & y\_offset  = Real  &  \ref{s:offset}  \\ \hline
  charge   = Real       &                 &  mat6\_calc\_method = Switch  & \ref{s:method} & s\_offset  = Real  &  \ref{s:offset}  \\ \hline
  n\_slice = Integer    &                 &  tracking\_method = Switch    & \ref{s:method} & x\_pitch = Real    &  \ref{s:offset}  \\ \hline
  symplectify = Logical & \ref{s:symp}    &  x\_limit = Real              & \ref{s:limit}  & y\_pitch = Real    &  \ref{s:offset}  \\ \hline
  is\_on = Logical      & \ref{s:is_on}   &  y\_limit = Real              & \ref{s:limit}  &                    &                  \\ \hline
                        &                 &  aperture = Real              & \ref{s:limit}  &                    &                  \\ \hline
\end{tabular}
}}
\end{table}

The magnitude of the strong beam's charge is set by the \vn{beam}
command (see \ref{s:beam}).  The sign of the strong beam's charge is
set by the \vn{charge} attribute.  Thus if \vn{charge} = -1 then the
strong beam has the opposite charge. This is the default.

\vn{sig_x}, \vn{sig_y}, \vn{sig_z} are the strong beam sigmas. 
In \bmad, \vn{x_offset} and \vn{y_offset} are used to offset the
\vn{BeamBeam} element instead of the \mad\ standard attributes
\vn{xma} and \vn{yma}.

\vn{x_pitch} and \vn{y_pitch} gives the beam--beam interaction a
crossing angle. This is the full crossing angle, not the half-angle.

The strong beam is divided up into \vn{n_slice} equal charge (not
equal thickness) slices. The default for \vn{n_slice} is 1.
Propagation through the strong beam involves a kick at the charge
center of each slice with drifts inbetween the kicks. The kicks are
calculated using the standard Bassetti--Erskine formula.  Even though
the strong beam can have a finite \vn{sig_z} the length of the element
is always considered to be zero. The longitudinal $s$--position of the
\vn{BeamBeam} element is at the spot where the reference particle's
position coinsides with the center of the strong bunch. For example,
with \vn{n_slice} = 2 the calculation would proceed as follows:
\begin{example}
  0) Start with the reference particle at the center of the strong bunch.
  1) Propagate (drift) backwards to the center of the first slice.
  2) Apply the beam--beam kick due to the first slice.
  3) Propagate (drift) forwards to the center of the second slice.
  4) Apply the beam--beam kick due to the second slice.
  5) Propagate (drift) backwards to end up with the reference particle
     at the center of the strong bunch.
\end{example}

\vskip0.2in \noindent
Possible \vn{mat6_calc_method} and \vn{tracking_method} values are:
\vskip 0.01in
\begin{example}
   bmad\_standard  (default) 
\end{example}

\vskip0.2in \noindent
Example:
\begin{example}
  bbi: beambeam, sig\_x = 3e-3, sig\_y = 3e-4, x\_offset = 0.05
\end{example}

\vskip0.2in \noindent
Dependent attributes:
\begin{example}
  beam\_energy  ! See section \ref{s:energy}
  bbi\_constant 
\end{example}
\vn{bbi_constant} = $N \, m_e \, r_e / (2 \, \pi \, \gamma \, (\sigma_x + \sigma_y))$ 
is a measure of the beam--beam interaction strungth. For example,
in the linear region near $x = y = 0$ the horizontal component of the
beam--beam kick is approximately 
$k_x = -4\, \pi \, x \, \mbox{bbi\_constant} / \sigma_x$ and the
horizontal beam--beam tune shift is 
$dQ_x = \mbox{bbi\_constant} \, \beta_x / \sigma_x$.

%-----------------------------------------------------------------
\section{Custom}
\label{s:custom}

A \vn{Custom} element is an element whose properites are defined
outside of the standard \bmad\ subroutine library. That is, to use a
custom element some programmer must write the appropriate custom
routines which are then linked with the \bmad\ subroutines into a
program. \bmad\ will call the custom routines at the appropriate time
to do tracking and transfer matrix calculations. See the programmer
who wrote the custom routines for more details!

\begin{table}[h]
{\centering {
\begin{tabular}{|l|l||l|l||l|l|} \hline
  {\sl Attribute} & {\sl Section}  & {\sl Attribute} & {\sl Section} & {\sl Attribute} & {\sl Section} \\ \hline
  l        = Real       & \ref{s:l}       &  type = String                & \ref{s:string} & tilt = Real        &  \ref{s:offset}  \\ \hline
  val$n$, $n$ = 1 - 12 = Real &           &  alias = String               & \ref{s:string} & y\_offset  = Real  &  \ref{s:offset}  \\ \hline
  rel\_tol = Real       & \ref{s:tol}     &  descrip = String             & \ref{s:string} & s\_offset  = Real  &  \ref{s:offset}  \\ \hline
  abs\_tol = Real       & \ref{s:tol}     &  mat6\_calc\_method = Switch  & \ref{s:method} & x\_offset  = Real  &  \ref{s:offset}  \\ \hline
  num\_steps = Integer  & \ref{s:tol}     &  tracking\_method = Switch    & \ref{s:method} & x\_pitch = Real    &  \ref{s:offset}  \\ \hline
  symplectify = Logical & \ref{s:symp}    &  x\_limit = Real              & \ref{s:limit}  & y\_pitch = Real    &  \ref{s:offset}  \\ \hline
  is\_on = Logical      & \ref{s:is_on}   &  y\_limit = Real              & \ref{s:limit}  & integration\_ord   &  \ref{s:ord}     \\ \hline
                        &                 &  aperture = Real              & \ref{s:limit}  &                    &                  \\ \hline
\end{tabular}
}}
\end{table}

\vskip0.2in \noindent
Possible \vn{mat6_calc_method} and \vn{tracking_method} values are:
\vskip 0.01in
\begin{example}
  custom  (default)
  runge\_kutta
  boris
\end{example}

\vskip0.2in \noindent
Example:
\begin{example}
  c1: custom, l = 3, v4 = 5.6, v12 = 0.9, num_steps = 12, tracking_method = boris
\end{example}

\vskip0.2in \noindent
Dependent attributes:
\begin{example}
  beam\_energy  ! See section \ref{s:energy}
\end{example}


%-----------------------------------------------------------------
\section{Drift}
\label{s:drift}

A \vn{Drift} element is just a space free and clear.

\begin{table}[h]
{\centering {
\begin{tabular}{|l|l||l|l||l|l|} \hline
  {\sl Attribute} & {\sl Section}  & {\sl Attribute} & {\sl Section} & {\sl Attribute} & {\sl Section} \\ \hline
  l        = Real       & \ref{s:l}       &  type = String                & \ref{s:string} & x\_limit = Real              & \ref{s:limit}  & 
                        &                 &  alias = String               & \ref{s:string} & y\_limit = Real              & \ref{s:limit}  & 
  rel\_tol = Real       & \ref{s:tol}     &  descrip = String             & \ref{s:string} & aperture = Real              & \ref{s:limit}  & 
  abs\_tol = Real       & \ref{s:tol}     &  mat6\_calc\_method = Switch  & \ref{s:method} & symplectify = Logical & \ref{s:symp}    &  
  num\_steps = Integer  & \ref{s:tol}     &  tracking\_method = Switch    & \ref{s:method} & integration\_ord = Integer & \ref{s:int}&  
                        &                 &  
\end{tabular}
}}
\end{table}

\vskip0.2in \noindent
Possible \vn{mat6_calc_method} and \vn{tracking_method} values are:
\vskip 0.01in
\begin{example}
  bmad\_standard
  symp\_lie\_ptc
  taylor
\end{example}

\vskip0.2in \noindent
Example:
\begin{example}
  d21: drift, l = 4.5
\end{example}

\vskip0.2in \noindent
Dependent attributes:
\begin{example}
  beam\_energy  ! See section \ref{s:energy}
\end{example}


%-----------------------------------------------------------------
\section{Ecollimator and Rcollimator}
\label{s:col}

An \vn{Ecollimator} is a drift with elliptic collimation.
An \vn{Rcollimator} is a drift with rectangular collimation.

\begin{table}[h]
{\centering {
\begin{tabular}{|l|l||l|l||l|l|} \hline
  {\sl Attribute} & {\sl Section}  & {\sl Attribute} & {\sl Section} & {\sl Attribute} & {\sl Section} \\ \hline
  l        = Real       & \ref{s:l}       &  type = String                & \ref{s:string} & tilt = Real        &  \ref{s:offset}  \\ \hline
                        &                 &  alias = String               & \ref{s:string} & y\_offset  = Real  &  \ref{s:offset}  \\ \hline
  rel\_tol = Real       & \ref{s:tol}     &  descrip = String             & \ref{s:string} & s\_offset  = Real  &  \ref{s:offset}  \\ \hline
  abs\_tol = Real       & \ref{s:tol}     &  mat6\_calc\_method = Switch  & \ref{s:method} & x\_offset  = Real  &  \ref{s:offset}  \\ \hline
  num\_steps = Integer  & \ref{s:tol}     &  tracking\_method = Switch    & \ref{s:method} & x\_pitch = Real    &  \ref{s:offset}  \\ \hline
  symplectify = Logical & \ref{s:symp}    &  x\_limit = Real              & \ref{s:limit}  & y\_pitch = Real    &  \ref{s:offset}  \\ \hline
  integration\_ord = Integer & \ref{s:int}&  y\_limit = Real              & \ref{s:limit}  &                    &                  \\ \hline
                        &                 &  aperture = Real              & \ref{s:limit}  &                    &                  \\ \hline
\end{tabular}
}}
\end{table}

\vskip0.2in \noindent
Possible \vn{mat6_calc_method} and \vn{tracking_method} values are:
\vskip 0.01in
\begin{example}
  bmad\_standard
  symp\_lie\_ptc
  taylor
\end{example}

\vskip0.2in \noindent
Example:
\begin{example}
  d21: ecollimator, l = 4.5, x_limit = 0.09/2, y_limit = 0.05/2
\end{example}

\vskip0.2in \noindent
Dependent attributes:
\begin{example}
  beam\_energy  ! See section \ref{s:energy}
\end{example}


%-----------------------------------------------------------------
\section{Elseperator}
\label{s:elsep}

A \vn{ElSeperator} is an electrostatic separator.

\begin{table}[h]
{\centering {
\begin{tabular}{|l|l||l|l||l|l|} \hline
  {\sl Attribute} & {\sl Section}  & {\sl Attribute} & {\sl Section} & {\sl Attribute} & {\sl Section} \\ \hline
  l        = Real       & \ref{s:l}       & type = String                & \ref{s:string} & tilt = Real        &  \ref{s:offset}  \\ \hline
  hkick    = Real       & \ref{s:kick}    & alias = String               & \ref{s:string} & y\_offset  = Real  &  \ref{s:offset}  \\ \hline
  vkick    = Real       & \ref{s:kick}    & descrip = String             & \ref{s:string} & s\_offset  = Real  &  \ref{s:offset}  \\ \hline
  gap      = Real       &                 & mat6\_calc\_method = Switch  & \ref{s:method} & x\_offset  = Real  &  \ref{s:offset}  \\ \hline
  tilt     = Real       & \ref{s:tilt}    & tracking\_method = Switch    & \ref{s:method} & x\_pitch = Real    &  \ref{s:offset}  \\ \hline
  rel\_tol = Real       & \ref{s:tol}     & x\_limit = Real              & \ref{s:limit}  & y\_pitch = Real    &  \ref{s:offset}  \\ \hline
  abs\_tol = Real       & \ref{s:tol}     & y\_limit = Real              & \ref{s:limit}  & a$n$, b$n$         &  \ref{s:ab}      \\ \hline
  num\_steps = Integer  & \ref{s:tol}     & aperture = Real              & \ref{s:limit}  & radius             &  \ref{s:ab}      \\ \hline
  integration\_ord = Integer & \ref{s:int}& symplectify = Logical        & \ref{s:symp}   &                    &                  \\ \hline
\end{tabular}
}}
\end{table}

For an \vn{Elseparator}, the kick is determined by \vn{hkick} and
\vn{vkick}. The \vn{gap} for an \vn{Elseparator} is used to compute
the electric field for a given kick. The voltage is a dependent
attribute determined by:
\begin{example}
  Voltage (V) = kick * E [ev] * gap [m] / L [m] 
\end{example}


\vskip0.2in \noindent
Possible \vn{mat6_calc_method} and \vn{tracking_method} values are:
\vskip 0.01in
\begin{example}
  bmad\_standard
  symp\_lie\_ptc
  taylor
\end{example}

\vskip0.2in \noindent
Example:
\begin{example}
  h_sep: elsep, l = 4.5, hkick = 0.003, gap = 0.11
\end{example}

\vskip0.2in \noindent
Dependent attributes:
\begin{example}
  beam\_energy  ! See section \ref{s:energy}
  ??? voltage ???
\end{example}

%-----------------------------------------------------------------
\section{Hkicker, Vkicker, and Kicker}
\label{s:kicker}

A \vn{Hkicker} is a horizontal bend. 

%-----------------------------------------------------------------
\section{Instrument}
\label{s:inst}

%-----------------------------------------------------------------
\section{Lcavity}
\label{s:lcav}

%-----------------------------------------------------------------
\section{Marker}
\label{s:mark}

%-----------------------------------------------------------------
\section{Monitor}
\label{s:mon}

%-----------------------------------------------------------------
\section{Multipole}
\label{s:m}

%-----------------------------------------------------------------
\section{Octupole}
\label{s:oct}

%-----------------------------------------------------------------
\section{Overlay}
\label{s:over}

%-----------------------------------------------------------------
\section{Patch}
\label{s:patch}

%-----------------------------------------------------------------
\section{Quadrupole}
\label{s:quad}

%-----------------------------------------------------------------
\section{Rbend}
\label{s:rbend}

%-----------------------------------------------------------------
\section{Rfcavity}
\label{s:rfcav}

%-----------------------------------------------------------------
\section{Sbend}
\label{s:sbend}

%-----------------------------------------------------------------
\section{Sextupole}
\label{s:sex}

%-----------------------------------------------------------------
\section{Solenoid}
\label{s:sol}

%-----------------------------------------------------------------
\section{Sol\_Quad}
\label{s:sq}

%-----------------------------------------------------------------
\section{Taylor}
\label{s:tay}

%-----------------------------------------------------------------
\section{Wiggler} 
\label{s:wig}

A \vn{Group} does not represent a physical element. Rather a
\vn{Group} element's purpose is to control the attributes of other elements.
This is akin to a knob in the control room.
\chapter{Element Sequencing}
\chapter{Superposition of Elements}
\chapter{Elements Controlling Other Elements}
\chapter{Tracking Methods}
\chapter{Transfer Matrix Calculation Methods}


%----------------------------------------------------------------
\part{Programmer's Guide}
%----------------------------------------------------------------
\chapter{The \bmad\ Distribution}

\section{Libraries in the \bmad\ Distribution}

When installing \bmad\ on a computer what one gets is not only the \bmad\
subroutine library but additionally subsidiary libraries upon which
subroutines in the \bmad\ library depend. There are 5 other libraries
that are used: cesr\_utils, forest, numerical\_recipes, pgplot, and dcslib.
\begin{description}
\item[cesr\_utils] This is a small low level library that primarily defines 
the precision that \bmad\ works at (see below) and defines the physical
and mathematical constants (pi, c\_light, etc.) that \bmad\ knows
about.
\item[dcslib] This library defines a set of miscellaneous helper routines. 
Routines include spline fitting, Gaussian random number generation,
etc. The library name comes from its creator.
\item[forest] This is the FPP/PTC 
(Fully Polymorphic Package/ Polymorphic Tracking Code) library of
Etienne Forest that handles Taylor maps to any arbitrary order (this
is also known as Truncated Power Series Algebra (TPSA)). The FPP part
handles the TPSA and the PTC part does the physics of tracking through
elements using Lie Algebra with a Hamiltonian.  \bmad\ uses this
software to crate Taylor Maps, track particles, etc.  FPP/PTC is a
very general package and \bmad\ only makes use of a small part of its
features. For more information see the FPP/PTC web site at
\begin{verbatim} 
    <http://bc1.lbl.gov/CBP_pages/educational/TPSA_DA/Introduction.html>
\end{verbatim}
\item[recipes] Numerical Recipes is a set of subroutines for doing 
scientific computing including Runge--Kutta integration, FFT's,
interpolation and extrapolation, etc., etc. The writeup for this
library is the book ``Numerical Recipes, The Art of Scientific
Computing''\cite{?}. For \bmad\ this library has been modified to handle
both single and double precision reals.
\item[pgplot] The PGPLOT Graphics Subroutine Library is a Fortran or 
C-callable, device-independent graphics package for making simple
scientific graphs.  One
disadvantage of PGPLOT is that it is not the most friendly software
for the programmer. To remidy this, there is a set of Fortran90
wrapper subroutines called quick\_plot. The quick\_plot suite is part
of the dcslib library. More information may be obtained from the PGPLOT
web site at 
\begin{verbatim}
    <http://www.astro.caltech.edu/~tjp/pgplot>.
\end{verbatim}

\end{description}

\section{Precision}

\bmad\ comes in two flavors: One where the real numbers are single
precision and a version with double precision reals. Which version you
are working with is controlled by the parameter \rp\ (Real Precision)
which is defined in cesr\_utils. [Note: For compatibility with older
programs the parameter \rdef\ is defined to be equal to \rp.]  On most
machines single precision has \rp\ = 4 and double precision has \rp\ =
8. Normally the double precision version is used since round-off
errors can be significant in some calculations. Long--term tracking is
an example where the single precision version is not adequate. 

To define your variables with the correct precision use the syntax
{\it real(rp)}. For example:
\begin{verbatim}
    real(rp) var1, var2, var3
\end{verbatim}
When you want to define a literal constant, for example to pass an
argument to a subroutine, add the suffix {\it \_rp} to the end of the
constant. For example: {\it 2.0\_rp} is equivalent to {\it 2.0D0} if
\rp is defined to be double precision. Notice that this is not
equivalent to {\it 2\_rp} which defines an integer (not a real) constant.



\chapter{The Ele\_struct}
\section{overview}

This chapter describes the \tn{ele_struct} which is the structure that
holds all the information about an individual element: quadrupoles,
separators, wigglers, etc. Discussion on how one \tn{ele_struct} controls
another is deferred to the chapter on the \tn{ring_struct}. Also deferred
is a discussion of how to compute global parameters such as the 
Twiss parameters, etc.

Part of the substructure of the \tn{ele_struct} is shown
in figure~\ref{f:ele_struct} (use \vn{getf} to see the entire structure 
definition). Some of the components of the \vn{ele_struct} like \vn{%name}, 
\vn{%tracking_method}, etc.\ have an obvious correspondance with 
attributes set in the lattice file and will not be discussed further.

\begin{figure}[tb]
\centering
\small
\begin{verbatim}
  type ele_struct
    character(16) name                ! name of element
    character(16) type                ! type name
    character(16) alias               ! Another name
    type (twiss_struct)  x,y,z        ! Twiss parameters at end of element
    type (floor_position_struct) position
    real(rp) value(n_attrib_maxx)     ! attribute values
    real(rp) vec0(6)                  ! 0th order transport vector
    real(rp) mat6(6,6)                ! 1st order transport matrix
    real(rp) c_mat(2,2)               ! 2x2 C coupling matrix
    real(rp) gamma_c                  ! gamma associated with C matrix
    real(rp) s                        ! longitudinal position at the end
    type (taylor_struct) :: taylor(6) ! Taylor terms
    type (wake_struct) wake           ! Wakefields
    integer key                       ! key value
    integer sub_key                   ! For wigglers: map_type$, periodic_type$
    integer control_type              ! SUPER_SLAVE$, OVERLAY_LORD$, etc.
    integer mat6_calc_method          ! bmad_standard$, taylor$, etc.
    integer tracking_method           ! bmad_standard$, taylor$, etc.
    integer field_calc                ! Used with integrators (Runge-Kutta et. al)
    integer num_steps                 ! number of slices for DA_maps
    integer integration_ord           ! For Etiennes' PTC: 2, 4, or 6.
    logical symplectify               ! Symplectify mat6 matrices.
    logical exact_rad_int_calc        ! Exact radiation integral calculation?
    logical field_master              ! Calculate strength from the field value?
    logical is_on                     ! For turning element on/off.
  end type
\end{verbatim}
\caption{The \tn{ele\_struct}. Only part of the substructure is shown.}
\label{f:ele_struct}
\end{figure}

%--------------------------------------------------------------------------
\section{Twiss Parameters, etc.}

There are a class of components of the \vn{ele_struct} whose values 
vary along the length of the element. In such a case the value of the 
component will be the value at the exit edge of the element. The 
components are:
\begin{example}
  %x, %y, %z      ! Twiss parameters
  %position       ! Floor position
  %c_mat(2,2)     ! Coupling c matrix
  %gamma_c        ! Coupling parameter
\end{example}
To get the Twiss parameters, etc.\ for the beginning of the element you
need to look at the preceding element in the \vn{ring%ele_(:)} array. To
get the parameters at a position within an element you can use the routines
\vn{

\vn{%x}, \vn{%y}, and \vn{%z} hold the Twiss parameters for the 
$a$, $b$ and $z$ modes respectively. [Yes it is known that the
labeling is misleading. Unfortunately it is a bit entrenched now.]
The $a$ mode is the ``nearly horizontal'' mode and the $b$ mode is the 
``nearly vertical'' mode. Remember: The Twiss parameters are associated
with the normal modes. With coupling there is no Twiss parameter associated
soley with the horizontal axis.

%--------------------------------------------------------------------------
\section{Transfer Maps}

The first order transfer map through a element is stored in \vn{vec0}
and \vn{mat6}. Thus with \vn{Linear} tracking the appropriate formula is
\begin{example}
  orbit_out = %vec0 + %mat6 * orbit_in
\end{example}
The \bmad\ routines that compute \vn{%mat6} (for example \vn{ring_make_mat6})
take a reference orbit as an argument and the resulting \vn{%mat6} matrix
is the Jacobian about the reference orbit.

%--------------------------------------------------------------------------
\section{Taylor Maps}

\vn{taylor_order} is the order of the Taylor Map. The map itself is stored 
in \vn{%taylor(1:6)}. Each \vn{%taylor(i)} is a Taylor series. 

%--------------------------------------------------------------------------
\section {multipoles}

%--------------------------------------------------------------------------
\section{General Use Components}

%--------------------------------------------------------------------------
\section{Floor Position}

%--------------------------------------------------------------------------
\section{Initializing}

Generally most \tn{ele_struct} components are stored within a
\tn{ring_struct} so you don't have to worry about
allocation/deallocation issues directly. In case you do have an local
\tn{ele_struct} variable within a subroutine then you either have do
deallocate the pointers within it with a call to
\rn{deallocate_ele_pointers} or you use the save attribute.
\begin{example}
  type (ele_struct), save :: ele     ! Either this or
  call deallocate_ele_pointers (ele) ! Do this at the end.
\end{example}


%--------------------------------------------------------------------------
\section{Dependent and Independent Variables}

Some attributes of an element are designated as "dependent variables"
which are dependent upon other independent variables. The dependent
and independent variables are: \hfil\break
\begin{table}[h]
\centering {
\begin{tabular}{|l|l|l|} \hline
           & {\em Dependent Variables}  & {\em Independent Variables}\\ \hline
  Rbend    & Rho, Angle, L\_Cord    & G, L                         \\ \hline
  Sbend    & Rho, Angle, L\_Cord    & G, L                         \\ \hline
  RFCavity & RF\_Wavelength         & Harmon                       \\ \hline
  BeamBeam & BBI\_Const             & Charge, Sig\_x, Sig\_y       \\ \hline
  Wiggler  & K1, Rho                & B\_max                       \\ \hline
\end{tabular}
}
\end{table}

When \rn{attribute_bookkeeper} routine is called (this is called by,
for example, \rn{make_mat6}) the values of the dependent variables
will be set based upon the values of the independent variables. Thus
trying to vary the strength of a bend by varying, say, the Rho
attribute is an exercise in futility. Also remember that routines are
allowed to assume that the dependent variables are consistent with the
independent variables. Thus you need to call either

\vn{b_field_master} ...

------------------------------------------------------------------------


  How to locate attributes

  Allocation/deallocation

%--------------------------------------------------------------------------
\section{Element Control}

    * Overlays
    * Superimpose
    * Groups


\endinput

\item[\%attribute\_name] This is used by overlays. See below.
\item[\%value(:)] this array holds the attribute values for the element. 
For example, the value of the k1 attribute for a quadrupole element is
stored in \vn{%value(k1)} where \vn{k1$} is an integer parameter. In
general to get the correct index for the \vn{%value(:)} for a given
attribute just use a "\$" as a suffix.
\item[\%gen0] Constant part of a \genfield.



\chapter{The ring\_struct}
\label{c:ring_struct}
\index{Ring_struct|textbf}

The \vn{ring_struct} is the structure that holds of all the information 
about a lattice.   Despite its name, \bmad\
makes no assumption about whether an \vn{ring_struct} is circular as
with a storage ring or open as with a LINAC.

%----------------------------------------------------------------------------
\section{Elements Within the ring\_struct}
\label {s:ring_ele}

\begin{figure}[htb]
\centering
\begin{verbatim}
type ring_struct
  type (mode_info_struct)  x, y, z  ! tunes, etc.
  character*16 name            ! Name in USE statement
  character*40 lattice         ! Lattice name
  character*80 input_file_name ! Lattice input file name
  character*80 title           ! From TITLE statement
  type (param_struct) param    ! parameters
  integer version              ! Version number for digested files
  integer n_ele_use            ! number of physical lattice elements
  integer n_ele_max            ! Index of last element used
  integer n_control_array      ! last index used in CONTROL_ array
  integer n_ic_array           ! last index used in IC_ array
  integer input_taylor_order   ! As set in the input file
  integer ic_(n_control_maxx)  ! index to %control_(:)
  type (ele_struct), pointer :: ele_(:)    ! Array of lattice elements
  type (ele_struct)  ele_init              ! For use by any program
  type (control_struct)  control_(n_control_maxx)  ! control list
  real(rp), pointer :: beam_energy ! = %ele_(0)%value(beam_energy$)
end type
\end{verbatim}
\caption{Definition of the \vn{ring\_struct}.}
\label{f:ring_struct}
\end{figure}

\index{Ring_struct!%ele_(:)}
\index{Ring_struct!%n_ele_use}
\index{Ring_struct!%n_ele_max}
\index{Ele_struct!in ring_struct} 
\index{I_Beam}
\index{Overlay}
\index{Group}
The definition of the \vn{ring_struct} is shown in
Figure~\ref{f:ring_struct}. The array \vn{%ele_(:)} holds the elements
of the lattice. This array is always allocated with zero as the lower
bound.  \vn{%ele_(0)} is a marker element with the name
\vn{BEGINNING}.  \vn{%ele_(0)%mat6} is always the unit
matrix. \vn{%ele_(0:)} is divided up into two parts: A ``physical''
part (also called the ``regular'' part) and a ``control'' part (also
called the ``lord'' part). The physical part of this array holds the
elements that are tracked through. The control part holds the
\vn{Overlay}, \vn{I_Beam} and \vn{Group} elements, and those elements that are
``mangled'' when elements are superimposed upon other elements.  The
bounds of these two parts is given in Table~\ref{tab:part_extent}.
\begin{table}[htb]
\begin{center}
\begin{tabular}{|l|l|l|}
\hline
              & \multicolumn{2}{c|} {\em index n}         \\ \hline
{\em section} & {\em min}          & {\em max}            \\ \hline
physical      & 0                  & \vn{%n_ele_use}     \\ \hline
control       & \vn{%n_ele_use}+1 & \vn{%n_ele_max}      \\ \hline
\end{tabular} 
\caption{Bounds of the regular and control parts 
of the array \vn{\%ele(:)}.}
\end{center}
\label{tab:part_extent}
\end{table}

\index{Ring_struct!%ele_init}
The \vn{%ele_init} component within the \vn{ring_struct} is not used
by \bmad\ and is available for general program use.

%----------------------------------------------------------------------------
\section{Param\_struct Structure}
\index{Param_struct}

The \vn{%param} variable within the \vn{ring_struct} is a
\vn{param_struct} structure whose definition is shown in
Figure~\ref{f:param_struct}
\begin{figure}[htb]
\centering
\begin{verbatim}
  type param_struct
    real(rp) n_part             ! Particles/bunch (for BeamBeam elements).
    real(rp) total_length       ! total_length of lattice
    real(rp) growth_rate        ! growth rate/turn if not stable
    real(rp) t1_with_RF(6,6)    ! Full 1-turn 6x6 matrix
    real(rp) t1_no_RF(6,6)      ! Transverse 1-turn 4x4 matrix (RF off).
    integer particle            ! +1 = positrons, -1 = electrons
    integer ix_lost             ! If lost at what element?
    integer end_lost_at         ! entrance_end$ or exit_end$
    integer lattice_type        ! linac_lattice$, circular_lattice$, etc...
    integer ixx                 ! Integer for general use
    logical stable              ! is closed lattice stable?
    logical aperture_limit_on   ! use apertures in tracking?
    logical lost                ! for use in tracking
  end type
\end{verbatim}
\caption{Definition of the \vn{param\_struct}.}
\label{f:param_struct}
\end{figure}

\index{Param_struct!%total_length}
\vn{%param%total_length} is the length of the lattice that a beam
tracks through
\begin{example}
  %param%total_length = %ele_(n_use)%s - %ele_(0)%s
\end{example}
Normally \vn{%ele_(0)%s} = 0 so \vn{%param%total_length} =
\vn{%ele_(n_use)%s} but this is not always the case.

\index{Param_struct!%n_part}
\index{BeamBeam}
\index{Lcavity}
\vn{%param%n_part} is the number of particles in a bunch and is used
by \vn{BeamBeam} element to determine the strength of the beambeam
interaction. \vn{%param%n_part} is also used by \vn{Lcavity} elements
for wakefield calculations.

\index{Param_struct!%lost}
\index{Param_struct!%end_lost_at}
\index{Param_struct!%aperture_limit_on}
When tracking particles through a lattice the variable
\vn{%param%aperture_limit_on} determines if apertures are checked.
\vn{%param%lost} is used to signal if a particle is lost and
\vn{%param%ix_lost} gives the index of the element at which a particle
is lost. Additionally, \vn{%param%end_lost_at} is used to indicate at
which end the particle was lost at. See Chapter~\ref{c:tracking} for
more details.

\index{Param_struct!%t1_with_RF}
\index{Param_struct!%t1_no_RF}
\index{Param_struct!%stable}
\vn{%param%t1_with_RF} and \vn{%param%t1_no_RF} are the 1--turn transfer
matrices from the start of the lattice to the end. \vn{%param%t1_with_RF}
is the full transfer matrix with RF on. \vn{%param%t1_no_RF} is the
transverse transfer matrix with RF off. \vn{%param%t1_no_RF} is used to
compute the Twiss parameters. When computing the Twiss parameters
\vn{%param%stable} is set according to whether the matrix is stable or
not. If the matrix is not stable the Twiss parameters cannot be
computed. If unstable, \vn{%param%growth_rate} will be set to the
growth rate per turn of the unstable mode.  \vn{%param%t1_with_RF} and
\vn{%param%t1_no_RF} are set by various routines. Other routines use
these matrices as input for calculations.


%----------------------------------------------------------------------------
\section{Finding Elements and Changing Their Value}
\label{s:ring_ele_change}

\bmad has two routines for searching for an element in a lattice:
\begin{example}
  \vnr{element_locator}(ele_name, ring, ix_ele) ! locate by name
  \vnr{elements_locator}(key, ring, indx)        ! locate by key
\end{example}
\vn{element_locator} searches the \vn{ring%ele_(:)} array for an
element of the correct name. Alternatively, \vn{elements_locator}
searches for all elements of a given key and returns an array of
indexes. To see how simple searching a lattice is consider the heart
of \vn{element_locator}:
\begin{example}
  do ix_ele = 0, ring%n_ele_max
    if (ring%ele_(ix_ele)%name == ele_name) return
  enddo
\end{example}

Once an element or elements are identified in the lattice,
changing an element attribute generally involves changing values in the 
\vn{ring%ele_(i)%value(:)} array. This is done using the 
\vnr{set_ele_attribute} routine. For example:
\begin{example}
  type (ring_struct) lat
  logical err_flag, make_xfer_mat
  ...
  call element_locator ('Q01W', lat, ix_ele)
  call set_ele_attribute (lat, ix_ele, 'K1', 0.1_rp, err_flag, make_xfer_mat)
\end{example}
\index{Overlay}
This example sets the \vn{K1} attribute of an element named \vn{Q01W}.
\vn{set_ele_attribute} checks whether this element is actually free to
be varied and sets the \vn{err_flag} logical accordingly. An element's
attribute may not be freely varied if, for example, the attribute is
controlled via an \vn{Overlay}.

\index{Bmad_com_struct!auto_bookkeeper}
There is a global variable \vn{bmad_com%auto_bookkeeper} that 
controls whether bookkeeping routines are called by routines that do tracking
and routines that calculate transfer matrices. 
This is done to make sure that the control 
and attribute information in a lattice is
up--to--date. Since this bookkeeping needs only be done when element attributes
are changed, this can lead to a significant waste of time. Setting 
\vn{bmad_com%auto_bookkeeper} to \vn{.false.} turns off this
automatic bookkeeping.
As a result, if an attribute value is changed directly the appropriate bookkeeping
routines need to be called:
\begin{example}
  type (ring_struct) lat
  bmad_com%auto_bookkeeper = .false.
  lat%ele_(ix_ele)%value(k1$) = 0.1
  call control_bookkeeper (lat, ix_ele)
\end{example}
Alternatively, if multiple elements are modified, then \vn{lattice_bookkeeper} can be
used to do bookkeeping for the entire lattice:
\begin{example}
  type (ring_struct) lat
  ... set some attributes ...
  call lattice_bookkeeper (lat)
\end{example}

%----------------------------------------------------------------------------
\section{Elements Controlling Other Elements}
\label{s:ring_control}

Generally anyone using \bmad routines will not need to bother with
the details of how elements control other elements (\sref{s:ring_ele_change}).
It is always safer to use the bookkeeping routines provided by \bmad rather 
than directly manipulating the control information. The following is thus
meant for those unusual situations where knowledge of how \bmad implements
control is needed.

The element control information is stored in the \vn{%control_(:)} array, 
Each element of this array is a \vn{control_struct} structure 
\begin{example}
  type control_struct
    real(rp) coef                ! control coefficient
    integer ix_lord                ! index to lord element
    integer ix_slave               ! index to slave element
    integer ix_attrib              ! index of attribute controlled
  end type
\end{example}
\index{I_Beam}
\index{Overlay}
\index{Group}
\vn{%ix_lord} and \vn{%ix_slave} give the indices in the \vn{%ele_(:)}
array of a lord element and an element it controls. A lord element
\vn{%ele_(i_lord)} has assigned to it a block \vn{%control_(:)}
elements.  The following example prints the names and controlled
attributes of the slaves of a particular lord element. If the lord is
an \vn{Overlay} or \vn{Group} then \vn{%control_(:)%ix_attrib} and 
\vn{%control_(:)%coef} give the attribute index of the controlled
attribute and the appropriate  coefficient 
\begin{example}
  type (ring_struct) lattice
  ...
  ix1 = lattice%ele_(i_lord)%ix1_slave  ! start of block
  ix2 = lattice%ele_(i_lord)%ix2_slave  ! end of block
  print *, 'Slaves for lord: ', lattice%ele_(i_lord)%name
  do i = ix1, ix2
    i_slave = lattice%control_(i)%ix_slave
    if (lattice%ele_(i_lord)%control_type == super_lord$) then
      print *, '  ', i, '  ', lattice%ele_(i_slave)%name
    else   ! must be overlay or slave
      ixa = lattice%control_(i)%ix_attrib
      attrib_name = attribute_name (lattice%ele_(i_slave), ixa)
      print *, '  ', i, '  ', lattice%ele_(i_slave)%name, attrib_name
    endif
  enddo
\end{example}

Going backward from a slave to its lords goes through one level of
indirection using the \vn{%ic_} array as shown in the next example
\begin{example}
  ix1 = lattice%ele_(i_slave)%ic1_lord
  ix2 = lattice%ele_(i_slave)%ic2_lord
  print *, 'Lords for slave: ', lattice%ele_(i_slave)%name
  do i = ix1, ix2
    ix = lattice%ic_(i)
    i_lord = lattice%control_(ix)%ix_lord
    print *, '  ', i, '  ', lattice%ele_(i_lord)%name
  enddo
\end{example}
\index{Group}
For historical reasons, since a \vn{Group} element only makes changes
in the values of the attributes it controls, \vn{Group} elements are
not included in the list of lords generated by the above example.

%----------------------------------------------------------------------------
\section{Pointers}
\label{s:ring:point}
\index{Ring_struct!pointers}

Since the \vn{ring_struct} has pointers within it there is an extra burden on
the programmer to make sure that allocation and deallocation is done
properly. To this end the equal sign has been overloaded by the
routine \vn{ring_equal_ring} so that when one writes
\begin{example}
    type (ring_struct) lattice1, lattice2
    ! ... some calculations ...
    lattice1 = lattice2
\end{example}
the pointers in the \vn{ring_structs} will be handled properly. The
result will be that lattice1 will hold the same information as
\vn{lattice2} but the pointers in \vn{lattice1} will point to
different locations in physical memory so that changes to one lattice
will not affect the other.

Initial allocation of the pointers in a \vn{ring_struct} variable is
generally handled by the \vnr{bmad_parser} and \vn{ring_equal_ring}
routines.  Once allocated, local \vn{ring_struct} variables must have
the save attribute or the pointers within must be appropriately
deallocated before leaving the routine.
\begin{example}
  type (ring_struct), save :: lattice     ! Either do this at the start or ...
  ...
  call deallocate_ring_pointers (lattice) ! ... Do this at the end.
\end{example}
Using the save attribute will generally be faster but will use more
memory. Typically using the save attribute will be the best choice.

\chapter{The Coord\_struct}

  Meaning of vec elements

  Meaning of vec elements in PTC.


\chapter{Reading and Writing Lattices}

%----------------------------------------------------------------------------
\section{Reading in Lattices}
\index{Lattice files!reading}

The subroutine to read in an XSIF lattice file is \vn{xsif_parser}.
There are two subroutines in \bmad\ to read in a \bmad standard
lattice file: \vn{bmad_parser} and \vn{bmad_parser2}.

\vn{bmad_parser} is used to initialize a \vn{ring_struct}
structure from scratch using the information from a lattice
file. Unless told otherwise, after reading in the lattice
\vn{bmad_parser} will compute the 6x6 transfer matrices for each
element. Normally you want to do this but there are exceptions where
you don't need it and for particular lattices the computation can take
a long time (especially if there are Taylor maps to be
computed). Notice that \vn{bmad_parser} does {\em not} compute any
Twiss parameters.

\vn{bmad_parser2} is typically used after \vn{bmad_parser} if there is
additional information that needs to be added to the lattice. For
example, aperture limits for the elements are often stored in a
separate file. In this case there are two possibilities: The first is
to use \vn{bmad_parser2}
\begin{verbatim}
  call bmad_parser ('lattice_file', ring)       ! read in a lattice.
  call bmad_parser2 ('aperture_file', ring)     ! read in the aperture limits.
\end{verbatim}
The second alternative is to create a third file that calls the first two
\begin{verbatim}
 ! This is a file to be called by bmad_parser
 call, file = 'lattice_file'
 call, file = 'aperture_file'
\end{verbatim}
and then just use \vn{bmad_parser} to parse this third file.

%----------------------------------------------------------------------------
\section{Digested Files}
\index{Lattice files!digested files}

Since parsing can be slow, once \vn{bmad_parser} has transfered the
information from a lattice file into the \vn{ring_struct} it will make
what is called a digested file. A digested file is an image of the
\vn{ring_struct} in binary form. When \vn{bmad_parser} is called it
(actually it is a the subroutine \vn{read_digested_bmad_file} that
does all the work) first looks in the same directory as the lattice
file for a digested file whose name is of the form
\begin{verbatim}
  'digested_' // LAT_FILE   ! for single precision BMAD 
  'digested8_' // LAT_FILE  ! for double precision BMAD 
\end{verbatim}
where \vn{LAT_FILE} is the lattice file name. If it finds the digested
file it checks that the file is not out--of--date. It can do this
since the digested file stores the names and the dates of all the
lattice files that were used when the digested file was made. The
\bmad\ version number stored in the digested file is also checked. The
\bmad\ version number is a global parameter that is increased (not too
frequently) each time the structure of the \vn{ring_struct} or
\vn{ele_struct} is modified. If the \bmad\ version number in the
digested file does not agree with the current or the digested file is
out--of--date a warning will be printed and the digested file will not
be used.

\index{Transfer map!Taylor!with digested files}
Since computing Taylor Maps can be very time intensive,
\vn{bmad_parser} tries to reuse Taylor Maps it finds in the digested
file even if the lattice file has been changed in the meantime. To
make sure that everything is OK it will check that the attribute
values of an element needing a Taylor map are the same as the
attribute values of a corresponding element in the digested file
before it reuses the map. Element names are not a factor in this
decision.

This leads to the following trick: If you want to read in a lattice
where there is no corresponding digested file, and if there is another
digested file that has elements with the correct Taylor Maps, then to
save on the map computation time simply make a copy of the digested
file with the digested file name corresponding to the first lattice.

The digested file is in binary format and is not human readable but it
provides a convenient mechanism for transporting lattices between
programs. For example, say you have read in a lattice, changed
some parameters in the \vn{ring_struct}, and now you want to do some
analysis on this modified \vn{ring_struct} using a different program. The 
answer is to have the first program create a digested file
\begin{example}
  call write_digested_bmad_file ('digested_file_of_mine', ring)
\end{example}
and then read the digested file in with the second program
\begin{example}
  call read_digested_bmad_file ('digested_file_of_mine', ring)
\end{example}
An alternative to writing a digested file is to write a lattice file
using \vn{write_bmad_lattice_file}

%----------------------------------------------------------------------------
\section{Writing MAD files}
\index{Lattice files!MAD files}
\index{MAD}

\MAD--8 compatable lattice files can be created using the routine \vnr{bmad_to_mad}:
\begin{example}
  type (ring_struct) lat             ! lattice
  ...
  call bmad_to_mad ('lat.mad', lat)  ! create MAD file
\end{example}

Since \MAD has no concept of things such as \vn{overlay}s and \vn{group}s, such 
information is lost in translation.


\chapter{Tracking and Transfer Maps}
\label{c:tracking}
\index{tracking}

%----------------------------------------------------------------
\section{The coord_struct}
\label{s:coord.struct}
\index{coord_struct}

\index{spin}
\index{phase space coordinates}
The \vn{coord_struct} holds the coordinates of a particle The definition of the \vn{coord_struct} is
\begin{example}
  type coord_struct
    real(rp) vec(6)     ! (x, px, y, py, z, pz)
    real(rp) s          ! Longitudinal position.
    real(rp) t          ! Absolute time (not relative to reference).
    real(rp) spin(3)    ! (x, y, z) Spin vector
    real(rp) field(2)   ! Photon (x, y) field intensity.
    real(rp) phase(2)   ! Photon (x, y) phase.
    real(rp) charge     ! charge in a particle (Coul).
    real(rp) dt_dref    ! path length (used by coherent photons).
    real(rp) r          ! For general use. Not used by Bmad.
    real(rp) p0c        ! For non-photons: Reference momentum. Negative -> going backwards.
                        !     For photons: Photon momentum (not reference).
    real(rp) beta       ! Velocity / c_light. 
    integer ix_ele      ! Index of the lattice element the particle is in.
                        !   May be -1 or -2 if element is not associated with a lattice.
    integer ix_branch   ! Index of the lattice branch the particle is in.
    integer ix_user     ! Not used by \bmad
    integer state       ! alive\$, lost\$, lost_neg_x\$, etc.
    integer direction   ! +1 or -1. Sign of longitudinal direction of motion (ds/dt).
                        !  This is independent of the element orientation.
    integer time_dir    ! +1 or -1. Time direction. -1 => Traveling backwards in time.
    integer species     ! Positron\$, proton\$, etc.
    integer location    ! upstream_end\$, inside\$, or downstream_end\$
end type
\end{example}
Definitions:
\begin{description}
\item{Direction of Travel} \Newline
The ``\vn{direction of travel}'', also called the ``\vn{direction of motion}'' is the direction
that the particle is moving in when traveling forward in time.
%
\item{Propagation Direction} \Newline
The ``\vn{propagation direction}'' is the direction that a particle will be propagated in during
tracking. The propagation direction will be in the same direction as the direction of travel when
propagating a particle forward in time and will be opposite the direction of travel when
propagating a particle backwards in time.
%
\item{Reverse Tracking} \Newline
``\vn{Reverse tracking} refers to tracking a particle with \vn{%direction} set to -1. That is,
tracking in the reverse direction longitudinally. The opposite to reverse tracking is called
``\vn{forward direction}'' tracking.
%
\item{Backwards Tracking} \Newline
``\vn{Backwards Tracking} refers to tracking a particle backwards in time. That is, with
\vn{%time_dir} = -1. The opposite to backwards tracking is called ``\vn{forward time}'' tracking.
\end{description}
The components of the \vn{coord_struct}:
\begin{description}
\item{\%beta} \Newline
The normalized velocity $v/c$ is stored in \vn{%beta}. \vn{%beta} is always positive.
%
\item{\%direction} \Newline
Longitudinal forward time ``direction of travel''. A setting of +1 (the default) is in the forward
+s (downstream) direction and a setting of -1 is in the reverse -s (upstream) direction
(\sref{s:ref.construct}).  Notice that the setting of \vn{direction} is independent of the
orientation of the lattice element the particle is traveling through. That is, for an element with
reversed \vn{orientation} (\vn{ele%orientation} = -1), a particle with \vn{direction} = 1 will be
traveling towards the \vn{entrance} end of the element (-z direction in body coordinates) and with
\vn{direction} = -1 the particle will be traveling towards the \vn{exit} (+z direction in body
coordinates) end (\sref{s:ref.construct}). See \vn{%time_dir}.
%
\item{\%time_dir} \Newline
Time direction that a particle is propagated through. A value of +1 (the default) is forward time
and a value of -1 is backwards time.
%
\item{\%field_x, \%field_y} \Newline
The \vn{%field_x} and \vn{%field_y} components are for photon tracking and are in units of
field/sqrt(cross-section-area). That is, the square of these units is an intensity. It is up to
individual programs to define an overall scaling factor for the intensity if desired.
%
\item{\%ix_branch} \Newline
The \vn{%ix_branch} component gives the index of the lattice branch in the \vn{lat%branch(ib)} array
that the particle is in.
%
\item{\%ix_ele} \Newline
The \vn{%ix_ele} component gives the index of the element in the \vn{lat%branch(ib)%ele(:)} array
that the particle is in. If the element is not associated with a lattice, \vn{%ix_ele} is set to
-1. When initializing a \vn{coord_struct} (see below), \vn{%ix_ele} will be initialized to
\vn{not_set\$}.
%
\item{\%ix_user} \Newline
The \vn{%ix_user} component is for use by code outside of the \bmad library.  This component will
not be modified by \bmad.
%
\item{\%location} \Newline
The \vn{%location} component indicates where a particle is longitudinally with respect to the
element being tracked. \vn{%location} will be on of:
\begin{example}
  entrance_end\$
  inside\$
  exit_end\$
\end{example}
\vn{entrance_end\$} indicates that the particle is at the element's entrance ($-s$) end and
\vn{exit_end\$} indicates that the particle is at the element's exit ($+s$) end.  \vn{inside\$}
indicates that the particle is in between. If the element has edge fields (for example, the \vn{e1}
and \vn{e2} edge fields of a bend), a particle at the \vn{entrance_end\$} or \vn{exit_end\$} is
considered to be just outside the element.
%
\item{\%p0c} \Newline
For charged-particles, the reference momentum in eV is stored in the \vn{%p0c} component. For
photons, \vn{%p0c} is the actual (not reference) momentum. For charged-particles, \vn{%p0c} may be
negative if the particle is traveling backwards longitudinally. For photons, \vn{%vec(6)}
($\beta_z$) will be negative if the photon is going backward.
%
\item{\%r} \Newline
The \vn{%r} component is for use by code outside of the \bmad library. \bmad will not modify this
component.
%
\item{\%s} \Newline
The \vn{%s} component gives the absolute $s$-position of the particle. When tracking through an
element (say with Runge-Kutta tracking), and when the particle coordinates is expressed in element
body coordinates (\sref{s:lab.body.transform}), the $s$-position at any point within the element, by
definition, is independent of any misalignments the element has as long as the element is not
reversed. If the element is reversed, the $s$-position is reversed as well.
%
\item{\%spin(3)} \Newline
The \vn{%spin(3)} component gives a particle's $(x, y, z)$ spin vector (\sref{s:spin.dyn}).
%
\item{\%state} \Newline
The \vn{%state} component will be one of:
\begin{example}
  not_set\$
  pre_born\$
  alive\$
  lost\$
  lost_neg_x\$
  lost_pos_x\$
  lost_neg_y\$
  lost_pos_y\$
  lost_z\$
  lost_pz\$
\end{example}
The \vn{not_set\$} setting indicates that the \vn{coord_struct} has not yet been used in
tracking. The \vn{alive\$} setting indicates that the particle is alive. If a particle is ``dead'',
the \vn{%state} component will be set to one of the other settings. The \vn{lost_neg_x\$}
setting indicates that the particle was lost at an aperture on the $-x$ side of the element. The
\vn{lost_z\$} setting is used to indicate that the particle tried to ``turn around''. This
can happen, for example, with strong magnetic fields or when a particle has been decelerated too
much.  The reason why the particle is marked lost in this case is due to the fact that $s$-based
tracking algorithms cannot handle particles that reverse direction. The exception is that the
\vn{time_runge_kutta} (\sref{s:tkm}) tracking method can handle particle reversal so in this case,
particles will not be declared lost if they reverse direction.

The \vn{lost\$} setting is used when neither of the other \vn{lost_*\$} settings are not
appropriate. For example, \vn{lost\$} is used in Runge-Kutta tracking when the adaptive step size
becomes too small (this may happen if the fields do not obey Maxwell's equations).

To convert the integer value of \vn{%state} to a string that can be printed, use the function
\Hyperref{r:coord.state.name}{coord_state_name}
\begin{example}
  type (coord_struct) orbit
  print *, "State of the orbit: ", coord_state_name(orbit%state)
\end{example}
%
\item{\%t} \Newline
\vn{%t} gives the absolute time.
%
\item{\%vec(:)} \Newline
The \vn{%vec(:)} array defines the phase space coordinants (\sref{s:phase.space}). Note that for
photons, the definition of the phase space coordinates (\sref{s:photon.phase.space}) is different
from that used for charged particles. The signs of \vn{%vec(2)} and \vn{%vec(4)} are such that, for
the signs of the change in \vn{%vec(1)} and \vn{%vec(3)} during propagation will be equal to the
product \vn{%direction * %time_dir * sign_of(%vec(2)} and \vn{%direction * %time_dir *
sign_of(%vec(2)} respectively.
%
\end{description}

To initialize a \vn{coord_struct} so it can be used as the start of tracking, the
\Hyperref{r:init.coord}{init_coord} routine can be used:
\begin{example}
  type (coord_struct) start_orb
  real(rp) phase_space_start(6)
  ...
  phase_space_start = [...]
  call init_coord (start_orb, phase_space_start, lat%ele(i), lat%param%particle)
\end{example}
Here \vn{init_coord} will initialize \vn{start_orb} appropriately for 
tracking through element \vn{lat%ele(i)} with the particle species set to the 
species of the reference particle given by \vn{lat%param%particle}. 

%----------------------------------------------------------------
\section{Tracking Through a Single Element}
\label{s:track1}

\Hyperref{r:track1}{track1} is the routine used for tracking through a
single element
\begin{example}
  type (coord_struct), start_orb, end_orb
  type (ele_struct) ele
  real(rp) start_phase_space(6)
  logical err
  ...
  start_phase_space = [...]
  call init_coord (start_orb, start_phase_space, ele, photon\$) 
  call track1 (start_orb, ele, end_orb, err_flag = err)
  if (.not. particle_is_moving_forward(end_orb)) then
    print *, "Particle is lost and gone forever..."
  endif
\end{example}
To check if a particle is still traveling in the forward direction,
the \Hyperref{r:particle.is.moving.forward}{particle_is_moving_forward} 
routine can be used as shown in the above example.

The ``virtual'' entrance and exit ends of a lattice element are, by definition, where the physical
ends of the element would be if there were no offsets. In particular, if an element has a finite
\vn{z_offset} (\sref{s:ele.offset}), the physical ends will be displaced from the virtual ends. The
position \vn{ds} of a particle with respect to the physical entrance end of the element is
\begin{example}
  ds = coord%s - (ele%s + ele%value(z_offset_tot\$) - ele%value(l\$))
\end{example}
When tracking through an element, the starting and ending positions always correspond to the virtual
ends. If there is a finite \vn{z_offset}, the tracking of the element will involve tracking through
drifts just before and just after the tracking of the body of the element so that the particle ends
at the proper virtual exit end.

Note: The $z$ phase space component of the orbit (\vn{%vec(5)}) is independent of the value of
\vn{ele%ref_time} even though the reference time is used to define $z$ (See \Eq{zbctt}). This is
true since the starting reference time that is used for a particle is arbitrary. For example, when
tracking multiple bunches, the reference time is typically set so that a particle at the center of
a bunch has $z = 0$. Also, in a ring, \vn{ele%ref_time} is only the reference time for the first turn
through an element. Since \bmad does not keep track of turn number, there is no way for \bmad to know
what the true reference time is other than to calculate it from the value of $z$!

%----------------------------------------------------------------
\section{Tracking Through a Lattice Branch}

When tracking through a lattice branch, one often defines an array of \vn{coord_struct}s -- one for
each element of the lattice branch. In this case, the $i$\Th \vn{coord_struct} corresponds to the
particle coordinates at the end of the $i$\Th element. Since the number of elements in the lattice
is not known in advance, the array must be declared to be allocatable. The lower bound of the array
must be set to zero to match a \vn{lat%branch(i)%ele(:)} array.  The upper bound should be the upper
bound of the \vn{%branch(i)%ele(:)} array.  The routine
\Hyperref{r:reallocate.coord}{reallocate_coord} will allocate an array of \vn{coord_struct}s:
\begin{example}
  type (coord_struct), allocatable :: orbit(:)
  type (lat_struct) lat
  ...
  call reallocate_coord (orbit, lat, ix_branch)
\end{example}
Alternatively, the \vn{save} attribute can be used so that the array
stays around until the next time the routine is called
\begin{example}
  type (coord_struct), allocatable, save :: orb(:) 
\end{example}
Saving the \vn{coord_stuct} is faster but leaves memory tied up. Note
that in the main program, the \vn{save} attribute is not permitted If
a \vn{coord_struct} array is passed to a routine, the routine must
explicitly set the lower bound to zero if the array is not declared as
allocatable:
\begin{example}
  subroutine my_routine (orbit1, orbit2)
    use bmad
    implicit none
    type (coord_struct), allocatable :: orbit1(:)  ! OK
    type (coord_struct) orbit2(0:)                 ! Also OK
    ...
\end{example}
Declaring the array allocatable is mandatory if the array is to be resized
or the array is passed to a routine that declares it allocatable.

\index{coord_array_struct}
For an entire lattice, the \vn{coord_array_struct} can be used to define an array
of \vn{coord_array} arrays:
\begin{example}
  type coord_array_struct
    type (coord_struct), allocatable :: orb(:)
  end type
\end{example}
The routine \Hyperref{r:reallocate.coord.array}{reallocate_coord_array} will allocate an
\vn{coord_array_struct} instance
\begin{example}
  type (coord_array_struct), allocatable :: all_orbit(:)
  type (lat_struct) lat
  ...
  call reallocate_coord_array (all_orbit, lat)
  ...
\end{example}

\index{lat_param_struct!ix_track}
Once an array of \vn{coord_struct} elements is defined, the \Hyperref{r:track.all}{track_all} 
routine can be used to track through a given lattice branch
\begin{example}
  type (coord_struct), allocatable :: orbit(:)
  integer ib, track_state
  ...
  ib = 1                      ! Branch to track through
  call init_coord(orbit(0), init_phase_space, lat%branch(ib)%ele(0), proton$) 
  call track_all (lat, orbit, ib, track_state, err_flag)
  if (track_state /= moving_forward\$) then
    print *, "Particle lost at element:", track_state
    print *, "Aperture lost at: ", coord_state_name(orbit(track_state)%state) 
\end{example}
After tracking, \vn{orbit(i)} will correspond to the particles orbit
at the end of \vn{lat%branch(ib)%ele(i)}.  

For routines like \vn{track_all} where an array of \vn{coord_struct}s
is used, an integer \vn{track_state} argument is provided that is set
to \vn{moving_forward\$} if the particle survives to the end, or is
set to the index of the element at which the particle either hit an
aperture or the particle's longitudinal velocity is reversed. 

The reason why the reversal of the particle's longitudinal velocity
stops tracking is due to the fact that the standard tracking routines,
which are $s$-based (that is, use longitudinal position $s$ as the
independent coordinate), are not designed to handle particles that
reverse direction. To properly handle this situation, time-based
tracking needs to be used (\sref{s:time.tracking}). Notice that this is
different from tracking a particle in the reversed ($-s$) direction.

Alternatively to \vn{track_all}, the routine
\Hyperref{r:track.many}{track_many} can be used to track through a
selected number of elements or to track backwards (See
\sref{s:back.track}).

The \vn{track_all} routine serves as a good example of how tracking
works. A condensed version of the code is shown in
\fig{f:track.all}. The call to \vn{track1} (line~18) tracks
through one element from the exit end of the $n-1$\St\ element to the
exit end of the $n$\Th element.

\begin{figure}[h!]
\begin{centering}
\small
\begin{listing}{1}
  subroutine track_all (lat, orbit, ix_branch, track_state, err_flag)
    use bmad
    implicit none
    type (lat_struct), target :: lat
    type (branch_struct), pointer :: branch
    type (coord_struct), allocatable :: orbit(:)
    integer, optional :: ix_branch, track_state
    logical, optional :: err_flag
    logical err

    ! 

    branch => lat%param(integer_option(0, ix_branch))
    branch%param%ix_track = moving_forward
    if (present(track_state)) track_state = moving_forward\$

    do n = 1, branch%n_ele_track
      call track1 (orbit(n-1), branch%ele(n), branch%param, orbit(n), err_flag = err)
      if (.not. particle_is_moving_forward(orbit(n))) then
        if (present(track_state)) track_state = n
        orbit(n+1:)%status = not_set$
        return
      endif
    enddo
  end subroutine
\end{listing}
\caption{Condensed track_all code.}
\label{f:track.all}
\end{centering}
\end{figure}

%----------------------------------------------------------------
\section{Forking from Branch to Branch}

\index{fork}\index{photon_fork}
Tracking from a \vn{fork} or \vn{photon_fork} (\sref{s:fork}) element
to the downstream \vn{branch} is not ``automatic''. That is, since the
requirements of how to handle forking can vary greatly from one
situation to the next, \bmad does not try to track from one \vn{branch}
to the next in any one of its tracking routines. 

The discussion here is restricted to the case where the particle being
tracked is simply transferred from the forking element to the downstream
branch. [Thus the subject of photon generation is not covered here.]

There are two cases discussed here. The first case is when a given
branch (called \vn{to_branch}) has an associated forking element in
the \vn{from_branch} that forks to the beginning of the
\vn{to_branch}. Appropriate code is:
\begin{example}
  type (lat_struct), target :: lat   ! Lattice 
  type (branch_struct) :: to_branch  ! Given forked-to branch
  type (branch_struct), pointer  :: from_branch ! Base branch
  type (ele_struct), pointer :: fork_ele
  type (coord_struct), allocatable :: from_orbit(:), to_orbit(:)
  integer ib_from, ie_from

  ib_from  = to_branch%ix_from_branch

  if (ib_from < 0) then
    ! Not forked to ...

  else
    from_branch => lat%branch(ib_from)
    ie_from = to_branch%ix_from_ele
    fork_ele => from_branch%ele(ie_from)
    to_orbit(0) = from_orbit(ie_from)
    call transfer_twiss (fork_ele, to_branch%ele(0))
  endif
\end{example}
\vn{from_orbit(0:)} and \vn{to_orbit(0:)} are arrays holding the
orbits at the exit end of the elements for the \vn{from_branch} and
\vn{to_branch} respectively. The call to \Hyperref{r:transfer.twiss}{transfer_twiss}
transfers the Twiss values to the \vn{to_branch} which can then be
propagated through the \vn{to_branch} using \vn{twiss_propagate_all}.

The second case starts with the \vn{fork_ele} forking element.  This
is similar to the first case but is a bit more general since here the
element, called \vn{to_ele} in the \vn{to_branch} that is connected to
\vn{fork_ele} need not be the starting element of \vn{to_branch}.
\begin{example}
  type (lat_struct), target :: lat   ! Lattice 
  type (branch_struct), pointer :: to_branch  ! forked-to branch
  type (ele_struct), pointer :: to_ele
  type (coord_struct), allocatable :: from_orbit(:), to_orbit(:)
  integer ib_to, ie_to

  ib_to  = nint(fork_ele%value(ix_to_branch\$))
  ie_to  = nint(fork_ele%value(ix_to_element\$))

  to_branch => lat%branch(ib_to)
  to_ele => to_branch%ele(ie_to)
  to_orbit(to_ele%ix_ele) = from_orbit(fork_ele%ix_ele)
\end{example}  
Notice that, by convention, the transferred orbit is located at the
exit end of the \vn{to_ele}.

%----------------------------------------------------------------
\section{Multi-turn Tracking}

Multi-turn tracking over a branch is simply a matter of
setting the coordinates at the beginning zeroth element equal to the
last tracked element within a loop:
\begin{example}
  type (lat_struct) lat             ! lattice to track through
  type (coord_struct), allocatable :: orbit(:)
  ...
  call reallocate_coord (orbit, lat, ix_branch = 1)
  orbit(0)%vec = [0.01, 0.2, 0.3, 0.4, 0.0, 0.0] ! init
  do i = 1, n_turns
    call track_all (lat, orbit, 1)
    orbit(0) = orbit(lat%branch(1)%n_ele_track)
  end do
\end{example}
Often times it is only the root branch, \vn{branch(0)}, that is to be tracked.
In this case, the above reduces to
\begin{example}
  type (lat_struct) lat             ! lattice to track through
  type (coord_struct), allocatable :: orbit(:)
  ...
  call reallocate_coord (orbit, lat%n_ele_max)
  orbit(0)%vec = [0.01, 0.2, 0.3, 0.4, 0.0, 0.0] ! init
  do i = 1, n_turns
    call track_all (lat, orbit)
    orbit(0) = orbit(lat%n_ele_track)
  end do
\end{example}

%----------------------------------------------------------------
\section{Closed Orbit Calculation}

\index{closed orbit}
For a circular lattice the closed orbit may be calculated using
\vn{closed_orbit_calc}. By default this routine will track in the
forward direction which is acceptable unless the particle you are
trying to simulate is traveling in the reverse direction and there is
radiation damping on. In this case you must tell
\vn{closed_orbit_calc} to do backward tracking. This routine works by
iteratively converging on the closed orbit using the 1--turn matrix to
calculate the next guess. On rare occasions if the nonlinearities are
strong enough, this can fail to converge. An alternative routine is
\vn{closed_orbit_from_tracking} which tries to do things in a more
robust way but with a large speed penalty.

%----------------------------------------------------------------
\section{Partial Tracking through elements}
\label{s:tracking.partial}
\index{tracking!partial}

There are several routines for tracking partially through an element:
\begin{example}
  \Hyperref{r:twiss.and.track.at.s}{twiss_and_track_at_s}
  \Hyperref{r:twiss.and.track.intra.ele}{twiss_and_track_intra_ele}
  \Hyperref{r:track.from.s.to.s}{track_from_s_to_s}
  \Hyperref{r:twiss.and.track.from.s.to.s}{twiss_and_track_from_s_to_s}
  \Hyperref{r:mat6.from.s.to.s}{mat6_from_s_to_s}
\end{example}
These routines make use of element ``slices'' (\sref{s:ele.slice}) which
are elements that represent some sub-section of an element. There are two
routines for creating slices:
\begin{example}
  \Hyperref{r:create.element.slice}{create_element_slice}
  \Hyperref{r:create.uniform.element.slice}{create_uniform_element_slice}
\end{example}

It is important to note that to slice up a given element, the \vn{s_to_s} tracking
routines will not always work. For example, consider the case where a given element is
followed by a zero length multipole. If \vn{track_from_s_to_s} is called with a value for
\vn{s2} (the value at the end of the track) which corresponds to the exit end of this
element, the result will also include tracking through the zero length multipole. Thus, in
the case where a given element is to be sliced, one or the other of the two slice routines
given above must be first used to create an element slice then this slice can be used for
tracking.

%----------------------------------------------------------------
\section{Apertures}
\label{s:tracking.apertures}
\index{tracking!apertures}

\index{ele_struct!\%aperture_type}
The routine \Hyperref{r:check.aperture.limit}{check_aperture_limit}
checks the aperture at a given element. The \vn{ele%aperture_type}
component determines the type of aperture. Possible values for
\vn{ele%aperture_type} are
\begin{example}
  rectangular$
  elliptical$
  custom$
\end{example} %$
With \vn{custom\$}, a program needs to be linked with a custom version
of
\Hyperref{r:check.aperture.limit.custom}{check_aperture_limit_custom}.

\index{bmad_common_struct!aperture_limit_on}
\index{bmad_common_struct!max_aperture_limit}
The logical \vn{bmad_com%aperture_limit_on} determines if element
apertures (See \sref{s:limit}) are used to determine if a
particle has been lost in tracking.  The default
\vn{bmad_com%aperture_limit_on} is True.  Even if this is False
there is a ``hard'' aperture limit set by
\vn{bmad_com%max_aperture_limit}. This hard limit is used to prevent
floating point overflows. The default hard aperture limit is 1000
meters. Additionally, even if a particle is within the hard limit,
some routines will mark a particle as lost if the tracking calculation
will result in an overflow.

\index{lat_param_struct!end_lost_at}
\index{lat_param_struct!lost}
\index{lat_param_struct!ix_lost}
\index{entrance_end}
\index{exit_end}
\vn{lat%param%lost} is the logical to check to see if a particle has
been lost. \vn{lat%param%ix_lost} is set by \vn{track_all} and gives
the index of the element at which a particle is lost.
\vn{%param%end_lost_at} gives which end the particle was lost at. 
The possible values for \vn{lat%param%end_lost_at} are:
\begin{example}
  entrance_end$
  exit_end$
\end{example}
When tracking forward, if a particle is lost at the exit end of an
element then the place where the orbit was outside the aperture is at
\vn{orbit(ix)} where \vn{ix} is the index of the element where the
particle is lost (given by \vn{lat%param%ix_lost}). If the
particle is lost at the entrance end then the appropriate index is one
less (remember that \vn{orbit(i)} is the orbit at the exit end of an
element). 

To tell how a particle is lost, check the \vn{lat%param%plane_lost_at}
parameter. Possible values for this are:
\begin{example}
  x_plane$
  y_plane$
  z_plane$
\end{example}               % $
\vn{x_plane\$} and \vn{y_plane\$} indicate that the particle was lost either horizontally, or
vertically. \vn{z_plane\$} indicates that the particle was turned around in an \vn{lcavity}
element. That is, the cavity was decelerating the particle and the particle did not not have enough
energy going into the cavity to make it to the exit.

%----------------------------------------------------------------
\section {Custom Tracking}

Custom code can be used for tracking. This is discussed in detail in sections \sref{s:custom.ele}
and \sref{s:hook}.

%----------------------------------------------------------------
\section {Tracking Methods}

\index{ele_struct!\%tracking_method}
For each element the method of tracking may be set either via the
input lattice file (see \sref{s:tkm}) or directly in the
program by setting the \vn{%tracking_method} attribute of an element
\begin{example}
  type (ele_struct) ele
  ...
  ele%tracking_method = symp_lie_ptc$  ! for symp_lie_ptc, tracking
  print *, "Tracking_method: ", calc_method_name(ele%tracking_method)
\end{example}
To form the corresponding parameter to a given tracking method just
put ``\$'' after the name. For example, the \vn{bmad_standard}
tracking method is specified by the \vn{bmad_standard\$} parameter. To
convert the integer \vn{%tracking_method} value to a string suitable
for printing, use the \vn{tracking_method_name} array.

\index{ele_struct!\%mat6}\index{linear}
It should be noted that except for \vn{linear} tracking, none of the
\bmad tracking routines make use of the \vn{ele%mat6} transfer
matrix. The reverse, however, is not true.  The transfer matrix
routines (\vn{lat_make_mat6}, etc.)  will do tracking.

For determining what tracking methods are valid for a given element, use
\Hyperref{r:valid.tracking.method}{valid_tracking_method} and
\Hyperref{r:valid.mat6.calc.method}{valid_mat6_calc_method} functions
\begin{example}
  print *, "Method is valid: ", valid_tracking_method(ele, symp_lie_ptc\$)
\end{example}

\index{synchrotron radiation!calculating}
\bmad simulates radiation damping and excitation by applying a kick
just before and after each element. 

%----------------------------------------------------------------
\section{Using Time as the Independent Variable}
\label{s:time.tracking}

Time tracking uses time as the independent variable as opposed to the standard $s$ based
tracking. Time tracking is useful when a particle's trajectory can reverse itself
longitudinally. For example, low energy particles generated when a relativistic particle hits the
vacuum chamber wall are good candidates for time tracking.

Currently, the only \vn{ele%tracking_method} available for time tracking is
\vn{time_runge_kutta\$}. Time tracking needs extra bookkeeping due to the fact that the particle may
reverse directions.  See the \vn{dark_current_tracker} program as an example.

Note: Using time as the independent variable can be used with both absolute and relative time
tracking (\sref{s:rf.time}).

%----------------------------------------------------------------
\section{Absolute/Relative Time Tracking}
\label{s:abs.time}
\index{absolute time tracking}

\index{bmad_common_struct!absolute_time_tracking}
Absolute or relative time tracking (\sref{s:rf.time}) can be set after the lattice file is parsed,
by setting the \vn{%absolute_time_tracking} component of the \vn{lat_struct}. when
\vn{%absolute_time_tracking} is toggled, the
\Hyperref{r:autoscale.phase.and.amp}{autoscale_phase_and_amp} must be called to reset the
appropriate phase offsets and scale amplitudes.

%----------------------------------------------------------------
\section{Taylor Maps}
\label{s:taylor.track}
\index{taylor Map}

A list of routines for manipulating Taylor maps is given in~\sref{r:taylor}. The order of the Taylor
maps is set in the lattice file using the \vn{parameter} statement (\sref{s:param}). In a program
this can be overridden using the routine \Hyperref{r:set.ptc}{set_ptc}. The routine
\Hyperref{r:taylor.coef}{taylor_coef} can be used to get the coefficient of any given term in a
Taylor map.
\begin{example}
  type (taylor_struct) t_map(6)
  ...
  print *, "out(4)=coef * in(1)^2:", taylor_coef(t_map(4), 1, 1)
  print *, "out(4)=coef * in(1)^2:", taylor_coef(t_map(4), [2,0,0,0,0,0])
\end{example}

\index{symp_lie_Bmad}
\index{symp_lie_PTC}
\index{taylor}
\index{taylor!deallocating}
Transfer Taylor maps for an element are generated as needed when the
\vn{ele%tracking_method} or \vn{ele%mat6_calc_method} is set to
\vn{Symp_Lie_Bmad}, \vn{Symp_Lie_PTC}, or
\vn{Taylor}. Since generating a map can take an appreciable time,
\bmad follows the rule that once generated, these maps are never
regenerated unless an element attribute is changed.  To generate a
Taylor map within an element irregardless of the
\vn{ele%tracking_method} or \vn{ele%mat6_calc_method} settings use the
routine \Hyperref{r:ele.to.taylor}{ele_to_taylor}. This routine will kill any old Taylor map
before making any new one. To kill a Taylor map (which frees up the
memory it takes up) use the routine \Hyperref{r:kill.taylor}{kill_taylor}.

To test whether a \vn{taylor_struct} variable has an associated Taylor
map. That is, to test whether memory has been allocated for the map,
use the Fortran associated function:
\begin{example}
  type (bmad_taylor) taylor(6)
  ...
  if (associated(taylor(1)%term)) then  ! If has a map ...
    ...
\end{example}

To concatenate the Taylor maps in a set of elements the routine
\Hyperref{r:concat.taylor}{concat_taylor} can be used
\begin{example}
  type (lat_struct) lat          ! lattice
  type (taylor_struct) taylor(6)  ! taylor map
  ...
  call taylor_make_unit (taylor)  ! Make a unit map
  do i = i1+1, i2
    call concat_taylor (taylor, lat%ele(i)%taylor, taylor)
  enddo
\end{example}
The above example forms the transfer Taylor map starting at the end of
element \vn{i1} to the end of element \vn{i2}. Note: This example
assumes that all the elements have a Taylor map. The problem with
concatenating maps is that if there is a constant term in the map
``feed down'' can make the result inaccurate (\sref{s:taylor.phys}. To
get around this one can ``track'' a taylor map through an element
using symplectic integration.
\begin{example}
  type (lat_struct) lat          ! lattice
  type (taylor_struct) taylor(6)  ! taylor map
  ...
  call taylor_make_unit (taylor)  ! Make a unit map
  do i = i1+1, i2
    call call taylor_propagate1 (taylor, lat%ele(i), lat%param)
  enddo
\end{example}
\index{ds_step}
\index{integrator_order}
Symplectic integration is typically much slower than concatenation.  The width of an integration
step is given by \vn{%ele%value(ds_step\$}.  The attribute \vn{%ele%value(num_steps\$)}, which gives
the number of integration steps, is a dependent variable (\sref{s:depend}) and should not be set
directly.  The order of the integrator (\sref{s:taylor.phys}) is given by
\vn{%ele%integrator_order}.  PTC (\sref{c:ptc.use}) currently implements integrators of order 2, 4, or
6.

%----------------------------------------------------------------
\section{Tracking Backwards}
\label{s:back.track}
\index{tracking!backwards}

Tracking backwards happens when a particle goes in the direction of decreasing \vn{s}. This is
indicated in the \vn{coord_struct} by \vn{coord%direction = -1}. 

The \vn{time_runge_kutta} \vn{tracking_method} is able to handle the situation where a particle
would reverse direction due to string electric or magnetic fields. All other tracking methods are
not able to handle this since they are position ($s$) based, instead of time based. With non
\vn{time_runge_kutta} tracking methods, the equations of motion become singular when a particle
``tries'' to reverse direction. In such a situation, the particle will be marked as lost and the
\vn{coord_struct} will have \vn{%status} /= \vn{alive\$}.

The ``problem'' with tracking backwards is that the reference time $t_0(s)$ that is used to compute
the $z$ phase space coordinate (\Eq{zbctt}) is independent of the motion of any particle. That is, a
particle traveling backwards will have a large negative $z$. As an alternative to tracking
backwards, reversing the lattice and tracking forwards is possible (\sref{s:reverse.track}).

One restriction with backwards tracking is that, for simplicity's sake, \bmad does not
compute transfer matrices for propagation in the backwards direction. Tracking with reversed
elements does not have this restriction.

%----------------------------------------------------------------
\section{Reversed Elements and Tracking}
\label{s:reverse.track}
\index{reversed elements}

With a lattice element that is reversed (\vn{s:ele.reverse}), the transfer map and transfer matrix
that is stored in the element is, just like for a non-reversed element, appropriate for a particle
traveling in the $+s$ direction.

%----------------------------------------------------------------
\section{Beam (Particle Distribution) Tracking}
\label{s:part.track}
\index{tracking!particle distributions}

Tracking with multiple particles is done with a \vn{beam_struct} instance:
\begin{example}
  type beam_struct
    type (bunch_struct), allocatable :: bunch(:)
  end type
\end{example}
A \vn{beam_struct} is composed of an array of bunches of type
\vn{bunch_struct}:
\begin{example}
  type bunch_struct
    type (coord_struct), allocatable :: particle(:)
    integer, allocatable :: ix_z(:)  ! bunch%ix_z(1) is index of head particle, etc.
    real(rp) charge_tot  ! Total charge in bunch (Coul).
    real(rp) charge_live ! Total charge of live particles in bunch (Coul).
    real(rp) z_center    ! Longitudinal center of bunch (m). Note: Generally, z_center of 
                         !   bunch #1 is 0 and z_center of the other bunches is negative.
    real(rp) t_center    ! Center of bunch creation time relative to head bunch.
    integer species      ! electron\$, proton\$, etc.
    integer ix_ele       ! Element this bunch is at.
    integer ix_bunch     ! Bunch index. Head bunch = 1, etc.
  end type
\end{example}
The \vn{bunch_struct} has an array of particles of type
\vn{coord_struct} (\sref{s:coord.struct}).

Initializing a \vn{beam_struct} to conform to some initial set of
Twiss parameters and emittances is done using the routine
\Hyperref{r:init.beam.distribution}{init_beam_distribution}: 
\begin{example}
  type (lat_struct) lat
  type (beam_init_struct) beam_init
  type (beam_struct) beam
  ...
  call init_beam_distribution (lat%ele(0), lat%param, beam_init, beam)
\end{example}
The \vn{lat%ele(0)} argument, which is of type \vn{ele_struct}, gives the twiss parameters to
initialize the beam to. In this case, we are starting tracking from the beginning of the
lattice. The \vn{beam_init} argument which is of type \vn{beam_init} gives additional information,
like emittances, which is needed to initialize the beam. See chapter~\sref{c:beam.init} for more
details.

Tracking a beam is done using the \Hyperref{r:track.beam}{track_beam} routine
\begin{example}
  type (lat_struct) lat
  type (beam_struct) beam
  ...
  call track_beam (lat, beam)
\end{example}
or, for tracking element by element, \Hyperref{r:track1.bunch}{track1_bunch} can be used.

For analyzing a bunch of particles, that is, for computing such things as the sigma matrix from the
particle distribution, the \Hyperref{r:calc.bunch.params}{calc_bunch_params} routine can be used.

Notice that when a particle bunch is tracked to a given longitudinal position in the lattice, all
the particles of the bunch are at that longitudinal position (this is no different if particles are
tracked individually independent of the bunch). Given that the bunch has a non-zero bunch length,
the current time $t(s)$ associated with the particles will be different for different particles (See
\Eq{zbctt}). If it is desired to reconstruct the shape of the bunch at {\em constant time}, each
particle must be tracked either forward or backwards by an appropriate amount. Since this tracking
generally involves only very short distances, it is usually acceptable to ignore any fields and to
propagate the particles as if they were in a field free region.

%----------------------------------------------------------------
\section{Spin Tracking}
\label{s:spin.track}
\index{tracking!spin}

See Section~\sref{s:spin.methods} for a list of spin tracking methods available. To turn spin
tracking on, use the \vn{bmad_com%spin_tracking_on} flag. \vn{ele%spin_tracking_method} sets the
method used for spin tracking. After properly initializing the spin in the \vn{coord_struct}, calls
to \vn{track1} will track both the particle orbit and the spin.

The Sokolov-Ternov effect\cite{b:barber99} is the self-polarization of charged particle beams due to
asymmetric flipping of a particle's spin when the particle is bent in a magnetic field. Whether this
effect is included in a simulation is determined by the setting of
\vn{bmad_com%spin_sokolov_ternov_flipping_on}.  Also, spin flipping will {\em not} be done if spin
tracking is off or both radiation damping and excitation are off.


%-------------------------------------------------------------------------
\section{X-ray Targeting}
\label{s:targeting.code}

X-rays can have a wide spread of trajectories resulting in many
``doomed'' photons that hit apertures or miss the detector with only a
small fraction of ``successful'' photons actually contributing to the
simulation results. The tracking of doomed photons can therefore
result in an appreciable lengthening of the simulation time. To get
around this, \bmad can be setup to use what is called ``targeting'' to
minimize the number of doomed photons generated. 

This is explained in detail in \sref{s:targeting}. The coordinates of
the four or eight corner points and the center target point are 
stored in:
\begin{example}
  gen_ele%photon%target%corner(:)%r(1:3)
  gen_ele%photon%target%center%r(1:3)
\end{example}
where \vn{gen_ele} is the 
generating element (not the element with the aperture).

%-------------------------------------------------------------------------
%\section{Recording the Track Through an Element}
%\label{s:track.track}
%
%Occasionally it is useful to record the track through an element when tracking with Runge-Kutta,
%etc. This can be done by calling \Hyperref{r:track1}{track1} and supplying the \vn{track} argument. Example:
%\begin{example}
%  type (track_struct) track
%  ...
%  track%n_pt = -1
%  call track1 (start_orb, ele, param, end_orb, track)
%\end{example}
%
%A \vn{track_struct} structure has components:
%\begin{example}
%  type track_struct
%    type (coord_struct), allocatable :: orb(:)      ! An array of track points: %orb(0:)
%    type (em_field_struct), allocatable:: field(:)  ! An array of em fields: %field(0:)
%    type (track_map_struct), allocatable :: map(:)  ! An array of maps: %cylindrical_map(0:)
%    real(rp) :: ds_save = 1e-3                      ! Min distance between points. Not positive => Save at all points.
%    integer :: n_pt = -1                            ! Track upper bound for %orb(0:), etc. arrays.
%    integer :: n_bad = 0                            ! Number of bad steps when adaptive tracking is done.
%    integer :: n_ok = 0                             ! Number of good steps when adaptive tracking is done.
%  end type
%\end{example}
%The \vn{%orb} and other arrays 
%The \vn{%n_pt} component gives the upp


\chapter{Transfer Matrices}
\chapter{Twiss Parameters}
\chapter{Interface to FPP/PTC}
\chapter{CESR Centric Routines}
\chapter{Routines Sorted by Functionality}

include dcslib and cesr\_utils routines



%----------------------------------------------------------------
\part{Physics Notes}
%----------------------------------------------------------------
\chapter{Emittances and Synchrotron Radiation}


%----------------------------------------------------------------
\begin{theindex}
\end{theindex}

\end{document}
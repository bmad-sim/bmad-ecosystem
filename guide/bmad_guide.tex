\documentclass{book}
\usepackage{graphicx}

%\makeatletter    % internal ``@'' commands can now be used. 
%\@addtoreset{chapter}{part}
%\makeatother     % Internal ``@'' commands are locked.
%\renewcommand{\thechapter}{\thepart.\arabic{chapter}}

\newcommand{\ringstruct}{{\bf ring\_struct}}
\newcommand{\elestruct}{{\bf ele\_struct}}
\newcommand{\coordstruct}{{\bf coord\_struct}}
\newcommand{\bmad}{{\bf BMAD}}
\newcommand{\rp}{{\bf rp}}
\newcommand{\rdef}{{\bf rdef}}


%----------------------------------------------------------------

\begin{document}

\title{The \bmad Reference Manual}

\date{11 June, 2003}
\maketitle

%----------------------------------------------------------------
\section*{Overview}

\bmad (Otherwise known as "Baby MAD" or "Better MAD" or just plain "BE MAD!")
is a software subroutine library used for simulating 
relativistic charged--particle dynamics in high energy accelerators
and storage rings. \bmad can be used to read in lattice files, compute 
Twiss parameters, track particles, etc. 
These subroutines, written in  Fortran90, have been developed to:
\begin{enumerate}
\item Cut down on the time needed to develop programs,
\item Cut down on programming errors, and
\item Provide a standard input format for specifying lattices.
\end{enumerate}
\bmad has been developed at Cornell University's Laboratory for Elementary
Particle Physics and has been in use since the 1990's.
\vfill
\break
%----------------------------------------------------------------
\section*{Introduction}

The strength of \bmad is that as a subroutine library it provides a flexible
framework from which sophisticated simulation programs may easily be developed.
The weakness of \bmad comes from its strength: Someone must put the pieces 
together into a program. As a consequence this manual serves two masters:

The \bmad lattice input standard was developed using the MAD lattice
input standard as a starting point. MAD (Methodical Accelerator
Design) is a widely used stand--alone program developed at CERN by
Christoph Iselin for charged--particle optics calculations. The
limitations of the MAD program was the impetus for writing
\bmad. Since it can be convenient to do simulations with both MAD and
\bmad, differences and similarities between the two input formats are
noted in this guide.

Errors and omissions are a fact of life for any reference work and
comments from you, dear reader, are therefore most welcome. Please
send any missives (or chocolate, etc.) to:
\begin{verbatim}
  David Sagan <dcs16@cornell.edu>
\end{verbatim}


%----------------------------------------------------------------
\tableofcontents

\listoffigures

\listoftables

%----------------------------------------------------------------
%----------------------------------------------------------------
\part{Conventions and Physics}

%----------------------------------------------------------------
\chapter{Conventions}


%----------------------------------------------------------------
%----------------------------------------------------------------
\part{Language Reference}


%----------------------------------------------------------------
\chapter{Syntax}

%----------------------------------------------------------------
\chapter{Arithmetic Expressions}

%----------------------------------------------------------------
\chapter{Physical Units and Constants}

%----------------------------------------------------------------
\chapter{Parameters}

%----------------------------------------------------------------
\chapter{Elements}

%----------------------------------------------------------------
\chapter{Element Sequencing}

%----------------------------------------------------------------
\chapter{Superposition of Elements}

%----------------------------------------------------------------
\chapter{Elements Controlling Other Elements}

%----------------------------------------------------------------
\chapter{Tracking Methods}

%----------------------------------------------------------------
\chapter{Transfer Matrix Calculation Methods}

%----------------------------------------------------------------
%----------------------------------------------------------------
\part{Programmer's Guide}

%----------------------------------------------------------------
\chapter{The \bmad Distribution}



\section{Libraries in the \bmad Distribution}

When installing \bmad on a computer what one gets is not only the \bmad
subroutine library but additionally subsidiary libraries upon which
subroutines in the \bmad library depend. There are 5 other libraries
that are used: cesr\_utils, forest, numerical\_recipes, pgplot, and dcslib.
\begin{description}
\item[cesr\_utils] This is a small low level library that primarily defines 
the precision that \bmad works at (see below) and defines the physical
and mathematical constants (pi, c\_light, etc.) that \bmad knows
about.
\item[dcslib] This library defines a set of miscellaneous helper routines. 
Routines include spline fitting, Gaussian random number generation,
etc. The library name comes from its creator.
\item[forest] This is the FPP/PTC 
(Full Polymorphic Package/ Polymorphic Tracking Code) library of
Etienne Forest that handles Taylor maps to any arbitrary order (this
is also known as Truncated Power Series Algebra (TPSA)). The FPP part
handles the TPSA and the PTC part does the physics of tracking through
elements using Lie Algebra with a Hamiltonian.  \bmad uses this
software to crate Taylor Maps, track particles, etc.  FPP/PTC is a
very general package and \bmad only makes use of a small part of its
features. See the FPP/PTC documentation at
<http://bc1.lbl.gov/CBP\_pages/educational/TPSA\_DA/Introduction.html>
for more information.
\item[recipes] Numerical Recipes is a set of subroutines for doing 
scientific computing including Runge--Kutta integration, FFT's,
interpolation and extrapolation, etc., etc. The writeup for this
library is the book ``Numerical Recipes, The Art of Scientific
Computing''\cite{?}. For \bmad this library has been modified to handle
both single and double precision reals.
\item[pgplot] The PGPLOT Graphics Subroutine Library is a Fortran or 
C-callable, device-independent graphics package for making simple
scientific graphs. More information may be obtained from the PGPLOT
web site at <http://www.astro.caltech.edu/\~tjp/pgplot/>. One
disadvantage of PGPLOT is that it is not the most friendly software
for the programmer. To remidy this, there is a set of Fortran90
wrapper subroutines called quick\_plot. The quick\_plot suite is part
of the dcslib library.
\end{description}

\section{Precision}

\bmad comes in two flavors: One where the real numbers are single
precision and a version with double precision reals. Which version you
are working with is controlled by the parameter \rp (Real Precision)
which is defined in cesr\_utils. [Note: For compatibility with older
programs the parameter \rdef is defined to be equal to \rp.]  On most
machines single precision has \rp = 4 and double precision has \rp =
8. Normally the double precision version is used since round-off
errors can be significant in some calculations. Long--term tracking is
an example where the single precision version is not adequate. 

To define your variables with the correct precision use the syntax
{\it real(rp)}. For example:
\begin{verbatim}
  real(rp) var1, var2, var3
\end{verbatim}
When you want to define a literal constant, for example to pass an
argument to a subroutine, add the suffix {\it \_rp} to the end of the
constant. For example: {\it 2.0\_rp} is equivalent to {\it 2.0D0} if
\rp is defined to be double precision. Notice that this is not
equivalent to {\it 2\_rp} which defines an integer (not a real) constant.


\section{Helper programs}

The {\it listf} command is used to locate routines and structures in
the cesr\_utils, dcslib and \bmad libraries. The form of the command is
\begin{verbatim}
  listf <name>
\end{verbatim}
This searches for any routine or structure with the name
<name>. <name> may contain the wild--cards ``*'' and ``\%'' where
``*'' matches to any number of characters and ``\%'' matches to any
single character. For example:
\begin{verbatim}
  listf ring_struct
  listf twiss_at_%
\end{verbatim}
The second example will match to {\it twiss\_at\_s} but not {\it
twiss\_at\_start}.

The {\it getf} command is like the {\it listf} command with the
addition that the header comments that are in the source code files
will be printed out for each routine match and the structure definition
will be printed for each structure matched. The {\it getf} command is
thus more verbose than the {\it listf} command.

%----------------------------------------------------------------
\chapter{The \bmad Module Hierarchy}


%----------------------------------------------------------------
\chapter{The Ring\_struct}

\section{overview}

\ringstruct is structure that holds all the information about a lattice.
The definition of \ringstruct is shown in figure~\ref{fig:ring_struct}
along with the definitions of its substructures (except for the
\elestruct definition which is given in the next section). Despite its
name there is no assumption about whether the lattice is circular as
with a storage ring or open as with a LINAC.

\begin{figure}[tb]
\centering
\begin{verbatim}
  type ring_struct
    character*16 name                 ! Name of ring as given by USE statement
    character*40 lattice              ! Lattice
    character*80 input_file_name      ! name of the lattice input file
    character*80 title                ! general title
    type (mode_info_struct)  x, y, z  ! tunes, etc.
    type (param_struct) param         ! parameters
    integer version                   ! Version number
    integer n_ele_ring                ! number of regular ring elements
    integer n_ele_symm                ! symmetry point for rings w/ symmetry
    integer n_ele_use                 ! number of elements used
    integer n_ele_max                 ! Index of last element used
    integer n_ele_maxx                ! Index of last element allocated
    integer n_control_array           ! last index used in CONTROL_ array
    integer n_ic_array                ! last index used in IC_ array
    integer input_taylor_order        ! As set in the input file
    type (ele_struct), pointer :: ele_(:)    ! Array of ring elements
    type (ele_struct)  ele_init              ! For use by any program
    type (control_struct)  control_(n_control_maxx)  ! control list
    integer ic_(n_control_maxx)                      ! index to %control_(:)
  end type

  type control_struct
    real(rp) coef                  ! control coefficient
    integer ix_lord                ! index to lord element
    integer ix_slave               ! index to slave element
    integer ix_attrib              ! index of attribute controlled
  end type

  type param_struct
    real(rp) energy             ! USE BEAM_ENERGY INSTEAD ! energy in GeV.
    real(rp) beam_energy        ! beam energy in eV
    real(rp) n_part             ! Number of particles in a bunch
    real(rp) total_length       ! total_length of ring
    real(rp) growth_rate        ! growth rate/turn if not stable
    real(rp) t1_mat6(6,6)       ! Full 1-turn 6x6 matrix
    real(rp) t1_mat4(4,4)       ! Transverse 1-turn 4x4 matrix (RF off).
    integer particle            ! +1 = positrons, -1 = electrons
    integer symmetry            ! symmetry of the ring (e/w symm, etc.)
    integer ix_lost             ! If lost at what element?
    integer lattice_type        ! LINAC_lattice$, circular_lattice$, etc...
    integer ixx                 ! Integer for general use
    logical stable              ! is closed ring stable?
    logical aperture_limit_on   ! use apertures in tracking?
    logical lost                ! for use in tracking
  end type

  type mode_info_struct
    real(rp) tune          ! "fractional" tune in radians: 0 < tune < 2pi
    real(rp) emit          ! Emittance
    real(rp) chrom         ! Chromaticity
  end type
\end{verbatim}
\caption{Definition of the \ringstruct and substructures 
(except the \elestruct).}
\label{fig:ring_struct}
\end{figure}

The array {\sl \%ele\_(n)} holds the elements of the lattice. The
array is divided up into two parts: A ``physical'' part and a
``control'' part (Cf.~(ref in lang section). The physical part of this
array holds the elements that are tracked through. The control part
holds the superimpose, overlay and group elements. The extent of these
parts is given in table~\ref{tab:part_extent}. 

\begin{table}[tb]
\begin{center}
\begin{tabular}{l|l|l}
\hline
              & \multicolumn{2}{c|} {\em index n}        \\ \hline
{\em section} & {\em min}         & {\em max}          \\ \hline
physical      & 0                 & \%n\_ele\_ring     \\ \hline
control       & \%n\_ele\_ring+1  & \%n\_ele\_ring+1   \\ \hline
\end{tabular} 
\caption{Extent of the physical and control parts 
of the array {\sl \%ele\_(n)}.}
\end{center}
\label{tab:part_extent}
\end{table}



\section{Element Control}

    * Overlays
    * Superimpose
    * Groups

Local \ringstruct variables must have the save attribute or
the pointers within must be appropriately deallocated
before leaving the routine.

\section{\%ele\_(:) Size}

How to allocate \coordstruct arrays. see the tracking section


%----------------------------------------------------------------
\chapter{The Ele\_struct}


\section{Initializing}

  Dependent and independent variables

  How to locate attributes

  Allocation/deallocation

%----------------------------------------------------------------
\chapter{The Coord\_struct}

  Meaning of vec elements

  Meaning of vec elements in PTC.

%----------------------------------------------------------------
\chapter{Reading and Writing Lattices}

\section{Digested files}

\section{bmad\_parser and bmad\_parser2}

%----------------------------------------------------------------
\chapter{Tracking}

When using radiation excitation the random number generator is based
upon the fortran90 intrinsic random\_number. If random\_seed is not called
then the fluctuations from run to run will be exactly the same.


How to allocate coord\_struct arrays. 

  Switches to switch tracking

  Runge Kutta

  Boris

  Custom
    * how to setup



%----------------------------------------------------------------
\chapter{Transfer Matrices}

%----------------------------------------------------------------
\chapter{Twiss Parameters}

%----------------------------------------------------------------
\chapter{Interface to FPP/PTC}

%----------------------------------------------------------------
\chapter{CESR Centric Routines}

%----------------------------------------------------------------
\chapter{Routines Sorted by Functionality}

include dcslib and cesr\_utils routines

%----------------------------------------------------------------
%----------------------------------------------------------------
\part{Physics Notes}

%----------------------------------------------------------------
\chapter{Emittances and Synchrotron Radiation}



%----------------------------------------------------------------
%----------------------------------------------------------------
\begin{theindex}


\end{theindex}

\end{document}
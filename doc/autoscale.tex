\chapter{Automatic Scaling of Accelerating Fields}
\label{c:autoscale}
\index{automatic field scaling|hyperbf}

\index{e_gun}\index{em_field}\index{lcavity}\index{rfcavity}
The elements that can have accelerating fields are:
\begin{example}
  e_gun       ! \sref{s:e.gun}
  lcavity     ! \sref{s:lcav}
  rfcavity    ! \sref{s:rfcav}
\end{example}

The problem that arises with accelerating fields is how to set the
overall amplitude (and phase if the fields are oscillating) of the
field so that the reference particle has the desired acceleration. The
problem becomes even more complicated at non-ultra relativistic
energies where the velocity is not a constant. In this case, the
proper amplitude and/or phase settings will depend upon what the
incoming energy of the reference particle is.

The scaling problem is not present when \vn{bmad_standard} tracking
(\sref{s:tkm}) is used since \vn{bmad_standard} tracking uses an
integrated formula that is designed to give the proper acceleration.

To help with the scaling problem, \bmad has the capability to automatically
scale an accelerating field's amplitude and/or phase. The two
lattice element parameters that turn on/off auto scaling are (\sref{s:param}):
\begin{example}
  autoscale_phase    = <Logical>  ! Automatic phase scaling.
  autoscale_amplitude      = <Logical>  ! Automatic amplitude scaling.
\end{example}
Example:
\begin{example}
  rf2: rfcavity, autoscale_phase = F
\end{example}
The default value is True for both parameters. 

Scaling takes place during program execution when a lattice is
initially created (that is, when the lattice file is parsed) and when
parameters in the lattice that would change the scaling are varied.

The parameters that are varied when the field is auto scaled depend
upon how the field is calculated. When the fields are specified using
modes (\sref{s:em.fields}), each mode has a \vn{field_scale} that
varies the field amplitude and each mode has a phase offset
\vn{phi0_ref}. Not all modes are involved with auto scaling. For
example, an \vn{em_field} element may have an accelerating mode along
with a mode for a solenoid field which is independently powered. In
this instance, only the accelerating mode should be auto scaled. The
``primary'' mode that is auto scaled is set by the
\vn{mode_to_auto scale} component of the \vn{field}
(\sref{v:field}). Additionally, any other mode (called ``secondary''
modes) whose \vn{master_scale} is the same as the \vn{master_scale} of
the primary auto scaled mode will be auto scaled. All modes involved in
auto scaling have their \vn{field_scale} adjusted by the same ratio.
Additionally, when the phase is auto scaled, all secondary modes will
have their \vn{phi0_ref} phases adjusted by the same difference that
the \vn{phi0_ref} of the primary mode is changed. Example:
\begin{example}
  rf2334: lcavity, field_calc = grid, field = \{
    mode_to_autoscale = 1,
    mode = \{master_scale = gradient, ...\},   ! Primary mode
    mode = \{master_scale = gradient, ...\},   ! Secondary mode
    mode = \{master_scale = none, ...\} \}      ! Not involved in auto scale
\end{example}

When the fields are {\em not} specified using modes, the element's
\vn{field_factor} parameter will be used for scaling the amplitude and
the \vn{phi0_ref} parameter (which is distinct from the \vn{phi0_ref}
associated with a mode) will be used for varying the phase.









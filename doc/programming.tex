%-----------------------------------------------------------------
%\chapter{Introduction}
%\label{c:prog_intro} 

\tao has been designed to be ready extensible with a minimum of
effort. The tutorial providesa simple example of a custom data type. Here each
the hook routines are explained and pointers are given for writing code. 

\chapter{Creating a Custom Version of \tao}
\label{c:prog_intro} 

The process is summarized as follows: For the purposes of this
discussion assume that the directory that you are developing \tao in
is called \vn{ROOT}. 
\begin{enumerate}
\item 
The first step is to checkout from CVS (or download a
copy) of \tao. You will now have \tao in
\vn{ROOT/tao}. The library source code will be sitting in \vn{ROOT/tao/code}
and the vanilla \tao program is \vn{ROOT/tao/program/tao_cl.f90}
\item 
From the \vn{ROOT}/tao area build the \tao library using the
\vn{gmake} command. This will create a \tao library that includes dummy hook
routines and structures. When creating custom hooks these are overiden.
\item
To extend \tao you will want to make a new
directory, say, called \vn{ROOT/my_tao}. In this directory you write
the necessary routines to extend \tao. You will also need a standard \vn{Makefile}
 for building programs.
\item
You can compile and link your routines with the \tao routines using
\vn{gmake} in \vn{ROOT/my_tao}.
\end{enumerate}

The tutorial in Part I of this manual gave an example of how to carry out the
above steps. This is repeated below but in greater detail.

\section{Creating the \tao Library and a Custom \tao Directory}
After obtaining the \tao distribution the \tao library is created by typing 
\cmd{gmake} in the \cmd{ROOT/tao} directory. This will create two libraries
called \cmd{libtao.a} and \cmd{libtao_g.a} where the second is a debug version
in the directory \cmd{ROOT/lib}. If you then type \cmd{gmake -f M.tao} then
"vanilla" \tao will be compiled called \cmd{tao} and \cmd{tao_g} and placed in
\cmd{ROOT/bin}


\section{Modifying the Hook Routines and Structures}

The golden rule when writing routines to extend \tao is that you are
only allowed to replace routines or redefine structures that have the
name ``hook'' in them. The reason for this is to ensure that, as time
goes by, and revisions are made to the \tao routines to extend the
usefulness of \tao and to eliminate bugs, that these changes will
have a minimum impact on the specialized routines that will be written
by various people to extend \tao.  What happens if you need to replace
or modify a non--hook routine or structure?  The answer is to contact
the \tao programming team and we will modify \tao and create the hooks 
you need.

Before one can begin writing code one must understand the structures
that \tao uses. The structures are defined in a file
\vn{tao/code/tao_struct.f90}.  \tao is based upon the \bmad software
package for simulations of relativistic charged particles and the
\tao structures have components that are defined in \bmad. For
information on these structures see the \bmad Reference Manual. Note
that the hook--structures are defined in a file \vn{tao_hook_mod.f90}.

%-----------------------------------------------------------------
\chapter{Plotting}
\label{s:prog_plotting} 

Plotting is based upon the \vn{quick_plot} subroutines which are
documented in the \bmad reference manual and you should review this
material if you are not familiar with concepts of ``graph'', ``box'',
and ``page''. 

\fbox{this chapter is yet to be completed!} 


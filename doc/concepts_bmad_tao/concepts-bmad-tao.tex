%-----------------------------------------
% Note: Use pdflatex to process this file.
%-----------------------------------------

%\documentclass{article}
\documentclass{hitec}

\usepackage{setspace}
\usepackage{graphicx}
\usepackage{moreverb}    % Defines {listing} environment.
\usepackage{amsmath, amsthm, amssymb, amsbsy, mathtools}
\usepackage{alltt}
\usepackage{rotating}
\usepackage[TABBOTCAP]{subfigure}
\usepackage{toc-bmad}
\usepackage{xspace}
%%\usepackage{makeidx}
\usepackage[section]{placeins}   % For preventing floats from floating to end of chapter.
\usepackage{longtable}  % For splitting long vertical tables into pieces
\usepackage{index}
\usepackage{multirow}
\usepackage{booktabs}   % For table layouts
\usepackage{yhmath}     % For widehat
\usepackage{xcolor}      % Needed for listings package.
\usepackage{listings}
\usepackage[T1]{fontenc}   % so _, <, and > print correctly in text.
\usepackage[strings]{underscore}    % to use "_" in text
\usepackage[pdftex,colorlinks=true]{hyperref}   % Must be last package!

%----------------------------------------------------------------

\newcommand{\sref}[1]{\S\ref{#1}}
\newcommand{\Sref}[1]{Sec.~\sref{#1}}

\newcommand{\vn}{\begingroup\catcode`\_=11 \catcode`\%=11 \dottcmd}
\newcommand\dottcmd[1]{{\usefont{T1}{lmss}{bx}{n} #1}\endgroup}

\newenvironment{example}
  {\vspace{-3.0ex} \begin{alltt}}
  {\end{alltt} \vspace{-2.5ex}}


\definecolor{light-gray}{gray}{0.95}
\lstset{backgroundcolor=\color{light-gray}}
\lstset{xleftmargin=0cm}
\lstset{framexleftmargin=0.3em}

\lstnewenvironment{Xcode}{}{}

\definecolor{lightcyan}{rgb}{0.88, 1.0, 1.0}
\newcounter{main}
\setcounter{main}{1}
\lstnewenvironment{code}[1][firstnumber=\themain,name=main]
  {\lstset{ %language=haskell,
           %columns=fullflexible,
           columns=fixed,
           basicstyle=\small\ttfamily,
           %numbers=left,
           numberstyle=\tiny\color{gray},
           backgroundcolor=\color{lightcyan},
           #1
          }
}
{\setcounter{main}{\value{lstnumber}}}



\newcommand{\BF}[1]{{\normalfont\textbf{#1}}}

\renewcommand{\ttdefault}{txtt}

\renewcommand{\textfraction}{0.1}
\renewcommand{\topfraction}{1.0}
\renewcommand{\bottomfraction}{1.0}

\settextfraction{0.9}  % Width of text
\setlength{\parindent}{0pt}
\setlength{\parskip}{1ex}
\newcommand{\Section}[1]{\section{#1}\vspace*{-1ex}}

\newenvironment{display}
  {\vspace*{-1.5ex} \begin{alltt}}
  {\end{alltt} \vspace*{-1.0ex}}

\title{Bmad and Tao Concepts}
\author{}
\date{David Sagan \\ June 24, 2016}
\begin{document}
\maketitle

\tableofcontents

%------------------------------------------------------------------------------
\Section{A Guide for the Perplexed}
\label{s:guide}

This is a guide to introduce the reader to some of the concepts that are used by the \bmad toolkit
for relativistic charged--particle and X-Ray simulations and as an initial training manual
for using the \tao simulation program.

%------------------------------------------------------------------------------
\Section{Overview: What is Bmad? What is Tao?}
\label{s:overview}

\bmad is an open-source subroutine library for relativistic
charged--particle and X-Ray simulations in accelerators and
storage rings. \bmad has been developed at Cornell University's
Laboratory for Elementary Particle Physics.
Over the years, \bmad has been used for a wide range of charged-particle and X-ray
simulations. The following list gives some idea as to the range that \bmad has been put to use
for:
\begin{display}
  Lattice design                              X-ray simulations
  Spin tracking                               Wakefields and HOMs
  Beam breakup simulations in ERLs            Touschek Simulations
  Intra-beam scattering (IBS) simulations     Dark current tracking
  Coherent Synchrotron Radiation (CSR)        Frequency map analysis
\end{display}

The advantage of \bmad over a stand-alone simulation program is that when new types of simulations
need to be developed, \bmad can be used to cut down on the time needed to develop such programs
with the added benefit that the number of programming errors will be reduced
as well. The disadvantage of \bmad is that, as a toolkit, one cannot proform any calculations
without first developing a program. To get around this, a program called \tao was developed.
\tao is a general purpose simulation program, based upon \bmad. \tao can be used to view
lattices, do Twiss and orbit calculations, nonlinear optimization on lattices, etc., etc.
Additionally, \tao's object oriented design makes it relatively easy to extend it. For
example, it can be used for orbit flattening in a machine control system.

%------------------------------------------------------------------------------
\Section{Resources}
\label{s:resources}

More information is readily available at the \bmad and \tao web site:
\begin{display}
  \url{http://www.lepp.cornell.edu/~dcs/bmad}
\end{display}
Links to the \bmad and \tao manuals can be found there as well as instructions for
downloading and setup if needed, etc.

%------------------------------------------------------------------------------
\Section{How to Use this Manual}

It is assumed that you are able to run \tao.

%------------------------------------------------------------------------------
\Section{Introduction to Bmad Lattices}

The basis of any \bmad based simulation is a lattice file. So the following is a
simple example:
\begin{code}
beginning[beta_a] = 10.   ! m  a-mode beta function
beginning[beta_b] = 10.   ! m  b-mode beta function
beginning[e_tot] = 10e6   ! eV   Or can set p0c

parameter[geometry] = open      ! or closed
parameter[particle] = electron  ! Default is positron

d: drift, L = 0.5
b: sbend, L = 0.5, g = 1    ! g = 1 / bending_radius
q: quadrupole, L = 0.6, k1 = 0.23

lat: line = (d, b, q)   ! List of lattice elements
use, lat                ! Line used to construct the lattice
\end{code}

%------------------------------------------------------------------------------
\Section{Introduction to Tao}

This section is a quick introduction to \tao. To run \tao, at minimum needs a lattice file as input.

\end{document}
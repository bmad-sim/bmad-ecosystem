\chapter{Wave Analysis}
\label{c:wave}


%----------------------------------------------------------------
\section{General Description}
\label{s:wave.general}

A ``wave analysis'' is method for finding isolated ``kick errors'' in
a machine by analyzing the appropriate data. Types of data that can be
analyzed and the associated error type is shown in
Table~\ref{t:wave0}.  

The analysis works on difference quantities. For example, the
difference between measurement and theory or the difference between
two measurements. Orbit and vertical dispersion measurements are the
exception here since an analysis of, say, just an orbit mesurement can
be considered to be the difference between the measurement and a
perfectly flat (zero) orbit.

\begin{table}[h]
\centering{\tt
\begin{tabular{|l|l|} \hline
  {\it Measurement Type}  & {\it Error Type}           \\ \hline
  Orbit                   & Steering errors            \\ \hline
  Betatron phase          & Quadrupolar errors         \\ \hline
  Beta function           & Quadrupolar errors         \\ \hline
  Coupling                & Skew quadrupolar errors    \\ \hline
  Dispersion              & Sextupole errors           \\ \hline
\end{tabular}}
\caption[Wave measurement types.]
{Types of measurements that can be used in a wave analysis and the 
types of errors that can be diagnosed.}
\label{t:wave0}
\end{table}

The formulation of the wave analysis for quadrupolar and skew
quadrupolar errors is presented by Sagan\cite{b:wave}.  Although not
discussed in the paper, the wave analysis for orbit and dispersion
measurements is similar to the beta function analysis that is
presented. 

The wave analysis is similar for all the measurement types. How the
wave analysis works is illustrated in Figure~\ref{f:wave0}.
Figure~\ref{f:wave0}a shows a simulated orbit for the Cesr Storage
Ring. The horizontal axis is the detector index. In this example, the
orbit represents the effect of two kicks generated by steerings near
detector 40 and detector 80.

For the wave analysis, two regions of the machine, labeled $A$ and $B$
in the figure, are chosen (more on this later). For each region in
turn, the data in that region is fit using a functional form that
assumes that there are no kick errors in the regions. For orbits, this
functional form is the standard equation
\Begineq
  x(s) = A \, \sqrt{\beta(s)} \, \sin(\phi(s) \, s + phi_0)
  \label{xabps}
\Endeq
where $\beta$ is the beta function and $\phi$ is the phase
advance. The quantities $A$ and $\phi_0$ are varied to give the best
fit.  Once $A$ and $\phi_0$ are fixed, \Eq{xabps} can be evaluated at
any point. Figure~\ref{f:wave0}b shows the orbit of \ref{f:wave0}a
with the fit to the $A$ region subtracted off. Similarly,
Figure~\ref{f:wave0}c shows the orbit of Figure~\ref{f:wave0}a with
the fit to the $B$ region subtracted off. Concentrating on
Figure~\ref{f:wave0}b, since there are no kick errors in the $A$
region, the fit is perfect and hence the difference between the data
and the fit is zero. Moving to the right from the $A$ region in
Figure~\ref{f:wave0}b, this difference is zero up to where the
assumption of no kick errors is violated. That is, at the location of
the kicker near detector 40. Similarly, since there are no kick errors
in region $B$, the difference between the data and the $B$ region fit
is zero in Figure~\ref{f:wave0}c and this remains true moving leftward
from region $B$ up to the kicker near detector 40.

By taking the fitted values for $A$ and $\phi_0$ for the regions $A$
and $B$, the point between the regions where the kick is generated and
the amplitude of the kick can be calculated. This calculation is
similar to that used to find quadrupolar errors from beta
data\cite{f:wave0}. The one difference is a factor of 2 that appears
in the beta calculation due to the fact that a freely propagating beta
wave oscillates at $2\phi(s)$. 

The success of the wave analysis in finding a kick error depends upon
whether there are regions of sufficient size on both sides of the kick
that are kick error free. That is, whether the kick error is
``isolated''. The locations of the $A$ and $B$ regions are set by the
user and the general strategy is to try to find, by varying the
location of the regions, locations where the data is well fit within
the regions. The data is well fit if the difference between data and
fit is small compared to the data itself. If there are multiple
isolated kick errors, then each error in turn can be bracketed and
analyzed. If there are multiple errors so close together that they
cannot be resolved, this will throw off the analysis, but it may still
be possible to give bounds for the location where the kicks are at and
an ``effective'' kick amplitude can be calculated.

For circular machines, to be able to analyze kicks near the beginning
or end of the lattice, the wave analysis can be done by ``wrapping''
the data past the end of the lattice for another 1/2 turn. This is
illustrated in Figure~\ref{f:wave0}. In the Cesr machine, there are
approximately 100 detectors labeled from 0 to 99.  The detectors from
100 to 150 are just the detectors from 0 to 50 shifted by 100. Thus,
for example, the detector labeled 132 in the figure is actually
detector 32.

%----------------------------------------------------------------
\section{Wave Analysis in Tao}
\label{s:wave.tao}

Proforming a wave analysis in \tao is a three step process:
\begin{example}
  1) Plot the data to be analyzed.
  2) Use the \vn{wave} command to select the data.
  3) Use the \vn{set wave} command to vary the fit regions.
\end{example}

In general, the accuracy of the wave analysis depends upon the
accuracy with which the beta function and phase advances are known in
the baseline lattice used. \tao uses the \vn{model} lattice for the
baseline. If possible, One strategy to improve the accuracy of the
wave analysis is first use a measurement to calculate what the
quadrupole strengths in the \vn{model} lattice should be. Possible
measurements that can give this information include an orbit response
matrix (ORM) analysis, fits to beta or betatron phase measurements, etc.

%----------------------------------------------------------------
\subsection{Prepairing the Data} 
\label{ss:wave.data}

At present (due to limited manpower to do the
coding), the wave analysis is restricted to data that is stored in a
d1_data array (\sref{c:data}). That is, the plotted curve to be
analyzed must have its \vn{data_type} parameter set to
\vn{'data_array'} (\sref{s:init.data}). The possible data types that
can be analyzed are:
\begin{example}
  orbit.x, orbit.y
  beta.a,  beta.b
  phase.a, phase.b
  eta.x, eta.y
  cbar.12
\end{example}
The curve to be analyzed must be visible. Any combination of data
components may be used:. "meas", "meas-ref", "model", etc.

If data from a circular machine is being analyzed, the data is wrapped
past the end of the lattice for another 1/2 turn. The translation from
the data index in the wrapped section to the first 1/2 section of the
lattice is determined by the values of \vn{ix_min_data} and
\vn{ix_max_data} of the d1_data array under consideration
(\sref{s:init.data}):
\begin{example}
  index_wrap \longrightarrow index_wrap - (ix_max_data - ix_min_data + 1)
\end{example}
For example, for the Cesr example in the previous section,
\vn{ix_min_data} was 0 and \vn{ix_max_data} was 99 to the translation
was
\begin{example}
  index_wrap \longringtarrow index_wrap - 100
\end{example}

%----------------------------------------------------------------
\subsection{Wave Analysis Commands}
\label{ss:wave.cmd}

The \vn{wave} command (\sref{s:wave}) sets which plotted data curve
is used for the wave analysis. The \vn{set wave} command (\sref{s:set}) 
is used for setting the $A$ and $B$ region locations. Finally the 
\vn{show wave} command (\sref{s:show}) prints analysis results. 

%----------------------------------------------------------------
\subsection{The Wave Analysis Output}
\label{ss:wave.cmd}


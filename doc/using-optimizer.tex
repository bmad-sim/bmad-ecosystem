%----------------------------------------------------------------
\chapter{Using the Optimizer for Lattice Correction and Design}
\index{optimization}
\label{c:optimizer}

\index{lm!optimizer}\index{de!optimizer}
There are three non-linear optimizers included with \tao: Two
optimizers are based on the Levenburg-Marquardt method. These are
referred to as `\vn{lm}', and \vn{lmdif}', the third optimizer is
based upon \vn{Differential Evolution} and is called `\vn{de}'. This example
will use the Levenburg-Marquardt optimizer which first uses steepest
decent to zero in on the region containing the minimum then uses the
inverse-Hessian to converge on the minimum. See Numerical Recipes in
Fortran (or C or C++) for a detailed explanation. There's no need to
know the details in order to use either optimizer. Once you set up the
problem \tao has the proper wrapper routines to do the
optimization. Of course, you are not limited to using the included
optimizers. Custom analysis can be done using custom routines but
these two optimizers have been integrated `out of the box' with the
\tao data and variable structures to make quick optimization possible.

Basically, the `\vn{lm}' is typically faster since it uses a Jacobian
or ``dmerit'' matrix to find the data derivatives versus each variable
before starting the optimization process.  However it assumes the
second derivative is fairly smooth, so for very complex function
spaces the `\vn{de}' may work better. But because `\vn{lm}' typically
converges much faster (for functions it can handle) it is recommended
to try this one first and only use `\vn{de}' if it fails.

%----------------------------------------------------------------
\section{Fix a Messed Up lattice}
\label{s:fix_it}

Let's mess the lattice up a little and see if the optimizer can
``fix'' the lattice. First transfer the ``correct'' or \vn{design}
lattice to the \vn{meas} data area.
\begin{example}
  set data *.*|meas = *.*|design
\end{example}
Now mess up the lattice a bit. We'll be messing with quadrupoles so
plot beta and phase.
\begin{example}
  place top beta
  place bottom phase
  plot * meas - model
  change var quad\_k1[10] 0.001
  change var quad\_k1[21] -0.001
  change var quad\_k1[67] -0.005
  scale
\end{example}
The lattice is now sufficiently screwed up.

Now specify what variables and data to use in the optimization. First type
\begin{example}
  show top10
\end{example}
to see what data is effecting the merit function the most. The merit
function is defined by
\Begineq
  {\cal M} \equiv \sum_{i} w_i \,
    \bigl[ \data_\model(i) -  \data_\meas(i) \bigr]^2 + 
  \sum_{j} w_j \,
    \bigl[ \var_\model(j) - \var_\meas(j) \bigr]^2
  \label{eq:merit}
\Endeq
where $w_{i}$ and $w_{j}$ are the weights given to each component.
The optimizer tries to minimize the merit function by changing the
model to look like the measured data. From the \vn{top10} output we
see that the beta function is effecting the merit function the
most. Since we are looking at beta and phase let's only use that data
in the optimization.
\begin{example}
  veto data *         ! Veto all the data
  use  data beta      ! Use all the beta data
  use  data phase     ! And use all the phase data
\end{example}
We also know that we need to change quadrupoles to correct the lattice.
\begin{example}
  veto var *           ! veto all the variables
  restore var quad\_k1 ! restore just the quad\_k1 variables
\end{example}
Note that we need to specify what data and variables we will be using
beforehand in the initialization files. This is already taken care of
in the demo initialization files. You can view these files to see how
the data and variables were initialized. Raw lattice elements cannot
be used by the included optimizer but there is no such restriction on
custom optimizers.

Now let's see if we have the optimizer set up correctly.
\begin{example}
  sho optimizer
\end{example}
Whoops! we want to use the Levenburg - Marquardt optimizer so
\begin{example}
  set global optimizer = lm
  show opti
\end{example}
The second command is short-hand. Most \tao commands can be shortened
to the least number of characters needed to distinguish the command
from all others.

Now we're ready to run the optimizer or ``fit'' the model to the
`measured' data.
\begin{example}
  run
\end{example}
You see the optimizer going through its cycles and it did it! The
model is now ``fitted.'' We can see what changes where done to the
quadrupoles by typing
\begin{example}
  sho var quad\_k1
\end{example}
The optimizer came very close to finding the ``design''
lattice. However, it changed more quadrupoles than just 10, 21 and
67. This isn't surprising. The optimizer finds the minimum of the
merit function and there are potentially many minima, or
degeneracies.  It does it's best not to get stuck in a local minimum
and as we can see by the plotted data, the minimum found is very close
-- virtually identical -- to the design lattice optics.  A good hint
as to what variables will be adjusted is the output of \cmd{show
top10}.  The top 3 derivatives were not the quadrupoles we
adjusted. Nevertheless, the final result was a darn near perfect
match!

%----------------------------------------------------------------
\section{Now Not Using all of the Variables}
\label{s:fix_it_not_all}

Alternatively, we could have used only a subset of the
quadrupoles. Say we know approximately which quadrupoles should be
adjusted. We can then specify these variables ranges.
\begin{example}
  change var quad\_k1[10]   0.001
  change var quad\_k1[21]  -0.001
  change var quad\_k1[67]  -0.005
  scale
  use var quad\_k1[8:12,20:25,65:70]
  run
  show constraints
  sho var quad\_k1[8:12,20:25,65:70]
\end{example}
Different quadrupoles than the ones we initially changed were still
adjusted by the optimizer. The end result is again very close to the
design lattice.

%----------------------------------------------------------------
\section{Lattice Design}
\label{s:lattice_design}

In \vn{lattice design} (\sref{s:lattice.design}) it is generally
easiest to specify constraint data using the ``\vn{single line}''
input format (\sref{s:init.data}). For example:
\begin{example}
&tao_d2_data
  d2_data%name = 'c1' 
  universe = '1'
  default_merit_type = "max"
  n_d1_data = 1
/

&tao_d1_data
  ix_d1_data = 1
  d1_data%name = 'xx'
  default_weight = 0.1
  ix_min_data = 1
  ix_max_data = 50
  data( 1) = 'beta.a'  'end_arc' 'end_lin3' 'max'    60   1.0   T
  data( 2) = 'beta.b'  'end_arc' 'end_lin3' 'max'    60   1.0   T
  data( 3) = 'beta.a'  ''        'end_lin3' 'max'    60   1.0   T
  data( 4) = 'beta.b'  ''        'end_lin3' 'max'    60   1.0   T
  data( 5) = 'alpha.a' ''        'end_lin3' 'max'   -0.6  1e3   T 
  data( 6) = 'alpha.b' ''        'end_lin3' 'max'   -0.6  1e3   T 
/
\end{example}


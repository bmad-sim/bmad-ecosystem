\chapter{Lattice Concepts and Structure}
\label{c:lat.concepts}

This chapter presents the basic concepts, such as \vn{element},
\vn{branch}, and \vn{lattice}, that \bmad uses to describe 
such things as LINACs, storage rings, X-ray beam lines, etc.
In fact, \bmad is capable of 
simulating a whole machine complex of 
interconnected parts: Including transfer lines connected to 
storage rings and storage rings interconnected together for
colliding beams.

\index{element}
The basic component \bmad uses to describe a machine is the
\vn{element} (also called a ``lattice element''). 
An element can be a physical thing like a bending magnet,
quadrupole or Bragg crystal, or something like a \vn{marker} 
element that is used
to mark a particular point in the machine. Chapter~\sref{c:elements}
lists the complete set of different element types that \bmad knows
about.

\index{branch}\index{end element}
\index{beginning element}\index{end element}
The next level up from an \vn{element} is the 
\vn{branch} (\sref{s:branching}).
A \vn{branch} can also be called a ``lattice branch'' to 
distinguish it from a \vn{branch} element.
A lattice \vn{branch} is just an ordered  sequence of elements that a 
particle will travel through. A branch can represent a LINAC, X-Ray
line, storage ring or anything else that can be represented as a 
simple ordered list of elements. All elements in a branch
are assigned a number
starting from zero. The 0\Th \vn{init_ele}
(\sref{s:init.ele}) element is automatically included
in every branch and is used as a marker for the beginning of the branch.
The 0\Th element is always named \vn{BEGINNING}. 
Additionally, every branch will,
by default have a final marker (\sref{s:mark}) element named \vn{END}. 

\index{lattice}
A \vn{lattice} is just an interconnected collection of branches. 
As such, a \vn{lattice} can describe an entire machine complex.


--> If there is only one branch then it can be ignored.


along with the attributes (length, strength, orientation,
etc.) of the elements.  A lattice file (or files) (\sref{c:lat.file}
is a file that is used to describe an accelerator or storage ring.



\index{MAD}
\index{lattice}
To \bmad, a ``lattice'' is the sequence of physical elements that is
to be studied. As explained below, the lattice is constructed in the
input lattice file using what are known as \vn{beam lines} and
\vn{replacement lists}. A simple example:
\begin{example}
  q1: quadrupole
  d1: drift
  my_machine: line = (q1, d1)
  use, my_machine
\end{example}
This lattice has two elements: A quadrupole \vn{q1} and a drift
\vn{d1}.  The line named \vn{my_machine} gives the ordered list of
elements and the \vn{use} statement chooses \vn{my_machine} as the
line to be used for the lattice.


If there are multiple branches (\sref{s:branching}) in the lattice,
beginning and end markers will be placed at the beginning and end of
all branches.

One is also allowed in \bmad to work with multiple lattices that can
be interconnected togeter using \vn{branching} (\sref{s:branching})
and multipass \vn{s:multipass} to form a description of an accelerator
complex.



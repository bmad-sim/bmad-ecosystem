\chapter{Using PTC/FPP}
\label{c:ptc.use}
\index{PTC/FPP}

%----------------------------------------------------------------------------

The PTC/FPP library of \'Etienne Forest handles Taylor maps to any arbitrary order. This is also
known as Truncated Power Series Algebra (TPSA). The core Differential Algebra (DA) package used by
PTC/FPP was developed by Martin Berz\cite{b:berz}. The PTC/FPP libraries are interfaced to \bmad so
that calculations that involve both \bmad and PTC/FPP can be done in a fairly seamless manner.

The FPP (``Fully Polymorphic Package'') part of the code handles Taylor map manipulation
and Lie algebraic operations. This is purely mathematical. FPP has no knowledge of accelerators,
magnetic fields, particle tracking etc. PTC (``Polymorphic Tracking Code'') implements the physics
and uses FPP to handle the Taylor map manipulation. Since the distinction between \vn{FPP} and
\vn{PTC} is irrelevant to the non-programmer, ``PTC'' will be used to refer to the entire package.

PTC is used by \bmad when constructing Taylor maps and when the \vn{tracking_method}
\sref{s:tkm}) is set to \vn{symp_lie_ptc}. All Taylor maps above first order are calculated
via PTC. No exceptions.

For the programmer, see Chapter~\sref{c:ptc.program} for more information.

%----------------------------------------------------------------------------
\section{PTC Tracking Versus Bmad Tracking}
\label{s:ptc.bmad.track}

While such things as magnet strengths will be the same, the model that PTC uses when it tracks
through a lattice element is independent of \bmad. That is, what approximations are made can be
different. Generally the agreement between PTC and \bmad tracking is quite good. But there are
situations where there is a noticeable difference. In such cases, one thing to do is to vary
parameters that affect PTC tracking. Two parameters that affect PTC accuracy are the integration
step size and the order of the integrator. The step size is set by each lattice element's \vn{ds_step}
parameter (\sref{s:ds.step}). 

%----------------------------------------------------------------------------
\section{PTC / Bmad Interfacing}
\label{s:ptc.interface}

\index{PTC!single element mode}
\bmad interfaces to PTC in two ways: One way, called ``single element'' mode, uses PTC on a per
element basis. In this case, the method used for tracking a given element can be selected on an
element-by-element basis so non-PTC tracking methods can be mixed with PTC tracking methods to
optimize speed and accuracy. [PTC tends to be accurate but slow.] The advantage of single element
mode is the flexibility it affords. The disadvantage is that it precludes using PTC's analysis tools
which rely on the entire lattice being tracked via PTC. Such tools include normal form analysis beam
envelope tracking, etc.

\index{PTC!whole lattice mode}
The alternative to single element mode is ``whole lattice'' mode where a series of PTC \vn{layout}s
(equivalent to a \bmad branch) are created from a \bmad lattice. Whether single element or whole
lattice mode (or both) is used is determined by the program being run.


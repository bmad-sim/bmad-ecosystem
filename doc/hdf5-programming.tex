\chapter{HDF5}
\label{c:hdf5}
\index{hdf5}

\vn{HDF5}, which stands for ``Hierarchical Data Format'' version 5\cite{b:hdf5}, is a set of file
formats designed to store and organize large amounts of data. HDF5 has been developed by scientists
from a number of institutions including the National Center for Supercomputing Applications, the
University of Illinois at Urbana-Champaign, and Sandia National Laboratories. Tools for viewing and
editing HDF5 files are available from the HDF Group\cite{b:hdf5}. Programs include \vn{h5dump} and
\vn{HDFView} which can be used to directly view files. Interfaces so that HDF5 files can accessed
via Java or Python also exist.

\bmad uses HDF5 for storing beam particle (positions, spin, etc.) and \vn{grid_field}
(\sref{s:grid.field} data. Storage details are given in sections \sref{s:hdf5.beam} and
\sref{s:hdf5.grid} respectively. While \vn{HDF5} defines how data is formatted, \vn{HDF5} does not
define the syntax for how data is to be stored. For that, \bmad uses the syntax defined by the
\vn{Beam Physics} extension to the \vn{openPMD} standard\cite{b:openpmd}. To understand the rest of
this chapter, the reader should familiarize themselves with the \vn{openPMD} and \vn{Beam Physics}
standards.

%-----------------------------------------------------------------
\section{HDF5 Particle Beam Data Storage}
\label{s:hdf5.beam}
\index{hdf5 and particle beam data}

As per the \vn{openPMD}/\vn{Beam Physics} standard, particle beam data is stored in a tree structure
within a data file. The root ``\vn{group}'' (tree node) for each bunch of the beam has the path
within the file:
\begin{example}
  /data/%T/particles/
\end{example}
where \vn{%T} is an integer starting from ``1''.

Bunch parameters (``attributes'') stored in the root groop are:
\begin{example}
  speciesType     ! The name of the particle species using the \vn{SpeciesType} syntax.
  totalCharge     ! Total bunch charge.
  chargeLive      ! Charge of live particles.
  numParticles    ! Number of particles.
\end{example}

What per-particle data is stored is determined by whether the bunch particles are photons or
not. The following particle parameters are common for both types:
\begin{table}[h]
\centering
{\tt
\begin{tabular}{lll} \toprule
  {\em Beam Physics Parameter} & {\em Bmad Equivalent} & {\em Notes}                      \\ \midrule
  time                         & -\%z / (c \%beta)     & time - ref_time. See \Eq{zbctt}  \\
  timeOffset                   & \%t - time            & reference time                   \\
  totalMomentumOffset          & \%p0c                 &                                  \\
  sPosition                    & \%s                   & See Fig.~\ref{f:local.coords}    \\
  weight                       & \%charge              & Macro bunch charge               \\
  branchIndex                  & \%ix_branch           &                                  \\
  elementIndex                 & \%ix_ele              &                                  \\
  locationInElement            & See below             &                                  \\
  particleStatus               & \%state               & See the \%state table in \sref{s:coord.struct} \\ \bottomrule
\end{tabular}
}
\end{table}
The \vn{formula} column gives the conversion between the Beam Physics Parameters and the
\vn{coord_struct} (\sref{s:coord.struct}) structure components that contains the particle position
information.  parameters with a ``\%'' suffix are \vn{coord_struct} components. For clarity's sake,
the \vn{%vec(1)} through \vn{%vec(6)} phase space coordinate components has been replaced by
\vn{%x}, \vn{%px}, $\ldots$, \vn{%z}, \vn{%pz}.

The \vn{locaionInElement} Beam Physics parameter is related to the \vn{%location} parameter of the
\vn{coord_struct} using the following transformation:
\begin{table}[h]
\centering
{\tt
\begin{tabular}{ll} \toprule
  {\em locationInElement Value} & {\em \%location Value}  \\ \midrule
  -1                            & upstream_end\$          \\
   0                            & inside\$                \\
   1                            & downstream_end\$        \\ \bottomrule
\end{tabular}
}
\end{table}

For photons, additional per-particle data is:
\\
\begin{table}[h]
\centering
{\tt
\begin{tabular}{ll} \toprule
  {\em Beam Physics Parameter}     & {\em Bmad Equivalent}  \\ \midrule
  velocity/x, y, z                 & (\%vx, \%vy, \%vz)     \\
  position/x, y, z                 & (\%x, \%y, \%z)        \\
  pathLength                       & \%path_len             \\
  photonPolarizationAmplitude/x, y & \%field                \\
  photonPolarizationPhase/x, y     & \%phase                \\ \bottomrule
\end{tabular}
}
\end{table}
Here \vn{%vx}, \vn{%vy}, and \vn{%vz} represent \vn{%vec(2)}, \vn{%vec(4)}, \vn{%vec(6)}

For non-photons, additional per-particle data is:
\begin{table}[h]
\centering
{\tt
\begin{tabular}{ll} \toprule
  {\em Beam Physics Parameter}    & {\em Bmad Equivalent}     \\ \midrule
  momentum/x, y, z                & \%p0c (\%px, \%py, sqrt((1 + \%pz)$^2$ - \%px$^2$ - \%py$^2$)) \\
  totalMomentum                   & \%p0c \%pz                \\
  position/x, y, z                & (\%x, \%y, 0)             \\
  spin/x, y, z                    & \%spin                    \\
  chargeState                     & Derived from \%species    \\ \bottomrule
\end{tabular}
}
\end{table}
Notice that the Beam Physics \vn{z} position (not to be confused with phase space \vn{z}) is always zero
by construction as shown in Fig.~\ref{f:local.coords}.

%-----------------------------------------------------------------
\section{HDF5 Grid\_Field Data Storage}
\label{s:hdf5.grid}
\index{hdf5 and grid_field data}


\chapter{GUI Installation}
\label{s:gui.install}

%-----------------------------------------------------------------
\section{Obtaining the Source Code}

Source code for both \tao and the GUI 

%-----------------------------------------------------------------
\section{Requirements}

There are a number of python modules needed for the GUI:
\begin{example}
  tkinter
  ttk (may be called pyttk)
  pexpect         # If using pexpect instead of ctypes.
  matplotlib, cycler, dateutil, tkagg
\end{example}
Note: The GUI uses the TkAgg backend for matplotlib. There may be a problem with Python finding the
TkAgg backend. On the mac, using macports, the solution is to install matplotlib with the
\vn{tkinter} variant. Something like:
\begin{example}
  sudo port uninstall py36-matplotlib           # May not be needed.
  sudo port install  py36-matplotlib +tkinter   # This is when using Python version 3.6
\end{example}
For more information see:
\begin{example}
  https://matplotlib.org/tutorials/introductory/usage.html#backends
\end{example}

If one of the modules is missing, python will generate an error message. For example:
\begin{example}
> python ../../gui/main.py
Exception in Tkinter callback
Traceback (most recent call last):
  File "/opt/local/Library/Frameworks/Python.framework/Versions/3.7/lib/python3.7/tkinter/__init__.py", line 1705, in __call__
    return self.func(*args)
  File "../../gui/main.py", line 372, in param_load
    from tao_interface import tao_interface
  File "/Users/dcs16/Bmad/bmad_dist/tao/gui/tao_interface.py", line 4, in <module>
    from tao_pipe import tao_io
  File "/Users/dcs16/Bmad/bmad_dist/tao/python/tao_pexpect/tao_pipe.py", line 14, in <module>
    import pexpect
ModuleNotFoundError: No module named 'pexpect'
\end{example}
Notice that the last line shows that the pexpect module is needed.

How to install missing modules on the mac: [Note: The exact installation commands will depend upon
which version of python is being used. Use the "python --version" command to see what version you
are using.

Using macports and python 3.6:
\begin{example}
  sudo port install py36-tkinter
  sudo port install py36-pexpect
\end{example}

Using pip (or pip3):
\begin{example}
  sudo pip install pytkk
  sudo pip install pexpect
\end{example}

WARNING: it can be dangerous to use pip to install/modify modules in your system python.
A much safer way to install the modules you need is to set up a python virtual environment.
On Linux, you may also be able to find versions of the required modules in your system package manager,
which are tailored to your Linux distribution and will not break your system python.

%-----------------------------------------------------------------
\section{Installation Troubleshooting}
\label{s:gui.trouble}

Got error:
\begin{example}
  ImportError: cannot import name ‘_tkagg'
\end{example}

Solution: Uninstall and then reinstall matplotlib. For example, if using pip:
\begin{example}
  sudo pip uninstall matplotlib
  sudo pip install matplotlib
\end{example}

%-----------------------------------------------------------------
\section{Developer Setup}
\label{s:gui.develop}

For GUI development, it may be desireable to specify a local build tree as the place for the python
scripts to find the \tao executable or \tao shared object library. To accomplish this, set the environmental
variable ACC_LOCAL_ROOT to the base directory of your local build tree.
\begin{example}
  export ACC_LOCAL_ROOT=/home/dcs16/bmad_dist
\end{example}

The standard place for the GUI script files is at:
\begin{example}
  "${ACC_ROOT_DIR}/tao/python/pytao/gui
\end{example}
When doing GUI development work, the default location can be changed by setting \vn{ACC_PYTHONPATH}. Example:
\begin{example}
  export ACC_PYTHONPATH=/home/dcs16/bmad_dist/tao/python/pytao/gui
\end{example}
\vn{ACC_PYTHONPATH} must be set before \bmad is initalized. That is, if \bmad is initalized in the \vn{.bashrc} file,
\vn{ACC_PYTHONPATH} must be initalized in the \vn{.bashrc} file before the \bmad initialization.

To check that \vn{PYTHONPATH} has the correct value use the command:
\begin{example}
  printenv |grep PYTHONPATH
\end{example}


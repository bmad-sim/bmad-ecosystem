\documentclass[11pt]{article}
\usepackage{tocloft}
\usepackage{geometry}            % See geometry.pdf to learn the layout options. There are lots.
\usepackage{xspace}
\geometry{letterpaper}           % ... or a4paper or a5paper or ... 
%\usepackage[parfill]{parskip}   % To begin paragraphs with an empty line rather than an indent
\usepackage{graphicx}
\usepackage{amssymb}
\usepackage{amsmath}
\usepackage{alltt}
\usepackage[T1]{fontenc}   % so _, <, and > print correctly in text.
\usepackage[strings]{underscore}    % to use "_" in text
\usepackage[pdftex,colorlinks=true]{hyperref}

%---------------------------------------------------------------------------------

\newcommand{\sref}[1]{$\S$\ref{#1}}
\newcommand{\srthree}{\texttt{Synrad3D}\xspace}
\newcommand\ttcmd{\begingroup\catcode`\_=11 \catcode`\%=11 \dottcmd}
\newcommand\dottcmd[1]{\texttt{#1}\endgroup}
\newcommand{\Begineq}{\begin{equation}}
\newcommand{\Endeq}{\end{equation}}
\newcommand{\fig}[1]{Figure~\ref{#1}}
\newcommand{\vn}{\ttcmd}           
\newcommand{\Th}{$^{th}$\xspace}
\newcommand{\Newline}{\hfil \\}


\newcommand{\bearray}{\begin{eqnarray}}
\newcommand{\eearray}{\end{eqnarray}}
\newcommand{\be}{\begin{equation}}
\newcommand{\ee}{\end{equation}}
\newcommand{\bearraynn}{\begin{eqnarray*}}
\newcommand{\eearraynn}{\end{eqnarray*}}
\newcommand{\benn}{\begin{displaymath}}
\newcommand{\eenn}{\end{displaymath}}
\newcommand{\eq}[1]{{Eq.~(\ref{#1})}}
\newcommand{\eqs}[2]{{Eqs.~(\ref{#1}--\ref{#2})}}

\newlength{\dPar}
\newlength{\ExBeg}
\newlength{\ExEnd}
\setlength{\dPar}{1.5ex}
\setlength{\ExBeg}{-\dPar}
\addtolength{\ExBeg}{-0.5ex}
\setlength{\ExEnd}{-\dPar}
\addtolength{\ExEnd}{-0.0ex}

\newenvironment{example}
  {\vspace{\ExBeg} \begin{alltt}}
  {\end{alltt} \vspace{\ExEnd}}

%---------------------------------------------------------------------------------

\setlength{\textwidth}{6.25in}
\setlength{\hoffset}{0.0in}
\setlength{\oddsidemargin}{0.25in}
\setlength{\evensidemargin}{0.0in}
\setlength{\textheight}{8.5in}
\setlength{\topmargin}{0in}

\setlength{\parskip}{\dPar}
\setlength{\parindent}{0ex}

\setlength\cftparskip{0pt}
\setlength\cftbeforesecskip{3pt}
\setlength\cftaftertoctitleskip{15pt}

%---------------------------------------------------------------------------------

\title{ibs_ring Simulation Program}
\author{Michael Ehrlichman, D. Sagan}
\date{February 28, 2014}

%---------------------------------------------------------------------------------

\begin{document}
\maketitle

\pdfbookmark[1]{Contents}{Contents}
\tableofcontents

%------------------------------------------------------------------
\section{Introduction} 
\label{s:intro}

\vn{ibs_ring} is a program for simulating intra-beam scattering (IBS).
The source code for this program lives in the \vn{bsim} directory in
the standard Bmad\cite{b:bmad} distribution.

%------------------------------------------------------------------
\section{Running ibs_ring} 
\label{s:run}

The input for the \vn{ibs_ring} program uses Fortran90 namelist
syntax: The data begins with the string \vn{\&parameters} and ends
with a slash \vn{/}. Everything outside this is ignored.
The input parameters are:
\begin{example}
\&parameters
  lat_file = <lattice-file-name>
  ibs_formula = <type>  ! 'cimp', 'bjmt', 'bane', 'mpzt', 'mpxx', or 'kubo'
  eqb_method = <type>   ! 'der' or 'rlx'
  clog_to_use = <int>   ! 1=no tail cut, 2=raubenheimer, 3=kubo, 4=kubo w/vertical
  inductance = <real>
  resistance = <real>
  set_dispersion = <logical> ! applied only for kubo method.
  eta_set = <real>
  etap_set = <real>
  b_emit = <real>
  a_emit = <real>
  energy_spread = <real>
  ratio = <real>
  granularity = <real>   ! -1 for element-by-element.
  x_view = <int>  ! element lattice index 
  y_view = <int>  
  z_view = <int>  
  mA_per_bunch = <real>  ! highest current
  stop_mA = <real>       ! lowest current
  delta_mA = <real>      ! step size
/
\end{example}

  \begin{description}
  \item[\vn{lat_file}] \Newline
Bmad lattice file describing the ring.

  %------------------------------
  \item[\vn{ibs_formula}] \Newline
Name of the algorithm to use for the calculation. Possibilities are:
\begin{example}
  'cimp'
  'bjmt'
  'bane'
  'mpzt'
  'mpxx'
  'kubo'
\end{example}

  %------------------------------
  \item[\vn{eqb_method}] \Newline
Method used for finding the equilibrium solution. Possibilities are:
\begin{example}
  'der'
  'rlx'
\end{example}
'der' finds the equilibrium emittances using differential equations.
The differential is with respect to time.

'rlx finds the equilibrium emittances by iterating to find the
solution to an analytic form for the equilibrium emittance.  The 'rlx'
method includes the controversial 'coupling parameter' which says that
the vertical emittance has contributions from both the vertical and
horizontal IBS rates and damping rates.

In the limit that the coupling parameter is zero, 'der' and 'rlx' are
equivalent.  In fact, the equations used for 'rlx' are the equilibrium
solution to the 'der' method's differential equations.

'der' is much faster and robust.  However, it does not allow for the
'coupling parameter', which has its uses.  Also, the two different
methods can be useful in diagnosing the code.

  %------------------------------
  \item[\vn{clog_to_use}] \Newline
Logarithmic cutoff to use. Possibilities are:
\begin{example}
  1  ! Classic, no tail cut.  
  2  ! Raubenheimer.  
  3  ! Oide
  4  ! Bane.  
\end{example} 

  %------------------------------
  \item[\vn{inductance}] \Newline
Longitudinal inductance for PWD calc.
Effects bunch length vs. current.

  %------------------------------
  \item[\vn{resistance}] \Newline
Resistive inductance for PWD calc.
Currently not used. 

  %------------------------------
  \item[\vn{set_dispersion}] \Newline
If true, then apply eta_set and etap_set.
If false, then do not.

  %------------------------------
  \item[\vn{eta_set}] \Newline
Used only if ibs_formula set to 'kubo'.
Applies x-pz coupling to each
element of lattice when calculating IBS rates.

  %------------------------------
  \item[\vn{etap_set}] \Newline
Used only if ibs_formula set to 'kubo'. Applies px-pz coupling to
each element of lattice when calculating IBS rates.

  %------------------------------
  \item[\vn{a_emit}] \Newline
Zero current horizontal emittance.
Set to -1 for rad int calc.

  %------------------------------
  \item[\vn{b_emit}] \Newline
Zero current vertical emittance.
Set to -1 for rad int calc.

  %------------------------------
  \item[\vn{energy_spread}] \Newline
Zero current energy spread.
Set to -1 for rad int calc.

  %------------------------------
  \item[\vn{ratio}] \Newline
"Coupling parameter r" hack for including coupling.

  %------------------------------
  \item[\vn{granularity}] \Newline
Step size along lattice in meters.
Set to -1 for element-by-element.

  %------------------------------
  \item[\vn{x_view}] \Newline
Index of element where projection is taken for horizontal beam size
calculation.

  %------------------------------
  \item[\vn{y_view}] \Newline
Index of element where projection is taken for vertical beam size
calculation.

  %------------------------------
  \item[\vn{z_view}] \Newline
Index of element where projection is taken for longitudinal beam size
calculation.

  %------------------------------
  \item[\vn{mA_per_bunch}] \Newline
Largest current per bunch in mA.

  %------------------------------
  \item[\vn{stop_mA}] \Newline
Smallest current per bunch in mA.

  %------------------------------
  \item[\vn{delta_mA}] \Newline
mA step size.

  \end{description}


%------------------------------------------------------------------
\begin{thebibliography}{9}

\bibitem{b:bmad}
D. Sagan,
"Bmad: A Relativistic Charged Particle Simulation Library"
Nuc.\ Instrum.\ \& Methods Phys.\ Res.\ A, {\bf 558}, pp 356-59 (2006).
The Bmad web site:
\hfill\break
\hspace*{0.3in} \url{http://www.lepp.cornell.edu/~dcs/bmad}

\end{thebibliography}
\end{document}  

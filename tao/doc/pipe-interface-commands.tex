% WARNING: this is automatically generated. DO NOT EDIT.

%% pipe beam ------------------------------------
\subsection{pipe beam}
\index{pipe!beam}
\label{p:beam}


Output beam parameters that are not in the beam_init structure.

\begin{example}
   pipe beam \{ix_uni\}@\{ix_branch\}
\end{example}
\begin{verbatim}
Where:
  {ix_uni} is a universe index. Defaults to s%global%default_universe.
  {ix_branch} is a lattice branch index. Defaults to s%global%default_branch.

Note: To set beam_init parameters use the "set beam" command.
\end{verbatim}

%% pipe beam_init ------------------------------------
\subsection{pipe beam_init}
\index{pipe!beam_init}
\label{p:beam.init}


Output beam_init parameters.

\begin{example}
   pipe beam_init \{ix_uni\}@\{ix_branch\}
\end{example}
\begin{verbatim}
Where:
  {ix_uni} is a universe index. Defaults to s%global%default_universe.
  {ix_branch} is a lattice branch index. Defaults to s%global%default_branch.

Note: To set beam_init parameters use the "set beam_init" command
\end{verbatim}

%% pipe bmad_com ------------------------------------
\subsection{pipe bmad_com}
\index{pipe!bmad_com}
\label{p:bmad.com}


Output bmad_com structure components.

\begin{example}
   pipe bmad_com
\end{example}
\begin{verbatim}

\end{verbatim}

%% pipe branch1 ------------------------------------
\subsection{pipe branch1}
\index{pipe!branch1}
\label{p:branch1}


Output lattice branch information for a particular lattice branch.

\begin{example}
   pipe branch1 \{ix_uni\}@\{ix_branch\}
\end{example}
\begin{verbatim}
Where:
  {ix_uni} is a universe index. Defaults to s%global%default_universe.
  {ix_branch} is a lattice branch index. Defaults to s%global%default_branch.
\end{verbatim}

%% pipe bunch_comb ------------------------------------
\subsection{pipe bunch_comb}
\index{pipe!bunch_comb}
\label{p:bunch.comb}


Outputs bunch parameters at a comb point. 
Also see the "write bunch_comb" and "show bunch -comb" commands.

\begin{example}
   pipe bunch_comb \{flags\} \{who\} \{ix_uni\}@\{ix_branch\} \{ix_bunch\}
\end{example}
\begin{verbatim}
Where:
  {flags} are optional switches:
      -array_out : If present, the output will be available in the 
             tao_c_interface_com%c_real array.
  {ix_uni} is a universe index. Defaults to s%global%default_universe.
  {ix_branch} is a branch index. Defaults to s%global%default_branch.
  {ix_bunch} is the bunch index. Defaults to 1.
  {who} is one of:
      x, px, y, py, z, pz, t, s, spin.x, spin.y, spin.z, p0c, beta     -- centroid 
      x.Q, y.Q, z.Q, a.Q, b.Q, c.Q where Q is one of: beta, alpha, gamma, phi, 
                                      eta, etap, sigma, sigma_p, emit, norm_emit
    sigma.IJ where I, J in range [1,6]
    rel_min.I, rel_max.I where I in range [1,6]
    charge_live, n_particle_live, n_particle_lost_in_ele, ix_ele

  Note: If ix_uni or ix_branch is present, "@" must be present.

Example:
  pipe bunch_comb py 2@1 1
\end{verbatim}

%% pipe bunch_params ------------------------------------
\subsection{pipe bunch_params}
\index{pipe!bunch_params}
\label{p:bunch.params}


Outputs bunch parameters at the exit end of a given lattice element.

\begin{example}
   pipe bunch_params \{ele_id\}|\{which\}
\end{example}
\begin{verbatim}
Where:
  {ele_id} is an element name or index.
  {which} is one of: "model", "base" or "design"

Example:
  pipe bunch_params end|model  ! parameters at model lattice element named "end".
\end{verbatim}

%% pipe bunch1 ------------------------------------
\subsection{pipe bunch1}
\index{pipe!bunch1}
\label{p:bunch1}


Outputs Bunch parameters at the exit end of a given lattice element.

\begin{example}
   pipe bunch1 \{ele_id\}|\{which\} \{ix_bunch\} \{coordinate\}
\end{example}
\begin{verbatim}
Where:
  {ele_id} is an element name or index.
  {which} is one of: "model", "base" or "design"
  {ix_bunch} is the bunch index.
  {coordinate} is one of: x, px, y, py, z, pz, "s", "t", "charge", "p0c", 
                                                                "state", "ix_ele"

For example, if {coordinate} = "px", the phase space px coordinate of each particle
of the bunch is displayed. The "state" of a particle is an integer. 
A value of 1 means alive and any other value means the particle has been lost.
\end{verbatim}

%% pipe building_wall_list ------------------------------------
\subsection{pipe building_wall_list}
\index{pipe!building_wall_list}
\label{p:building.wall.list}


Output List of building wall sections or section points

\begin{example}
   pipe building_wall_list \{ix_section\}
\end{example}
\begin{verbatim}
Where:
  {ix_section} is a building wall section index.

If {ix_section} is not present, a list of building wall sections is given.
If {ix_section} is present, a list of section points is given.
\end{verbatim}

%% pipe building_wall_graph ------------------------------------
\subsection{pipe building_wall_graph}
\index{pipe!building_wall_graph}
\label{p:building.wall.graph}


Output (x, y) points for drawing the building wall for a particular graph.

\begin{example}
   pipe building_wall_graph \{graph\}
\end{example}
\begin{verbatim}
Where:
  {graph} is a plot region graph name.

Note: The graph defines the coordinate system for the (x, y) points.
\end{verbatim}

%% pipe building_wall_point ------------------------------------
\subsection{pipe building_wall_point}
\index{pipe!building_wall_point}
\label{p:building.wall.point}


add or delete a building wall point

\begin{example}
   pipe building_wall_point \{ix_section\}^^\{ix_point\}^^\{z\}^^\{x\}^^\{radius\}^^
                                                               \{z_center\}^^\{x_center\}
\end{example}
\begin{verbatim}
Where:
  {ix_section}    -- Section index.
  {ix_point}      -- Point index. Points of higher indexes will be moved up 
                       if adding a point and down if deleting.
  {z}, etc...     -- See tao_building_wall_point_struct components.
                  -- If {z} is set to "delete" then delete the point.
\end{verbatim}

%% pipe building_wall_section ------------------------------------
\subsection{pipe building_wall_section}
\index{pipe!building_wall_section}
\label{p:building.wall.section}


Add or delete a building wall section

\begin{example}
   pipe building_wall_section \{ix_section\}^^\{sec_name\}^^\{sec_constraint\}
\end{example}
\begin{verbatim}
Where:
  {ix_section}      -- Section index. Sections with higher indexes will be
                         moved up if adding a section and down if deleting.
  {sec_name}        -- Section name.
  {sec_constraint}  -- A section constraint name or "delete". Must be one of:
      delete          -- Delete section. Anything else will add the section.
      none
      left_side
      right_side
\end{verbatim}

%% pipe constraints ------------------------------------
\subsection{pipe constraints}
\index{pipe!constraints}
\label{p:constraints}


Output optimization data and variable parameters that contribute to the merit function.

\begin{example}
   pipe constraints \{who\}
\end{example}
\begin{verbatim}
Where:
  {who} is one of: "data" or "var"

Data constraints output is:
  data name
  constraint type
  evaluation element name
  start element name
  end/reference element name
  measured value
  ref value (only relavent if global%opt_with_ref = T)
  model value
  base value (only relavent if global%opt_with_base = T)
  weight
  merit value
  location where merit is evaluated (if there is a range)
Var constraints output is:
  var name
  Associated varible attribute
  meas value
  ref value (only relavent if global%opt_with_ref = T)
  model value
  base value (only relavent if global%opt_with_base = T)
  weight
  merit value
  dmerit/dvar
\end{verbatim}

%% pipe da_aperture ------------------------------------
\subsection{pipe da_aperture}
\index{pipe!da_aperture}
\label{p:da.aperture}


Output dynamic aperture data

\begin{example}
   pipe da_aperture \{ix_uni\}
\end{example}
\begin{verbatim}
Where:
  {ix_uni} is a universe index. Defaults to s%global%default_universe.
\end{verbatim}

%% pipe da_params ------------------------------------
\subsection{pipe da_params}
\index{pipe!da_params}
\label{p:da.params}


Output dynamic aperture input parameters

\begin{example}
   pipe da_params \{ix_uni\}
\end{example}
\begin{verbatim}
Where:
  {ix_uni} is a universe index. Defaults to s%global%default_universe.
\end{verbatim}

%% pipe data ------------------------------------
\subsection{pipe data}
\index{pipe!data}
\label{p:data}


Output Individual datum parameters.

\begin{example}
   pipe data \{ix_uni\}@\{d2_name\}.\{d1_name\}[\{dat_index\}]
\end{example}
\begin{verbatim}
Where:
  {ix_uni} is a universe index. Defaults to s%global%default_universe.
  {d2_name} is the name of the d2_data structure the datum is in.
  {d1_datum} is the name of the d1_data structure the datum is in.
  {dat_index} is the index of the datum.

Use the "pipe data-d1" command to get detailed info on a specific d1 array.

Example:
  pipe data 1@orbit.x[10]
\end{verbatim}

%% pipe data_d_array ------------------------------------
\subsection{pipe data_d_array}
\index{pipe!data_d_array}
\label{p:data.d.array}


Output list of datums for a given d1_data structure.

\begin{example}
   pipe data_d_array \{ix_uni\}@\{d2_name\}.\{d1_name\}
\end{example}
\begin{verbatim}
Where:
  {ix_uni} is a universe index. Defaults to s%global%default_universe.
  {d2_name} is the name of the containing d2_data structure.
  {d1_name} is the name of the d1_data structure containing the array of datums.

Example:
  pipe data_d_array 1@orbit.x
\end{verbatim}

%% pipe data_d1_array ------------------------------------
\subsection{pipe data_d1_array}
\index{pipe!data_d1_array}
\label{p:data.d1.array}


Output list of d1 arrays for a given data_d2.

\begin{example}
   pipe data_d1_array \{d2_datum\}
\end{example}
\begin{verbatim}
{d2_datum} should be of the form
  {ix_uni}@{d2_datum_name}
\end{verbatim}

%% pipe data_d2 ------------------------------------
\subsection{pipe data_d2}
\index{pipe!data_d2}
\label{p:data.d2}


Output information on a d2_datum.

\begin{example}
   pipe data_d2 \{ix_uni\}@\{d2_name\}
\end{example}
\begin{verbatim}
Where:
  {ix_uni} is a universe index. Defaults to s%global%default_universe.
  {d2_name} is the name of the d2_data structure.
\end{verbatim}

%% pipe data_d2_array ------------------------------------
\subsection{pipe data_d2_array}
\index{pipe!data_d2_array}
\label{p:data.d2.array}


Output data d2 info for a given universe.

\begin{example}
   pipe data_d2_array \{ix_uni\}
\end{example}
\begin{verbatim}
Where:
  {ix_uni} is a universe index. Defaults to s%global%default_universe.

Example:
  pipe data_d2_array 1
\end{verbatim}

%% pipe data_d2_create ------------------------------------
\subsection{pipe data_d2_create}
\index{pipe!data_d2_create}
\label{p:data.d2.create}


Create a d2 data structure along with associated d1 and data arrays.

\begin{example}
   pipe data_d2_create \{ix_uni\}@\{d2_name\}^^\{n_d1_data\}^^\{d_data_arrays_name_min_max\}
\end{example}
\begin{verbatim}
Where:
  {ix_uni} is a universe index. Defaults to s%global%default_universe.
  {d2_name} is the name of the d2_data structure to create.
  {n_d1_data} is the number of associated d1 data structures.
  {d_data_arrays_name_min_max} has the form
    {name1}^^{lower_bound1}^^{upper_bound1}^^....
                                           ^^{nameN}^^{lower_boundN}^^{upper_boundN}
  where {name} is the data array name and 
  {lower_bound} and {upper_bound} are the bounds of the array.

Example:
  pipe data_d2_create 2@orbit^^2^^x^^0^^45^^y^^1^^47
This example creates a d2 data structure called "orbit" with 
two d1 structures called "x" and "y".

The "x" d1 structure has an associated data array with indexes in the range [0, 45].
The "y" d1 structure has an associated data arrray with indexes in the range [1, 47].

Use the "set data" command to set created datum parameters.

Note: When setting multiple data parameters, 
      temporarily toggle s%global%lattice_calc_on to False
  ("set global lattice_calc_on = F") to prevent Tao trying to 
      evaluate the partially created datum and generating unwanted error messages.
\end{verbatim}

%% pipe data_d2_destroy ------------------------------------
\subsection{pipe data_d2_destroy}
\index{pipe!data_d2_destroy}
\label{p:data.d2.destroy}


Destroy a d2 data structure along with associated d1 and data arrays.

\begin{example}
   pipe data_d2_destroy \{ix_uni\}@\{d2_name\}
\end{example}
\begin{verbatim}
Where:
  {ix_uni} is a universe index. Defaults to s%global%default_universe.
  {d2_name} is the name of the d2_data structure to destroy.

Example:
  pipe data_d2_destroy 2@orbit
This destroys the orbit d2_data structure in universe 2.
\end{verbatim}

%% pipe data_parameter ------------------------------------
\subsection{pipe data_parameter}
\index{pipe!data_parameter}
\label{p:data.parameter}


Output an array of values for a particular datum parameter for a given array of datums, 

\begin{example}
   pipe data_parameter \{data_array\} \{parameter\}
\end{example}
\begin{verbatim}
{parameter} may be any tao_data_struct parameter.
Example:
  pipe data_parameter orbit.x model_value
\end{verbatim}

%% pipe data_set_design_value ------------------------------------
\subsection{pipe data_set_design_value}
\index{pipe!data_set_design_value}
\label{p:data.set.design.value}


Set the design (and base \& model) values for all datums.

\begin{example}
   pipe data_set_design_value
\end{example}
\begin{verbatim}
Example:
  pipe data_set_design_value

Note: Use the "data_d2_create" and "datum_create" first to create datums.
\end{verbatim}

%% pipe datum_create ------------------------------------
\subsection{pipe datum_create}
\index{pipe!datum_create}
\label{p:datum.create}


Create a datum.

\begin{example}
   pipe datum_create \{datum_name\}^^\{data_type\}^^\{ele_ref_name\}^^\{ele_start_name\}^^
                       \{ele_name\}^^\{merit_type\}^^\{meas\}^^\{good_meas\}^^\{ref\}^^
                       \{good_ref\}^^\{weight\}^^\{good_user\}^^\{data_source\}^^
                       \{eval_point\}^^\{s_offset\}^^\{ix_bunch\}^^\{invalid_value\}^^
                       \{spin_axis%n0(1)\}^^\{spin_axis%n0(2)\}^^\{spin_axis%n0(3)\}^^
                       \{spin_axis%l(1)\}^^\{spin_axis%l(2)\}^^\{spin_axis%l(3)\}
\end{example}
\begin{verbatim}
Note: The 3 values for spin_axis%n0, as a group, are optional. 
      Also the 3 values for spin_axis%l are, as a group, optional.
Note: Use the "pipe data_d2_create" command first to create a d2 structure 
      with associated d1 arrays.
Note: After creating all your datums, use the "pipe data_set_design_value" routine
      to set the design (and model) values.
\end{verbatim}

%% pipe datum_has_ele ------------------------------------
\subsection{pipe datum_has_ele}
\index{pipe!datum_has_ele}
\label{p:datum.has.ele}


Output whether a datum type has an associated lattice element

\begin{example}
   pipe datum_has_ele \{datum_type\}
\end{example}
\begin{verbatim}

\end{verbatim}

%% pipe derivative ------------------------------------
\subsection{pipe derivative}
\index{pipe!derivative}
\label{p:derivative}


Output optimization derivatives

\begin{example}
   pipe derivative
\end{example}
\begin{verbatim}
Note: To save time, this command will not recalculate derivatives. 
Use the "derivative" command beforehand to recalcuate if needed.
\end{verbatim}

%% pipe ele:ac_kicker ------------------------------------
\subsection{pipe ele:ac_kicker}
\index{pipe!ele:ac_kicker}
\label{p:ele:ac.kicker}


Output element ac_kicker parameters

\begin{example}
   pipe ele:ac_kicker \{ele_id\}|\{which\}
\end{example}
\begin{verbatim}
Where: 
  {ele_id} is an element name or index.
  {which} is one of: "model", "base" or "design"

Example:
  pipe ele:ac_kicker 3@1>>7|model
This gives element number 7 in branch 1 of universe 3.
\end{verbatim}

%% pipe ele:cartesian_map ------------------------------------
\subsection{pipe ele:cartesian_map}
\index{pipe!ele:cartesian_map}
\label{p:ele:cartesian.map}


Output element cartesian_map parameters

\begin{example}
   pipe ele:cartesian_map \{ele_id\}|\{which\} \{index\} \{who\}
\end{example}
\begin{verbatim}
Where:
  {ele_id} is an element name or index
  {which} is one of: "model", "base" or "design"
  {index} is the index number in the ele%cartesian_map(:) array
  {who} is one of: "base", or "terms"

Example:
  pipe ele:cartesian_map 3@1>>7|model 2 base
This gives element number 7 in branch 1 of universe 3.
\end{verbatim}

%% pipe ele:chamber_wall ------------------------------------
\subsection{pipe ele:chamber_wall}
\index{pipe!ele:chamber_wall}
\label{p:ele:chamber.wall}


Output element beam chamber wall parameters

\begin{example}
   pipe ele:chamber_wall \{ele_id\}|\{which\} \{index\} \{who\}
\end{example}
\begin{verbatim}
Where:
  {ele_id} is an element name or index.
  {which} is one of: "model", "base" or "design"
  {index} is index of the wall.
  {who} is one of:
    "x"       ! Return min/max in horizontal plane
    "y"       ! Return min/max in vertical plane
\end{verbatim}

%% pipe ele:control_var ------------------------------------
\subsection{pipe ele:control_var}
\index{pipe!ele:control_var}
\label{p:ele:control.var}


Output list of element control variables.
Used for group, overlay and ramper type elements.

\begin{example}
   pipe ele:control_var \{ele_id\}|\{which\}
\end{example}
\begin{verbatim}
Where:
  {ele_id} is an element name or index.
  {which} is one of: "model", "base" or "design"

Example:
  pipe ele:control_var 3@1>>7|model
This gives control info on element number 7 in branch 1 of universe 3.
\end{verbatim}

%% pipe ele:cylindrical_map ------------------------------------
\subsection{pipe ele:cylindrical_map}
\index{pipe!ele:cylindrical_map}
\label{p:ele:cylindrical.map}


Output element cylindrical_map

\begin{example}
   pipe ele:cylindrical_map \{ele_id\}|\{which\} \{index\} \{who\}
\end{example}
\begin{verbatim}
Where 
  {ele_id} is an element name or index.
  {which} is one of: "model", "base" or "design"
  {index} is the index number in the ele%cylindrical_map(:) array
  {who} is one of: "base", or "terms"

Example:
  pipe ele:cylindrical_map 3@1>>7|model 2 base
This gives map #2 of element number 7 in branch 1 of universe 3.
\end{verbatim}

%% pipe ele:elec_multipoles ------------------------------------
\subsection{pipe ele:elec_multipoles}
\index{pipe!ele:elec_multipoles}
\label{p:ele:elec.multipoles}


Output element electric multipoles

\begin{example}
   pipe ele:elec_multipoles \{ele_id\}|\{which\}
\end{example}
\begin{verbatim}
Where:
  {ele_id} is an element name or index.
  {which} is one of: "model", "base" or "design"

Example:
  pipe ele:elec_multipoles 3@1>>7|model
This gives element number 7 in branch 1 of universe 3.
\end{verbatim}

%% pipe ele:floor ------------------------------------
\subsection{pipe ele:floor}
\index{pipe!ele:floor}
\label{p:ele:floor}


Output element floor coordinates. The output gives four lines. "Reference" is
without element misalignments and "Actual" is with misalignments. The lines with "-W"
give the W matrix. The exception is that if ele is a multipass_lord, there will be 4*N
lines where N is the number of slaves.

\begin{example}
   pipe ele:floor \{ele_id\}|\{which\} \{where\}
\end{example}
\begin{verbatim}
Where:
  {ele_id} is an element name or index.
  {which} is one of: "model", "base" or "design"
  {where} is an optional argument which, if present, is one of
    beginning  ! Upstream end.
    center     ! Middle of the element. This is the surface of element when used 
               !  with photonic reflecting elements such as crystal and mirror elements.
    end        ! Downstream end (default).

Example:
  pipe ele:floor 3@1>>7|model
This gives element number 7 in branch 1 of universe 3.
\end{verbatim}

%% pipe ele:gen_attribs ------------------------------------
\subsection{pipe ele:gen_attribs}
\index{pipe!ele:gen_attribs}
\label{p:ele:gen.attribs}


Output element general attributes

\begin{example}
   pipe ele:gen_attribs \{ele_id\}|\{which\}
\end{example}
\begin{verbatim}
Where: 
  {ele_id} is an element name or index.
  {which} is one of: "model", "base" or "design"

Example:
  pipe ele:gen_attribs 3@1>>7|model
This gives element number 7 in branch 1 of universe 3.
\end{verbatim}

%% pipe ele:gen_grad_map ------------------------------------
\subsection{pipe ele:gen_grad_map}
\index{pipe!ele:gen_grad_map}
\label{p:ele:gen.grad.map}


Output element gen_grad_map 

\begin{example}
   pipe ele:gen_grad_map \{ele_id\}|\{which\} \{index\} \{who\}
\end{example}
\begin{verbatim}
Where: 
  {ele_id} is an element name or index.
  {which} is one of: "model", "base" or "design"
  {index} is the index number in the ele%gen_grad_map(:) array
  {who} is one of: "base", or "derivs".

Example:
  pipe ele:gen_grad_map 3@1>>7|model 2 base
This gives element number 7 in branch 1 of universe 3.
\end{verbatim}

%% pipe ele:grid_field ------------------------------------
\subsection{pipe ele:grid_field}
\index{pipe!ele:grid_field}
\label{p:ele:grid.field}


Output element grid_field

\begin{example}
   pipe ele:grid_field \{ele_id\}|\{which\} \{index\} \{who\}
\end{example}
\begin{verbatim}
Where:
  {ele_id} is an element name or index.
  {which} is one of: "model", "base" or "design"
  {index} is the index number in the ele%grid_field(:) array.
  {who} is one of: "base", or "points"

Example:
  pipe ele:grid_field 3@1>>7|model 2 base
This gives grid #2 of element number 7 in branch 1 of universe 3.
\end{verbatim}

%% pipe ele:head ------------------------------------
\subsection{pipe ele:head}
\index{pipe!ele:head}
\label{p:ele:head}


Output "head" Element attributes

\begin{example}
   pipe ele:head \{ele_id\}|\{which\}
\end{example}
\begin{verbatim}
Where: 
  {ele_id} is an element name or index.
  {which} is one of: "model", "base" or "design"

Example:
  pipe ele:head 3@1>>7|model
This gives element number 7 in branch 1 of universe 3.
\end{verbatim}

%% pipe ele:lord_slave ------------------------------------
\subsection{pipe ele:lord_slave}
\index{pipe!ele:lord_slave}
\label{p:ele:lord.slave}


Output the lord/slave tree of an element.

\begin{example}
   pipe ele:lord_slave \{ele_id\}|\{which\}
\end{example}
\begin{verbatim}
Where: 
  {ele_id} is an element name or index.
  {which} is one of: "model", "base" or "design"

Example:
  pipe ele:lord_slave 3@1>>7|model
This gives lord and slave info on element number 7 in branch 1 of universe 3.
Note: The lord/slave info is independent of the setting of {which}.

The output is a number of lines.
Each line gives information on an element (element index, etc.).
Some lines begin with the word "Element". 
After each "Element" line, there are a number of lines (possibly zero) 
that begin with the word "Slave or "Lord".
These "Slave" and "Lord" lines are the slaves and lords of the "Element" element.
\end{verbatim}

%% pipe ele:mat6 ------------------------------------
\subsection{pipe ele:mat6}
\index{pipe!ele:mat6}
\label{p:ele:mat6}


Output element mat6

\begin{example}
   pipe ele:mat6 \{ele_id\}|\{which\} \{who\}
\end{example}
\begin{verbatim}
Where: 
  {ele_id} is an element name or index.
  {which} is one of: "model", "base" or "design"
  {who} is one of: "mat6", "vec0", or "err"

Example:
  pipe ele:mat6 3@1>>7|model mat6
This gives element number 7 in branch 1 of universe 3.
\end{verbatim}

%% pipe ele:methods ------------------------------------
\subsection{pipe ele:methods}
\index{pipe!ele:methods}
\label{p:ele:methods}


Output element methods

\begin{example}
   pipe ele:methods \{ele_id\}|\{which\}
\end{example}
\begin{verbatim}
Where: 
  {ele_id} is an element name or index.
  {which} is one of: "model", "base" or "design"

Example:
  pipe ele:methods 3@1>>7|model
This gives element number 7 in branch 1 of universe 3.
\end{verbatim}

%% pipe ele:multipoles ------------------------------------
\subsection{pipe ele:multipoles}
\index{pipe!ele:multipoles}
\label{p:ele:multipoles}


Output element multipoles

\begin{example}
   pipe ele:multipoles \{ele_id\}|\{which\}
\end{example}
\begin{verbatim}
Where: 
  {ele_id} is an element name or index.
  {which} is one of: "model", "base" or "design"

Example:
  pipe ele:multipoles 3@1>>7|model
This gives element number 7 in branch 1 of universe 3.
\end{verbatim}

%% pipe ele:orbit ------------------------------------
\subsection{pipe ele:orbit}
\index{pipe!ele:orbit}
\label{p:ele:orbit}


Output element orbit

\begin{example}
   pipe ele:orbit \{ele_id\}|\{which\}
\end{example}
\begin{verbatim}
Where: 
  {ele_id} is an element name or index.
  {which} is one of: "model", "base" or "design"

Example:
  pipe ele:orbit 3@1>>7|model
This gives element number 7 in branch 1 of universe 3.
\end{verbatim}

%% pipe ele:param ------------------------------------
\subsection{pipe ele:param}
\index{pipe!ele:param}
\label{p:ele:param}


Output lattice element parameter

\begin{example}
   pipe ele:param \{ele_id\}|\{which\} \{who\}
\end{example}
\begin{verbatim}
Where: 
  {ele_id} is an element name or index.
  {which} is one of: "model", "base" or "design"
  {who} values are the same as {who} values for "pipe lat_list".
        Note: Here {who} must be a single parameter and not a list.

Example:
  pipe ele:param 3@1>>7|model e_tot
This gives E_tot of element number 7 in branch 1 of universe 3.

Note: On output the {variable} component will always be "F" (since this 
command cannot tell if a parameter is allowed to vary).

Also see: "pipe lat_list".
\end{verbatim}

%% pipe ele:photon ------------------------------------
\subsection{pipe ele:photon}
\index{pipe!ele:photon}
\label{p:ele:photon}


Output element photon parameters

\begin{example}
   pipe ele:photon \{ele_id\}|\{which\} \{who\}
\end{example}
\begin{verbatim}
Where: 
  {ele_id} is an element name or index.
  {which} is one of: "model", "base" or "design"
  {who} is one of: "base", "material", or "curvature"

Example:
  pipe ele:photon 3@1>>7|model base
This gives element number 7 in branch 1 of universe 3.
\end{verbatim}

%% pipe ele:spin_taylor ------------------------------------
\subsection{pipe ele:spin_taylor}
\index{pipe!ele:spin_taylor}
\label{p:ele:spin.taylor}


Output element spin_taylor parameters

\begin{example}
   pipe ele:spin_taylor \{ele_id\}|\{which\}
\end{example}
\begin{verbatim}
Where: 
  {ele_id} is an element name or index.
  {which} is one of: "model", "base" or "design"

Example:
  pipe ele:spin_taylor 3@1>>7|model
This gives element number 7 in branch 1 of universe 3.
\end{verbatim}

%% pipe ele:taylor ------------------------------------
\subsection{pipe ele:taylor}
\index{pipe!ele:taylor}
\label{p:ele:taylor}


Output element taylor map 

\begin{example}
   pipe ele:taylor \{ele_id\}|\{which\}
\end{example}
\begin{verbatim}
Where: 
  {ele_id} is an element name or index.
  {which} is one of: "model", "base" or "design"

Example:
  pipe ele:taylor 3@1>>7|model
This gives element number 7 in branch 1 of universe 3.
\end{verbatim}

%% pipe ele:twiss ------------------------------------
\subsection{pipe ele:twiss}
\index{pipe!ele:twiss}
\label{p:ele:twiss}


Output element Twiss parameters

\begin{example}
   pipe ele:twiss \{ele_id\}|\{which\}
\end{example}
\begin{verbatim}
Where: 
  {ele_id} is an element name or index.
  {which} is one of: "model", "base" or "design"

Example:
  pipe ele:twiss 3@1>>7|model
This gives element number 7 in branch 1 of universe 3.
\end{verbatim}

%% pipe ele:wake ------------------------------------
\subsection{pipe ele:wake}
\index{pipe!ele:wake}
\label{p:ele:wake}


Output element wake.

\begin{example}
   pipe ele:wake \{ele_id\}|\{which\} \{who\}
\end{example}
\begin{verbatim}
Where: 
  {ele_id} is an element name or index.
  {which} is one of: "model", "base" or "design"
  {Who} is one of:
      "sr_long"        "sr_long_table"
      "sr_trans"       "sr_trans_table"
      "lr_mode_table"  "base"

Example:
  pipe ele:wake 3@1>>7|model
This gives element number 7 in branch 1 of universe 3.
\end{verbatim}

%% pipe ele:wall3d ------------------------------------
\subsection{pipe ele:wall3d}
\index{pipe!ele:wall3d}
\label{p:ele:wall3d}


Output element wall3d parameters.

\begin{example}
   pipe ele:wall3d \{ele_id\}|\{which\} \{index\} \{who\}
\end{example}
\begin{verbatim}
Where: 
  {ele_id} is an element name or index.
  {which} is one of: "model", "base" or "design"
  {index} is the index number in the ele%wall3d(:) array 
            The size of this array is obtained from "pipe ele:head".
  {who} is one of: "base", or "table".
Example:
  pipe ele:wall3d 3@1>>7|model 2 base
This gives element number 7 in branch 1 of universe 3.
\end{verbatim}

%% pipe evaluate ------------------------------------
\subsection{pipe evaluate}
\index{pipe!evaluate}
\label{p:evaluate}


Output the value of an expression. The result may be a vector.

\begin{example}
   pipe evaluate \{flags\} \{expression\}
\end{example}
\begin{verbatim}
Where:
  Optional {flags} are:
      -array_out : If present, the output will be available in the 
                    tao_c_interface_com%c_real array.
  {expression} is expression to be evaluated.

Example:
  pipe evaluate 3+data::cbar.11[1:10]|model
\end{verbatim}

%% pipe em_field ------------------------------------
\subsection{pipe em_field}
\index{pipe!em_field}
\label{p:em.field}


Output EM field at a given point generated by a given element.

\begin{example}
   pipe em_field \{ele_id\}|\{which\} \{x\} \{y\} \{z\} \{t_or_z\}
\end{example}
\begin{verbatim}
Where:
  {which} is one of: "model", "base" or "design"
  {x}, {y}  -- Transverse coords.
  {z}       -- Longitudinal coord with respect to entrance end of element.
  {t_or_z}  -- time or phase space z depending if lattice is setup for 
            --   absolute time tracking.
\end{verbatim}

%% pipe enum ------------------------------------
\subsection{pipe enum}
\index{pipe!enum}
\label{p:enum}


Output list of possible values for enumerated numbers.

\begin{example}
   pipe enum \{enum_name\}
\end{example}
\begin{verbatim}
Example:
  pipe enum tracking_method
\end{verbatim}

%% pipe floor_plan ------------------------------------
\subsection{pipe floor_plan}
\index{pipe!floor_plan}
\label{p:floor.plan}


Output (x,y) points and other information that can be used for drawing a floor_plan.

\begin{example}
   pipe floor_plan \{graph\}
\end{example}
\begin{verbatim}

\end{verbatim}

%% pipe floor_orbit ------------------------------------
\subsection{pipe floor_orbit}
\index{pipe!floor_orbit}
\label{p:floor.orbit}


Output (x, y) coordinates for drawing the particle orbit on a floor plan.

\begin{example}
   pipe floor_orbit \{graph\}
\end{example}
\begin{verbatim}

\end{verbatim}

%% pipe global ------------------------------------
\subsection{pipe global}
\index{pipe!global}
\label{p:global}


Output global parameters.

\begin{example}
   pipe global
\end{example}
\begin{verbatim}
Output syntax is parameter list form. See documentation at the beginning of this file.

Note: The follow is intentionally left out:
  optimizer_allow_user_abort
  quiet
  single_step
  prompt_color
  prompt_string
\end{verbatim}

%% pipe global:optimization ------------------------------------
\subsection{pipe global:optimization}
\index{pipe!global:optimization}
\label{p:global:optimization}


Output optimization parameters.
Also see global:opti_de.

\begin{example}
   pipe global:optimization
\end{example}
\begin{verbatim}
Output syntax is parameter list form. See documentation at the beginning of this file.
\end{verbatim}

%% pipe global:opti_de ------------------------------------
\subsection{pipe global:opti_de}
\index{pipe!global:opti_de}
\label{p:global:opti.de}


Output DE optimization parameters.

\begin{example}
   pipe global:opti_de
\end{example}
\begin{verbatim}
Output syntax is parameter list form. See documentation at the beginning of this file.
\end{verbatim}

%% pipe help ------------------------------------
\subsection{pipe help}
\index{pipe!help}
\label{p:help}


Output list of "help xxx" topics

\begin{example}
   pipe help
\end{example}
\begin{verbatim}

\end{verbatim}

%% pipe inum ------------------------------------
\subsection{pipe inum}
\index{pipe!inum}
\label{p:inum}


Output list of possible values for an INUM parameter.
For example, possible index numbers for the branches of a lattice.

\begin{example}
   pipe inum \{who\}
\end{example}
\begin{verbatim}

\end{verbatim}

%% pipe lat_calc_done ------------------------------------
\subsection{pipe lat_calc_done}
\index{pipe!lat_calc_done}
\label{p:lat.calc.done}


Output if a lattice recalculation has been proformed since the last 
  time "pipe lat_calc_done" was called.

\begin{example}
   pipe lat_calc_done
\end{example}
\begin{verbatim}

\end{verbatim}

%% pipe lat_ele_list ------------------------------------
\subsection{pipe lat_ele_list}
\index{pipe!lat_ele_list}
\label{p:lat.ele.list}


Output lattice element list.

\begin{example}
   pipe lat_ele_list \{branch_name\}
\end{example}
\begin{verbatim}
{branch_name} should have the form:
  {ix_uni}@{ix_branch}
\end{verbatim}

%% pipe lat_branch_list ------------------------------------
\subsection{pipe lat_branch_list}
\index{pipe!lat_branch_list}
\label{p:lat.branch.list}


Output lattice branch list

\begin{example}
   pipe lat_branch_list \{ix_uni\}
\end{example}
\begin{verbatim}
Output syntax:
  branch_index;branch_name;n_ele_track;n_ele_max
\end{verbatim}

%% pipe lat_list ------------------------------------
\subsection{pipe lat_list}
\index{pipe!lat_list}
\label{p:lat.list}


Output list of parameters at ends of lattice elements

\begin{example}
   pipe lat_list \{flags\} \{ix_uni\}@\{ix_branch\}>>\{elements\}|\{which\} \{who\}
\end{example}
\begin{verbatim}
Where:
 Optional {flags} are:
  -no_slaves   - If present, multipass_slave and super_slave elements will not 
               -   be matched to.
  -track_only  - If present, lord elements will not be matched to.
  -index_order - If present, order elements by element index instead of the 
               -   standard s-position.
  -array_out   - If present, the output will be available in the 
    tao_c_interface_com%c_real or tao_c_interface_com%c_integer arrays. 
    See the code below for when %c_real vs %c_integer is used.
    Note: Only a single {who} item permitted when -array_out is present.

  {which} is one of: "model", "base" or "design"

  {who} is a comma deliminated list of:
    orbit.floor.x, orbit.floor.y, orbit.floor.z    ! Floor coords at particle orbit.
    orbit.spin.1, orbit.spin.2, orbit.spin.3,
    orbit.vec.1, orbit.vec.2, orbit.vec.3, orbit.vec.4, orbit.vec.5, orbit.vec.6,
    orbit.t, orbit.beta,
    orbit.state,     ! Note: state is an integer. alive$ = 1, anything else is lost.
    orbit.energy, orbit.pc,
    ele.name, ele.key, ele.ix_ele, ele.ix_branch
    ele.a.beta, ele.a.alpha, ele.a.eta, ele.a.etap, ele.a.gamma, ele.a.phi,
    ele.b.beta, ele.b.alpha, ele.b.eta, ele.b.etap, ele.b.gamma, ele.b.phi,
    ele.x.eta, ele.x.etap,
    ele.y.eta, ele.y.etap,
    ele.ref_time, ele.ref_time_start
    ele.s, ele.l
    ele.e_tot, ele.p0c
    ele.mat6      ! Output: mat6(1,:), mat6(2,:), ... mat6(6,:)
    ele.vec0      ! Output: vec0(1), ... vec0(6)
    ele.c_mat     ! Output: c_mat11, c_mat12, c_mat21, c_mat22.
    ele.gamma_c   ! Parameter associated with coupling c-matrix.
    ele.XXX       ! Where XXX is a Bmad syntax element attribute. 
                  !   EG: ele.beta_a, ele.k1, etc.

  {elements} is a string to match element names to.
    Use "*" to match to all elements.

Examples:
  pipe lat_list -track 3@0>>Q*|base ele.s,orbit.vec.2
  pipe lat_list 3@0>>Q*|base real:ele.s    

Also see: "pipe ele:param"
\end{verbatim}

%% pipe lat_param_units ------------------------------------
\subsection{pipe lat_param_units}
\index{pipe!lat_param_units}
\label{p:lat.param.units}


Output units of a parameter associated with a lattice or lattice element.

\begin{example}
   pipe lat_param_units \{param_name\}
\end{example}
\begin{verbatim}

\end{verbatim}

%% pipe matrix ------------------------------------
\subsection{pipe matrix}
\index{pipe!matrix}
\label{p:matrix}


Output matrix value from the exit end of one element to the exit end of the other.

\begin{example}
   pipe matrix \{ele1_id\} \{ele2_id\}
\end{example}
\begin{verbatim}
Where:
  {ele1_id} is the start element.
  {ele2_id} is the end element.
If {ele2_id} = {ele1_id}, the 1-turn transfer map is computed.
Note: {ele2_id} should just be an element name or index without universe, 
      branch, or model/base/design specification.

Example:
  pipe matrix 2@1>>q01w|design q02w
\end{verbatim}

%% pipe merit ------------------------------------
\subsection{pipe merit}
\index{pipe!merit}
\label{p:merit}


Output merit value.

\begin{example}
   pipe merit
\end{example}
\begin{verbatim}

\end{verbatim}

%% pipe orbit_at_s ------------------------------------
\subsection{pipe orbit_at_s}
\index{pipe!orbit_at_s}
\label{p:orbit.at.s}


Output twiss at given s position.

\begin{example}
   pipe orbit_at_s \{ix_uni\}@\{ele\}->\{s_offset\}|\{which\}
\end{example}
\begin{verbatim}
Where:
  {ix_uni}   - Universe index. Defaults to s%global%default_universe.
  {ele}      - Element name or index. 
                 Default at the Beginning element at start of branch 0.
  {s_offset} - Offset of the evaluation point from the downstream end of ele. 
                 Default is 0. If {s_offset} is present, the preceeding "->" sign
                 must be present. EG: Something like "23|model" will {which} is 
                 one of: "model", "base" or "design".

Example:
  pipe orbit_at_s Q10->0.4|model   ! Orbit at 0.4 meters from Q10 element exit end in model lattice.
\end{verbatim}

%% pipe place_buffer ------------------------------------
\subsection{pipe place_buffer}
\index{pipe!place_buffer}
\label{p:place.buffer}


Output the place command buffer and reset the buffer.
The contents of the buffer are the place commands that the user has issued.
See the Tao manual for more details.

\begin{example}
   pipe place_buffer
\end{example}
\begin{verbatim}

\end{verbatim}

%% pipe plot_curve ------------------------------------
\subsection{pipe plot_curve}
\index{pipe!plot_curve}
\label{p:plot.curve}


Output curve information for a plot.

\begin{example}
   pipe plot_curve \{curve_name\}
\end{example}
\begin{verbatim}

\end{verbatim}

%% pipe plot_graph ------------------------------------
\subsection{pipe plot_graph}
\index{pipe!plot_graph}
\label{p:plot.graph}


Output graph info.

\begin{example}
   pipe plot_graph \{graph_name\}
\end{example}
\begin{verbatim}
{graph_name} is in the form:
  {p_name}.{g_name}
where
  {p_name} is the plot region name if from a region or the plot name if a template plot.
  This name is obtained from the pipe plot_list command.
  {g_name} is the graph name obtained from the pipe plot1 command.
\end{verbatim}

%% pipe plot_histogram ------------------------------------
\subsection{pipe plot_histogram}
\index{pipe!plot_histogram}
\label{p:plot.histogram}


Output plot histogram info.

\begin{example}
   pipe plot_histogram \{curve_name\}
\end{example}
\begin{verbatim}

\end{verbatim}

%% pipe plot_lat_layout ------------------------------------
\subsection{pipe plot_lat_layout}
\index{pipe!plot_lat_layout}
\label{p:plot.lat.layout}


Output plot Lat_layout info

\begin{example}
   pipe plot_lat_layout \{ix_uni\}@\{ix_branch\}
\end{example}
\begin{verbatim}
Note: The returned list of element positions is not ordered in increasing
      longitudinal position.
\end{verbatim}

%% pipe plot_list ------------------------------------
\subsection{pipe plot_list}
\index{pipe!plot_list}
\label{p:plot.list}


Output list of plot templates or plot regions.

\begin{example}
   pipe plot_list \{r_or_g\}
\end{example}
\begin{verbatim}
where "{r/g}" is:
  "r"      ! list regions of the form ix;region_name;plot_name;visible;x1;x2;y1;y2
  "t"      ! list template plots of the form ix;name
\end{verbatim}

%% pipe plot_template_manage ------------------------------------
\subsection{pipe plot_template_manage}
\index{pipe!plot_template_manage}
\label{p:plot.template.manage}


Template plot creation or destruction.

\begin{example}
   pipe plot_template_manage \{template_location\}^^\{template_name\}^^
                          \{n_graph\}^^\{graph_names\}
\end{example}
\begin{verbatim}
Where:
  {template_location} - Location to place or delete a template plot. 
                          Use "@Tnnn" syntax for the location.
  {template_name}     - The name of the template plot. 
                          If deleting a plot this name is immaterial.
  {n_graph}           - The number of associated graphs. 
                          If set to -1 then any existing template plot is deleted.
  {graph_names}       - Names of the graphs. graph_names should be in the form:
                            graph1_name^^graph2_name^^...^^graphN_name
                          where N=n_graph names
\end{verbatim}

%% pipe plot_curve_manage ------------------------------------
\subsection{pipe plot_curve_manage}
\index{pipe!plot_curve_manage}
\label{p:plot.curve.manage}


Template plot curve creation/destruction

\begin{example}
   pipe plot_curve_manage \{graph_name\}^^\{curve_index\}^^\{curve_name\}
\end{example}
\begin{verbatim}
If {curve_index} corresponds to an existing curve then this curve is deleted.
In this case the {curve_name} is ignored and does not have to be present.
If {curve_index} does not not correspond to an existing curve, {curve_index}
must be one greater than the number of curves.
\end{verbatim}

%% pipe plot_graph_manage ------------------------------------
\subsection{pipe plot_graph_manage}
\index{pipe!plot_graph_manage}
\label{p:plot.graph.manage}


Template plot graph creation/destruction

\begin{example}
   pipe plot_graph_manage \{plot_name\}^^\{graph_index\}^^\{graph_name\}
\end{example}
\begin{verbatim}
If {graph_index} corresponds to an existing graph then this graph is deleted.
In this case the {graph_name} is ignored and does not have to be present.
If {graph_index} does not not correspond to an existing graph, {graph_index}
must be one greater than the number of graphs.
\end{verbatim}

%% pipe plot_line ------------------------------------
\subsection{pipe plot_line}
\index{pipe!plot_line}
\label{p:plot.line}


Output points used to construct the "line" associated with a plot curve.

\begin{example}
   pipe plot_line \{region_name\}.\{graph_name\}.\{curve_name\} \{x_or_y\}
\end{example}
\begin{verbatim}
Optional {x-or-y} may be set to "x" or "y" to get the smooth line points x or y 
component put into the tao_c_interface_com%c_real array buffer.
Note: The plot must come from a region, and not a template, since no template plots 
      have associated line data.
Examples:
  pipe plot_line r13.g.a   ! String array output.
  pipe plot_line r13.g.a x ! x-component of line points put in array buffer.
  pipe plot_line r13.g.a y ! y-component of line points put in array buffer.
\end{verbatim}

%% pipe plot_symbol ------------------------------------
\subsection{pipe plot_symbol}
\index{pipe!plot_symbol}
\label{p:plot.symbol}


Output locations to draw symbols for a plot curve.

\begin{example}
   pipe plot_symbol \{region_name\}.\{graph_name\}.\{curve_name\} \{x_or_y\}
\end{example}
\begin{verbatim}
Optional {x_or_y} may be set to "x" or "y" to get the symbol x or y 
positions put into the real array buffer.
Note: The plot must come from a region, and not a template, 
      since no template plots have associated symbol data.
Examples:
  pipe plot_symbol r13.g.a       ! String array output.
  pipe plot_symbol r13.g.a x     ! x-component of the symbol positions 
                                     loaded into the real array buffer.
  pipe plot_symbol r13.g.a y     ! y-component of the symbol positions 
                                     loaded into the real array buffer.
\end{verbatim}

%% pipe plot_transfer ------------------------------------
\subsection{pipe plot_transfer}
\index{pipe!plot_transfer}
\label{p:plot.transfer}


Output transfer plot parameters from the "from plot" to the "to plot" (or plots).

\begin{example}
   pipe plot_transfer \{from_plot\} \{to_plot\}
\end{example}
\begin{verbatim}
To avoid confusion, use "@Tnnn" and "@Rnnn" syntax for {from_plot}.
If {to_plot} is not present and {from_plot} is a template plot, the "to plots" 
 are the equivalent region plots with the same name. And vice versa 
 if {from_plot} is a region plot.
\end{verbatim}

%% pipe plot1 ------------------------------------
\subsection{pipe plot1}
\index{pipe!plot1}
\label{p:plot1}


Output info on a given plot.

\begin{example}
   pipe plot1 \{name\}
\end{example}
\begin{verbatim}
{name} should be the region name if the plot is associated with a region.
Output syntax is parameter list form. See documentation at the beginning of this file.
\end{verbatim}

%% pipe ptc_com ------------------------------------
\subsection{pipe ptc_com}
\index{pipe!ptc_com}
\label{p:ptc.com}


Output Ptc_com structure components.

\begin{example}
   pipe ptc_com
\end{example}
\begin{verbatim}

\end{verbatim}

%% pipe ring_general ------------------------------------
\subsection{pipe ring_general}
\index{pipe!ring_general}
\label{p:ring.general}


Output lattice branch with closed geometry info (emittances, etc.)

\begin{example}
   pipe ring_general \{ix_uni\}@\{ix_branch\}|\{which\}
\end{example}
\begin{verbatim}
where {which} is one of:
  model
  base
  design
Example:
  pipe ring_general 1@0|model
\end{verbatim}

%% pipe shape_list ------------------------------------
\subsection{pipe shape_list}
\index{pipe!shape_list}
\label{p:shape.list}


Output lat_layout or floor_plan shapes list

\begin{example}
   pipe shape_list \{who\}
\end{example}
\begin{verbatim}
{who} is one of:
  lat_layout
  floor_plan
\end{verbatim}

%% pipe shape_manage ------------------------------------
\subsection{pipe shape_manage}
\index{pipe!shape_manage}
\label{p:shape.manage}


Element shape creation or destruction

\begin{example}
   pipe shape_manage \{who\} \{index\} \{add_or_delete\}
\end{example}
\begin{verbatim}
{who} is one of:
  lat_layout
  floor_plan
{add_or_delete} is one of:
  add     -- Add a shape at {index}. 
             Shapes with higher index get moved up one to make room.
  delete  -- Delete shape at {index}. 
             Shapes with higher index get moved down one to fill the gap.

Example:
  pipe shape_manage floor_plan 2 add
Note: After adding a shape use "pipe shape_set" to set shape parameters.
This is important since an added shape is in a ill-defined state.
\end{verbatim}

%% pipe shape_pattern_list ------------------------------------
\subsection{pipe shape_pattern_list}
\index{pipe!shape_pattern_list}
\label{p:shape.pattern.list}


Output list of shape patterns or shape pattern points

\begin{example}
   pipe shape_pattern_list \{ix_pattern\}
\end{example}
\begin{verbatim}
If optional {ix_pattern} index is omitted then list all the patterns.
If {ix_pattern} is present, list points of given pattern.
\end{verbatim}

%% pipe shape_pattern_manage ------------------------------------
\subsection{pipe shape_pattern_manage}
\index{pipe!shape_pattern_manage}
\label{p:shape.pattern.manage}


Add or remove shape pattern

\begin{example}
   pipe shape_pattern_manage \{ix_pattern\}^^\{pat_name\}^^\{pat_line_width\}
\end{example}
\begin{verbatim}
Where:
  {ix_pattern}      -- Pattern index. Patterns with higher indexes will be moved up 
                                      if adding a pattern and down if deleting.
  {pat_name}        -- Pattern name.
  {pat_line_width}  -- Line width. Integer. If set to "delete" then section 
                                            will be deleted.
\end{verbatim}

%% pipe shape_pattern_point_manage ------------------------------------
\subsection{pipe shape_pattern_point_manage}
\index{pipe!shape_pattern_point_manage}
\label{p:shape.pattern.point.manage}


Add or remove shape pattern point

\begin{example}
   pipe shape_pattern_point_manage \{ix_pattern\}^^\{ix_point\}^^\{s\}^^\{x\}
\end{example}
\begin{verbatim}
Where:
  {ix_pattern}      -- Pattern index.
  {ix_point}        -- Point index. Points of higher indexes will be moved up
                                    if adding a point and down if deleting.
  {s}, {x}          -- Point location. If {s} is "delete" then delete the point.
\end{verbatim}

%% pipe shape_set ------------------------------------
\subsection{pipe shape_set}
\index{pipe!shape_set}
\label{p:shape.set}


Set lat_layout or floor_plan shape parameters.

\begin{example}
   pipe shape_set \{who\}^^\{shape_index\}^^\{ele_name\}^^\{shape\}^^\{color\}^^
                    \{shape_size\}^^\{type_label\}^^\{shape_draw\}^^
                    \{multi_shape\}^^\{line_width\}
\end{example}
\begin{verbatim}
{who} is one of:
  lat_layout
  floor_plan
\end{verbatim}

%% pipe show ------------------------------------
\subsection{pipe show}
\index{pipe!show}
\label{p:show}


Output the output from a show command.

\begin{example}
   pipe show \{line\}
\end{example}
\begin{verbatim}
{line} is the string to pass through to the show command.
Example:
  pipe show lattice -pipe
\end{verbatim}

%% pipe space_charge_com ------------------------------------
\subsection{pipe space_charge_com}
\index{pipe!space_charge_com}
\label{p:space.charge.com}


Output space_charge_com structure parameters.

\begin{example}
   pipe space_charge_com
\end{example}
\begin{verbatim}
Output syntax is parameter list form. See documentation at the beginning of this file.
\end{verbatim}

%% pipe species_to_int ------------------------------------
\subsection{pipe species_to_int}
\index{pipe!species_to_int}
\label{p:species.to.int}


Convert species name to corresponding integer

\begin{example}
   pipe species_to_int \{species_str\}
\end{example}
\begin{verbatim}
Example:
  pipe species_to_int CO2++
\end{verbatim}

%% pipe species_to_str ------------------------------------
\subsection{pipe species_to_str}
\index{pipe!species_to_str}
\label{p:species.to.str}


Convert species integer id to corresponding

\begin{example}
   pipe species_to_str \{species_int\}
\end{example}
\begin{verbatim}
Example:
  pipe species_to_str -1     ! Returns 'Electron'
\end{verbatim}

%% pipe spin_invariant ------------------------------------
\subsection{pipe spin_invariant}
\index{pipe!spin_invariant}
\label{p:spin.invariant}


Output closed orbit spin axes n0, l0, or m0 at the ends of all lattice elements in a branch.
n0, l0, and m0 are solutions of the T-BMT equation.
n0 is periodic while l0 and m0 are not. At the beginning of the branch, the orientation of the 
l0 or m0 axes in the plane perpendicular to the n0 axis is chosen a bit arbitrarily.
See the Bmad manual for more details.

\begin{example}
   pipe spin_invariant \{flags\} \{who\} \{ix_uni\}@\{ix_branch\}|\{which\}
\end{example}
\begin{verbatim}
Where:
  {flags}       - Optional flags (currently there is only one):
                    -array_out  If present, the output will be available in 
                                                the tao_c_interface_com%c_real.
  {who}         - One of: l0, n0, or m0
  {ix_uni}      - A universe index. Defaults to s%global%default_universe.
  {ix_branch}   - A branch index. Defaults to s%global%default_branch.
  {which}       - Switch which is one of:
                     model
                     base
                     design

Example:
  pipe spin_invariant 1@0|model

Note: This command is under development. If you want to use please contact David Sagan.
\end{verbatim}

%% pipe spin_polarization ------------------------------------
\subsection{pipe spin_polarization}
\index{pipe!spin_polarization}
\label{p:spin.polarization}


Output spin polarization information

\begin{example}
   pipe spin_polarization \{ix_uni\}@\{ix_branch\}|\{which\}
\end{example}
\begin{verbatim}
Where:
  {ix_uni} is a universe index. Defaults to s%global%default_universe.
  {ix_branch} is a branch index. Defaults to s%global%default_branch.
  {which} is one of:
    model
    base
    design

Example:
  pipe spin_polarization 1@0|model

Note: This command is under development. If you want to use please contact David Sagan.
\end{verbatim}

%% pipe spin_resonance ------------------------------------
\subsection{pipe spin_resonance}
\index{pipe!spin_resonance}
\label{p:spin.resonance}


Output spin resonance information

\begin{example}
   pipe spin_resonance \{ix_uni\}@\{ix_branch\}|\{which\} \{ref_ele\}
\end{example}
\begin{verbatim}
Where:
  {ix_uni} is a universe index. Defaults to s%global%default_universe.
  {ix_branch} is a lattice branch index. Defaults to s%global%default_branch.
  {which} is one of: "model", "base" or "design"
  {ref_ele} is an element name or index.
This will return a string_list with the following fields:
  spin_tune                   -- Spin tune
  dq_X_sum, dq_X_diff         -- Tune sum Q_spin+Q_mode and tune difference 
                                   Q_spin-Q_mode for modes X = a, b, and c.
  xi_res_X_sum, xi_res_X_diff -- The linear spin/orbit sum and difference resonance 
                                   strengths for X = a, b, and c modes.
\end{verbatim}

%% pipe super_universe ------------------------------------
\subsection{pipe super_universe}
\index{pipe!super_universe}
\label{p:super.universe}


Output super_Universe parameters.

\begin{example}
   pipe super_universe
\end{example}
\begin{verbatim}

\end{verbatim}

%% pipe taylor_map ------------------------------------
\subsection{pipe taylor_map}
\index{pipe!taylor_map}
\label{p:taylor.map}


Output Taylor map between two points.

\begin{example}
   pipe taylor_map \{ele1_id\} \{ele2_id\} \{order\}
\end{example}
\begin{verbatim}
Where:
  {ele1_id}   - The start element.
  {ele2_id}   - The end element.
  {order}     - The map order. Default is order set in the lattice file. 
                  {order} cannot be larger than what is set by the lattice file. 

If {ele2_id} = {ele1_id}, the 1-turn transfer map is computed.
Note: {ele2_id} should just be an element name or index without universe, 
      branch, or model/base/design specification.
Example:
  pipe taylor_map 2@1>>q01w|design q02w  2
\end{verbatim}

%% pipe twiss_at_s ------------------------------------
\subsection{pipe twiss_at_s}
\index{pipe!twiss_at_s}
\label{p:twiss.at.s}


Output twiss parameters at given s position.

\begin{example}
   pipe twiss_at_s \{ix_uni\}@\{ele\}->\{s_offset\}|\{which\}
\end{example}
\begin{verbatim}
Where:
  {ix_uni}    - A universe index. Defaults to s%global%default_universe.
  {ele}       - An element name or index. Default is the Beginning element of branch 0.
  {s_offset}  - Evaluation point offset from the downstream end of ele. Default is 0.
                  If {s_offset} is present, "->" must also be present. 
  {which}     - One of: "model", "base" or "design".
\end{verbatim}

%% pipe universe ------------------------------------
\subsection{pipe universe}
\index{pipe!universe}
\label{p:universe}


Output universe info.

\begin{example}
   pipe universe \{ix_uni\}
\end{example}
\begin{verbatim}
Use "pipe global" to get the number of universes.
\end{verbatim}

%% pipe var ------------------------------------
\subsection{pipe var}
\index{pipe!var}
\label{p:var}


Output parameters of a given variable.

\begin{example}
   pipe var \{var\} \{slaves\}
\end{example}
\begin{verbatim}
Note: use "pipe var_general" to get a list of variables.
\end{verbatim}

%% pipe var_create ------------------------------------
\subsection{pipe var_create}
\index{pipe!var_create}
\label{p:var.create}


Create a single variable

\begin{example}
   pipe var_create \{var_name\}^^\{ele_name\}^^\{attribute\}^^\{universes\}^^
                     \{weight\}^^\{step\}^^\{low_lim\}^^\{high_lim\}^^\{merit_type\}^^
                     \{good_user\}^^\{key_bound\}^^\{key_delta\}
\end{example}
\begin{verbatim}
{var_name} is something like "kick[5]".
Before using var_create, setup the appropriate v1_var array using 
the "pipe var_v1_create" command.
\end{verbatim}

%% pipe var_general ------------------------------------
\subsection{pipe var_general}
\index{pipe!var_general}
\label{p:var.general}


Output list of all variable v1 arrays

\begin{example}
   pipe var_general
\end{example}
\begin{verbatim}
Output syntax:
  {v1_var name};{v1_var%v lower bound};{v1_var%v upper bound}
\end{verbatim}

%% pipe var_v_array ------------------------------------
\subsection{pipe var_v_array}
\index{pipe!var_v_array}
\label{p:var.v.array}


Output list of variables for a given data_v1.

\begin{example}
   pipe var_v_array \{v1_var\}
\end{example}
\begin{verbatim}
Example:
  pipe var_v_array quad_k1
\end{verbatim}

%% pipe var_v1_array ------------------------------------
\subsection{pipe var_v1_array}
\index{pipe!var_v1_array}
\label{p:var.v1.array}


Output list of variables in a given variable v1 array

\begin{example}
   pipe var_v1_array \{v1_var\}
\end{example}
\begin{verbatim}

\end{verbatim}

%% pipe var_v1_create ------------------------------------
\subsection{pipe var_v1_create}
\index{pipe!var_v1_create}
\label{p:var.v1.create}


Create a v1 variable structure along with associated var array.

\begin{example}
   pipe var_v1_create \{v1_name\} \{n_var_min\} \{n_var_max\}
\end{example}
\begin{verbatim}
{n_var_min} and {n_var_max} are the lower and upper bounds of the var
Example:
  pipe var_v1_create quad_k1 0 45
This example creates a v1 var structure called "quad_k1" with an associated
variable array that has the range [0, 45].

Use the "pipe var_create" and "set variable" commands to set variable parameters.
Note: When setting multiple variable parameters, first set
  set global lattice_calc_on = F")
to prevent Tao trying to evaluate the 
partially created variable and generating unwanted error messages.
\end{verbatim}

%% pipe var_v1_destroy ------------------------------------
\subsection{pipe var_v1_destroy}
\index{pipe!var_v1_destroy}
\label{p:var.v1.destroy}


Destroy a v1 var structure along with associated var sub-array.

\begin{example}
   pipe var_v1_destroy \{v1_datum\}
\end{example}
\begin{verbatim}

\end{verbatim}

%% pipe wall3d_radius ------------------------------------
\subsection{pipe wall3d_radius}
\index{pipe!wall3d_radius}
\label{p:wall3d.radius}


Output vaccum chamber wall radius for given s-position and angle in (x,y) plane.
The radius is with respect to the local wall origin which may not be the (x,y) = (0,0) origin.

\begin{example}
   pipe wall3d_radius \{ix_uni\}@\{ix_branch\} \{s_position\} \{angle\}
\end{example}
\begin{verbatim}
Where:
  {ix_uni} is a universe index. Defaults to s%global%default_universe.
  {ix_branch} is a lattice branch index. 
  {s_position} is the s-position to evaluate at.
  {angle} is the angle to evaluate at.
\end{verbatim}

%% pipe wave ------------------------------------
\subsection{pipe wave}
\index{pipe!wave}
\label{p:wave}


Output Wave analysis info.

\begin{example}
   pipe wave \{who\}
\end{example}
\begin{verbatim}
Where {who} is one of:
  params
  loc_header
  locations
  plot1, plot2, plot3
\end{verbatim}

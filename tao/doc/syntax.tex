\chapter{Syntax}
\label{c:syntax}

%------------------------------------------------------------------------
\section{Command Line Syntax}
\label{s:com.syntax}

In ``\vn{line mode}'' (\sref{c:command}), commands are case sensitive. Multiple commands may be
entered on one line using the semicolon ``;'' character as a separator.  [However, a semicolon used
as as part of an \vn{alias} (\sref{s:alias}) definition is part of that definition.]  An exclamation
mark ``\vn{!}''  denotes the beginning of a comment and the exclamation mark and everything after it
to the end of the line is ignored.  Example:
\begin{example}
  set default uni = 2; show global  ! Two commands and a comment
\end{example}

The length of a command on a single line is currently limited to 1000 characters. Multiple lines may
be used for a single command by putting a ``\&'' character at the end of a line to be
continued. Example:
\begin{example}
  set default &    ! Continue command to next line
  uni = 2
\end{example}

Note that, for historical reasons, Bmad itself is case insensitive. Thus things like lattice element 
names within \tao commands will similarly be case insensitive.

%------------------------------------------------------------------------
\section{Specifying a Single Lattice Element}
\label{s:ele.name}

A full description of how to specify a lattice element is given in section \extref{B-s:ele.match}
``\vn{Matching to Lattice Element Names}'' in the \bmad manual. Generally, elements are specified
using either their names or by their index number. Additionally, in \tao, the universe in which
the element exists may be specified by prepending the element name by the universe number followed by
the ``\vn{@}'' sign.
Examples:
\begin{example}
  Q3##2      ! 2nd instance of element named Q3 in branch 0 of the default universe.
  134        ! Element with index 134 in branch 0 of the default universe.
  1>>13      ! Element with index 13 in branch 1 of the default universe.
  2@1>>TZ    ! Element named TZ in branch 1 of universe 2.
  B37        ! Element named B37 of the default universe.
  0@B37      ! Same as the previous example.
\end{example}
Note: element names are {\em not} case sensitive.

%------------------------------------------------------------------------
\section{Lattice Element List Format}
\label{s:ele.list.format}

The syntax for specifying a set of lattice elements is called \vn{element list} format. 
A element list is a list of items separated by a comma.\footnote
  {
 A blank space may be acceptable in some circumstances but a comma is always safe.
  }
Each item of the list is one of:
\begin{center}
\begin{tabular}{ll}
  {\it Item Type} & {\it Example} \\ \hline     
  A single element (\sref{s:ele.name})                & "1>>Q10W"            \\
  A name with wild card characters                    & "5@q*"               \\
  A range of elements in the form <ele1>:<ele2>       & "b23w:67"            \\
  A class::name specification                         & "sbend::b*"          \\
\end{tabular}
\break
\end{center}

Example element list:
\begin{example}
  23, 45:74, quad::q*
\end{example}

The wild card characters ``*'' and/or ``\%'' can be used. The ``*'' wildcard matches any number of
characters, The ``\%'' wildcard matches a single character. For example, ``q\%1*'' matches any
element whose name begins with ``q'' and whose third character is ``1''.  If there are multiple
elements in the lattice that match a given name, all such elements are included. Thus ``d12'' will
match to all elements of that name. Examples
\begin{example}
  "134"        ! Element with index 134 in branch 0 of the default universe.
  "1>>13"      ! Element with index 13 in branch 1 of the default universe.
  "5@q*"       ! All elements whose name begins with "q" of universe 5.
  "2@3>>q1##4" ! The fourth element named "q1" in branch 3 of universe 2.
  "*@sex10w"   ! Element "sex10w" of all universes.
  "b37"        ! Element "b37" of the default universe.
  "0@b37"      ! Same as the previous example.
\end{example}
Note: element names are {\em not} case sensitive.

An element index item is simply the index of the number in the lattice list of elements. A prefix
followed by the string ">>" can be used to specify a branch. As with element names, a universe
prefix can be given. Example
\begin{example}
  2@3>>183   ! Element number 183 of branch \# 3 of universe 2.
\end{example}

A range of elements is specified using the format:
\begin{example}
  \{<class>::\}<ele1>:<ele2>
\end{example}
\vn{<ele1>} is the element at the beginning of the range and \vn{<ele2>} is the element at the end
of the range. Either an element name or index can be used to specify \vn{<ele1>} and
\vn{<ele2>}. Both \vn{<ele1>} and \vn{<ele2>} are part of the range. The optional \vn{<class>}
prefix can be used to select only those elements in the range that match the class.  Example:
\begin{example}
  quad::sex10w:sex20w
\end{example}
This will select all quadrupoles between elements \vn{sex10w} and \vn{sex20w}.

\index{class::name}
A \vn{class::name} item
selects elements based upon their class (Eg: \vn{quadrupole},
\vn{marker}, etc.), and their name. The syntax is:
\begin{example}
  <element class>::<element name>
\end{example}
where \vn{<element class>} is an element class and \vn{<element name>} is the element name that can
(and generally does) contain the wild card characters ``\%'' and ``*''. Essentially this is an
extension of the \vn{element name} format. As with element names, a universe prefix can be
given. Example:
\begin{example}
  "4@quad::q*"   ! All quadrupole whose name starts with "q" of universe 4.
\end{example}

%------------------------------------------------------------------------
\section{Arithmetic Expressions}
\index{arithmetic Expressions}
\label{s:arithmetic.exp}

\tao is able to handle arithmetic expressions within commands (\sref{c:command}) and in strings in a
\tao initialization file.  Arithmetic expressions can be used in a place where a real value or an
array of real values are required.  The standard binary operators are defined: \hfil\break \hspace*{0.15in}
\begin{tabular}{ll}
  $a + b$           & Addition        \\
  $a - b$           & Subtraction     \\
  $a \, \ast \, b$  & Multiplication  \\
  $a \; / \; b$     & Division        \\
  $a \, \land \, b$ & Exponentiation  \\
\end{tabular} \newline
The following intrinsic functions are also recognized (this is the same list as the \bmad parser):
\hfil\break
\index{intrinsic functions}
\hspace*{0.15in}
\begin{tabular}{ll}
  \vn{sqrt}(x)                & Square Root                      \\
  \vn{log}(x)                 & Logarithm                        \\
  \vn{exp}(x)                 & Exponential                      \\
  \vn{sin}(x), \vn{cos}(x)    & Sine and Cosine                  \\
  \vn{tan}(x), \vn{cot}(x)    & Tangent and Cotangent            \\
  \vn{asin}(x), \vn{acos}(x)  & Arc sine and Arc Cosine          \\
  \vn{atan}(y)                & Arc Tangent                      \\
  \vn{atan2}(y, x)            & Arc Tangent                      \\
  \vn{abs}(x)                 & Absolute Value                   \\
  \vn{factorial(x)}           & Factorial                        \\
  \vn{ran}()                  & Random number between 0 and 1    \\
  \vn{ran_gauss}()            & Gaussian distributed random number with unit RMS \\
  \vn{ran_gauss}(sig_cut)     & Gaussian distributed random number truncated at sig_cut. \\
  \vn{int(x)}                 & Nearest integer with magnitude less then x \\
  \vn{nint(x)}                & Nearest integer to x             \\
  \vn{floor(x)}               & Nearest integer less than x      \\
  \vn{ceiling(x)}             & Nearest integer greater than x   \\
  \vn{modulo(a, p)}           & a - floor(a/p) * p. Will be in range [0, p]. \\
  \vn{average(arr)}           & Average value of an array        \\
  \vn{rms(arr)}               & RMS value of an array            \\
  \vn{sum(arr)}               & Sum of array values.             \\
  \vn{min(arr)}               & Minimum of array values.         \\
  \vn{max(arr)}               & Maximum of array values.         \\
  \vn{mass_of}(A)               & Mass of particle A             \\
  \vn{charge_of}(A)             & Charge, in units of the elementary charge, of particle A \\
  \vn{anomalous_moment_of}(A)   & Anomalous magnetic moment of particle A        \\
  \vn{species}(A)               & Integer ID associated with species A
\end{tabular} \newline
Both \vn{ran} and \vn{ran_gauss} use a seeded random number generator.  Setting the seed is
described in Section~\sref{s:globals}.

Expressions may involve arrays of values. For example:
\begin{example}
  lat::orbit.x[5:8]     ! X-orbit at lattice elements 5 through 8.
  [1, 2, 3]             ! A vector of size three.
\end{example}
When using vectors with binary operators or intrinsic functions, the standard rules apply. For example:
\begin{example}
  s * [a, b, c]         = [s*a, s*b, s*c]
  [a, b, c] - [x, y, z] = [a-x, b-y, c-z]
  tan([a, b, c])        = [tan(a), tan(b), tan(c)]
  sum([a, b, c])        = a+b+c
  min(a, b, c)          ! Error: Correct is min([a, b, c])
\end{example}
Note that \tao does not make a distinction between a scalar and a vector of length one.

See the following sections for the syntax for using data, variable, and lattice parameters in an
expression. Use the \vn{show value} command (\sref{s:show.value}) to show the results of expressions.

%------------------------------------------------------------------------
\section{Specifying Data Parameters in Expressions}
\label{s:data.token}

A data (\sref{s:data.org}) parameter ``\vn{token}'' is a string that specifies a scalar or an array
of data parameters.  The general form for data tokens in expressions (\sref{s:arithmetic.exp}) is:
\begin{example}
  \{[<universe(s)>]@\}data::<d2.d1_name>[<index_list>]|<component>
\end{example}
where:
\begin{example}
  <universe(s)>       Optional universe specification (\sref{s:universe})
  <d2.d1_name>        D2.D1 data name
  <index_list>        List of indexes.
  <component>         Component. 
\end{example}
examples:
\begin{example}
  [2:4,7]@data::orbit.x      ! The \vn{orbit.x} data in universes 2, 3, 4 and 7.
  [2]@data::orbit.x          ! The \vn{orbit.x} data in universe 2. 
  2@data::orbit.x[4]         ! Fourth \vn{orbit.x} datum in universe 2.
  data::orbit.x[4,7:9]|meas  ! Default uni measured values of datums 4, 7, 8, and 9.
  -1@data::orbit.x           ! Same as "orbit.x".
  *@data::orbit.x            ! orbit.x data in all the universes.
  *@data::*                  ! All the data in all the universes.
\end{example}

It is important to keep in mind that data must be defined at startup in the appropriate
initialization file as discussed in \Sref{s:init.data} before reference is made to data in an
expression. The \vn{<d2.d1_name>} data names that have been defined at initialization time may be
viewed using the \vn{show data} command. Note that these names are user defined and do not have to
correspond to the data types given in \Sref{s:data.types}. See \Sref{s:lat.token} for how to use
``lattice parameters'' that correspond to the data types given in \Sref{s:data.types} and that do
not need to be defined at initialization.

See \Sref{s:data.anatomy} for a list of datum \vn{<component>}s (when running \tao, view a
particular datum with the \vn{show data} command to see the list).

\vn{<index_list>} is a list of indexes. \vn{<index_list>} will determine how many elements are in
the array. For example, \vn{orbit.x[10:21,44]} represents an array of 13 elements.

Depending upon the context, some parts of a token may be omitted. For example, with the \vn{set
data} command the ``\vn{data::}'' part of the token may be omitted.  Example:
\begin{example}
  set data 2@orbit.x|meas = var::quad_k1[5]|model - orbit.y[3]|ref
\end{example}
Here \tao will default to evaluating a token as data. In general, what may be omitted
should be clear in context.

Data components that are computed by \tao may be used on the right hand side of an equal sign but
may not be set. For example, the \vn{model} value of a datum is computed by \tao but the \vn{ref}
value is not.

If multiple tokens are used in an expression, all tokens must evaluate to arrays of the same size.

%------------------------------------------------------------------------
\section{Specifying Variable Parameters in Expressions}
\label{s:var.token}

A variable (\sref{c:var}) parameter ``\vn{token}'' is a string that specifies a scalar or an array
of variable parameters. The general form for variable tokens in expressions
(\sref{s:arithmetic.exp}) is:
\begin{example}
  var::<v1_name>[<index_list>]<component>
\end{example}
where:
\begin{example}
  <universe(s)>       Optional universe specification (\sref{s:universe})
  <v1_name>           V1 variable name.
  <index_list>        List of indexes.
  <component>         component. 
\end{example}
Examples:
\begin{example}
  var::*                     ! All the variables
  var::quad_k1[*]|design     ! All design values of quad_k1.
  var::quad_k1[]|model       ! No values. That is, the empty set.
  var::quad_k1|model         ! Same as quad_k1[*]|model
\end{example}

It is important to keep in mind that variables must be defined at startup in the appropriate
initialization file as discussed in \Sref{s:init.var} before reference is made to them in an
expression.  The defined \vn{<v1_name>} variable names can be viewed using the \vn{show variable}
command. Since these names are user defined they will change if different initialization files are
used.

See \Sref{c:var} for a list of \vn{<components>} of a variable.

\vn{<index_list>} is a list of indexes. \vn{<index_list>} will determine how many elements are in
the array. For example, \vn{k_quad[10:21,44]} represents an array of 13 elements.

Depending upon the context, some parts of a token may be omitted. For example, with the \vn{set
variable} command the ``\vn{var::}'' part of the token may be omitted.  Example:
\begin{example}
  set var quad_k1[5]|meas = data::2@orbit.x|meas
\end{example}
Here \tao will default to evaluating a token as a variable component. In general, what may be
omitted should be clear in context.

Variable components that are computed by \tao may be used on the right hand side of an equal sign
but may not be set. For example, the \vn{design} value of a variable is computed by \tao but the
\vn{meas} value is not.

If multiple tokens are used in an expression, all tokens must evaluate to arrays of the same size.

%------------------------------------------------------------------------
\section{Specifying Lattice Parameters in Expressions}
\label{s:lat.token}

``Lattice parameters'' are like \vn{data} parameters (\sref{s:data.token}) except lattice parameters
are calculated from the lattice and do not have to be defined at initialization time.  A lattice
parameter ``\vn{token}'' is a string that specifies a scalar or an array of lattice parameters. The
general form for data tokens in expressions (\sref{s:arithmetic.exp}) is:
\begin{example}
  \{[<universe(s)>]@\}lat::<eval_param>[\{<ref_ele>&\}<element_list>\{-><s_offset>\}]\{|<component>\}
\end{example}
where:
\begin{example}
  <universe(s)>       Optional universe specification (\sref{s:universe})
  <eval_param>        Name of the parameter to evaluate. 
                        Possible data types listed in \Sref{s:data.types}. 
  <ref_ele>           Optional reference element.
  <element_list>      Evaluation point or points.
  <s_offset>          Longitudinal offset to evaluate at.
  <component>         Optional component. 
\end{example}
The \vn{<s_offset>} string can be an expression. Any parameter in this expression, if not qualified,
will be interpreted as a parameter of the element containing the evaluation point. For example
\begin{example}
  3@lat::orbit.x[q10w->-L/2]|model
\end{example}
Here ``\vn{L}'' in the \vn{<s_offset>} string ``\vn{-L/2}'' is interpreted as the length of the element
\vn{q10w}. Other examples:
\begin{example}
  3@lat::orbit.x[34:37]            ! Array of orbits at element 34 through 37 in universe 3.
  3@lat::orbit.x[q10w]|model       ! Orbit.x model value at exit end of element q10w
  3@lat::orbit.x[q10w->0.1]|model  ! Same as above except eval point is shifted 0.1 meters.
  lat::sigma.12[q10w]              ! Beam sigma matrix component at element q10w computed 
                                   !  from lattice parameters.
\end{example}

The list of possible lattice \vn{<eval_param>} names is given in \Sref{s:data.types}. The table
\ref{t:data.types} shows which data names are associated with the lattice. Lattice parameters are
independent of \vn{data} parameters. For example, \vn{lat::orbit.x} refers to the horizontal orbit
while \vn{data::orbit.x} refers to user defined data whose name corresponds to \vn{orbit.x} and in
fact there is nothing to prevent a user from assigning the name \vn{orbit.x} to data that is derived
from, say, the Twiss beta function.

Also notice the difference between, say, ``\vn{lat::orbit.x[10]}'' and ``\vn{data::orbit.x[10]}''.
With the ``\vn{lat::}'' source, the element index, in this case \vn{10}, refers to the 10th lattice
element. With the ``\vn{data::}'' source, ``\vn{10}'' refers to the 10\Th element in the
\vn{orbit.x} data array which may or may not correspond to the 10\Th lattice element.

The optional \vn{<ref_ele>} specifies a reference element for the evaluation. For example
\begin{example}
  lat::r.56[q0\&qa:qb]
\end{example}  
is an array of the $r(5,6)$ matrix element of the transport map between element \vn{q0} and each
element in the range from element \vn{qa} and \vn{qb}.

The optional \vn{s_offset>} specifies a longitudinal offset for the evaluation point. This may be
an expression.

%------------------------------------------------------------------------
\section{Specifying Beam Parameters in Expressions}
\label{s:beam.token}

Beam parameters are like lattice parameters (\sref{s:lat.token}) except beam parameters are derived
from tracking a beam of particles and may only be used in an expression if beam tracking is turned
on.  A beam parameter ``\vn{token}'' is a string that specifies a scalar or an array of beam
parameters. The general form for data tokens in expressions (\sref{s:arithmetic.exp}) is:
\begin{example}
  \{[<universe(s)>]@\}beam::<eval_param>[\{<ref_ele>&\}<element_list>]\{|<component>\}
\end{example}
where:
\begin{example}
  <universe(s)>       Optional universe specification (\sref{s:universe})
  <eval_param>        Name of the parameter
  <ref_ele>           Optional reference element.
  <element_list>      Evaluation point or points.
  <component>         Component. 
\end{example}
Examples:
\begin{example}
  2@beam::sigma.x[q10w]           Beam sigma at element q10w.
  beam::n_particle_loss[2&56]     Particle loss between elements 2 and 56.
\end{example}

The list of possible beam \vn{<eval_param>} names is given in \Sref{s:data.types}. The table
\ref{t:data.types} shows which data names are associated with beam tracking.

%------------------------------------------------------------------------
\section{Specifying Element Parameters in Expressions}
\label{s:ele.token}

``Element parameters'' are parameters associated with lattice elements like the quadrupole strength
associated with an element. Element parameters also include derived quantities like the computed
Twiss parameters and the beam orbit. An element parameter ``\vn{token}'' is a string that specifies
a scalar or an array of element parameters. The general form for element tokens in expressions is:
\begin{example}
  \{<universe(s)>@\}ele::<element_list>[<parameter>]\{|<component>\}
  \{<universe(s)>@\}ele_mid::<element_list>[<parameter>]\{|<component>\}
\end{example}
where:
\begin{example}
  <universe(s)>       Optional universe specification (\sref{s:universe})
  <element_list>      List of element names or indexes.
  <parameter>         Name of the element parameter
  <component>         Component. 
\end{example}
Examples:
\begin{example}
  3@ele_mid::34[orbit_x]     Orbit at middle of element with index 34 in universe 3.
  ele::sex01w[k2]            Sextupole component of element \vn{sex01w}
  ele::Q01W[is_on]|model     The on/off status of element \vn{Q01W}.
\end{example}

There is some overlap between element parameters and lattice parameters (\sref{s:lat.token}).  For
historical reasons, the \vn{element} parameter syntax roughly follows a convention developed for
\bmad lattice files which is somewhat different from the convention developed for \tao data. For
example, the $a$-mode beta is named \vn{beta.a} in \tao while \bmad uses the name \vn{beta_a}. See
the \bmad manual for more information on the \bmad lattice file syntax. The following table lists
the parameters that have both \tao datum and \bmad element parameter names
\begin{table}[ht] 
\centering 
{\tt
\begin{tabular}{lll} \toprule
  \vn{\tao Datum}                   & \vn{\bmad Element Parameter}        \\ \midrule
  alpha.a, alpha.b                  & alpha_a, alpha_b                    \\
  beta.a, beta.b                    & beta_a, beta_b                      \\
  cmat.11, etc.                     & cmat_11, etc.                       \\
  e_tot                             & e_tot                               \\
  eta.a, eta.b                      & eta_a, eta_b                        \\
  eta.x, eta.y                      & eta_x, eta_y                        \\
  etap.a, etap.b                    & etap_a, etap_b                      \\
  etap.x, etap.y                    & etap_x, etap_y                      \\
  floor.x, floor.y, floor.z         & x_position, y_position, z_position  \\
  floor.theta, floor.phi, floor.psi & theta_position, phi_position, psi_position \\
  gamma.a, gamma.b                  & gamma_a, gamma_b                    \\
  phase.a, phase.b                  & phi_a, phi_b                        \\
\bottomrule
\end{tabular}
} 
\caption{\tao datums that have equivalent \bmad element parameters.}  
\label{t:bmad.equiv1}
\end{table}

The following table lists the parameters that have both \tao datum and \bmad particle orbit names
\begin{table}[ht] 
\centering 
{\tt
\begin{tabular}{lll} \toprule
  \vn{\tao Datum}               & \vn{\bmad Orbit Parameter}         \\ \midrule
  orbit.x, orbit.y, orbit.z     & x, y, z                            \\
  orbit.px, orbit.py, orbit.pz  & px, py, pz                         \\
  spin.x, spin.y, spin.z        & spin_x, spin_y, spin_z             \\
  spin.amp spin.theta, spin.phi & spinor_polarization, spinor_theta, spinor_phi \\
\bottomrule
\end{tabular}
} 
\caption{\tao datums that have equivalent \bmad orbital parameters.}  
\label{t:bmad.equiv2}
\end{table}

For parameters that are varying throughout the element, like the Twiss parameters, \vn{ele::} will
evaluate the parameter at the exit end of the element and \vn{ele_mid::} will evaluate the parameter
at the middle of the element. For parameters that do not vary, like the quadrupole strength, use the
\vn{ele::} syntax.

Element list format (\sref{s:ele.list.format}) is used for the \vn{<element_list>} so an array of
elements can be defined.

For element parameter that evaluate to a logical, if they are used on the right hand side of an
expression where the result is a real number, a \vn{True} value will be converted to a value of
\vn{1} and a \vn{False} value is converted to a value of \vn{0}.

%------------------------------------------------------------------------
\section{Format Descriptors}
\label{s:edit.descrip}

Some \tao commands like \vn{show lattice} (\sref{s:show.lattice}) have optional arguments where the
format output of various quantities can be specified. \tao follows Fortran format descriptor
notation.  Since complete information is available on the Web (do a search for ``fortran edit
descriptor''), only a brief introduction will be given here.

Format descriptors are case insensitive. The commonly used descriptors with \tao are:
\begin{example}
  Form     Output
  ----     ----------------------------
  Aw       String
  Fw.d     Real numbers. Fixed point (no exponent).
  nPFw.d   Real numbers. Fixed point with the decimal point shifted \vn{n} places.
  ESw.d    Real numbers. Floating point (with exponent).
  Lw       Logicals.
  Iw       Integers.
  Iw.r     Integers padded with zeros to width \vn{r}.
  wX       White space.
  Tc       Tab to column \vn{c}.
\end{example}
In the above, ``\vn{w}'' is the width of the field (number of charactgers in the printed string) and 
``\vn{d}'' is the number of digits to the right of the decimal place, 

Examples:
\begin{example}
           Internal
  Format   Quantity   Output String   Comment
  ----     --------   -------------   -----------
  F7.2     76.1234    "  76.12"       Right justified.
  1PF7.2   76.1234    " 761.23"       Shifted decimal place 1 digit.
  F0.2     76.1234    "76.12"         0 Field width => Output width exactly fits.
  F3.2     76.1234    "***"           Number overflows field width.
  ES9.2    76.1234    " 7.61E+01"     Right justified.
  L3       True       "  T"           Right justified.
  I0       34         "34"            0 Field width => Actual width = number of digits.
  I4       34         "  34"          Right justified.
  I4.3     34         " 034"          Number padded with a zero to three digits.
  A3       "abcdef"   "abc"           String truncated.
  A3       "ab    "   "ab "           String truncated but looks left justified.
  A        "abcdef"   "abcdef"        Output width exactly fits string.
  A8       "abcdef"   "  abcdef"      Right justified.
  4x                  "   "           Four spaces.
  T45                                 Next output string starts at column 45
\end{example}

Note: When a format descriptor is being used to construct a table (EG \vn{show lattice} command),
using a ``\vn{0}'' for the field width is ill-advised since columns will not be properly aligned.

A comma delimited list is used for outputting multiple quantities. For example, the format ``\vn{I4,
A}'' is used to output an integer followed by a string.

If multiple quantities with the same format are to be outputted a \vn{multiplier} prefix number can
be used. For example, ``\vn{3A}'' is equivalent to ``\vn{A, A, A}''. If the format has a \vn{P}
prefix then parentheses can be used to separate the multiplier from the \vn{P} prefix. Example:
``\vn{2(3PF7.2)}'' is equivalent to ``\vn{3PF7.2, 3PF7.2}''.

Note to programmers: In a code file, a format string must always be enclosed in parentheses.

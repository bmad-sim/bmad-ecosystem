\chapter{Customizing Tao}
\index{customizing}
\label{c:custom.tao}

\tao has been designed to be readily extensible with a minimum of effort when certain rules are
followed. This chapter discusses how this is done. This is separate from using \tao's \vn{pipe}
command (\sref{s:pipe}) to control \tao.

%----------------------------------------------------------------
\section{Initial Setup}
\label{s:cust.init}

Creating a custom version of \tao involves creating custom code that is put in a directory that is
distinct from the \vn{tao} directory that contains the standard \tao code files.

\textbf{It is important to remember that the code in the \vn{tao} directory is not to be modified.
This ensures that, as time goes on, and as \tao is developed by the "Taoist" developers, changes to
the code in the \vn{tao} directories will have a minimal chance to break your custom code.} If you do
feel you need to change something in the \vn{tao} directory, please seek help first.

To setup a custom \tao version do the following:
  \begin{enumerate}
  \item
Establish a base directory in which things will be built. This directory can have any name. Here we
will call this directory \vn{ROOT}.
  \item
Make a subdirectory of \vn{ROOT} that will contain the custom code.  This directory can have any
name.  Here this directory will be called \vn{tao_custom}.
  \item
Copy the files from the directory \vn{tao/customization} to \vn{ROOT/tao_custom}. The \vn{tao}
directory is part of the \bmad package. If you do not know where to find it, ask your local Guru
where it is. Along with a \vn{README} file, there are two CMake\footnote
  {
CMake is a program used for compiling code.
  } 
script files in the \vn{customization} directory:
\begin{example}
  CMakeLists.txt
  cmake.custom_tao
\end{example}
These scripts are setup to make an executable called \vn{custom_tao}. This name can be changed by
modifying the \vn{cmake.custom_tao} file.
  \item
Copy the file \vn{tao/program/tao_program.f90} to \vn{ROOT/tao_custom}.
  \item
Copy as needed \vn{hook} files from \vn{tao/hook} to \vn{ROOT/tao_custom}. The hook files you will
need are the hook files you will want to modify to customize \tao. See below for details. See
\sref{s:cust.example} for an example.
  \item
Go to the \vn{ROOT/tao_custom} directory and use the command \vn{mk} to create the
executable 
\begin{example}
    \vn{ROOT/production/bin/custom_tao}. 
\end{example}
If a debug executable is wanted, the command \vn{mkd} will create one at: 
\begin{example}
    \vn{ROOT/debug/bin/custom_tao}
\end{example}
	\end{enumerate}
A debug executable is only needed if you are debugging the code. The debug exe will run much
slower than the production version.

%----------------------------------------------------------------
\section{It's All a Matter of Hooks}
\index{customizing!hooks}

The golden rule when extending \tao is that you are only allowed to customize routines that have the
name ``hook'' in them. These files are located in the directory \vn{tao/hook}.  To customize one of
these files, copy it from \vn{tao/hook} to \vn{ROOT} and then make modifications to the copy.

The reason for this golden rule is to ensure that, as time goes by, and revisions are made to the
\tao routines to extend \tao's usefulness and to eliminate bugs, these changes will have a minimum
impact on the specialized routines you write.  What happens if the modification you want to do
cannot be accomplished by customizing a hook routine? The answer is to contact the \tao programming
team and we will modify \tao and provide the hooks you need so that you can then do your
customization.

%----------------------------------------------------------------
\section{Implementing a Hook Routine in Tao}

Function pointers are used by \tao to call customized hook routines. \tao uses the same system as \bmad
where an abstract interface with a \vn{_def} suffix in the name is defined along with a function
pointer with a \vn{_ptr} suffix. See the section ``Custom and Hook Routines'' in the \bmad manual
for more details. For example, the \vn{tao_hook_command} routine has the function pointer (defined
in \vn{/tao/code/tao_interface.f90}):
\begin{example}
  procedure(tao_hook_command_def), pointer :: tao_hook_command_ptr => null()
\end{example}
To use a customized \vn{tao_hook_command} routine, the following can be put in the
\vn{tao_program.f90} that was copied to your area:
\begin{example}
  tao_hook_command_ptr => tao_hook_command
\end{example}

%----------------------------------------------------------------
\section{Initializing Hook Routines}

One way to initialize a hook routine is to read in parameters from an initialization file.  If an
initialization file is used, the filename may be set using the \vn{s%global%hook_init_file}
string. This string may be set in the \vn{tao_params} namelist (\sref{s:globals}) or may be set on
the command line using the \vn{-hook_init_file} option (\sref{s:command.line}).

%----------------------------------------------------------------
\section{Hook Routines}

To get a good idea of how \tao works it is recommended to spend a little bit of time going through
the source files. This may also provide pointers on how to make customizations in the hook
routines. Of particular interest is the module \vn{tao_lattice_calc_mod.f90} where tracking and
lattice parameters are computed.

Plotting is based upon the \vn{quick_plot} subroutines which are documented in the \bmad reference
manual. If custom plotting is desired this material should be reviewed to get familiar with the
concepts of ``graph'', ``box'', and ``page''.

The following is a run through of each of the hook routines. Each routine is in a separate file
called \vn{tao/hook/<hook_routine_name>.f90}. See these files for subroutine headers and plenty of
comments throughout the dummy code to aid in the modification of these subroutines.

%-----------------------------------------------------------------
\subsection{tao_hook_graph_setup}
\index{customizing!tao_hook_graph_data_setup}

Use this to setup custom graph data for a plot.

%-----------------------------------------------------------------
\subsection{tao_hook_command}\index{customizing!tao_hook_commad}
\label{s:hook.command}

Any custom commands are placed here. The dummy subroutine already has a bit of code that replicates
what is performed in \vn{tao_command}. Commands placed here are searched before the standard \tao
commands. This allows for the overwriting of any standard \tao command.

By default, there is one command included in here: \vn{`hook'}. This is just a simple command that
doesn't really do anything and is for the purposes of demonstrating how a custom command would be
implemented.

The only thing needed to be called at the end of a custom command is \vn{tao_cmd_end_calc}. This
will perform all of the steps listed in Section~\sref{s:lat.calc}.

See Sec.~\sref{s:cust.read.example} for an example of how to use this hook.

%-----------------------------------------------------------------
\subsection{tao_hook_evaluate_a_datum}
\index{customizing!tao_hook_evaluate_a_datum}

Any custom data types are defined and calculated here. If a non-standard data type is listed in the
initialization files, then a corresponding data type must be placed in this routine. The tutorial
uses this hook routine when calculating the emittance.

Dependent lattice parameters (such as closed orbits, beta functions, etc.) are recalculated every
time \tao believes the lattice has changed (for example, after a \vn{change} command).  This is done
in \vn{tao_lattice_calc}. \vn{tao_lattice_calc} in turn calls \vn{tao_evaluate_a_datum} for each
datum. \vn{tao_evaluate_a_datum} in turn calls \vn{tao_hook_evaluate_a_datum} to allow for custom
data evaluations. 

See the \vn{tao_evaluate_a_datum} routine as an example as how to handle datums.  The arguments for
\vn{tao_hook_evaluate_a_datum} is
\begin{example}
  tao_hook_evaluate_a_datum (found, datum, u, tao_lat, datum_value, valid_value)
\end{example}
The \vn{found} logical argument should be set to \vn{True} for datums that are handled by this hook
routine and \vn{found} should be set to \vn{False} for all other datums.

%-----------------------------------------------------------------
\subsection{tao_hook_init1 and tao_hook_init2}
\label{s:hook.init}
\index{customizing!tao_hook_init}

After the \vn{design} lattice and the global and universe structures are initialized,
\vn{tao_hook_init1} is called from the \vn{tao_init} routine. Here, any further initializations can
be added. In particular, if any custom hook structures need to be initialized, here's the place to
do it.

Further down in \vn{tao_init}, \vn{tao_hook_init2} is called. Normally you will want to use
\vn{tao_hook_init1}. However, \vn{tao_hook_init2} can be used, for example, ! to set model variable
values different from design variable values since when \vn{tao_hook_init1} is called the \vn{model}
lattice has not yet been initialized.

%-----------------------------------------------------------------
\subsection{tao_hook_init_design_lattice}
\index{customizing!tao_hook_init_design_lattice}

This will do a custom lattice initialization. The standard lattice initialization just calls
\vn{bmad_parser}. If anything more complex needs to be done then do it here. This is also where any
custom overlays or other elements would be inserted after the parsing is complete. But in general,
anything placed here should, in principle, be something that can be placed in a lattice file.

\textbf{This is the only routine that should insert elements in the ring}. This is because the \tao
data structures use the element index for each element associated with the datum. If all the element
indexes shift then the data structures will break. If new elements need to be inserted then modify
this routine and recompile. You can alternatively create a custom initialization file used by this
routine that reads in any elements to be inserted.

%-----------------------------------------------------------------
\subsection{tao_hook_lattice_calc}
\index{customizing!tao_hook_lattice_calc}

The standard lattice calculation can be performed for single particle, particle beam tracking and
will recalculate the orbit, transfer matrices, twiss parameters and load the data arrays. If
something else needs to be performed whenever the lattice is recalculated then it is placed here. A
custom lattice calculation can be performed on any lattice separately, this allows for the
possibility of, for example, tracking a single particle for one lattice and beams in another.

%-----------------------------------------------------------------
\subsection{tao_hook_merit_data}
\index{customizing!tao_hook_merit_data}

A custom data merit type can be defined here. Table~\ref{t:delta.v} lists the standard merit
types. If a custom merit type is used then \vn{load_it} in \vn{tao_hook_load_data_array} may also
need to be modified to handle this merit type, additionally, all standard data types may need to be
overridden in \vn{tao_hook_load_data_array} in order for the custom \vn{load_it} to be used.  See
\vn{tao_merit.f90} for how the standard merit types are calculated.

%-----------------------------------------------------------------
\subsection{tao_hook_merit_var}
\index{customizing!tao_hook_merit_var}

This hook will allow for a custom variable merit type. However, since there is no corresponding data
transfer, no \vn{load_it} routine needs to be modified.  See \vn{tao_merit.f90} for how the standard
merit types are calculated.

%-----------------------------------------------------------------
\subsection{tao_hook_optimizer}
\index{customizing!tao_hook_optimizer}

If a non standard optimizer is needed, then it can be implemented here. See the
\vn{tao_*_optimizer.f90} files for how the standard optimizers are implemented.

%-----------------------------------------------------------------
\subsection{tao_hook_parse_command_args}
\index{customizing!tao_hook_parse_command_args}

The \vn{tao_hook_parse_command_args} routine can be used to set the names of initialization
files. The file names are stored in the \vn{s%com} structure. For example, in the hook file, the
following changes the default plot initialization file:
\begin{example}
  s%com%hook_plot_file = '/nfs/acc/user/dcs16/my_plot_init.tao'
\end{example}
Note that if an initialization file name is given on the command line or in the root \tao
initialization file, that name will supersede the hook name.

%-----------------------------------------------------------------
\subsection{tao_hook_plot_graph}
\index{customizing!tao_hook_plot_graph}

This will customize the plotting of a graph. See the \tao module \vn{tao_plot_mod} for details on
what it normally done. You will also need to know how \vn{quick_plot} works (See the \bmad manual).

%-----------------------------------------------------------------
\subsection{tao_hook_plot_data_setup}
\index{customizing!tao_hook_plot_data_setup}

Use this routine to override the \vn{tao_plot_data_setup} routine which essentially transfers the
information from the \vn{s%u(:)%data} arrays to the
\vn{s%plot_page%region(:)%plot%graph(:)%curve(:)} arrays. This may be useful if you want to make a
plot that isn't simply the information in a data or variable array.

%-----------------------------------------------------------------
\subsection{tao_hook_post_process_data}
\index{customizing!tao_hook_post_process_data}

Here can be placed anything that needs to be done after the data arrays are loaded. This routine is
called immediately after the data arrays are called and before the optimizer or plotting is done, so
any final modifications to the lattice or data can be performed here.

%-----------------------------------------------------------------
%\chapter{Plotting}
%\label{s:prog.plotting} 

%\fbox{this chapter is yet to be completed!} 

%----------------------------------------------------------------
\section{Adding a New Data Type Example}
\label{s:cust.example}

As an example of a customization, let's include a new data type called \vn{particle_emittance}. This
will be the non-normalized x and y emittance as found from the Courant-Snyder invariant. This data
type will behave just like any other data type (i.e.  \vn{orbit}, \vn{phase} etc...).

This example will only require the modification of one file:
\vn{tao_hook_evaluate_a_datum.f90}. This file should be copied from the \vn{tao/hook} directory and
put in your \vn{ROOT/code} directory (\sref{s:cust.init}).

The formula for single particle emittance is
\begin{equation}
  \epsilon = \gamma x^{2} + 2 \alpha x x' + \beta x'^{2}
  \label{e:emittance}
\end{equation}
Place the following code in \vn{tao_hook_evaluate_a_datum.f90} in the \cmd{case select}
construct. Also add the necessary type declarations. See the routine \vn{tao_evaluate_a_datum} as an
example.
\begin{example}
  type (coord_struct), pointer :: orbit(:)
  type (ele_struct), pointer :: ele
  type (lat_struct), pointer :: lat
  integer ix_ele
  ...
  lat => tao_lat%lat
  orbit => tao_lat%tao_branch(0)%orbit
  ele => tao_pointer_to_datum_ele (lat, datum%ele_name, datum%ix_ele, datum, &
                                                          valid_value, why_invalid)
  ...
  select case (datum%data_type)
  case ('particle_emittance.x') 
    datum_value =  (ele%a%gamma * orbit(ix_ele)%vec(1)**2 + &
		     2 * ele%a%alpha * orbit(ix_ele)%vec(1) * orbit(ix_ele)%vec(2) + &
		     ele%a%beta * orbit(ix_ele)%vec(2)**2)
    
  case ('particle_emittance.y')
    datum_value = (ele%b%gamma * orbit(ix_ele)%vec(3)**2 + &
		     2 * ele%b%alpha * orbit(ix_ele)%vec(3) * orbit(ix_ele)%vec(4) + &
		     ele%b%beta * orbit(ix_ele)%vec(4)**2)
  end select
\end{example}
This defines what is to be calculated for each \vn{particle_emittance} datum.  There are two
transverse coordinates, so two definitions need to be made, one for each dimension.

Now you just need to declare the data types in the \cmd{tao.init} and \cmd{tao_plot.init} files. For
the sake of this example, modify the example files found in the \vn{bmad-doc/tao_examples} directory
\begin{example}
	mkdir ROOT/my_example
  cp tao/example/*.init ROOT/my_example
  cp tao/example/*.lat ROOT/my_example
\end{example}

In \cmd{ROOT/my_example/tao.init} add the following lines to the data declarations section
\begin{example}
  &tao_d2_data
    d2_data%name = "particle_emittance" 
    universe = 0 
    n_d1_data = 2
  /

  &tao_d1_data
    ix_d1_data = 1
    d1_data%name = "x"  
    default_weight = 1
    use_same_lat_eles_as = 'orbit.x"
  /

  &tao_d1_data
    ix_d1_data = 2
    d1_data%name = "y"  
    default_weight = 1
    use_same_lat_eles_as = 'orbit.x"
  /
\end{example}

In \cmd{ROOT/my_example/tao_plot.init} add the following lines to the end
of the file
\begin{example}
  &tao_template_plot
    plot%name = 'particle_emittance'
    plot%x_axis_type = 'index'
    plot%n_graph = 2
  /
  
  &tao_template_graph
    graph%name = 'x'
    graph_index = 1
    graph%box = 1, 2, 1, 2
    graph%title = 'Horizontal Emittance (microns)'
    graph%margin =  0.15, 0.06, 0.12, 0.12, '%BOX'
    graph%y%label = 'x'
    graph%y%max =  15
    graph%y%min =  0.0
    graph%y%major_div = 4
    curve(1)%data_source = 'data'
    curve(1)%data_type   = 'particle_emittance.x'
    curve(1)%y_axis_scale_factor = 1e6 !convert from meters to microns
  /

  &tao_template_graph
    graph%name = 'y'
    graph_index = 2
    graph%box = 1, 1, 1, 2
    graph%title = 'Vertical Emittance (microns)'
    graph%margin =  0.15, 0.06, 0.12, 0.12, '%BOX'
    graph%y%label = 'Y'
    graph%y%max =  15
    graph%y%min =  0.0
    graph%y%major_div = 4
    curve(1)%data_source = 'data'
    curve(1)%data_type = 'particle_emittance.y'
    curve(1)%units_factor = 1e6 !convert from meters to microns
  /
\end{example}
These namelists are described in detail in Chapter~\ref{c:init}.

We are now ready to compile and then run the program. The \tao library should have already been
created so all you need to do is
\begin{example}
	cd ROOT/code
	mk
  cd ROOT/my_example
  ../production/bin/custom_tao
\end{example}

After your custom \tao initializes type
\begin{example}
  place bottom particle_emittance
  scale
\end{example}
Your plot should look like Figure~\ref{f:plot.emittance}.

The emittance (as calculated) is not constant. This is due to dispersion and coupling throughout the
ring. \bmad provides a routine to find the particle emittance from the twiss parameters that
includes dispersion and coupling called \vn{orbit_amplitude_calc}.

\begin{figure}
  \centering
  \includegraphics[width=5in]{plot-emittance.pdf}
  \caption{Custom data type: non-normalized emittance}
  \label{f:plot.emittance}
\end{figure}

%----------------------------------------------------------------
\section{Reading in Measured Data Example}
\label{s:cust.read.example}

This section shows how to construct a customized version of \tao, called \vn{ping_tao}, to read in
measured data for analysis. This example uses data from the Fermilab proton recirculation. The data
is obtained by measuring the orbit turn-by-turn of a beam that has been initially pinged to give it
a finite oscillation amplitude.

The files for constructing \vn{ping_tao} can be found
in the directory
\begin{example}
  bmad-doc/tao_examples/custom_tao_with_measured_data
\end{example}
The files in this directory are as follows:
\begin{description}
  \item[CMakeLists.txt, cmake.ping_tao] \Newline
Script files for creating \vn{ping_tao}. See Sec.~\sref{s:cust.init}.
  \item[README] \Newline
The \vn{README} file gives some instructions on how to create \vn{ping_tao}
  \item[RRNOVAMU2E11172016.bmad] \Newline
Lattice file for the proton recirculation ring.
  \item[data] \Newline
Directory where some ping data is stored
  \item[tao.init] \Newline
\tao initialization file defining the appropriate data and variable structures (\sref{s:init.begin})
  \item[tao.startup] \Newline
File with some command that are executed when \tao is started. These commands will read in
and plot some data.
  \item[tao_hook_command.f90] \Newline
Custom code for reading in ping data. The template used to construct this file is at
\vn{tao/hook/tao_hook_command.f90} (\sref{s:hook.command}).
  \item[tao_plot.init] \Newline
File for defining plot parameters (\sref{s:init.plot}).
  \item[tao_program.f90] \Newline
copy of the \vn{tao/program/tao_program.f90} file (\sref{s:cust.init}).
\end{description}

After creating the \vn{ping_tao} program (see the \vn{README} file), the program can be run by going
to the custom_tao_with_measured_data directory and using the command:
\begin{example}
	../production/bin/ping_tao
\end{example}

The customized \vn{tao_hook_command} routine implements a custom command called
\vn{pingread}.  This command will read in ping data. Ping data is the amplitude and phase
of the beam oscillations at a BPM for either the \vn{a-mode} or \vn{b-mode} oscillations.
See the write up on ping data types in Sec.~\sref{s:data.types} under \vn{ping_a.amp_x},
and \vn{ping_b.amp_x} for more details.

The data files in the \vn{data} directory contain data for either the \vn{a-mode} or \vn{b-mode}
ping at either the horizontal or vertical BPMs.

The syntax of the \vn{pingread} command is:
\begin{example}
  pingread <mode> <filename> <data_or_ref>
\end{example}
The first argument, \vn{<mode>}, should be either ``\vn{a_mode}'' ``\vn{b_mode}'' indicating wether
the data is for the \vn{a-mode} \vn{b-mode} analysis (a better setup would encode this information
in the data file itself). The second argument, \vn{filename} is the name of the data file, and the
third argument, \vn{data_or_ref} should be ``\vn{data}'' or ``\vn{reference}'' indicating that the
data is to be read into the \vn{meas_value} or \vn{ref_value} of the appropriate
\vn{tao_data_struct}.

%----------------------------------------------------------------
\subsection{Analysis of the tao_hook_command.f90 File}
\label{s:hook.cmd.anal}

The first part of the \vn{tao_hook_command} routine parses the command line to see if the
\vn{pingread} command is present. The relevant code, somewhat condensed, is:
\begin{example}
  subroutine tao_hook_command (command_line, found)

  !!!! put your list of hook commands in here. 

  character(16) :: cmd_names(1) = [character(16):: 'pingread']  

  ! "found" will be set to TRUE if the command is found.

  found = .false.

  ! strip the command line of comments

  call string_trim (command_line, cmd_line, ix_line)
  ix = index(cmd_line, '!')
  if (ix /= 0) cmd_line = cmd_line(:ix-1)        ! strip off comments

  ! blank line => nothing to do

  if (cmd_line(1:1) == '') return

  ! match first word to a command name
  ! If not found then found = .false.

  call match_word (cmd_line(:ix_line), cmd_names, ix_cmd, .true., .true., cmd_name)
  if (ix_cmd < 0) then
    call out_io (s_error$, r_name, 'AMBIGUOUS HOOK COMMAND')
    found = .true.
    return
  endif

  found = .true.
  call string_trim (cmd_line(ix_line+1:), cmd_line, ix_line)
\end{example}

Note: To quickly find information on routines and structures, use the \vn{getf} and \vn{listf}
scripts as explained in the \bmad manual. For example, typing ``\vn{getf string_trim}'' on the
system command line will give information on the string_trim subroutine.

The above code tests to see if the command is \vn{pingread} and, if not, returns without doing
anything.

If the \vn{pingread} command is found, the rest of the command line is parsed to get the
\vn{<mode>}, \vn{<filename>}, and \vn{<data_or_ref>} arguments.

In the \vn{tao.init} file, a \vn{tune} d2 datum is setup to have two \vn{d1} datum arrays One for
the \vn{a}-mode tune and one for the \vn{b}-mode tune:
\begin{example}
  \&tao_d2_data
    d2_data%name = "tune"
    universe = '*'  ! apply to all universes
    n_d1_data = 2
  /

  \&tao_d1_data
    ix_d1_data = 1
    d1_data%name = "a"
    default_weight = 1e6
    ix_min_data = 1
    ix_max_data = 1
  /

  \&tao_d1_data
    ix_d1_data = 2
    d1_data%name = "b"
    default_weight = 1e6
    ix_min_data = 1
    ix_max_data = 1
  /
\end{example}
And each \vn{d1} array has only one datum since the \vn{a}-mode and \vn{b}-mode tunes have only one
value associated with them (as opposed to, say an orbit which will have multiple values from
different BPMs).

In a data file there is a header section which, among other things, records the tune.
In a line beginning with the word ``\vn{Tune}''. Example:
\begin{example}
                   Horz         Vert         Sync.                           
   Tune           ( .452444)   ( .404434)   ( 0      ) 2p                    
\end{example}

In the \vn{tao_hook_command} file, after the arguments are parsed, the header part of the
data file is read to extract the tune datums:
\begin{example}
  type (tao_d2_data_array_struct), allocatable :: d2(:)
  ...
  if (line(1:4) == 'Tune') then
    call tao_find_data (err, 'tune', d2_array = d2)
    if (size(d2) /= 1) then
      call out_io (s_fatal$, r_name, 'NO TUNE D2 DATA STRUCTURE DEFINED!')
      return
    endif
\end{example}
The call to \vn{tao_find_data} looks for a \vn{d2} data structure named \vn{tune}. This structure is
setup in the \vn{tao.init} file. Alternatively, the \vn{ping_tao} program could be configured to
automatically setup the appropriate data and/or variable structures via the \vn{tao_hook_init1}
routine (\sref{s:hook.init}).

The returned value from the call to \vn{tao_find_data} is an array called \vn{d2} of type
\vn{tao_d2_data_array_struct}. \vn{d2} holds an array of pointers to all \vn{d2_data_struct}
structures it can find. In general, there could be multiple such structures if multiple universes
are being used or if the match string, in this case \vn{'tune'}, contained wild card characters. In
this case, the expectation is that there will only one universe used and thus there should be one
and only one structure that matches the name \vn{tune}. This structure will be pointed to by
\vn{d2(1)%d2}. The appropriate datums, will be:
\begin{example}
  d2(1)%d2%d1(1)%d(1)   ! a-mode tune
  d2(1)%d2%d1(1)%d(2)   ! b-mode tune
\end{example}
The values read from the data file are put in these datums via the code:
\begin{example}
  if (data_or_ref == 'data') then
    d2(1)%d2%d1(1)%d(1)%meas_value = twopi * (data_tune_a + nint(design_tune_a))
    d2(1)%d2%d1(1)%d(1)%good_meas = .true.
    d2(1)%d2%d1(2)%d(1)%meas_value = twopi * (data_tune_b + nint(design_tune_b))
    d2(1)%d2%d1(2)%d(1)%good_meas = .true.
  else
    d2(1)%d2%d1(1)%d(1)%ref_value = twopi * (data_tune_a + nint(design_tune_a))
    d2(1)%d2%d1(1)%d(1)%good_ref = .true.
    d2(1)%d2%d1(2)%d(1)%ref_value = twopi * (data_tune_b + nint(design_tune_b))
    d2(1)%d2%d1(2)%d(1)%good_ref = .true.
  endif      
\end{example}

The next step is to setup pointers to the appropriate data arrays to receive the ping data.
In the data file the ping data looks like:
\begin{example}
  BPM           Phase    Ampl.   RMSdev     Beta  bml_psi *Calib Old_Cal     
  R:HP222    -0.27314  0.46085    0.078    1.863  0.35183                         
  R:HP224    -0.05939  0.28277    0.143    0.701 -0.43442                         
  R:HP226     0.23140  0.31712    0.075    0.882 -0.14363                         
  ... etc ...
\end{example}
The ``\vn{H}'' in \vn{R:HP222}, etc. indicates that the data is from BPMs that only measure the
horizontal displacement of the beam. Alternatively, a ``\vn{V}'' would indicate data from vertical
measurement BPMs.

In the \vn{tao_hook_command} file the data pointers are setup by the code:
\begin{example}
  type (tao_d1_data_array_struct), allocatable, target :: d1_amp_arr(:), d1_phase_arr(:)
  ...
  if (line(3:3) == 'H') then
    if (mode == 'a_mode') then
      call tao_find_data (err, 'ping_a.amp_x', d1_array = d1_amp_arr)
      call tao_find_data (err, 'ping_a.phase_x', d1_array = d1_phase_arr)
    else 
      call tao_find_data (err, 'ping_b.amp_x', d1_array = d1_amp_arr)
      call tao_find_data (err, 'ping_b.phase_x', d1_array = d1_phase_arr)
    endif
  elseif (line(3:3) == 'V') then
    if (mode == 'a_mode') then
      call tao_find_data (err, 'ping_a.amp_y', d1_array = d1_amp_arr)
      call tao_find_data (err, 'ping_a.phase_y', d1_array = d1_phase_arr)
    else 
      call tao_find_data (err, 'ping_b.amp_y', d1_array = d1_amp_arr)
      call tao_find_data (err, 'ping_b.phase_y', d1_array = d1_phase_arr)
    endif
\end{example}
\vn{line(3:3)} is either \vn{H} or \vn{V} indicating horizontal or vertical orbit measuring BPMs. In
this case, the call to the \vn{tao_find_data} routine returns \vn{d1} data arrays to the amplitude
data (\vn{d1_amp_arr}) and phase data (\vn{d1_phase_arr}).  Just like the tune data, since it is
assumed only one universe is being used, there should be one and only \vn{d1} structure for the
phase and only one \vn{d1} structure for the amplitude:
\begin{example}
  d1_amp_arr(1)%d1      ! d1 struucture for the amplitude data
  d1_phase_arr(1)%d1    ! d1 struucture for the phase data
\end{example}
To save on typing, and make the code clearer, pointers are used to point to these structures:
\begin{example}
  type (tao_d1_data_struct), pointer :: d1_phase, d1_amp
  ...
  d1_amp => d1_amp_arr(1)%d1
  d1_phase => d1_phase_arr(1)%d1
\end{example}
The array of datums for the amplitude and phase data will be \vn{d1_amp%d(:)} and
\vn{d1_phase%d(:)} respectively.

After the \vn{d1_amp} and \vn{d1_phase} pointers have been set, there is a loop over all the lines
in the file to extract the ping data. One problem faced is that the order of the data in the file is
not the same as the order of the data in \vn{d1} structures.  [The data in the file is sorded in
increasing numberical order in the BPM name while the order in the \vn{d1} structures is sorted by
increasing logitudinal s-position.]  To get around this problem, the BPM name in the file is used to
locate the appropriate datum (the associated BPM element name is stored in the \vn{%ele_name}
component of the datums):
\begin{example}
  character(140) :: cmd_word(12), ele_name
  ... 
  call tao_cmd_split (line, 4, cmd_word, .false., err)
  read (cmd_word(2), *) r1
  read (cmd_word(3), *) r2
  ele_name = cmd_word(1)
  datum_amp => tao_pointer_to_datum(d1_amp, ele_name(3:))
  datum_phase => tao_pointer_to_datum(d1_phase, ele_name(3:))
\end{example}
The \vn{line} string holds a line from the data file, the call to \vn{tao_cmd_split} splits the line
into word chunks and puts them into the array \vn{cmd_word(:)}.  \vn{cmd_word(1)} holds the first
word which is the BPM name with ``\vn{R:}'' prepended to the name. The calls to
\vn{tao_pointer_to_datum} return pointers, \vn{datum_amp} and \vn{datum_phase}, to the approbriate
datums given the BPM name.

After the appropriate datums have been identified, the ping data values read from the data
file, \vn{r1} and \vn{r2}, are used to set the appropriate components:
\begin{example}
  if (data_or_ref == 'data') then
    datum_phase%good_meas = .true.
    datum_amp%meas_value = r2
    datum_amp%good_meas = .true.
  else
    datum_phase%good_ref = .true.
    datum_amp%ref_value = r2
    datum_amp%good_ref = .true.
  endif
\end{example}

One problem is that individual data phase data points can be off by factors of $2\pi$. To correct
this, the measured phase values are shifted by factors of $2\pi$ so that they are within $\pm\pi$ of
the design values. There is an added ``branch cut'' problem here in that, even without the factors
of $2\pi$ problem, the measured phases will be off from the design values by some arbitrary amount
(determined by how the zero phase is defined in the program that created the data file). If this
difference between the zero phase of the data and the zero phase of design lattice (in the design
lattice, the phase is taken to be zero at the beginning of the lattice) is close enough to $\pi$,
the shifting of the phases by factors of $2\pi$ will not be correct. For this reason, a best guess
as to what the offset is is used in the calculation to avoid the branch cut problem:
\begin{example}
  rms_best = 1e30

  do i = 1, 20
    offset = i / 20.0
    data = data + nint(design + offset - data)
    rms = sum((data - design - offset)**2, mask = ok)
    if (rms < rms_best) then
      offset_best = offset
      rms_best = rms
    endif
  enddo

  data = data + nint(design + offset_best - data)
\end{example}

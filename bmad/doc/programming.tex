\chapter{Bmad Programming Overview}
\label{c:programming}

%-----------------------------------------------------------------------------
\section{Manual Notation}
\label{s:component}

\bmad defines a number of structures and these structures may contain
components which are structures, etc. In order to keep the text in
this manual succinct when referring to components, the enclosing
structure name may be dropped. For example, the \vn{lat_struct}
structure looks like
\begin{example}
  type lat_struct
    character(40) name               
    type (mode_info_struct) a, b, z  
    type (lat_param_struct) param    
    type (ele_struct), pointer ::  ele(:)
    type (branch_struct), allocatable :: branch(:)  
    ... etc. ...
  end type
\end{example}
In this example, ``\vn{%a}'' could be used to refer to, the \vn{a}
component of the \vn{lat_struct}.  To make it explicit that this is a
component of a \vn{lat_struct}, ``\vn{lat_struct%a}'' is an alternate
possibility. Since the vast majority of structures have the
``_struct'' suffix, this may be shortened to ``\vn{lat%a}''. A similar
notation works for subcomponents. For example, a \vn{branch_struct}
looks like
\begin{example}
  type branch_struct
    character(40) name
    integer ix_from_ele                  ! Index of branching element
    integer, pointer :: n_ele_track      ! Number of tracking elements
    integer, pointer :: n_ele_max
    type (ele_struct), pointer :: ele(:) ! Element array
    ... etc. ...
  end type
\end{example}
The \vn{ele} component of the \vn{branch} component of the
\vn{lat_struct} can be referred to using ``\vn{lat%branch%ele}'',
``\vn{%branch%ele}'', or ``\vn{%ele}''. Potentially, the last of these
could be confused with the ``\vn{lat%ele}'' component so ``\vn{%ele}''
would only be used if the meaning is unambiguous in the context.
%-----------------------------------------------------------------------
\section {The Bmad Libraries}
\label{s:libs}
\index{Bmad!distribution}

The code that goes into a program based upon \bmad is divided up into a number of libraries.
The \bmad libraries are divided into two groups. One group of libraries contains ``in-house''
developed code. The other \vn{``package''} libraries consist of ``external'' code that \bmad relies
upon.

The in-house developed code libraries are:
\begin{description}
  \index{sim_utils library}
  \item[bmad] \Newline
The \vn{bmad} library contains the routines for charged particle simulation including
particle tracking, Twiss calculations, symplectic integration, etc., etc.
  \item[cpp_bmad_interface]
The \vn{cpp_bmad_interface} library is for interfacing \bmad with C++.  This library defines a set
of C++ classes corresponding to the major \bmad structures. Along with this, the library contains
conversion routines to move information between the C++ classes and the corresponding \bmad
structures.
  \item[sim_utils] \Newline
The \vn{sim_utils} library contains a set of miscellaneous helper routines.  Included are routines
for string manipulation, file manipulation, matrix manipulation, spline fitting, Gaussian random
number generation, etc.
\end{description}
%  
The \vn{package} libraries are:
\begin{description}
  \index{PTC/FPP!library}
  \item[forest] \Newline
This is the PTC/FPP (Polymorphic Tracking Code / Fully Polymorphic Package) library of \'Etienne
Forest that handles Taylor maps to any arbitrary order (this is also known as Truncated Power Series
Algebra (TPSA)). See Chapter~\ref{c:ptc.use} for more details.  FPP/PTC is a very general
package. For more information see the FPP/PTC manual\cite{b:ptc}. The core Differential Algebra (DA)
package used by PTC was developed by Martin Berz\cite{b:berz}.
%
  \index{fftw!library}
  \item[fftw] \Newline
FFTW is a C subroutine library for computing the discrete Fourier transform in one or more
dimensions. FFTW has a Fortran 2003 API.
%
  \index{gsl!library}
  \index{fgsl!library}
  \item[gsl / fgsl] \Newline
The Gnu Scientific Library (GSL), written in C, provides a wide range of mathematical routines such
as random number generators, special functions and least-squares fitting. There are over 1000
functions in total. The FGSL library provides a Fortran interface to the GSL library.
%
  \index{hdf5!library}
  \item[hdf5] \Newline
\vn{hdf5} is a library for for storing and managing data\cite{b:hdf5}. In particular, \bmad uses
this library for storing particle position data and field grid data.
%
  \index{lapack!library}
  \index{lapack95!library}
  \item[lapack / lapack95] \Newline
\vn{lapack} is a widely used package of linear algebra routines written in Fortran77. The
\vn{lapack95} library provides a Fortran95 interface to \vn{lapack}.
%
  \index{mad_tpsa!library}
  \item[mad_tpsa] \Newline
\vn{mad_tpsa} is a subset of the MAD-NG (Next Generation MAD) code\cite{b:mad.ng}. Specifically, the
\vn{mad_tpsa} library implements TPSA (Truncated Power Series Algebra). This is similar to the
\vn{FPP} code (see above). There are several advantages to using \vn{mad_tpsa} over \vn{FPP}. For one, using
\vn{mad_tpsa} is independent of \vn{PTC} so \vn{mad_tpsa} can be used along side \vn{PTC}. Another
reason is that \vn{mad_tpsa} is more flexible and better structured.
%
  \index{open_spacecharge!library}
  \item[open_spacecharge] \Newline
The \vn{open_spacecharge} library provides low energy tracking with space charge effects.
%
  \index{pgplot!library}
  \item[PGPLOT] \Newline
The \vn{pgplot} Graphics Subroutine Library is a Fortran or C-callable, device-independent graphics
package for making simple scientific graphs. Documentation including a user's manual may be obtained
from the \vn{pgplot} web site at
\begin{verbatim}
    www.astro.caltech.edu/~tjp/pgplot.
\end{verbatim} 
One disadvantage of \vn{pgplot} for the programmer is that it is not the most user friendly. To
remedy this, there is a set of Fortran90 wrapper subroutines called \vn{quick_plot}.  The
\vn{quick_plot} suite is part of the \vn{sim_utils} library and is documented in
Chapter~\ref{c:quick.plot}.
%
  \index{plplot!library}
  \item[plplot] \Newline
The \vn{plplot} library is an updated version of \vn{pgplot}. The \vn{plplot} library can be used as
a replacement for \vn{pgplot}. The \vn{quick_plot} suite, which is part of the \vn{sim_utils}
library and is documented in Chapter~\ref{c:quick.plot}, provides wrapper routines for \vn{plplot}
to make things more programmer friendly.
%
  \index{xraylib!library}
  \item[xraylib] \Newline
The xraylib library provides routines for obtaining parameters pertinent to the X-ray interaction
with matter. xraylib is developed by Tom Schoonjans and is hosted on github\cite{b:xraylib}

\end{description}

%-----------------------------------------------------------------------------
\section{Using getf and listf for Viewing Routine and Structure Documentation}
\label{s:getf}
\index{getf}
\index{listf}

As can be seen from the program example in Chapter~\ref{c:program.info}
there is a lot going on behind the scenes even for this
simple program. This shows that programming with \bmad can be both easy
and hard. Easy in the sense that a lot can be done with just a few
lines. The hard part comes about since there are many details that
have to be kept in mind in order to make sure that the subroutines
are calculating what you really want them to calculate.

To help with the details, all \bmad routines have in their source
files a comment block that explains the arguments needed by the
subroutines and explains what the subroutine does. To help quickly
access these comments, there are two Python scripts that are supplied
with the \bmad distribution that are invoked with the commands
\vn{listf} (``list function'') and \vn{getf} (``get function'').

The \vn{getf} command is used to locate routines and structures, and
to type out information on them.  The form of the command is
\begin{verbatim}
    getf <name>
\end{verbatim}
This searches for any routine or structure with the name
\vn{<name>}. \vn{<name>} may contain the wild--cards ``*'' and ``.'' where
``*'' matches to any number of characters and ``.'' matches to any
single character. For example:
\begin{example}
    getf bmad_parser
    getf lat_struct
    getf twiss_at_.
\end{example}
The third line in this example will match to the routine
\vn{twiss_at_s} but not the routine \vn{twiss_at_start}. You may or
may not have to put quotation marks if you use wild card characters.
As an example, the command \vn{getf twiss_struct} produces:
\begin{example}
  /home/cesrulib/cesr_libs/devel/cvssrc/bmad/modules/twiss_mod.f90
    type twiss_struct
      real(rp) beta, alpha, gamma, phi, eta, etap
      real(rp) sigma, emit
    end type
\end{example}
The first line shows the file where the structure is located (This is
system and user dependent so don't be surprised if you get a different
directory when you use \vn{getf}). The rest of the output shows the
definition of the \vn{twiss_struct} structure.  The result of issuing
the command \vn{getf relative_tracking_charge} is:
\begin{example}
  File: ../../bmad/modules/bmad_utils_mod.f90
  !+
  ! Function relative_tracking_charge (orbit, param) result (rel_charge)
  !
  ! Routine to determine the relative charge/mass of the particle being
  ! tracked relative to the charge of the reference particle.
  !
  ! Input:
  !   orbit -- coord_struct: Particle position structure.
  !   param -- lat_param_struct: Structure holding the reference particle id.
  !
  ! Output:
  !   rel_charge -- real(rp): Relative charge/mass
  !-
  function relative_tracking_charge (orbit, param) result (rel_charge)
\end{example}
The first line again shows in what file the subroutine is located. The rest of the output explains
what the routine does and how it can be called.

The \vn{getf} command can also be used to search for global integer and real parameter constants. For example
\begin{example}
  getf c_light
\end{example}
will give the result:
\begin{example}
  File: ../sim_utils/interfaces/physical_constants.f90
      real(rp), parameter :: c_light = 2.99792458d8              ! speed of light
\end{example}
[Global constants are constants defined in a module that have global scope (defined before the
\vn{contains} statement).] For parameters whose name ends with a dollar sign ``\vn{\$}'' character
(\sref{s:prog.conventions}), the dollar sign suffix may be omitted in the search string. For example
the search
\begin{example}
  getf quadrupole
\end{example}
will give the result
\begin{example}
  File: ../../bmad/modules/bmad_struct.f90
      integer, parameter :: drift$ = 1, sbend$ = 2, quadrupole$ = 3, group$ = 4, ...
\end{example}
Since the dollar sign is a special character for the Python regexp module used by \vn{getf}, to
include a dollar sign in the search string the dollar sign must be prefixed by three back
slashes. Thus the search
\begin{example}
  getf quadrupole\B\B\B$
\end{example}
will also locate the value of \vn{quadrupole\$}.

The \vn{listf} command is like the \vn{getf} command except that only
the file name where a routine or structure is found is printed.
The \vn{listf} command is useful if you
want to just find out where a routine or structure definition lives.
For example, the \vn{listf relative*} command would produce
\begin{example}
  File: ../../bmad/code/relative_mode_flip.f90
      function relative_mode_flip (ele1, ele2) result (rel_mode)

  File: ../../bmad/modules/bmad_utils_mod.f90
      function relative_tracking_charge (orbit, param) result (rel_charge)
\end{example}

The way \vn{getf} and \vn{listf} work is that they search a list of
directories to find the \vn{bmad}, \vn{sim_utils}, and \vn{tao}
libraries. Currently the libraries in the \bmad distribution that were
not developed at Cornell are not searched. This is primarily due to
the fact that, to save time, \vn{getf} and \vn{listf} make assumptions
about how documentation is arranged in a file and the non--Cornell libraries 
do not follow this format.

%-----------------------------------------------------------------------
\section{Precision of Real Variables}
\label{s:precision}
\index{programming!precision (rp)}
\index{rp}


Historically, \bmad come in two flavors: One version where the real
numbers are single precision and a second version with double
precision reals. Which version you are working with is controlled by
the kind parameter \vn{rp}\ (Real Precision) which is defined in the
\vn{precision_def} module. On most platforms, single precision
translates to \vn{rp}\ = 4 and double precision to \vn{rp}\ = 8. The
double precision version is used by default since round-off errors can
be significant in some calculations. Long--term tracking is an example
where the single precision version is not adequate. Changing the
precision means recompiling all the libraries except \vn{PTC} and
\vn{pgplot}.  You cannot mix and match. Either you are using the
single precision version or you are using the double precision
version. Currently, \bmad is always compiled double precision and it
is a near certainty that there would have to be some fixes if there
was ever a need for compiling single precision.

To define floating point variables in Fortran with the correct precision,
 use the syntax {\tt ``real(rp)''}. For example:
\begin{example}
    real(rp) var1, var2, var3
\end{example}
When you want to define a literal constant, for example to pass an
argument to a subroutine, add the suffix \vn{_rp} to the end of the
constant. For example
\begin{example}
   var1 =  2.0_rp * var2
   call my_sub (var1, 1.0e6_rp)
\end{example}
Note that \vn{2_rp} is different from \vn{2.0_rp}. \vn{2_rp} is an
integer of kind \vn{rp}, not a real.

Independent of the setting of \vn{rp}, the parameters \vn{sp} and
\vn{dp} are defined to give single and double precision numbers
respectively.

%-----------------------------------------------------------------------------
\section{Programming Conventions}
\label{s:prog.conventions}
\index{programming!conventions}

\bmad subroutines follow the following conventions:

\begin{description}

\index{\$!character to denote a parameter}
\item[A ``\$'' suffix denotes a parameter:] 
A dollar sign ``\$'' at the end of a name denotes an 
parameter. For example, in the above program, to check
whether an element is a quadrupole one would write:
\begin{verbatim}
  if (lat%ele(i)%key == quadrupole$) ...
\end{verbatim}
Checking the source code one would find in the module \vn{bmad_struct}
\begin{verbatim}
  integer, parameter :: drift$ = 1, sbend$ = 2, quadrupole$ = 3, group$ = 4
\end{verbatim}
One should always use the parameter name instead of the integer it represents.
That is, one should never write
\begin{verbatim}
  if (lat%ele(i)%key == 3) ...  ! DO NOT DO THIS!
\end{verbatim}
For one, using the name makes the code clearer. However, more
importantly, the integer value of the parameters may at times be
shuffled for practical internal reasons. The use of the integer value
could thus lead to disastrous results. 

By convention all names ending in ``\$'' are parameters. And most ``dollar sign'' parameters are
integers but there are exceptions. For example, the parameter \vn{real_garbage\$} is a real number.
To find the value of a dollar sign parameter, the \vn{getf} or \vn{listf} (\sref{s:getf}) commands
can be used.

\index{structures}
\item[Structure names have a ``_struct'' suffix:]
For example: \vn{lat_struct}, \vn{ele_struct}, etc. Structures without a 
\vn{_struct} are usually part of \'Etienne's PTC/FPP package.

\end{description}

%-----------------------------------------------------------------------
\section {Using Modules}
\label{s:modules}
\index{modules}

When constructing a program unit,\footnote
  {
A program unit is a module such as a subroutine, function, or program.
  }
the appropriate \vn{use} statement(s) need to appear at the top of the unit to import the
appropriate modules. This should be a simple matter but, due to historical reasons, \bmad module
dependencies are somewhat convoluted. At some point in the future this will be straightened out
but, for now, the following serves as a guide on how to handle things.

In many cases, a ``\vn{use bmad}'' statement is all that is needed. If there is a problem, an error
message will be generated at the program linking stage. The error message might look like:
\begin{example}
  [100%] Linking Fortran executable /home/dcs16/linux_lib/production/bin/test
  CMakeFiles/test-exe.dir/test.f90.o: In function `MAIN__':
  test.f90:(.text+0x22e): undefined reference to `taylor_coef_'
  gmake[2]: *** [/home/dcs16/linux_lib/production/bin/test] Error 1
  gmake[1]: *** [CMakeFiles/test-exe.dir/all] Error 2
\end{example}
The ``\vn{undefined reference}'' line shows that the subroutine \vn{taylor_coef} was
called in the program (``\vn{MAIN}'') without the necessary \vn{use} statement. To find the correct \vn{use}
statement, use the \vn{getf} or \vn{listf} command (\sref{s:getf}) to find the file that the
subroutine is in. Example:
\begin{example}
  > listf taylor_coef

  File: ../bmad/modules/taylor_mod.f90
      subroutine taylor_coef (bmad_file, lat, err, output_form, orbit0)
\end{example}
This shows that the subroutine lives in the file \vn{taylor_mod.f90}. The coding rule that
\bmad follows is that the module name is the file name minus the \vn{.f90} suffix.\footnote
  {
Be aware. PTC code does not follow this logic.
  }
Thus the needed \vn{use} statement in this case is:
\begin{example}
  use taylor_mod
\end{example}

Note: if the linking error looks like:
\begin{example}
  [100%] Linking Fortran executable /home/dcs16/linux_lib/production/bin/test
  CMakeFiles/test-exe.dir/test.f90.o: In function `MAIN__':
  test.f90:(.text+0x230): undefined reference to 
                                     `taylor_mod_mp_taylor_coef_'
\end{example}
here the error message references 
\begin{example}
  taylor_mod_mp_taylor_coef_
\end{example}
The ``\vn{taylor_mod_mp}'' prefix shows that the linker knows to look in the module
\vn{taylor_mod} for the subroutine. Here the problem is not a missing use statement but
rather the problem is that the linker cannot find the module itself. This indicates something like a
library is missing in the list of libraries to link to or the order of libraries to link to is
wrong. This kind of problem generally happens with libraries other than the \bmad library
itself. Further documentation on how the list of libraries to link to is defined is contained on the
\bmad web site\cite{b:bmad.web}.

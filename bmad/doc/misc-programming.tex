\chapter{Miscellaneous Programming}

%-----------------------------------------------------------------------------
\section{Custom and Hook Routines}
\label{s:custom.hook}
\index{custom}

\bmad calculations, like particle tracking through a lattice element, can be customized using what
are called ``\vn{custom}'' and ``\vn{hook}'' routines. The general idea is that a programmer can
implement custom code which is linked into a program and this custom code will be called at the
appropriate time by \bmad. For example, custom code can be created for Runge-Kutta tracking that
calculates the electromagnetic field of some complicated electromagnet. Prototype custom and hook
routines are available in the \vn{/bmad/custom} directory and are discussed in detail below.

To enable \bmad to be able to call customized code, \vn{function pointers} are defined, one for each
custom or hook routine. At certain places in the \bmad code, the appropriate function pointer will
be checked. If the function pointer is associated with a routine, that routine will be called. By
default, the function pointers are not associated with any functions and the only way there will be
an association by user code modification.

The function pointers are defined in the file \vn{/bmad/modules/bmad_routine_interface.f90}. The
convention followed is that for any given custom or hook routine there is a base name, for example,
\vn{track1_custom}, and in \vn{bmad_routine_interface.f90} there will be an abstract interface which
with the base name with a \vn{_def} suffix (\vn{track1_custom_def} for this example). Additionally
the corresponding function pointer uses a \vn{_ptr} suffix and is defined like:
\begin{example}
  procedure(track1_custom_def), pointer :: track1_custom_ptr => null()
\end{example}

To implement custom code for, say, \vn{track1_custom}:
\begin{itemize}
%
\item 
Copy the file \vn{/bmad/custom/track1_custom.f90} to the area where the program is to be compiled.
%
\item
Customize file as desired. The name of the routine can be changed if desired as long as that name is used
consistently throughout the program. In fact, multiple custom
routines can be created and switched in and out as desired in the program.
%
\item
In the program, define an interface for the custom routine like:
\begin{example}
  procedure(track1_custom_def) :: track1_custom
\end{example}
This procedure statement must go in the declaration section above the executable section of the code.
Note: If the custom routine has been put in a module this will not be needed.
%
\item
Somewhere near the beginning of the program (generally before \vn{bmad_parser} is called), set the
function pointer to point to your custom routine:
\begin{example}
  track_custom_ptr => track1_custom
\end{example}
%
\item
If needed modify the compile script(s) (typically named something like ``cmake.XXX'') to compile the
file the custom routine is in.
%
\item
Compile the program using the \vn{mk} command.
%
\end{itemize}

While coding a custom routine, it is important to remember that it is {\em not} permissible to
modify any routine argument that does not appear in the list of output arguments shown in the
comment section at the top of the file.

Note: Custom and hook entry points are added to \bmad on an as-needed basis. If you have a need that
is not met by the existing set of entry points, please contact a \bmad maintainer.

Note: The custom and hook routines in \vn{/bmad/custom/} are not compiled with the \bmad library. Their
only purpose is to serve as prototypes for code development.

%-----------------------------------------------------------------------------
\section{Custom Calculations}
\label{s:custom.ele}
\index{custom}

There are essentially two ways to do \vn{custom} (as opposed to \vn{hook}) calculations. One way
involves using a \vn{custom} element (\sref{s:custom}). The other way involves setting the
appropriate \vn{method} component of an element to \vn{custom}. An appropriate method component is
one of
\begin{example}
  tracking_method       \sref{s:tkm}
  mat6_calc_method      \sref{s:xfer}
  field_calc            \sref{s:integ}
  aperture_type         \sref{s:limit}
\end{example}

There are eight routines that implement custom calculations:
\begin{example}
  \Hyperref{r:check.aperture.limit.custom}{check_aperture_limit_custom}
  \Hyperref{r:em.field.custom}{em_field_custom}
  \Hyperref{r:init.custom}{init_custom}
  \Hyperref{r:make.mat6.custom}{make_mat6_custom}
  \Hyperref{r:radiation.integrals.custom}{radiation_integrals_custom}
  \Hyperref{r:track1.custom}{track1_custom}
  \Hyperref{r:track1.spin.custom}{track1_spin_custom}
  \Hyperref{r:wall.hit.handler.custom}{wall_hit_handler_custom}
\end{example}
[Use \vn{getf} for more details about the argument lists for these
routines.]  

\index{descrip}
The \Hyperref{r:init.custom}{init_custom} routine is called by \Hyperref{r:bmad.parser}{bmad_parser}
at the end of parsing for any lattice element that is a \vn{custom} element or has set any one of
the element components as listed above to \vn{custom}. The \vn{init_custom} routine can be used to
initialize the internals of the element. For example, consider a \vn{custom} element defined in a
lattice file by
\begin{example}
  my_element: custom, val1 = 1.37, descrip = "field.dat", mat6_calc_method = tracking
\end{example}
In this example, the \vn{descrip} (\sref{s:alias}) component is used
to specify the name of a file that contains parameters for this
element. When \vn{init_custom} is called for this element (see below),
the file can be read and the parameters stored in the element
structure. Besides the \vn{ele%value} array, parameters may be stored
in the general use components given in \sref{s:ele.gen}.

The \Hyperref{r:make.mat6.custom}{make_mat6_custom} routine is called by the
\Hyperref{r:track1}{track1} routine when calculating the transfer matrix through an element.

The \Hyperref{r:track1.custom}{track1_custom} routine is called by the \Hyperref{r:track1}{track1}
routine when the \vn{tracking_method} for the element is set to \vn{custom}. Further customization
can be set by the routines \Hyperref{r:track1.preprocess}{track1_preprocess} and
\Hyperref{r:track1.postprocess}{track1_postprocess}. See Section~\sref{s:hook} for more details.

A potential problem with \vn{track1_custom} is that the calling routine, that is \vn{track1}, does
some work like checking aperture, etc. (see the \vn{track1} code for more details). If this is not
desired, the \vn{track1_preprocess} routine (\sref{s:hook}) can be used to do custom tracking and to
make sure that \vn{track1} does not do any extra calculations. This is accomplished by putting the
custom tracking code in \vn{track1_preprocess} and by setting the \vn{finished} argument of
\vn{track1_preprocess} to True.

The \Hyperref{r:check.aperture.limit.custom}{check_aperture_limit_custom} routine is used to check
if a particle has hit an aperture in tracking. It is called by the standard \bmad routine
\Hyperref{r:check.aperture.limit}{check_aperture_limit} when \vn{ele%aperture_type} is set to
\vn{custom\$}. A \vn{custom} element has the standard limit attributes (\sref{s:limit}) so a
\vn{custom} element does not have to implement custom aperture checking code.

The \Hyperref{r:em.field.custom}{em_field_custom} routine is called by the electro-magnetic field
calculating routine \Hyperref{r:em.field.calc}{em_field_calc} when \vn{ele%field_calc} is set to
\vn{custom\$}. As an alternative to \vn{em_field_custom}, a \vn{custom} element can use a field map
(\sref{s:fieldmap}) to characterize the element's electromagnetic fields.

Note: When tracking through a \vn{patch} element, the first step is to transform the particle's
coordinates from the entrance frame to the exit frame. This is done since it simplifies the
tracking. [The criterion for stopping the propagation of a particle through a \vn{patch} is that the
particle has reached the exit face and the calculation to determine if a particle has reached the
exit face is simplified if the particle's coordinates are expressed in the coordinate frame of the
exit face.] Thus for \vn{patch} elements, unlike all other elements, the particle coordinates passed
to \Hyperref{r:em.field.custom}{em_field_custom} are the coordinates with respect to the exit
coordinate frame and not the entrance coordinate frame. If field must be calculated in the entrance
coordinate frame, a transformation between entrance and exit frames must be done:
\begin{example}
  subroutine em_field_custom (ele, param, s_rel, time, orb, &
                                  local_ref_frame, field, calc_dfield, err_flag)
  use lat_geometry_mod
  ...
  real(rp) w_mat(3,3), w_mat_inv(3,3), r_vec(3), r0_vec(3)
  real(rp), pointer :: v(:)
  ...
  ! Convert particle coordinates from exit to entrance frame.
  v => ele%value   ! v helps makes code compact
  call floor_angles_to_w_mat (v(x_pitch\$), v(y_pitch\$), v(tilt\$), w_mat, w_mat_inv)
  r0_vec = [v(x_offset\$), v(y_offset\$), v(z_offset\$)]
  r_vec = [orb%vec(1), orb%vec(3), s_rel]  ! coords in exit frame
  r_vec = matmul(w_mat, r_vec) + r0_vec      ! coords in entrance frame

  ! Calculate field and possibly field derivative
  ...

  ! Convert field from entrance to exit frame
  field%E = matmul(w_mat_inv, field%E)
  field%B = matmul(w_mat_inv, field%B)
  if (logic_option(.false., calc_dfield)) then
    field%dE = matmul(w_mat_inv, matmul(field%dE, w_mat))
    field%dB = matmul(w_mat_inv, matmul(field%dB, w_mat))
  endif
\end{example}

The \Hyperref{r:wall.hit.handler.custom}{wall_hit_handler_custom} routine is called when the
Runge-Kutta tracking code \Hyperref{r:odeint.bmad}{odeint_bmad} detects that a particle has hit a
wall (\sref{s:wall}). [This is separate from hitting an aperture that is only defined at the
beginning or end of an lattice element.] The dummy \vn{wall_hit_handler_custom} routine does
nothing.  To keep tracking, the particle must be marked as alive
\begin{example}
  subroutine wall_hit_handler_custom (orb, ele, s, t)
    ...
    orb%state = alive\$   ! To keep on truckin'
    ...
\end{example}
Note: \vn{odeint_bmad} normally does not check for wall collisions.  To change the default behavior,
the \vn{runge_kutta_com} common block must modified. This structure is defined in
\vn{runge_kutta_mod.f90}:
\begin{example}
  type runge_kutta_common_struct
    logical :: check_wall_aperture = .false.
    integer :: hit_when = outside_wall\$   ! or wall_transition\$
  end type

  type (runge_kutta_common_struct), save :: runge_kutta_com
\end{example}
To check for wall collisions, the \vn{%check_wall_aperture} component must be set to true. The
\vn{%hit_when} components determines what constitutes a collision. If this is set to
\vn{outside_wall\$} (the default), then any particle that is outside the wall is considered to have
hit the wall. If \vn{%hit_when} is set to \vn{wall_transition\$}, a collision occurs when the
particle crosses the wall boundary. The distinction between \vn{outside_wall\$} and
\vn{wall_transition\$} is important if particles are to be allowed to travel outside the wall.

%-----------------------------------------------------------------------------
\section{Hook Routines}
\label{s:hook}

\index{track1_preprocess}\index{track1_postprocess}
\index{apply_element_edge_kick_hook}
\index{ele_geometry_hook}
\index{ele_to_fibre_hook}
A \vn{hook} routine is like a \vn{custom} routine in that a \vn{hook} routine can be used for
customizing a \bmad calculation by replacing the \vn{dummy} version of a \vn{hook} routine with
customized code. The difference is that the \vn{hook} routine is always called at the appropriate
time without regard to the type of lattice element under consideration or what tracking method is
being used.  The \vn{hook} routines that are available are:
\begin{example}
  \Hyperref{r:apply.element.edge.kick.hook}{apply_element_edge_kick_hook}
  \Hyperref{r:ele.geometry.hook}{ele_geometry_hook}
  \Hyperref{r:ele.to.fibre.hook}{ele_to_fibre_hook}
  \Hyperref{r:time.runge.kutta.periodic.kick.hook}{time_runge_kutta_periodic_kick_hook}
  \Hyperref{r:track1.bunch.hook}{track1_bunch_hook}
  \Hyperref{r:track1.preprocess}{track1_preprocess}
  \Hyperref{r:track1.postprocess}{track1_postprocess}
  \Hyperref{r:track1.wake.hook}{track1_wake_hook}
\end{example}

The \vn{apply_element_edge_kick_hook} routine can be used for custom tracking through a fringe field.
See the documentation in the file \vn{apply_element_edge_kick_hook.f90} for more details.

The \vn{ele_geometry_hook} routine can be used for custom calculations of the global geometry of an
element. This is useful, for example, for a support table on a kinematic mount since \bmad does not
have the knowledge to calculate the table orientation from the position of the mount points. See the
documentation in the file \vn{ele_geometry_hook.f90} for more details.

The {ele_to_fibre_hook} routine can be used to customize how the PTC fibre corresponding to a
\bmad lattice element is constructed. 

The \vn{time_runge_kutta_periodic_kick_hook} routine can be used to introduce a time dependent kick
when doing tracking with \vn{time_runge_kutta}. This routine could be used, for example, to add the
kick due to a passing beam ! on a residual gas ion that is being tracked. See the documentation in
the file \vn{time_runge_kutta_periodic_kick_hook.f90} for more details.

The \vn{track1_bunch_hook} routine can be used for custom bunch tracking through an element.

The \vn{track1_preprocess} and \vn{track1_postprocess} routines are called by the
\Hyperref{r:track1}{track1} routine. [Additionally, if the element being tracked through has its
tracking method set to \vn{custom}, the \vn{track1_custom} routine is called.] The
\vn{track1_preprocess} and \vn{track1_postprocess} routines are useful for a number of things. For
example, if the effect of an electron cloud is to be modeled, these two routines can be used to put
in half the electron cloud kick at the beginning of an element and half the kick at the end.

The routine \vn{track1_preprocess} has an additional feature in that it has an argument
\vn{radiation_included} that can be set to \vn{True} if the routine \vn{track1_custom} will be
called and \vn{track1_custom} will be handling radiation damping and excitation effects.

The \vn{track1_wake_hook} can be used to apply custom wakes.

%-----------------------------------------------------------------------------
\section{Nonstandard Init_Custom Use}

For back compatibility reasons, a nonstandard method for using \vn{init_custom} is supported.
This method involves not bothering to set the \vn{init_custom_ptr} (\sref{s:custom.hook})
and simply including a custom \vn{init_custom} routine that is included in the list
of routines to link to. The problem with this method is that it is not guaranteed to work
on all platforms. In particular, this method is known to fail with MacOS and the \vn{gcc}
compiler. Due to this, this method is discouraged. 

%-----------------------------------------------------------------------------
\section{Physical and Mathematical Constants}
\label{s:physical.constants}

\index{constants}
Common physical and mathematical constants that can be used in any expression
are defined in the file:
\begin{example}
 sim_utils/interfaces/physical_constants.f90
\end{example}

The following constants are defined
\begin{example}
  pi = 3.14159265358979d0
  twopi = 2 * pi
  fourpi = 4 * pi
  sqrt_2 = 1.41421356237310d0
  sqrt_3 = 1.73205080757d0
  complex: i_imaginary = (0.0d0, 1.0d0)

  e_mass = 0.51099906d-3   ! DO NOT USE! In GeV
  p_mass   = 0.938271998d0   ! DO NOT USE! In GeV

  m_electron = 0.51099906d6  ! Mass in eV
  m_proton   = 0.938271998d9 ! Mass in eV

  c_light = 2.99792458d8             ! speed of light
  r_e = 2.8179380d-15                ! classical electron radius
  r_p = r_e * m_electron / m_proton  ! proton radius
  e_charge = 1.6021892d-19           ! electron charge

  h_planck = 4.13566733d-15          ! eV*sec Planck's constant
  h_bar_planck = 6.58211899d-16      ! eV*sec h_planck/twopi

  mu_0_vac = fourpi * 1e-7                   ! Permeability of free space
  eps_0_vac = 1 / (c_light**2 * mu_0_vac)    ! Permittivity of free space

  classical_radius_factor = r_e * m_electron ! Radiation constant

  g_factor_electron = 0.001159652193    ! Anomalous gyro-magnetic moment
  g_factor_proton   = 1.79285           ! Anomalous gyro-magnetic moment
\end{example}

%-----------------------------------------------------------------------------
\section{Global Coordinates and S-positions}
\label{s:global.coords}

\index{global coordinates}
\index{s-positions}
The routine \Hyperref{r:lat.geometry}{lat_geometry} will compute the global floor coordinates at the
end of every element in a lattice.  \vn{lat_geometry} works by repeated calls to
\Hyperref{r:ele.geometry}{ele_geometry} which takes the floor coordinates at the end of one element
and calculates the coordinates at the end of the next. For conversion between orientation matrix
$\Bf W$ (\sref{s:global}) and the orientation angles $\theta, \phi, \psi$, the routines
\Hyperref{r:floor.angles.to.w.mat}{floor_angles_to_w_mat} and
\Hyperref{r:floor.w.mat.to.angles}{floor_w_mat_to_angles} can be used.

The routine \Hyperref{r:s.calc}{s_calc} calculates the longitudinal $s$ positions for the elements
in a lattice.

%-----------------------------------------------------------------------------
\section{Reference Energy and Time}
\label{s:ref.energy.prog}

\index {reference energy}
\index{lcavity!reference energy}\index{custom!reference energy}\index{hybrid!reference energy}
The reference energy and time for the elements in a lattice is calculated by
\Hyperref{r:lat.compute.ref.energy.and.time}{lat_compute_ref_energy_and_time}.  The reference energy
associated with a lattice element is stored in
\begin{example}
  ele%value(E_tot_start\$)   ! Total energy at upstream end of element (eV)
  ele%value(p0c_start\$)     ! Momentum * c_light at upstream end of element (eV)
  ele%value(E_tot\$)         ! Total energy at downstream end (eV)
  ele%value(p0c\$)           ! Momentum * c_light at downstream end(eV)
\end{example}
Generally, the reference energy is constant throughout an element so that \vn{%value(E_tot_start\$}
= \vn{%value(E_tot\$} and \vn{%value(p0c_start\$} = \vn{%value(p0c\$}. Exceptions are elements of
type:
\begin{example}
  custom,
  em_field,
  hybrid, or
  lcavity
\end{example}
In any case, the starting \vn{%value(E_tot_start\$} and \vn{%value(p0c_start\$} values of a given
element will be the same as the ending \vn{%value(E_tot\$} and \vn{%value(p0c\$} energies of the
previous element in the lattice.

The reference time and reference transit time is stored in
\begin{example}
  ele%ref_time                ! Ref time at downstream end
  ele%value(delta_ref_time\$)
\end{example}

The reference orbit for computing the reference energy and time is
stored in
\begin{example}
  ele%time_ref_orb_in        ! Reference orbit at upstream end
  ele%time_ref_orb_out       ! Reference orbit at downstream end
\end{example}
Generally \vn{ele%time_ref_orb_in} is the zero orbit. The exception comes when an element is a
\vn{super_slave}. In this case, the reference orbit through the super_slaves of a given
\vn{super_lord} is constructed to be continuous. This is done for consistency sake. For example, to
ensure that when a marker is superimposed on top of a wiggler the reference orbit, and hence the
reference time, is not altered.

\index{group!reference energy}\index{overlay!reference energy}
\index{superposition!reference energy}
\vn{group} (\sref{s:group}), \vn{overlay} (\sref{s:overlay}), and \vn{super_lord} elements inherit
the reference from the last slave in their slave list (\sref{s:lat.control}). For \vn{super_lord}
elements this corresponds to inheriting the reference energy of the slave at the downstream end of
the \vn{super_lord}. For \vn{group} and \vn{overlay} elements a reference energy only makes sense if
all the elements under control have the same reference energy.

Additionally, photonic elements like \vn{crystal}, \vn{capillary},
\vn{mirror} and \vn{multilayer_mirror} elements have an associated photon reference wavelength
\begin{example}
  ele%value(ref_wavelength\$)      ! Meters.
\end{example}

%-----------------------------------------------------------------------------
\section{Global Common Structures}
\label{s:com.struct}

\index{bmad_com}
\index{global_com}
There are two common variables used by Bmad for communication between routines. These are
\vn{bmad_com}, which is a \vn{bmad_common_struct} structure, and \vn{global_com} which is a
\vn{global_common_struct} structure. The \vn{bmad_com} structure is documented in
Section~\sref{s:bmad.ptc.com}.

\index{global_common_struct|hyperbf}
The \vn{global_common_struct} is meant to hold common parameters that should
not be modified by the user. 
\begin{example}
  type global_common_struct
    logical mp_threading_is_safe = T    ! MP threading safe?
    logical exit_on_error  = T          ! Exit program on error?
  end type
\end{example}
A global variable \vn{global_com} is defined in the \vn{sim_utils} library:
\begin{example}
  type (global_common_struct), save :: global_com
\end{example}
And various routines use the settings in \vn{global_com}.

\begin{description}
\item[\vn{\%mp_threading_is_safe}] \Newline
Toggle to prevent MP threading optimizations from being done. See Sec.~\sref{s:parallel.proc} for
more details.
\item[\vn{\%exit_on_error}] \Newline
The \vn{%exit_on_error} component tell a routine if it is OK to stop a program on a severe
error. Stopping is generally the right thing when a program is simply doing a calculation and
getting a wrong answer is not productive. In control system programs and in interactive programs
like \vn{Tao}, it is generally better not to stop on an error.
\end{description}

%-----------------------------------------------------------------------------
\section{Parallel Processing}
\label{s:parallel.proc}

\index{global_common_struct!parallel processing}
\index{global_com!parallel processing}
\bmad was initially developed without regard to parallel processing. When a demand for
multithreading capability arose, \bmad was modified to meet the need and uses both MP
(Multi-Processing) and MPI (Message Passing Interface) type threading. And sometimes both will be
used within the same program.

The general rule at present is that \bmad can be run multi-threaded as long as either lattice
parameters are not varied or an array of lattices is used, one for each thread. Thus multi-threading
with MPI is generally thread safe since, by default, different MPI threads do not share memory. With
MP, things are more complicated.  For example, tracking a particle through a lattice is generally
thread safe with MP. The exception is if there are \vn{ramper} (\sref{s:ramper}) elements since
ramping involves the modifying lattice element parameters while tracking and is thus not MP thread
safe. Another exception is that PTC code (\sref{c:ptc.use}) is not thread safe.

In order to signal routines that have MP code whether it is safe to using threading, there is the
global switch \vn{global_com%mp_threading_is_safe}. The default is True. 

%-----------------------------------------------------------------------------
\section{Expression Trees and Expression Stacks)

Expression trees and expression stacks are how \bmad and \tao handle arithmetical expressions. There
is a bit of history here. Originally, there was only expression stacks. An expression stack is an
array of \vn{expression_atom_struct}s. The main routines to manipulate stacks are:
\begin{example}
  expression_string_to_stack  ! Convert expression string to a reverse Polish stack.
  express_stack_value         ! Evaluate expression stack.
  expression_value            ! Combines expression_string_to_stack and expression_stack_value.
  expression_stack_to_string  ! Creates an expression string from a stack.
\end{example}

After the stack code was developed, and when \tao was created, since the needs of \tao were more
extensive (needed to handle arrays, needed to handle variable, data and lattice element names that
used the asterisk as a wild card while, at the same time the asterisk was used as the multiplication
symbol), a different evaluation method using expression trees was implemented. Note: While trees are
more flexible than stacks, they also are more complicated and for this reason, and the fact that
stacks work perfectly well for \bmad, the stacks that are built into \bmad have been left untouched.

The main routines to manipulate trees are:
\begin{example}
  expression_string_to_tree  ! Convert expression string to tree form.
  type_expression_tree       ! Print an expression tree in tree form.
  expression_tree_to_string  ! Convert a tree to a string.
\end{example}
Note These routines are in \bmad. \tao implements other routines. for evaluation.
 

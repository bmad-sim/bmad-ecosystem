\chapter{Wakefields}
\label{c:wakefields.eq}
\index{wakefields|hyperbf}

Wakefield effects are divided into short--range (within a bunch) and long--range (between
bunches). Short--range wakes are described in \Sref{s:sr.wake.eq} and long--range wakes are
described in \Sref{s:lr.wake.eq}. The syntax for describing wakes in a lattice file is given in
\Sref{s:wakes}.

%-----------------------------
\section{Short--Range Wakes}
\label{s:sr.wake.eq}
\index{wakefields!short-range}

The syntax for assigning short--range wakes to a lattice element is described in
\Sref{s:sr.wake.syntax}. Only the monopole and dipole wakefields are modeled.

Short--range wakes are divided into three classes: Those that are dependent linearly upon transverse
offset of the leading particle (but independent of the position of the trailing particle), those
that are dependent linearly upon the transverse offset of the trailing particle (but independent of
the position of the leading particle, and those wakes that are independent of the offset.

The longitudinal monopole energy kick $dE$ for the $i$\Th (trailing) macroparticle due to the wake
from the $j$\Th (leading) macroparticle, assuming the kick is independent of the transverse
positions, is computed from the equation
\begin{equation}
  \Delta p_z(i) = \frac{-e \, L}{v \, P_0} \, \Bigl( \frac{1}{2}\WlS(0) \,  |q_i| +
        \sum_{j \ne i} \WlS(dz_{ij}) \, |q_j| \Bigr)
  \label{delvp1}
\end{equation}
where $v$ is the particle velocity, $e$ is the charge on an electron, $q$ is the macroparticle
charge, $L$ is the cavity length, $dz_{ij}$ is the longitudinal distance of the $i$\Th particle
with respect to the $j$\Th macroparticle (positive $z$ means $i$\Th particle is in front of the
$j$\Th particle), $\WlS$ is the short--range longitudinal wakefield function.

If the beam chamber has azimuthal symmetry the energy kick will be independent of the transverse
positions. If this is not true, there can be some dependence. There are four cases that \bmad
simulates: Linear in the $x$ or $y$-position of the leading particle, or linear in the $x$ or
$y$-position of the trailing particle. For example, if the kick is linear in the $x$-position of the
leading particle the kick is
\begin{equation}
  \Delta p_z(i) = \frac{-e \, L \, x_i}{v \, P_0} \, \Bigl( \frac{1}{2}\WlS(0) \,  |q_i| +
        \sum_{j \ne i} \WlS(dz_{ij}) \, |q_j| \Bigr)
  \label{delvp2}
\end{equation}
And if the kick is linear in the $y$-position of the trailing particle the kick is
\begin{equation}
  \Delta p_z(i) = \frac{-e \, L}{v \, P_0} \, \Bigl( \frac{1}{2}\WlS(0) \,  |q_i| \, y_i +
        \sum_{j \ne i} \WlS(dz_{ij}) \, |q_j| \, y_j \Bigr)
  \label{delvp3}
\end{equation}

The kick $\Delta p_x(i)$ due to the transverse wake for the $i$\Th particle is modeled with the
equation
\begin{equation}
  \Delta p_x(i) = \frac{-e \, L \, \sum_j |q_j| \, x \, \WtS(dz_{ij})}{v \, P_0}
  \label{pelqxw1}
\end{equation}
Where $\WtS$ is the transverse short--range wake function and $x$ is the horizontal displacement of
the leading or trailing particle as appropriate. There is a similar equation for $\Delta p_y(i)$.
If the beam chamber has azimuthal symmetry, the only wakes present are those that are dependent upon
the offset of the leading particle. If the transverse wake is modeled as being independent of
position the above equation is modified:
\begin{equation}
  \Delta p_x(i) = \frac{-e \, L \, \sum_j |q_j| \, \WtS(dz_{ij})}{v \, P_0}
  \label{pelqxw2}
\end{equation}


With either the longitudinal wake $\WlS$ or the transverse $\WtS$ wake, the wake can be
approximated as a sum of what are called ``pseudo'' modes $W_i(z)$, $i = 1 \ldots M$:
\begin{equation}
  W(z) = A_a \, \sum_{i = 1}^M W_i(z)
  = A_a \, \sum_{i = 1}^M A_i \, e^{d_i z} \, \sin (k_i \, z + \phi_i)
  \label{wadzk}
\end{equation}
This is similar to approximating any function as a sum of Fourier terms. The parameters $(A_i, d_i,
k_i, \phi_i)$ are chosen by the person constructing the lattice to fit the calculated wake
potential. Since $z$ is negative for trailing particles, $d_i$ should be positive to get the wake to
decay exponentially with distance. The dimensionless overall amplitude scale $A_a$ is introduced
as a convenient way to scale the overall wake. The reason why the pseudo mode approach is used in \bmad is
due to the fact that, with pseudo modes, the calculation time scales as the number of particles $N$
while a calculation based upon a table of wake vs $z$ would scale as $N^2$. [The disadvantage is
that initially must perform a fit to the wake potential to generate the mode parameter
values.]

%-----------------------------
\section{Long--Range Wakes}
\label{s:lr.wake.eq}
\index{wakefields!long-range}

The lattice syntax for defining long-range wakes is discussed in \Sref{s:lr.wake.syntax}.

Following Chao\cite{b:chao} Eq.~2.88, the long--range wakefields are characterized by a set of
cavity modes. The wake function $W_i$ for the $i$\Th mode is
\begin{equation}
  W_i(t) = -c \, A_a \, \left( \frac{R}{Q} \right)_i \,\,
  \exp(-d_i \, t) \, \sin (\omega_i \, t + \phi_i)
  \label{wcrq}
\end{equation}
The order of the mode $m_i$ does not come into this equation but will appear in equations below.
The dimensionless overall amplitude scale $A_a$ is introduced as a convenient way to scale the
amplitude of all the wakes with just one parameter. Normally, for a wake that has a well defined
mode, The phase factor $\phi_i$ is zero. Finite $\phi_i$ is used for simulations of such things as
the long-range resistive wall wake. In this case, the resistive wall wake needs to be modeled as the
sum of a number of modes since the resistive wall wake is not well modeled by a single damped
sinusoid.

The mode strength $(R/Q)_i$ in the above equation has units of Ohms/meter$^{2m_i}$. Notice that
$R/Q$ is defined so that it includes the cavity length. Thus the long--range wake equations, as
opposed to the short--range ones, do not have any explicit dependence on $L$. To make life more
interesting, different people define $R/Q$ differently. A common practice is to define an $R/Q$ ``at
the beam pipe radius''. In this case the above equations must be modified to include factors of the
beam pipe radius. Another convention uses a ``linac definition'' which makes $R/Q$ twice as large
and adds a factor of 2 in \Eq{wcrq} to compensate.

Note: Originally, \bmad characterized the damping factor $d_i$ using the quality factor $Q_i$ via the
relationship
\begin{equation}
  d_i = \frac{\omega_i}{2 \, Q_i}
  \label{do2q}
\end{equation}
This proved to be inconvenient when modeling such things as the resistive wall wake (where it is convenient
to have modes where $\omega_i = 0$) so the lattice file syntax was modified to use $d_i$ directly.

Assuming that the macroparticle generating the wake is offset a distance $r_w$ along the $x$--axis,
a trailing macroparticle at transverse position $(r, \theta)$ will see a kick
\begin{align}
  \Delta {\Bf p}_\perp &= 
    -C \, I_m \, W(t) \, m \, r^{m-1} \, \left( 
    \bfhat r \cos m\theta - {\bf\hat\theta} \sin m\theta \right) \\
  &= -C \, I_m \, W(t) \, m \, r^{m-1} \, \left( \bfhat x \cos [(m-1) \theta] - 
    \bfhat y \sin [(m-1)\theta] \right) \CRNO
  \Delta p_z &= -C \, I_m \, W'(t) \, r^m \, \cos m\theta
\end{align}
where in this, and other equations below, the subscript $i$ has been dropped. $C$ is given by
\begin{equation}
  C = \frac{e}{c \, P_0}
\end{equation}
 and
\begin{equation}
  I_m = q_w \, r_w^m
\end{equation}
with $q_w$ being the magnitude of the charge on the particle.  Generalizing the above, a
macroparticle at $(r_w, \theta_w)$ will generate a wake
\begin{align}
  -\Delta p_x + i\Delta p_y &= C \, I_m \, W(t) \, 
    m \, r^{m-1} \, e^{-i m \theta_w} \, e^{i (m-1) \theta} 
    \label{ppcimr} \\
  \Delta p_z &= C \, I_m \, W'(t) \, r^m \, \cos [m(\theta - \theta_w)]
    \label{pciwr}
\end{align}
Comparing \Eq{ppcimr} to \eq{bib1nb}, and using the relationship between kick and field as given by
\eq{pqlbp1} and \eq{pqlbp2}, shows that the form of the wakefield transverse kick is the same as
for a multipole of order $n = m - 1$.

The wakefield felt by a particle is due to the wakefields generated by all the particles ahead of
it. If the wakefield kicks are computed by summing over all particle pairs, the computation will
scale as $N^2$ where $N$ is the number of particles. This quickly becomes computationally
exorbitant. A better solution is to keep track of the wakes in a cavity. When a particle comes
through, the wake it generates is simply added to the existing wake. This computation scales as $N$
and makes simulations with large number of particles practical.

To add wakes together, a wake must be decomposed into its components.  Spatially, there are normal
and skew components and temporally there are sin and cosine components. This gives 4 components
which will be labeled $a_{\cos}$, $a_{\sin}$, $b_{\cos}$, and $b_{\sin}$. For a mode of order $m$, a
particle passing through at a time $t_w$ with respect to the reference particle will produce wake
components
\begin{alignat}{2}
  \delta a_{\sin} &\equiv  &c \, A_a \, \left( \frac{R}{Q} \right) \,
    e^{d \, t_w} \, \cos (\omega \, t_w) \, I_m \, \sin(m \theta_w) 
    \CRNO
  \delta a_{\cos} &\equiv -&c \, A_a \, \left( \frac{R}{Q} \right) \,
    e^{d \, t_w} \, \sin (\omega \, t_w) \, I_m \, \sin(m \theta_w) 
    \label{ac2rq} 
    \\
  \delta b_{\sin} &\equiv  &c \, A_a \, \left( \frac{R}{Q} \right) \,
    e^{d \, t_w} \, \cos (\omega \, t_w) \, I_m \, \cos(m \theta_w) 
    \CRNO
  \delta b_{\cos} &\equiv -&c \, A_a \, \left( \frac{R}{Q} \right) \,
    e^{d \, t_w} \, \sin (\omega \, t_w) \, I_m \, \cos(m \theta_w) 
    \nonumber
\end{alignat}
These are added to the existing wake components. The total is
\begin{equation}
  a_{\sin} = \sum_{\text{particles}} \delta a_{\sin}
\end{equation}
with similar equations for $a_{\cos}$ etc. Here the sum is over all particles that cross the
cavity before the kicked particle. To calculate the kick due to wake, the normal and skew components
are added together
\begin{align}
  a_{tot} &= e^{-d \, t} \, \left( 
    a_{\cos} \, \cos (\omega \, t + \phi) + a_{\sin} \, \sin (\omega \, t + \phi) \right) 
    \label{akz2q} \\
  b_{tot} &= e^{-d \, t} \, \left(
    b_{\cos} \, \cos (\omega \, t + \phi) + b_{\sin} \, \sin (\omega \, t + \phi) \right) \nonumber 
\end{align}
Here $t$ is the passage time of the particle with respect to the reference particle. In analogy to
\Eq{ppcimr} and \eq{pciwr}, the kick is
\begin{align}
  -\Delta p_x + i\Delta p_y &= C \, 
    m \, (b_{tot} + i a_{tot}) \, r^{m-1} \, e^{i (m-1) \theta} 
    \label{ppcmbar} \\
  \Delta p_z &= -C \, r^m \, \left( 
    (b_{tot}' + i a_{tot}') e^{i m\theta} + (b_{tot}' - i a_{tot}') e^{-i m\theta} \right)
\end{align}
where $a' \equiv da/dt$ and $b' \equiv db/dt$.

When simulating trains of bunches, the exponential factor $d \, t_w$ in \Eq{ac2rq} can
become very large. To prevent numerical overflow, \bmad uses a reference time $z_\REF$ so
that all times $t$ in the above equations are replaced by
\begin{equation}
  t \longrightarrow t - t_\REF
  \label{ttt}
\end{equation}

The above equations were developed assuming cylindrical symmetry. With cylindrical symmetry, the
cavity modes are actually a pair of degenerate modes. When the symmetry is broken, the modes no
longer have the same frequency. In this case, one has to consider a mode's polarization angle
$\theta_p$. Equations \eq{akz2q} and \eq{ppcmbar} are unchanged.  In place of \Eq{ac2rq}, the
contribution of a particle to a mode is
\begin{alignat}{2}
  \delta a_{\sin} &=  & c \, A_a \, \left( \frac{R}{Q} \right) \,
    e^{d \, t_w} \, \cos (\omega \, t_w) \, I_m \, \left[
    \sin(m \theta_w) \, \sin^2(m \theta_p) + 
    \cos(m \theta_w) \, \sin(m \theta_p) \, \cos(m\theta_p) \right]
    \CRNO
  \delta a_{\cos} &= -& c \, A_a \, \left( \frac{R}{Q} \right) \,
    e^{d \, t_w} \, \sin (\omega \, t_w) \, I_m \, \left[ 
    \sin(m \theta_w) \, \sin^2(m \theta_p) + 
    \cos(m \theta_w) \, \sin(m \theta_p) \, \cos(m\theta_p) \right]
    \\
  \delta b_{\sin} &=  & c \, A_a \, \left( \frac{R}{Q} \right) \,
    e^{d \, t_w} \, \cos (\omega \, t_w) \, I_m \, \left[
    \cos(m \theta_w) \, \cos^2(m \theta_p) + 
    \sin(m \theta_w) \, \sin(m \theta_p) \, \cos(m\theta_p) \right]
    \CRNO
  \delta b_{\cos} &= -& c \, A_a \, \left( \frac{R}{Q} \right) \,
    e^{d \, t_w} \, \sin (\omega \, t_w) \, I_m \, \left[
    \cos(m \theta_w) \, \cos^2(m \theta_p) + 
    \sin(m \theta_w) \, \sin(m \theta_p) \, \cos(m\theta_p) \right]
    \nonumber
\end{alignat}
Note: Technically an unpolarized mode is actually two polarized modes perpendicular to each
other. The axes of these two normal modes can be chosen arbitrary as long as they are at
right angles.

\chapter{Lattice File Statements}
\label{c:lat.file}
\index{lattice files|hyperbf}

\index{Bmad!lattice file format}
A lattice (\sref{c:lat.concepts}) defines the sequence of elements that a particle will travel
through along with the attributes (length, strength, orientation, etc.) of the elements.  A lattice
file (or files) is a file that is used to describe an accelerator or storage ring.

When \bmad was first developed, The \bmad lattice file syntax was modeled upon the format defined
for the \mad program\cite{b:maduser}. Since then, the \bmad format has been developed to meet ever
increasing simulation needs so currently there are many differences between the two formats. One
difference, which has been present from the very start, is that there are no ``action'' commands
(action commands tell the program to calculate the Twiss parameters, do tracking, etc.) in a \bmad
lattice file. The reason for this is due to the fact that \bmad is a software library and not a
program. That is, interacting with the user to determine what actions a \bmad based program should
take is left to the program itself and is not part of the \bmad standard.

%---------------------------------------------------------------------------
\section{File Example and Syntax}
\index{Bmad!statement syntax}

\index{parameter statement}
\index{use statement}
The following (rather silly) example shows some of the features of a
\bmad lattice file:
\begin{example}
  ! This is a comment
  parameter[E_TOT] = 5e9                   ! Parameter definition
  pa1 = sin(3.47 * pi / c_light)                 ! Constant definition
  bend1: sbend, type = "arc bend", l = 2.3,      ! An element definition
      g = 2*pa1, tracking_method = bmad_standard
  bend2: bend1, l = 3.4                          ! Another element def
  bend2[g] = 105 - exp(2.3) / 37.5               ! Redefining an attribute
  ln1: line = (ele1, ele2, ele3)                 ! A line definition
  ln2: line = (ln1, ele4, ele5)                  ! Lines can contain lines
  arg_ln(a, b): line = (ele1, a, ele2, b)        ! A line with arguments.
  use, ln2                                       ! Which line to use for the lattice
\end{example}

\index{"! comment symbol}
\index{comment symbol ("!)}
A \bmad lattice file consists of a sequence of statements. An exclamation mark (!) denotes a comment
and the exclamation mark and everything after the exclamation mark on a line are ignored.  

\bmad is generally case insensitive. Most names are converted to uppercase. Exceptions are 
file names and atomic formulas for materials used in crystal diffraction. Also \vn{type}, \vn{alias},
and \vn{descrip} string labels which can be set for any element are not converted (\sref{s:alias}).

\index{\& continuation symbol}
\index{continuation symbol (\&)}
Normally a statement occupies a single line in the file. Several statements may be placed on the
same line by inserting a semicolon (``;'') between them. A long statement can occupy multiple lines
by putting an ampersand (``\&'') at the end of each line of the statement except for the last
line. Additionally, lines that end with an ``implicit end-line continuation character'' are automatically
continued to the next line and lines that begin with an ``implicit begin-line continuation character are
automatically appended to the previous line. The implicit end-line continuation characters are:
\begin{example}
  ,   (   \{   [   =
\end{example}
The implicit begin-line continuation characters are:
\begin{example}
  ,   )   \}   ]   =
\end{example}
The following example shows a command extending over four lines
\begin{example}
  wall = \{               ! End-line continuation character.
    section = \{s = 0.45 
    \}                    ! Begin-line continuation character.
  \}                      ! Begin-line continuation character.
\end{example}
Note: The symbols ``+'', ``-'', ``*'', and ``/'' are {\em not} valid implicit continuation
characters. The reason for this is to avoid confusion when a species name (for example ``He++'')
comes at the end of a line.

A single line in a lattice file is limited to 500 characters. Commands have no restrictions as to
the number of lines they may be continued over and there is no limit to the length of a command.

\index{lattice files!name syntax}
Names of constants, elements, lines, etc. are limited to 40 characters. The first character must be
a letter (\vn{A} --- \vn{Z}).  The other characters may be a letter, a digit (\vn{0} --- \vn{9}) or
an underscore (\vn{_}). Other characters may appear but should be avoided since they are used by
Bmad for various purposes. For example, the backslash (\vn{\B}) character is used to by Bmad when
forming the names of superposition slaves (\sref{s:super}) and dots (\vn{.}) are used by Bmad when
creating names of \vn{tagged} elements (\sref{s:tag}). Also use of special characters may make the
lattice files less portable to non-Bmad programs.

\index{parameter statement}
\index{parameter statement}
\index{parameter statement}
\index{beginning statement}
The following example constructs a linear lattice with two elements: 
\begin{example}
  parameter[geometry] = open
  parameter[e_tot] =2.7389062E9
  parameter[particle] = POSITRON
  beginning[beta_a] = 14.5011548
  beginning[alpha_a] = -0.53828197
  beginning[beta_b] = 31.3178048
  beginning[alpha_b] = 0.25761815
  q: quadrupole, l = 0.6, b1_gradient = 9.011
  d: drift, l = 2.5
  t: line = (q, d)
  use, t 
\end{example}
here \vn{parameter[geometry]} (\sref{s:param}) is set to \vn{open} which specifies that the lattice
is not circular. In this case, the beginning Twiss parameters need to be specified and this is done
by the \vn{beginning} statements (\sref{s:beginning}). A quadrupole named \vn{q} and a drift element
named \vn{d} are specified and the entire lattice consists of element \vn{q} followed by element
\vn{d}.

%----------------------------------------------------------------------------
\section{Digested Files}
\label{s:digested}
\index{digested files}

Normally the \bmad parser routine will create what is called a ``digested file'' after it has parsed
a lattice file so that when a program is run and the same lattice file is to be read in again, to
save time, the digested file can be used to load in the lattice information.  This digested file is
in binary format and is not human readable. The digested file will contain the transfer maps for all
the elements.  Using a digested file can save considerable time if some of the elements in the
lattice need to have Taylor maps computed.  (this occurs typically with map--type wigglers).

\bmad creates the digested file in the same area as the lattice file.  If \bmad is not able to
create a digested file (typically because it does not have write permission in the directory), an
error message will be generated but otherwise program operation will be normal.

Digested files contain the names of the lattice files used to create them. If a lattice file has
been modified since the digested file has been created then the lattice files will be reread and a
new digested file will be generated.

\index{ran}
\index{ran_gauss}
Note: If any of the random number functions (\sref{s:functions}) are used in the process of creating
the lattice, the digested file will be ignored. In this case, each time the lattice is read into a
program, different random numbers will be generated for expressions that use such random numbers.

Digested files can also be used for easy transport of lattices between programs or between sessions
of a program. For example, using one program you might read in a lattice, make some adjustments (say
to model shifts in magnet positions) and then write out a digested version of the lattice. This
adjusted lattice can now be read in by another program.

%---------------------------------------------------------------------------
\section{Element Sequence Definition}

\index{line}\index{use statement|hyperbf}
A \vn{line} defines a sequence of elements. \vn{lines} may contain other \vn{lines} and so a
hierarchy may be established. One line is selected, via a \vn{use} statement, that defines the
lattice. For example:
\begin{example}
  l3: line = (l1, l2)   ! Concatenate two lines
  l1: line = (a, b, c)  ! Line with 3 elements
  l2: line = (a, z)     ! Another line 
  use, l3               ! Use l3 as the lattice definition.
\end{example}
In this case the lattice would be
\begin{example}
  (a, b, c, a, z)
\end{example}
\vn{Lines} can be defined in any order. See \cref{c:sequence} for more
details.

\index{superimpose}
The \vn{superimpose} construct allows elements to be placed in a lattice at a definite longitudinal
position. What happens is that after a lattice is expanded, there is a reshuffling of the elements
to accommodate any new superimpose elements. See \sref{s:super} for more details.

%---------------------------------------------------------------------------
\section{Lattice Elements}

The syntax for defining a lattice element roughly follows the
\mad\cite{b:maduser} program:
\begin{example}
  ele_name: keyword [, attributes]
\end{example}
where \vn{ele_name} is the element name, \vn{keyword} is the type of element, and \vn{attributes} is
a list of the elements attributes. \cref{c:elements} gives a list of elements types with their
attributes.  \vn{Overlay} and \vn{group} type elements have a slightly different syntax:
\begin{example}
  ele_name: keyword = \{ list \}, master-attribute [= value] [, attributes]
\end{example}
and \vn{Girder} elements have the syntax
\begin{example}
  ele_name: keyword = \{ list \} [, attributes]
\end{example}  
For example:
\begin{example}
  q01w: quadrupole, type = "A String", l = 0.6, tilt = pi/4
  h10e: overlay = \{ b08e, b10e \}, var = \{hkick\}
\end{example}

The \vn{keyword} specifying the element type can be the name of an element that has already been
declared. In this case, the element being defined will inherit the attributes that have been
set for the element associated with \vn{keyword}. Example:
\begin{example}
  qa, quad, l = 0.6, tilt = pi/4  ! Define QA.
  qb: qa                          ! QB Inherits from QA.
  qa[k1] = 0.12                   ! QB unaffected by modifications of QA.
\end{example}
In this example, element \vn{QB} inherits the attributes of \vn{QA} which, in this case, are the
length and tilt parameters of \vn{QA}. Once \vn{QB} is defined, the elements are separate so 
modifications of the parameters of \vn{QA} after \vn{QB} is defined will not affect \vn{QB}.

\bmad allows element names to be abbreviations of element types. For example, ``\vn{Q}'', \vn{QU}'',
and ``\vn{QUADRUPOL}'' are all valid element names but ``\vn{QUADRUPOLE}'', being an exact match to
the corresponding element type, is not. Care must be used when using elements that are abbreviations
of element types since \bmad allows element type names to be abbreviated (but element names may not
be abbreviated when using inheritance). For example:
\begin{example}
  q1: quad          ! Q1 is a quadrupole since it comes before the next line.
  quad: sextupole   ! QUAD element is defined to be a sextupole.
  q2: quad          ! Q2 is a sextupole as it inherits from QUAD.
  q3: qua           ! Q3 is a quadrupole. Inherit from names cannot be abbreviated.
\end{example}

%---------------------------------------------------------------------------
\section{Lattice Element Names}
\label{s:ele.names}
\index{element!names}

\index{element!name}\index{superimpose}\index{multipass}
A valid element name may be up to 40 characters in length. The first character of the name must be a
letter [A-Z]. After that, the rest of the name can contain only letters, digits [0-9], underscore
``_'', period ``.'', backslash ``\B'', or a hash mark ``\#''. A double hash mark ``\#\#'' is not
permitted since this interfers with the notation for finding the $N$\Th element with a given name
(\sref{s:ele.match}). It is best to avoid these last three symbols since \bmad uses them to denote
``relationships''. Periods are used for tagging (\sref{s:tag}), and backslash and hash marks are
used for to compose names for superposition (\sref{s:super}) and multipass (\sref{s:multipass})
slave elements.

\index{reserved names}
\index{beam}\index{no_digested}\index{root}
\index{beginning}\index{no_superimpose}\index{slice_lattice}
\index{call}\index{parameter}\index{start_branch_at}
\index{calc_reference_orbit}\index{parser_debug}\index{superimpose}
\index{combine_consecutive_elements}\index{particle_start}\index{title}
\index{debug_marker}\index{print}\index{use}
\index{end_file}\index{redef}\index{use_local_lat_file}
\index{expand_lattice}\index{remove_elements}\index{write_digested}
\index{merge_elements}\index{return}
There is a short list of names that cannot be used as an element, line or list name. These reserved names are:
\begin{example}
  beam                             no_digested                  root
  beginning                        no_superimpose               slice_lattice
  call                             parameter                    start_branch_at
  calc_reference_orbit             parser_debug                 superimpose
  combine_consecutive_elements     particle_start               title
  debug_marker                     print                        use
  end_file                         redef                        use_local_lat_file
  expand_lattice                   remove_elements              write_digested
  merge_elements                   return
\end{example}
Note: The one exception is if \vn{end} is used to define a \vn{marker}. This exception is allowed
since \bmad uses the \vn{end} name to define a marker in any case.

It is perfectly acceptable for multiple lattice elements to have the same name. Example:
\begin{example}
  q: quadrupole, ...
  aline: line = (5*q)
  use, aline
\end{example}
This will produce a lattice with five elements named ``\vn{q}''. The exception is \vn{group}
(\sref{s:group}) and \vn{overlay} (\sref{s:overlay}) controller elements will always have unique
names. It is important to keep in mind that elements with the same name do not necessarily have
the same parameter values. See Section~\sref{s:ele.match} for an example. 

%---------------------------------------------------------------------------
\section{Matching to Lattice Element Names}
\label{s:ele.match}
\index{element!matching to names}

Where appropriate, for example when setting element attributes (\sref{s:lat.attribs}), the wild card
characters \vn{``*''} and \vn{``%''} can be used to select multiple elements.  The \vn{``*''}
character will match any number of characters (including zero) while \vn{``%''} maches to any single
character. Additionally, matching can be restricted to a certain element class using the syntax:
\begin{example}
  class::element_name
\end{example}
where \vn{class} is a class (EG: sextupole). For example:
\begin{example}
  m*              ! Match to all elements whose name begins with "m".
  a%c             ! Match to "abc" but not to "ac" or "azzc".
  quadrupole::*w  ! Match to all quadrupoles whose name ends in "w"
\end{example}
Note: The character \vn{``%''} can be used in expressions used for setting lattice element parameter
values. In this case, the \vn{``%''} character represents the name of the lattice element whose
parameter is being set. This is discussed in section~\sref{s:lat.attribs}.

Instead of matching to an element name, matches may be made to \vn{type}, \vn{alias},
or \vn{descrip} attributes (\sref{s:alias}) using the syntax:
\begin{example}
  attribute_name::string
\end{example}
where \vn{attribute_name} is one of:
\begin{example}
  type
  alias
  descrip
\end{example}
and \vn{string} is a string to match to which can include wild card characters \vn{``*''} and
\vn{``%''}. If \vn{string} contains blank characters, the string must be enclosed with single or
double quotation marks. Note: An element attribute that is blank will never match. Also note that
while \vn{type}, \vn{alias}, or \vn{descrip} strings may have lower case characters (unlike element
names, these strings are not converted to upper case), matching is always case insensitive. For Example:
\begin{example}
  type::"det bpm*"   ! Match to all elements whose \vn{type} string starts with "det bpm".
  alias::*           ! Match to all elements whose \vn{alias} string is not blank.
\end{example}

\vspace*{1pt}
After lattice expansion (\sref{s:expand}), the general syntax to specify a set of elements is:
\begin{example}
  \{branch_id>>\}\{class::\}element_id
  \{branch_id>>\}element_name##N                 ! N^th instance of element_name in 
                                               !        branch with branch_id
  \{branch_id>>\}\{class::\}element_id\{##N\}+M  ! M^th element after element
  \{branch_id>>\}\{class::\}element_id\{##N\}-M  ! M^th element before element
\end{example}
where \vn{\{...\}} marks an optional component, \vn{class} is a class name, \vn{branch_id} is a
branch name or index (\sref{s:branch.def}), \vn{element_id} is and element name or element index
(\sref{s:lines.wo.arg}), and \vn{\#\#N} indicates that the N\Th matching element in a branch is to be
used. Examples:
\begin{example}
  x_br>>quad::q*        ! All quadrupoles of branch "x_br" whose name begins with "q".
  2>>45                 ! element \#45 of branch \#2.
  q01##3                ! The 3rd element named q01 in each branch where there are at 
                        ! least three elements named q01.
  q01-1                 ! Elements just before all elements named q01.
  q01##3+0              ! Same as q01##3.
\end{example}
Note: \vn{Group} and \vn{overlay} elements have unique names so using \vn{\#\#} is unnecessary.

Note: When using the \vn{\#\#} construct, \vn{super_lord} and \vn{girder_lord} elements are considered 
to be situated where their slave elements are situated in the lattice. This is independent of where
they actually exist which is in the lord part of branch 0 (\sref{s:lord.slave}).

Note: When using \vn{+M} and \vn{-M} offsets, no space should be put around the plus or minus signs.
Offsets cannot be used with \vn{overlay}, \vn{group}, \vn{ramper_lord} or \vn{multipass_lord}
elements. If used with a \vn{super_lord} or \vn{girder} element, for a \vn{-M} offset the element
selected is the \vn{M}\Th element before the first \vn{slave} element of the lord and for a \vn{+M}
offset the element selected is the \vn{M}\Th element after the last \vn{slave} element of the
lord. Also, independent of the \vn{geometry} of the branch, the element selected will ``wrap''
around the ends of the branch.  For example, if a lattice branch looks like:
\pagebreak[2]
\begin{example}
  Index  Name
    0    Beginnning
    1    A
    2    B\B1       ! First super_slave of B super_lord
    3    M
    4    B\B2       ! Second super_slave of B super_lord
    5    C
\end{example}
then
\begin{example}
    Name  Translates to     Name  Translates to
    ----  -------------     ----  -------------
    B-1   A                 B+1   C
    B-2   Beginning         B+2   Beginning
    B-3   C                 B+3   A
    B-4   B\B2               B+4   B\B1
\end{example}

It is advised to avoid setting the parameters of differing elements that share the same
name to differing values since this can lead problems later on. For example, consider this in a
lattice file named, say, \vn{lat.bmad}:
\begin{example}
  q1: quadrupole, ...
  a: line = (..., q1, ...)
  b: line = (..., q1, ...)
  c: line = (a, b)  ! There are two q1 elements. 
                    ! One from A-line and one from B-line.
  use, c
  expand_lattice    ! Expand the lattice
  q1##1[k1] = 0.1   ! Set first element whose name is q1
  q1##2[k1] = 0.2   ! Set second element whose name is q1
\end{example}
Now if later on someone wants to study just the \vn{B} line that person could try to do this by
creating a second file with just two lines:
\begin{example}
  call, file = lat.bmad
  use, b
\end{example}
Normally this would work but in this case the lattice is invalid since there is only one \vn{q1}
element in line \vn{B}. A more flexible solution would be to use unique names for the two \vn{q1}
elements.

Multiple elements in a lattice may share the same name.  When multiple branches are present, to
differentiate elements that appear in different branches, the ``branch qualified'' element name may
be used. The branch qualified element name is of the form
\begin{example}
  branch_name>>element_name
\end{example}
where \vn{branch_name} is the name of the branch and \vn{element_name} is the ``regular'' name of
the element. Example:
\begin{example}
  root>>q10w
  x_branch>>crystal3
\end{example}

For \vn{branch} lines (\sref{s:branch.def}), the full ``branch qualified'' name of an element is of
the form
\begin{example}
  branch_name>>element_name
\end{example}
where \vn{branch_name} is the name of the branch and \vn{element_name} is the ``regular'' name of
the element. Example:
\begin{example}
  root>>q10w
  xline>>cryst3
\end{example}
Using the full name is only needed to distinguish elements that have the same regular name in
separate branches. When parsing a lattice file, branches are not formed until the lattice is
expanded (\sref{s:expand}). Therefore an \vn{expand_lattice} statement is required before full names
can be used in statements.

After lattice expansion (\sref{s:expand}), when setting element attributes (\sref{s:lat.attribs}, a
comma delimited list of names can be used (the commas are actually optional). Each item in a list is
either the name of an element or an element range. An element range has the syntax:
\begin{example}
  \{branch>>\}\{class::\}ele1:ele2
\end{example}
where
\begin{example}
  branch   ! Optional branch name or index.
  class    ! Optional element class name ("quadrupole", "sbend", etc.)
  ele1     ! Starting element of the range which includes ele1.
  ele2     ! Ending element of the range which includes ele2. 
\end{example}
For example:
\begin{example}
  3,15:17          ! Elements with index 3, 15, 16, and 17 in branch 0.
  3 15:17          ! Same as above (commas are optional).
  2>>45:51         ! Elements 45 through 51 of branch 2.
  q1:q5            ! Elements between q1 and q5.
  sbend::q1:q5     ! All sbend elements between q1 and q5 including q1 and q5 if they. 
                   ! are sbends. Notice that q1 and q5 do not have to be sbends
\end{example}
if the element index of \vn{ele1} is greater than \vn{ele2} then the range wraps around the end of
the lattice. For example, if branch0 has 360 tracking elements (\sref{s:lord.slave}), the range
\vn{321:72} is equivalent to \vn{372:360, 0:72}.

With a comma delimited list of names, a tilde prefix can be used to remove elements from the list.
Adding and subtracting elements is done left to right. For example:
\begin{example}
  *::*, ~quadrupole::*  ! All elements except quadrupoles
  b*, ~b1*, b13         ! All elements with names beginning with B except
                        !   elements with names beginning with B1. However,
                        !   elements named B13 are retained.
\end{example}

With a list of names, an ampersand ``\vn{\&}'' can be used to form the intersection of two sets.
\begin{example}
  100:200 \& sbend::*    ! All sbend elements whose index is between 100 and 200.
  q* \& quad::*          ! Equivalent to "quadrupole::q*".
  q* \& quad::* b1       ! Equivalent to "quadrupole::q*, b1".
\end{example}

When lattice expansion occurs during parsing of a lattice (\sref{s:expand.lat}), all \vn{rbend}
elements are converted to \vn{sbend} elements (\sref{s:bend}) but there is a \vn{sub_key} element
parameter that is used so \bmad knows which bend elements were defined as \vn{rbend}s in the lattice
file. After lattice expansion, the string \vn{sbend::*} will match to all bend elements irregardless
of whether a bend was defined to be a \vn{sbend} or a \vn{rbend}. On the other hand, after lattice
expansion the string \vn{rbend::*} will match to all bend elements that were defined as \vn{rbend}s
before expansion.

%---------------------------------------------------------------------------
\section{Lattice Element Parameters}
\label{s:lat.attribs}
\index{element attributes|hyperbf}

Any lattice element has various attributes like its name, its length, its strength, etc. The values
of element attributes can be specified when the element is defined. For example:
\begin{example}
  b01w: sbend, l = 6.0, rho = 89.0 ! Define an element with attributes.
\end{example}
After an element's definition, most attributes may be referred to using the syntax
\begin{example}
  class::element_name[attribute_name]
\end{example}
Examples:
\begin{example}
  q01[k1]                       ! K1 attribute of element Q01.
  sbend::b0%[dg]                ! DG attribute of sbend elements whose name
                                !   has three characters starting with "B0"
\end{example}
Some element parameters have a more complicated syntax and are listed in section \sref{s:attrib.non}.

Element attributes can be set or used in an algebraic expression:
\begin{example}
  b01w[roll] = 6.5                  ! Set an attribute value.
  b01w[L] = 6.5                     ! Change an attribute value.
  b01w[L] = b01w[rho] / 12          ! OK to reset an attribute value.
  my_const = b01w[rho] / b01w[L]    ! Use of attribute values in an expression.
\end{example}
Notice that there can be no space between the element name and the \vn{[} opening bracket.

Chapter \cref{c:elements} lists the attributes appropriate for each element class.

When setting an attribute value, if more than one element has the \vn{element_name} then {\it all}
such elements will be set. When setting an attribute value, if \vn{element_name} is the name of a
type of element, all elements of that type will be set. For example
\begin{example}
  q_arc[k1] = 0.234                      ! Set all elements named Q_ARC. 
  rfcavity::*[voltage] = 3.7             ! Set all RFcavity elements.
\end{example}

To set an attribute for multiple element at one time, The wild cards ``\vn{*}'', and ``\vn{\%}'' can
be used in element names (\sref{s:ele.match}). Examples:
\begin{example}
  *[tracking_method] = bmad_standard  ! Matches all elements.
  quadrupole::Q*[k1] = 0.234    ! Matches all quadrupoles with names beginning with Q.
  Q%1[k1] = 0.234               ! Matches to "Q01" but not "Q001".
\end{example}
\index{beginning element}
Unlike when there are no wild cards used in a name, it is not an error if a name with wild cards
does not match to any element.  Note: A name with wild cards will never match to the \vn{BEGINNING}
element (\sref{s:use}).

The ``\vn{\%}'' symbol can be used in expression to represent the lattice element whose parameter is
being set. For example:
\begin{example}
  s%z[k2] = %[k2] + 0.03 * ran_gauss()
\end{example}
The ``\vn{s\%{}z}'' on the left hand side matches to all three letter elements whose name begins with
``\vn{s}'' and ends with ``\vn{z}''. For each element that is matched, the ``\vn{\%[k2]}'' in the
expression on the right hand side will be the \vn{k2} value of that element. Thus, if there are three
elements, named \vn{s0z}, \vn{saz}, and \vn{s.z}, in the lattice that match \vn{s\%{}z}, the above command
is equivalent to the following three commands
\begin{example}
  s0z[k2] = s0z[k2] + 0.03 * ran_gauss()
  saz[k2] = saz[k2] + 0.03 * ran_gauss()
  s.z[k2] = s.z[k2] + 0.03 * ran_gauss()
\end{example}

After lattice expansion (\sref{s:expand}), the attributes of specific elements may be set using the
syntax as discussed in Section~\sref{s:ele.match}. Example:
\begin{example}
  expand_lattice              ! Expand the lattice.
  97[x_offset] = 0.0023       ! Set x_offset attribute of 97th element
  b2>>si_cryst##2[tilt] = 0.1 ! Tilt the 2nd instance of "si_cryst" in branch "b2" 
  5:32[x_limit] = 0.3         ! Sets elements with indexes 5 through 32 in branch 0.
\end{example}

%---------------------------------------------------------------------------
\section{Nonstandard Parameter Syntax}
\label{s:attrib.non}

This section lists parameters that have a ``nonstandard'' syntax (not in the form
``\vn{ename[parameter]''}'').  In the following, ``\vn{ename}'' is an element name, and ``\vn{N}'',
``\vn{N1}'', ``\vn{N2}'', ``\vn{N3}'' and ``\vn{M}'' along with \vn{<out>}, \vn{<n1>}, and \vn{<n2>} are
integers.

AC_kicker (\sref{s:ac.kick}):
\begin{example}
  ename[amp_vs_time(N)%time]
  ename[amp_vs_time(N)%amp] 
  ename[frequencies(N)%freq]
  ename[frequencies(N)%amp] 
  ename[frequencies(N)%phi] 
\end{example}

Cartesian map (\sref{s:cart.map}):
\begin{example}
  ename[cartesian_map(N)%field_scale]      
  ename[cartesian_map(N)%r0(1)]
  ename[cartesian_map(N)%r0(2)]
  ename[cartesian_map(N)%r0(3)]
  ename[cartesian_map(N)%master_parameter]
  ename[cartesian_map(N)%t(M)%A]                  -- M^th term in N^th map.
  ename[cartesian_map(N)%t(M)%kx]
  ename[cartesian_map(N)%t(M)%ky]
  ename[cartesian_map(N)%t(M)%kz]
  ename[cartesian_map(N)%t(M)%x0]
  ename[cartesian_map(N)%t(M)%y0]
  ename[cartesian_map(N)%t(M)%phi_z]
\end{example}

Controller knot points (\sref{s:go.syntax}):
\begin{example}
  ename[x_knot(N)]                        -- N^th x_knot point.
  ename[slave(M)%y_knot(N)]               -- N^th y_knot point for M^th slave.
\end{example}

Custom attributes (\sref{s:cust.att}):
\begin{example}
  ename[r_custom(N1, N2, N3)]
  ename[r_custom(N1, N2)]                 -- Equivalent to ename[r_custom(N1, N2,  0)]
  ename[r_custom(N1)]                     -- Equivalent to ename[r_custom(N1,  0,  0)]
\end{example}

Cylindrical map (\sref{s:cylind.map}):
\begin{example}
  ename[cylindrical_map(N)%phi0_fieldmap]  
  ename[cylindrical_map(N)%theta0_azimuth]
  ename[cylindrical_map(N)%field_scale]
  ename[cylindrical_map(N)%dz]
  ename[cylindrical_map(N)%r0(1)]
  ename[cylindrical_map(N)%r0(2)]
  ename[cylindrical_map(N)%r0(3)]
  ename[cylindrical_map(N)%master_parameter]
\end{example}

Gen_Grad map (\sref{s:gen.grad.map}):
\begin{example}
  ename[gen_grad_map(N)%field_scale]     
  ename[gen_grad_map(N)%r0(1)]
  ename[gen_grad_map(N)%r0(2)]
  ename[gen_grad_map(N)%r0(3)]
  ename[gen_grad_map(N)%master_parameter]
\end{example}

Grid_field (\sref{s:grid.field}):
\begin{example}
  ename[grid_field(N)%phi0_fieldmap] 
  ename[grid_field(N)%interpolation_order]
  ename[grid_field(N)%harmonic]
  ename[grid_field(N)%geometry]
  ename[grid_field(N)%ename_anchor_pt]
  ename[grid_field(N)%phi0_fieldmap]
  ename[grid_field(N)%field_scale]
  ename[grid_field(N)%dr(1)]
  ename[grid_field(N)%dr(2)]
  ename[grid_field(N)%dr(3)]
  ename[grid_field(N)%r0(1)]
  ename[grid_field(N)%r0(2)]
  ename[grid_field(N)%r0(3)]
  ename[grid_field(N)%master_parameter]
\end{example}

Long range wake (\sref{s:lr.wake.syntax}):
\begin{example}
  ename[lr_wake%amp_scale]
  ename[lr_wake%time_scale]
  ename[lr_wake%freq_spread
  ename[lr_wake%mode(N)%freq_in]       
  ename[lr_wake%mode(N)%freq]     
  ename[lr_wake%mode(N)%r_over_q]  
  ename[lr_wake%mode(N)%damp]       
  ename[lr_wake%mode(N)%phi]       
  ename[lr_wake%mode(N)%polar_angle]
  ename[lr_wake%mode(N)%polarized] 
\end{example}

Surface curvature (\sref{s:surface}):
\begin{example}
  ename[curvature%spherical]
  ename[curvature%elliptical_x]
  ename[curvature%elliptical_y]
  ename[curvature%elliptical_z]
  ename[curvature%x(N1)y(N2)]
\end{example}

Taylor terms (\sref{s:taylor}):
\begin{example}
  ename[tt<out><n1><n2>...]      ! Orbital terms. EG: rot[tt13] -> M13 matrix term
  ename[ttS0<n1><n2><n3>...]     ! S0 spin quaternion terms.
  ename[ttSx<n1><n2><n3>...]     ! Sx spin quaternion terms.
  ename[ttSy<n1><n2><n3>...]     ! Sy spin quaternion terms.
  ename[ttSz<n1><n2><n3>...]     ! Sz spin quaternion terms.
\end{example}

Wall for vacuum chambers and masks (\sref{s:wall}):
\begin{example}
  ename[wall%section(N)%s]
  ename[wall%section(N)%wall%dr_ds]
  ename[wall%section(N)%v(M)%x]
  ename[wall%section(N)%v(M)%y]
  ename[wall%section(N)%v(M)%radius_x]
  ename[wall%section(N)%v(M)%radius_y]
  ename[wall%section(N)%v(M)%tilt]
\end{example}

%---------------------------------------------------------------------------
\section{Custom Element Attributes}
\label{s:cust.att}
\index{element attributes!defining custom attributes}

Real scalar and vector custom element attributes may be defined for any class of element and real
scaler parameters can be defined for the lattice as a whole.  Custom element attributes are useful
with programs that need to associate ``extra'' information with particular lattice elements or the
lattice itself and it is desired that this extra information be settable from within a lattice
file. For example, a program might need an error tolerance for the strength of quadrupoles.

Adding custom attributes will not disrupt programs that are not designed to use the custom
attributes. Currently, up to 40 named custom attributes may be defined for any given element
class. The syntax for defining custom attributes is:
\begin{example}
  parameter[custom_attributeN] = \{class_name::\}attribute_name
\end{example}
Where ``\vn{N}'' is an integer between 1 and 40 and "\vn{attribute_name}" is the name of the
attribute. To restrict the custom attribute to a particular element class, the element class can be
prefixed to the attribute name. To define a global parameter for the lattice, use \vn{parameter}''
as the class name.
Examples:
\begin{example}
  parameter[custom_attribute1] = quadrupole::error_k1
  parameter[custom_attribute1] = mag_id
  parameter[custom_attribute1] = sextupole::error_k2
  parameter[custom_attribute2] = color
  parameter[custom_attribute2] = parameter::quad_mag_moment
\end{example}
The first line in the example assigns, for the first custom attribute group
(\vn{custom_attribute1}), a name of \vn{error_k1} to all quadrupoles. The second line in the example
assigns to the first custom attribute group the name \vn{mag_id} to all element classes except
quadrupoles since that class of element already has an assigned name. The third line assigns, for
the first custom attribute group, a name of \vn{error_k2} to all sextupoles overriding the
\vn{mad_id} name. The fourth line in the above example assigns, for the second custom attribute
group, a name of \vn{color} to all element classes. Finally, the last line defines a global parameter
called \vn{quad_mag_moment}.

Once a custom attribute has been defined it may be set for an element
of the correct class. Example:
\begin{example}
  parameter[custom_attribute2] = lcavity::rms_phase_err
  parameter[custom_attribute3] = parameter::cost
  ...
  parameter[cost] = 140000000
  l2a: lcavity, rms_phase_err = 0.0034, ...
\end{example}
Notice that defining the name for a custom attribute must come before its use.

Custom attributes that are assigned to an individual element class, like \vn{error_k1} above, are
called ``\vn{class-specific}'' attributes. Custom attributes, like \vn{mag_id} above, that are
assigned to all element classes, are called ``\vn{common}'' attributes. For a given custom attribute
group, The setting of a \vn{class-specific} attribute will take precedence over the setting of a
\vn{common} attribute. Thus, in the above example, the fact that \vn{quadrupole::error_k1} comes
before \vn{mag_id} and \vn{sextupole::error_k2} appears after does not affect anything. Once a
\vn{common} attribute is defined for a given custom attribute group, it cannot be
changed. Similarly, once a \vn{class-specific} attribute is defined for a given class for a given
custom attribute group it cannot be changed. Trying to redefine a given custom attribute using a 
new name that is the same as the old name is not considered an error. For example, the following is OK:
\begin{example}
  parameter[custom_attribute2] = color
  parameter[custom_attribute2] = color   ! OK since the same name is used.
\end{example}

Custom attributes are global in a program and not lattice-specific. That is, if a program reads in two
different lattices the custom attribute settings of both lattices will be combined.

For someone creating a program, section~\sref{s:ele.gen} describes how to make the appropriate
associations.

Note: If custom string information needs to be associated with an element, the \vn{type}, \vn{alias}
and \vn{descrip} element components (\sref{s:alias}) are available.

\index{r_custom}
Besides the named custom attributes described above, there is a three dimensional vector, called
\vn{r_custom}, associated with each element that can be used to store numbers.  
For example:
\begin{example}
  qq: quadrupole, r_custom(-2,1,5) = 34.5, r_custom(-3) = 77.9
\end{example}
Negative indices are accepted and if only one or two indices are present, the others
are assumed to be zero. Thus \vn{r_custom(-3)} is equivalent to \vn{r_custom(-3,0,0)}.

Note: When there is a superposition (\sref{s:super}), the \vn{super_slave} elements that are formed
do {\em not} have any custom attributes assigned to them even when their \vn{super_lord} elements
have custom attributes. This is done since the \bmad bookkeeping routines are not able to handle the
situation where a \vn{super_slave} element has multiple \vn{super_lord} elements and thus the custom
attributes from the different \vn{super_lord} elements have to be combined. Proper handling of this
situation is left to any custom code that a program implements to handle custom attributes.

%---------------------------------------------------------------------------
\section{Parameter Types}
\label{s:param.types}
\index{parameter types}

\index{logicals|hyperbf}
There are five types of parameters in \bmad: reals, integers, switches, logicals (booleans), and
strings. Acceptable logical values are
\begin{example}
   true    false
   t       f
\end{example}
For example
\begin{example}
  rf1[is_on] = False
\end{example}

\index{strings|hyperbf}
String literals can be quoted using double quotes (") or single quotes (').  If there are no blanks
or commas within a string, the quotes can be omitted. For example:
\begin{example}
  Q00W: Quad, type = "My Type", alias = Who_knows, &
                                  descrip = "Only the shadow knows"
\end{example}
Unlike most everything else, strings are not converted to uppercase.

\index{switches|hyperbf}
Switches are parameters that take discrete values. For example:
\begin{example}
  parameter[particle] = positron          
  q01w: quad, tracking_method = bmad_standard 
\end{example}
The name ``switch'' can refer to the parameter (for example, \vn{tracking_method}) or to a value
that it can take (for example, \vn{bmad_standard}). The name ``method'' is used interchangeably with
switch.

%---------------------------------------------------------------------------
\section{Particle Species Names}
\label{s:species.name}
\index{particle species name}
\index{positron}\index{electron}\index{proton}\index{antiproton}\index{photon}
\index{muon}\index{antimuon}\index{pion0}\index{pion-}\index{pion+}

For the purpose of assigning names to simulated particles, particles are divided into four
groups. One group are are ``fundamental particles''. These are:
\begin{example}
  electron,  positron
  muon,      antimuon
  proton,    antiproton
  neutron    anti_neutron
  deuteron   anti_deuteron
  pion+,     pion0,      pion-
  helion     anti_helion        ! #3He 
  photon
\end{example}
For historical reasons, names for the fundamental particles are {\em not} case sensitive.

Another group are atoms. The general syntax for atoms is:
\begin{example}
  \{\#nnn\}AA\{ccc\}
\end{example}
The curly brackets \{...\} denote optional prefixes and suffixes. \vn{AA} here is the
atomic symbol, \vn{\#nnn} is the number of nucleons, and \vn{ccc} is the charge. Examples:
\begin{example}
  parameter[particle] = \#12C+3       ! Triply charged carbon-12.
  parameter[p0c] = 12 * 500e6         ! Reference momentum is total momentum for particle.
  parameter[particle] = He--          ! Doubly charged He.
\end{example}
If the number of nucleons is given, the appropriate weight for that isotope is used. If the number
of nucleons is not present, the mass is an average weighted by the isotopic abundances of the
element. The charge may be given by using the appropriate number of plus (+) or minus (-) signs or
by using a plus or minus sign followed by a number. Thus ``\vn{-{-}-}'' is equivalent to
``-3''. Names here are case sensitive. ``@M'' must be used and not ``@m'' for specifying the mass.
The mass of an atom is adjusted by the number of electrons relative to neutral. That is
\begin{equation}
  m_{\mbox{atom}} = m_{\mbox{neutral atom}} - C \cdot m_{\mbox{electron}}
\end{equation}
where $C$ is the charge in units number of electrons relative to neutral. No adjustment is made for
mass shifts due to finite electron binding energies. This shift is small typically being well less
than 1\% of the mass of the electron.

Anti-atoms made with antimatter have names using the prefix \vn{``anti''}. For example, a bare
gold anti-atom nucleous would be designated \vn{antiAu-79}.

Note: When setting the reference momentum \vn{parameter[p0c]}, or reference total energy
\vn{parameter[E_tot]}, the total for the whole particle is used. {\em Not} the value per nucleon.

Another group of particles are the ``known'' molecules. The syntax for these are:
\begin{example}
  BBB\{@Mxxxx\}\{ccc\}
\end{example}
\vn{@Mxxxx} is the mass in AMU, \vn{ccc} is the charge, and \vn{BBB} is the molecular formula. The
mass may to specified to hundredths of an AMU. The known molecules are:
\begin{example}
  CO       CO2      
  D2       D2O      
  OH       O2      
  H2       H2O      HF
  N2       NH2      NH3      
  CH2      CH3      CH4      
  C2H3     C2H4     C2H5
\end{example}
Like with atoms, if the mass is not specified, the average isotopic mass is used. Examples:
\begin{example}
  C2H3@M28.4+     ! Singly charged C2H3 with mass of 28.4
  CH2             ! Neutral CH2
\end{example}
Like the atomic formulas, molecular formulas are case sensitive. Like atoms, the mass of a known molecule
is adjusted by the number of electrons relative to neutral.

The last group of particle are particles where only the mass and charge are specified.  The syntax
for these are:
\begin{example}
  @Mxxxx\{ccc\}
\end{example}
Example:
\begin{example}
  @M37.54++    ! Doubly charged molecule of mass 37.54 AMU.
\end{example}

Note: When setting the value of a variable to be a particle species ID, use the \vn{species} function as discussed
in \sref{s:functions}.

\newpage

%-----------------------------------------------------------------
\section{Units and Constants}
\label{s:constants}
\index{constants|hyperbf}

\index{MAD!units}
\index{units!with MAD}
\bmad uses SI (Syst\`eme International) units as shown in Table~\ref{t:units}.  Note that \mad uses
different units. For example, \mad's unit of Particle Energy is GeV not eV. 

Note: For compatibility with \mad, the \vn{beam, energy = xxx} command (\sref{s:beam.mad}) uses GeV
and the \vn{emass} and \vn{pmass} constants (see below this section) also use GeV. It is recommended that
the use of these constructs be avoided.

\begin{table}[ht]
\centering
\begin{tabular}{ll} \toprule
  {\em Quantity}     & {\em Units}       \\ \midrule
  Angles             &    radians        \\ 
  Betatron Phase     &    radians        \\
  Current            &    Amps           \\ 
  Frequency          &    Hz             \\ 
  Kick               &    radians        \\ 
  Length             &    meters         \\ 
  Magnetic Field     &    Tesla          \\ 
  Particle Energy    &    eV             \\ 
  RF Phase Angles    &    radians/2$\pi$ \\ 
  Voltage            &    Volts          \\ \bottomrule
\end{tabular}
\caption{Physical units used by \bmad.}
\label{t:units}
\end{table}

\bmad defines commonly used physical and mathematical constants shown in Table~\ref{t:constants}.
All symbols use straight SI units except for \vn{emass} and \vn{pmass} which are provided for
compatibility with \mad and {\large\em should be avoided}.

\index{mass_of}\index{anomalous_moment_of}
As an alternative, the \vn{mass_of}, and \vn{anomalous_moment_of} functions
(\sref{s:functions}) may be used in place of the defined constants for mass and anomalous
magnetic moment.

Note: The standard definition of the magnetic moment $g$-factor for spin 1/2 fundamental particles is
\begin{equation}
  \Bf\mu = g \, \frac{q}{2 \, m} \, \bfS
  \label{mgq2m}
\end{equation}
where $\Bf\mu$ is the magnetic moment, $q$ is the particle charge, and $m$ is the mass. The
anomalous moment $a$ is then defined to be
\begin{equation}
  a = \frac{g-2}{2}
\end{equation}
For nuclei and other composite baryonic particles, it is conventional to define the $g$-factor using
\begin{equation}
  \Bf\mu = g \, \frac{e}{2 \, m_p} \, \bfS
  \label{mgq2m2}
\end{equation}
where $m_p$ is the mass of the proton. This is inconvenient for calculations since an equation like
\Eq{orpt} would not work for all particles. To get around this, the $g$-factors used by Bmad are
always derived from \Eq{mgq2m} (think of this as an ``effective'' $g$-factor).

\begin{table}[h]
\centering
\begin{tabular}{llll} \toprule
  {\em Symbol}          & {\em Value}                & {\em Units} &  {\em Name}           \\ \midrule
  pi                    & 3.141592653589793          &             &                       \\
  twopi                 & 2 * pi                     &             &                       \\
  fourpi                & 4 * pi                     &             &                       \\
  e                     & 2.718281828459045          &             &                       \\
  e_log                 & 2.718281828459045          &             &                       \\
  sqrt_2                & 1.414213562373095          &             &                       \\
  degrad                & 180 / pi                   &             & From rad to deg       \\
  degrees               & pi / 180                   &             & From deg to rad       \\
  raddeg                & pi / 180                   &             & From deg to rad       \\
  anom_moment_deuteron  & $-0.1425617662$            &             & Deuteron anomalous magnetic moment$^*$ \\
  anom_moment_electron  & $0.00115965218128$         &             & Electron anomalous magnetic moment     \\
  anom_moment_muon      & $0.00116592089$            &             & muon anomalous magnetic moment         \\
  anom_moment_proton    & $1.792854734463$           &             & proton anomalous magnetic moment       \\
  anom_moment_he3       & $-4.184153686$             &             & He$^3$ anomalous magnetic moment$^*$   \\ 
  fine_struct_const     & $0.0072973525693$          &             & Fine structure constant                \\
  m_deuteron            & $1.87561294257 \pow{9}$    & eV          & Deuteron mass         \\
  m_electron            & $0.51099895000 \pow{6}$    & eV          & Electron mass         \\
  m_neutron             & $0.93956542052 \pow{9}$    & eV          & Neutron mass          \\
  m_muon                & $105.6583755 \pow{6}$      & eV          & Muon mass             \\
  m_pion_0              & $134.9766 \pow{6}$         & eV          & $\pi^0$ mass          \\
  m_pion_charged        & $139.57018 \pow{6}$        & eV          & $\pi^+$, $\pi^-$ mass \\
  m_proton              & $0.93827208816d \pow{9}$   & eV          & Proton mass           \\
  c_light               & $2.99792458 \pow{8}$       & m/sec       & Speed of light        \\
  r_e                   & $2.8179403262 \pow{-15}$   & m           & Electron radius       \\
  r_p                   & $1.5346982647 \pow{-18}$   & m           & Proton radius         \\
  e_charge              & $1.602176634 \pow{-19}$    & Coul        & Electron charge       \\
  h_planck              & $4.135667696 \pow{-15}$    & eV*sec      & Planck's constant     \\
  h_bar_planck          & $6.582118990 \pow{-16}$    & eV*sec      & Planck / $2\pi$       \\
  emass                 & $0.51099895000 \pow{-3}$   & GeV         & Electron mass (please avoid using)   \\
  pmass                 & $0.93827208816$            & GeV         & Proton mass (please avoid using)     \\ \bottomrule
\end{tabular}
$^*$ Effective anomalous moments. See the discussion after \Eq{mgq2m}.
\caption{Physical and mathematical constants recognized by \bmad.}
\label{t:constants}
\end{table}

%---------------------------------------------------------------------------
\section{Arithmetic Expressions}
\index{arithmetic expressions} 
\label{s:arith}

Arithmetic expressions can be used in a place where a real value is required.
The standard operators are defined: \hfil\break
\hspace*{0.15in}
\begin{tabular}{ll}
  $a + b$           & Addition        \\
  $a - b$           & Subtraction     \\
  $a \, \ast \, b$  & Multiplication  \\
  $a \; / \; b$     & Division        \\
  $a \, ^{\scriptstyle\wedge} \, b$ & Exponentiation  \\
\end{tabular}
\hfil\break
\bmad also has a set of intrinsic functions. A list of these is given
in \sref{s:functions}.

\index{arithmetic expressions!constants}
Literal constants can be entered with or without a decimal point. An
exponent is marked with the letter E. For example
\begin{example}
  1, 10.35, 5E3, 314.159E-2
\end{example}
Symbolic constants can be defined using the syntax
\begin{example}
  constant_name = expression
\end{example}
\index{MAD!syntax compatibility with BMAD}
Alternatively, to be compatible with \mad, using ``:='' instead of ``='' is accepted
\begin{example}
  constant_name := expression
\end{example}
Examples:
\begin{example}
  my_const = sqrt(10.3) * pi^3
  abc     := my_const * 23
\end{example}
\index{MAD!delayed substitution}
Unlike \mad, \bmad uses immediate substitution so that all constants in an expression must have been
previously defined. For example, the following is {\em not} valid:
\begin{example}
  abc      = my_const * 23      ! No: my_const needs to be defined first.
  my_const = sqrt(10.3) * pi^3
\end{example}
here the value of \vn{my_const} is not known when the line ``\vn{abc} = $\ldots$'' is parsed. Note:
To get the effect of delayed evaluation, use \vn{overlay} (\sref{s:overlay}) or \vn{group}
(\sref{s:group}) controller elements.

Once defined, symbolic constants cannot be redefined. For example:
\begin{example}
  my_const = 1
  my_const = 2  ! No! my_const cannot be redefined.
\end{example}
The restriction against redefining constants was implemented to avoid hard to find problems.
On very rare occasions, it is convenient to be able to redefine constants so if the redefinition
has a \vn{redef:} prefix, a constant can be redefined
\begin{example}
  my_const = 1
  redef: my_const = 2  ! OK!
\end{example}
It is advised not to use \vn{redef} unless there a very good reason for its use.

\index{group}\index{overlay}\index{lattice expansion}
\vn{group} (\sref{s:group}) and \vn{overlay} (\sref{s:overlay}) controller elements are an
exception to the immediate evaluation rule. Since controller elements may control elements
that do not exist until \vn{lattice expansion} (\sref{s:expand}), the arithmetic
expressions associated with controller elements are not evaluated until lattice expansion.
Example:
\begin{example}
  s_20W: sextupole, l = 0.27
  sk: overlay = \{s_20W[a1]:-2*s_20W[L]\}, var = \{k1\}, k1 = 0.2
  s_20W[L] = 0.34
  s_30E: s_20W
  ...
  expand_lattice
\end{example}
Here the expression of overlay \vn{sk} is evaluated, when the lattice is expanded, to be
\vn{-0.68 = -2*0.34}. This uses the length of element \vn{s_20W} at the point when
the lattice is expanded and not at the point when \vn{sk} was defined. Additionally, the
element \vn{s_30E}, which inherits the attributes of \vn{s_20W}, inherits a value of zero
for \vn{a1} (skew multipole moment) since inheritance uses immediate evaluation just like
the setting of constants.

Element attributes can be used after they have been defined but not
before.  Example:
\begin{example}
  sa: sextupole, l = 0.3, k2 = 0.01 * sa[L]  ! Good
  sb: sextupole, k2 = 0.01 * sb[L], l = 0.3  ! BAD SET OF K2. L IS DEFINED AFTER.
\end{example}
In this example, the \vn{k2} attribute of element \vn{sa} is correctly
set since \vn{k2} is defined after \vn{l}. On the other hand, \vn{k2}
of element \vn{sb} will have a value of zero since \vn{l} of \vn{sb}
defaults to zero before it is set.

One potential pitfall with immediate substitution is that when
an element attribute changes, it does not affect prior evaluations.
Example:
\begin{example}
  s1: sextupole, k2 = 2.3
  aa = s1[k2]              ! aa = 2.3
  s1[k2] = 1.7             ! value of aa does not change
\end{example}
Here the value of constant \vn{aa} will remain fixed at 2.3 no matter how
the value of \vn{s1[k2]} is altered after \vn{aa} is defined.

Another potential pitfall is when using
dependent element attributes (\sref{s:depend}). For example:
\begin{example}
  b01w: sbend, l = 0.5, angle = 0.02
  a_const = b01w[g]    ! No: bend g has not yet been computed!
\end{example}
Here the bend strength \vn{g} (\sref{s:bend}) will eventually be computed to be 0.04 (= angle / l)
but that computation does not happen until lattice expansion (\sref{s:expand}). In this case, the
value of \vn{a_const} will be the default value of \vn{g} which is zero.  As a rule of thumb, never
rely on dependent attributes having their correct value.

%---------------------------------------------------------------------------
\section{Intrinsic functions}
\label{s:functions}
\index{intrinsic functions}

\index{sqrt}\index{log}\index{exp}\index{sin}\index{cos}\index{tan}\index{factorial}
\index{asin}\index{acos}\index{atan}\index{abs}\index{ran}\index{ran_gauss}
\index{mass_of}\index{charge_of}\index{anomalous_moment_of}\index{species}
The following intrinsic functions are recognized by \bmad: \hfil\break
\hspace*{0.15in}
\begin{tabular}{ll}
  \vn{sqrt}(x)                  & Square Root                                    \\
  \vn{log}(x)                   & Logarithm                                      \\
  \vn{exp}(x)                   & Exponential                                    \\
  \vn{sin}(x), \vn{cos}(x)      & Sine and cosine                                \\
  \vn{tan}(x), \vn{cot}(x)      & Tangent and cotangent                          \\
  \vn{sinc}(x)                  & Sin(x)/x Function                              \\
  \vn{asin}(x), \vn{acos}(x)    & Arc sine and Arc cosine                        \\
  \vn{atan}(x)                  & Arc tangent                                    \\
  \vn{atan2}(y, x)              & Arc tangent of y/x                             \\

  \vn{sinh}(x), \vn{cosh}(x)    & Hyperbolic sine and cosine                     \\
  \vn{tanh}(x), \vn{coth}(x)    & Hyperbolic tangent and cotangent               \\
  \vn{asinh}(x), \vn{acosh}(x)  & Hyperbolic arc sine and Arc cosine             \\
  \vn{atanh}(x), \vn{acoth}(x)  & Hyperbolic arc tangent and cotangent           \\

  \vn{abs}(x)                   & Absolute Value                                 \\
  \vn{factorial}(n)             & Factorial                                      \\
  \vn{ran}()                    & Random number between 0 and 1                  \\
  \vn{ran_gauss}()              & Gaussian distributed random number             \\
  \vn{ran_gauss}(sig_cut)       & Gaussian distributed random number             \\
  \vn{int}(x)                   & Nearest integer with magnitude less then x     \\
  \vn{nint}(x)                  & Nearest integer to x                           \\
  \vn{sign}(x)                  & 1 if x positive, -1 if negative, 0 if zero     \\
  \vn{floor}(x)                 & Nearest integer less than x                    \\
  \vn{ceiling}(x)               & Nearest integer greater than x                 \\
  \vn{modulo}(a, p)             & a - floor(a/p) * p. Will be in range [0, p].   \\
  \vn{mass_of}(A)               & Mass of particle A                             \\
  \vn{charge_of}(A)             & Charge, in units of the elementary charge, of particle A \\
  \vn{anomalous_moment_of}(A)   & Anomalous magnetic moment of particle A        \\
  \vn{species}(A)               & Species ID of A
\end{tabular}

\index{ran_seed}
\vn{ran_gauss} is a Gaussian distributed random number with unit RMS.  Both \vn{ran} and
\vn{ran_gauss} use a seeded random number generator.  To choose the seed set
\begin{example}
  parameter[ran_seed] = <Integer>
\end{example}
A \vn{value} of zero will set the seed using the system clock so that different sequences of random
numbers will be generated each time a program is run.  The default behavior if
\vn{parameter[ran_seed]} is not present is to use the system clock for the seed.

The \vn{ran_gauss(cut)} function with an argument truncates the distribution so that no values are returned
with an absolute value greater than \vn{cut}. If \vn{cut} is non-positive, it is ignored so that, for example,
\vn{ran_gauss(-1)} is equivalent to \vn{ran_gauss()}.

\index{expand_lattice}
If an element is used multiple times in a lattice, and if \vn{ran} or \vn{ran_gauss} is used to set
an attribute value of this element, then to have all instances of the element have different
attribute values the setting of the attribute must be after the lattice has been expanded
(\sref{s:expand}). For example:
\begin{example}
  a: quad, ... 
  a[x_offset] = 0.001*ran_gauss()
  my_line: line = (a, a)
  use, my_line
\end{example}
Here, because \bmad does immediate evaluation, the \vn{x_offset} values for \vn{a} gets set in line
2 and so both copies of \vn{a} in the lattice get the same value. This is probably not what is
wanted.  On the other hand if the attribute is set after lattice expansion:
\begin{example}
  a: quad, ...
  my_line: line = (a, a)
  use, my_line
  expand_lattice
  a[x_offset] = 0.001*ran_gauss()
\end{example}
Here the two \vn{a} elements in the lattice get different values for \vn{x_offset}.

The following functions take a species ID as the argument:
\begin{example}
  mass_of(A),              charge_of(A)
  anomalous_moment_of(A),  species(A)
\end{example}
See \sref{s:species.name} for the syntax of naming particles.

The \vn{mass_of}, \vn{charge_of}, and \vn{anomalous_moment_of} functions give the mass of, charge of
(in units of the elementary charge), and anomalous moment of, a particle.  Example:
\begin{example}
  parameter[particle] = deuteron
  am = anomalous_moment_of(parameter[particle])^2
  my_particle = species(He++)      ! my_particle now represents He++
  chg1 = charge_of(my_particle)    ! chg = charge of He++
  chg2 = charge_of(He++)           ! Same as previous line
  chg3 = charge_of(species(He++))  ! Same as previous line
\end{example}
The \vn{species} function is needed in the definition of \vn{my_particle} in the above example so
that \bmad knows that the string ``He++'' represents a type of particle. Inside functions like
\vn{mass_of}, the use of \vn{species} is optional since, in this case, \bmad can correctly parse the
argument. 

The value returned by the \vn{mass_of} function accounts for the ionization state of a particle in
that there is a correction for the change in the number of electrons a particle has.  Thus the
values of \vn{mass_of{\#3He}} will be heavier than \vn{mass_of{\#3He++}} by two electron masses.
This correction only involves multiples of the electron mass and variations in the particle mass due
to electron binding energies is not accounted for. These binding energy corrections are generally
very small. If the isotopic state is not specified for an atom, the average weighted by the natural
abundance is used.

%-----------------------------------------------------------------------------
\section{Statement Order}
\label{s:state.order}
\index{statement order|hyperbf}

With some exceptions, statements in a lattice file can be in any order. For example, the lines
(\sref{s:lines.wo.arg}) specified in a \vn{use} statement (\sref{s:use}) can come after the \vn{use}
statement. And \vn{group} (\sref{s:group}) and \vn{overlay} (\sref{s:overlay}) controller elements
may be defined before the slave elements whose parameters they control are defined.

The exceptions to this rule are:
\begin{itemize}
\item 
If there is an \vn{expand_lattice} statement (\sref{s:expand.lat}), everything necessary for
lattice expansion must come before. In particular, all \vn{line}s
(\sref{s:lines.wo.arg}), \vn{list}s (\sref{s:replace.list}), and \vn{use} (\sref{s:use}) statements
necessary for lattice expansion must come before.
\item
Immediate evaluation of arithmetic expressions (\sref{s:arith}) mandates that values be defined
before use.
\item
A lattice element must be defined before any of its parameters are set. Example:
\begin{example2}
  pp[z_offset] = 0.1    ! WRONG! PP HAS NOT BEEN DEFINED YET!
  pp: patch             ! Here PP is defined
\end{example2}
In this example, the \vn{z_offset} of the element \vn{pp} is set before \vn{pp} has been
defined. This is an error. As a corollary to this rule, element parameters that are set 
using wild card characters will only affect those parameters that have been already defined. For
example:
\begin{example2}
  crystal::*[b_param] = 0.2
  c5: crystal
\end{example2}
In this example, the \vn{b_param} of all \vn{crystal} elements is set to \vn{0.2} {\em except} for
\vn{c5} and all other crystal elements that are defined after the set.
\end{itemize}

%-----------------------------------------------------------------------------
\section{Print Statement}
\label{s:print}
\index{print statement|hyperbf}

The \vn{print} statement prints a message at the terminal when the 
lattice file is parsed by a program. Syntax:
\begin{example}
  print <string>
\end{example}
Where \vn{<string>} is the string to be printed. Variable values can be printed by using
\textbf{back-tick} characters. For example:
\begin{example}
  print Remember! Q01 quad strength of \backtick{}q01[k1]\backtick not yet optimized!
  print Optimization is as easy as 2 + 2 = \backtick{}2+2\backtick.
\end{example}
will result in the following being printed:
\begin{example}
  Message in Lattice File: Remember! Q01 quad strength of 0.4526 not yet optimized!
  Message in Lattice File: Optimization is as easy as 2 + 2 = 4.
\end{example}

The \vn{print} statement is useful to remind someone using the lattice of important details.

%-----------------------------------------------------------------------------
\section{Title Statement}
\index{title statement|hyperbf}

The \vn{title} statement sets a title string which can be used by a program.  For consistency with
\mad there are two possible syntaxes
\begin{example}
  title, <String>
\end{example}
or the statement can be split into two lines
\begin{example}
  title
  <String>
\end{example}
For example
\begin{example}
  title
  "This is a title"
\end{example}

%--------------------------------------------------------------------------
\section{Call Statement}
\label{s:call}
\index{call statement|hyperbf}

It is frequently convenient to separate the lattice definition into several files.  Typically there
might be a file (or files) that define the layout of the lattice (something that doesn't change
often) and a file (or files) that define magnet strengths (something that changes more often).  The
\vn{call} is used to read in separated lattice files. The syntax is
\begin{example}
  call, filename = <file-name>
\end{example}
Example:
\begin{example}
  call, filename = "../layout/my_layout.bmad"      ! Relative pathname
  call, filename = "/nfs/cesr/lat/my_layout.bmad"  ! Absolute pathname
  call, filename = "$LATDIR/my_layout.bmad"        ! Absolute pathname
\end{example}
Environment variables in the file name will be expanded. \bmad will read the called file until a
\vn{return} or \vn{end_file} statement is encountered or the end of the file is reached.

For filenames that have a relative pathname, the called file will be searched for relative to the
directory of the calling file.  Thus, in the above example, if the file containing the call
statements is in the directory \vn{/path/to/lat_dir}, the first call will open the file:
\begin{example}
  /path/to/lat_dir/../layout/my_layout.bmad 
\end{example}
To call a file relative to the current working directory, use the environment variable
\vn{PWD}. Example:
\begin{example}
  call, filename = $PWD/here.bmad
\end{example}

Where a called file is searched for may be modified by using a \vn{use_local_lat_file}
statement. See Section~\sref{s:use.loc} for more details.

%--------------------------------------------------------------------------
\section{Inline Call}
\label{s:call.inline}
\index{call!inline}

Any lattice elements will have a set of attributes that need to be defined.  As a convenience, it is
possible to segregate an element attribute or attributes into a separate file and then ``call'' this
file using an ``inline call''. The inline call has three forms:
\begin{example}
  <ele_name>: <ele_type>, ..., call::<file_name>, ...                       ! or
  <ele_name>: <ele_type>, ..., <attribute_name> = call::<file_name>, ...    ! or
  <ele_name>[<attribute_name>] = call::<file_name>
\end{example}
where \vn{<attribute_name>} is the name of the attribute and \vn{<file_name>} is the name of the
where the attribute structure is given. The Environment variables in the file name will be expanded.
Example:
\begin{example}
  c: crystal, call::$AB/my_curvature.bmad, h_misalign = call::my_surface.bmad, ...
\end{example}  

For \vn{grid_fields}, which can take some time to parse, a \vn{HDF5} binary file can to created and then
the HDF5 file, which must have a \vn{.h5} or \vn{.hdf5} suffix, can be read in with an inline call Example:
\begin{example}
  qq: quadrupole, grid_field = call::my_grid.h5, ...
\end{example}
To create hdf5 files, first create a lattice with the grid_field defined with plain text. Next read
the lattice into any program that can create \bmad lattice files (for example, the \tao program
(\sref{s:tao.intro}) can do this) and have the program then generate a lattice file.

%--------------------------------------------------------------------------
\section{Use_local_lat_file Statement}
\label{s:use.loc}
\index{use_local_lat_file_statement|hyperbf}

It is sometimes convenient to override where \bmad looks for called files (see \sref{s:call}). For
example, suppose it is desired to temporarily override the settings in a called file without
modifying the called file itself. In this case, the \vn{use_local_lat_file} statement can be
used. When this statement is encountered in a lattice file, the local directory (that is, the
directory from which the program is run) is searched first for the called file and if a file of the
correct name is found, that file is used.

An example will make this clear. Suppose lattice file \vn{/A/lat.bmad}
contains the call:
\begin{example}
  call, filename = "/B/sub.bmad"
\end{example}
Now suppose that you want to use \vn{lat.bmad} with a modified \vn{sub.bmad} but you do not want to
modify \vn{/A/lat.bmad} or \vn{/B/sub.bmad}. The solution is to create two new files. One file, call
it \vn{new.bmad}, which can be situated in any directory, has two lines in it:
\begin{example}
  use_local_lat_file
  call, filename = "/A/lat.bmad"
\end{example}
The second new file is the modified \vn{sub.bmad} and it must be in the directory from which the
program is run.

%--------------------------------------------------------------------------
\section{No_Superimpose Statement}
\label{s:no.sup}
\index{no_superimpose statement}

In certain cases it is useful to turn off superposition (\sref{s:super}). The \vn{no_superposition}
statement will do this. To turn off all superpositioning, this statement can appear anywhere as long
as it is before any \vn{expand_lattice} (\sref{s:expand.lat}) statement. If the lattice has an
\vn{expand_lattice} statement, and the \vn{no_superposition} statement appears after, the
\vn{no_superposition} statement will only block superpositions that are defined after the
\vn{no_superposition} statement.

%--------------------------------------------------------------------------
\section{Return and End_File Statements}
\index{return statement|hyperbf}
\index{end_file statement|hyperbf}

\vn{Return} and \vn{end_file} have identical effect and tell \bmad to ignore anything beyond the
\vn{return} or \vn{end_file} statement in the file.

%----------------------------------------------------------------------------
\section{Expand_Lattice Statement}
\label{s:expand.lat}
\index{expand_lattice|hyperbf}

Normally, lattice expansion happens automatically at the end of the parsing of the lattice file but
an explicit \vn{expand_lattice} statement in a lattice file will cause immediate expansion. See
\sref{s:expand} for details. Subsequent \vn{expand_lattice} statements after the first one are
ignored and have no effect on the lattice.

%----------------------------------------------------------------------------
\section{Lattice Expansion}
\label{s:expand}
\index{lattice!expansion|hyperbf}

At some point in parsing a lattice file, the ordered sequence (or sequences if there are multiple
branches) of elements that form a lattice must be constructed. This process is called \vn{lattice
expansion} since the element sequence can be built up from sub--sequences
(\sref{c:sequence}). Normally, lattice expansion happens automatically at the end of the parsing of
the lattice file (or files) but an explicit \vn{expand_lattice} statement in a lattice file will
cause immediate expansion. The reason why lattice expansion may be necessary before the end of the
file is due to the fact that some operations need to be done after lattice expansion. This includes:
\begin{Itemize}
\item 
\index{ran}
\index{ran_gauss}
The \vn{ran} and \vn{ran_gauss} functions, when used with elements
that show up multiple times in a lattice, generally need to be used
after lattice expansion. See \sref{s:functions}.
\item
Some dependent parameters may be set as if they are independent
parameters but only if done before lattice expansion. See \sref{s:depend}.
\index{multipass}
\item 
Setting the \vn{phi0_multipass} attribute for an 
\vn{Lcavity} or \vn{RFcavity} multipass
slave may only be done after lattice expansion (\sref{s:multipass}).
\item
\index{tags for Lines and Lists}
Setting individual element attributes for tagged elements can only be done
after lattice expansion (\sref{s:tag}).
\end{Itemize}

Notice that all \vn{line}s (\sref{s:lines.wo.arg}), \vn{list}s (\sref{s:replace.list}), and \vn{use}
(\sref{s:use}) statements necessary for lattice expansion must come before an \vn{expand_lattice}
statement.

Lattice expansion is only done once so it is an error if multiple
\vn{expand_lattice} statements are present.

The steps used for lattice expansion are:
\begin{enumerate}
\item
Instantiate all of the lines listed in the last \vn{use} statement (\sref{s:use}). If an instantiated line has
\vn{fork} or \vn{photon_fork} (\sref{s:fork}) elements, instantiate the lines connected to the fork
elements if the \vn{fork} or \vn{photon_fork} is connected to a new branch. Instantiation of a given
line involves:
\begin{enumerate}
\item
Line expansion (\sref{c:sequence}) where the element sequence is constructed from the line and sub-lines.
\item
Adding any superpositions (\sref{s:super}).
\end{enumerate}
\item
Form multipass lords and mark the appropriate multipass slaves (\sref{s:multipass}).
\item
Add girder control elements (\sref{s:girder}).
\item
Add group (\sref{s:group}) and overlay (\sref{s:overlay}) control elements.
\end{enumerate}

\index{secondary lattice file}
A lattice file where all the statements are post lattice expansion valid is called a ``\vn{secondary
lattice file}''.  To promote flexibility, \bmad has methods for parsing lattices in a two step
process: First, a ``primary'' lattice file that defines the basic lattice is read. After the primary
lattice has been parsed and lattice expansion has been done, the second step is to read in one or
more secondary lattice files. Such secondary lattice files can be used, for example, to set such
things as element misalignments. The point here is that there are no calls (\sref{s:call}) of the
secondary files in the primary file so the primary lattice file does not have to get modified when
different secondary files are to be used.

%--------------------------------------------------------------------------
\section{Calc_Reference_Orbit Statement}
\label{s:calc.ref.orbit}

The \vn{calc_reference_orbit} statement triggers the computation of the ``standard'' reference orbit
which is defined to be the closed orbit if the geometry (set by \vn{parameter[geometry]}
(\sref{s:param})) is closed and which is defined to be the orbit as calculated from the starting
position (set by any \vn{particle_start} statements (\sref{s:beam.start})) if the geometry is open.

The \vn{calc_reference_orbit} statement is used before a \vn{merge_elements} statement
(\sref{s:merge.eles}) to signal that the maps of the Taylor elements produced by the
\vn{merge_elements} statement are computed from the standard reference orbit and not the zero
orbit.

The \vn{calc_reference_orbit} statement must come after an \vn{expand_lattice} command
(\sref{s:expand}).

The \vn{calc_reference_orbit} statement can only be used if the standard reference orbit can
be computed. For example, for a lattice with closed geometry, the closed orbit must exist.

The \vn{calc_reference_orbit} statement has no arguments. Example:
\begin{example}
  expand_lattice          ! Expand the lattice.
  calc_reference_orbit    ! Compute the reference orbit.
\end{example}

%--------------------------------------------------------------------------
\section{Merge_Elements Statement}
\label{s:merge.eles}

The \vn{merge_elements} statement is used to merge groups of consecutive elements into single
taylor elements (\sref{s:taylor}) for faster tracking. The syntax of this statement is
\begin{example}
  merge_elements <list>
\end{example}
where \vn{<list>} is a list of elements that are {\em not} to be combined.

Example:
\begin{example}
  call, file = my_ring.bmad     ! Read in a lattice
  parameter[taylor_order] = 3   ! Set the taylor order for the maps
  expand_lattice                ! Expand the lattice
  calc_reference_orbit          ! ref orbit calc.
  merge_elements sbend::*, bb ! Merge everything but sbend elements and 
                                !   any elements named BB.
\end{example}
In this example, groups of elements that are between bends (with the exception of any elements named
``BB''), are replaced by \vn{taylor} elements. The order of the Taylor maps is set by
\vn{parameter[taylor_order]} (\sref{s:param}).

The \vn{merge_elements} statement must come after an \vn{expand_lattice} command
(\sref{s:expand}).

If there is a \vn{calc_reference_orbit} statement before the \vn{merge_elements} statement, the
``standard'' reference orbit (\sref{s:calc.ref.orbit}) is used for the computation of the Taylor
maps. Otherwise the zero orbit is used as the reference orbit.

%--------------------------------------------------------------------------
\section{Combine_Consecutive_Elements Statement}
\label{s:merge.ele}

The \vn{combine_consecutive_elements} statement is used to combine consecutive elements with the same name into a
single element. If a marker element has been placed in between two element with the same name, the
marker element will be discarded. This can be a useful statement to add when given a lattice where
elements have been split into two.\footnote
  {
This is common in lattices translated from MAD.
  }
The \vn{combine_consecutive_elements} statement must come after lattice expansion (\sref{s:expand}). Example:
\begin{example}
  m: marker
  myline: line = (q1, m, q1)
  use, myline
  expand_lattice
  combine_consecutive_elements
\end{example}
In this case the finished line will will have a single \vn{q1} element whose length will be twice the length of q1

%--------------------------------------------------------------------------
\section{Remove_Elements Statement}
\label{s:remove.ele}

The \vn{remove_elements} statement is used to remove elements from the lattice. The
\vn{remove_elements} statement must come after lattice expansion (\sref{s:expand}). The syntax of
the \vn{remove_elements} statement is
\begin{example}
  remove_elements <element-list>
\end{example}
where \vn{<element-list>} is a list of elements (\sref{s:ele.match}). For example
\begin{example}
  expand_lattice              ! Lattice expansion must happen first
  remove_elements overlay::*
\end{example}
In this example, all overlay elements are to be removed. This is useful, for example, when direct control
of overlay slave parameters is desired.

%--------------------------------------------------------------------------
\section{Slice_Lattice Statement}
\label{s:slice}

The \vn{slice_lattice} statement is used to remove elements from the lattice. The \vn{slice_lattice}
is useful when analysis of only part of the lattice is desired and the analysis of the entire
lattice can take a significant amount of time.

The \vn{slice_lattice} statement must come after lattice expansion (\sref{s:expand}). The syntax of
the \vn{slice_lattice} statement is
\begin{example}
  slice_lattice <element-list>
\end{example}
where \vn{<element-list>} is a list of elements (\sref{s:ele.match}). For example
\begin{example}
  expand_lattice           ! Lattice expansion must happen first
  slice_lattice q1##2:357,end
\end{example}
In this example, all elements outside of the range from element \vn{q1\#\#2} (the second instance of
\vn{q1} in the lattice) to element with index number 357 are discarded except for the element named
\vn{end} (which is typically the last element in any lattice branch). Additionally, the lord
elements (\sref{s:lord.slave}) of any elements that remain are retained and the \vn{beginning}
element at the start of any branch is also retained.

For any lattice branch where elements are removed, the Twiss parameters and reference energy is
computed, and the Twiss parameters and reference energy at the entrance end of the first element
that is not removed is transferred to the beginning branch element. The branch \vn{geometry} is also
set to \vn{open}.

For a lattice branch with a closed geometry, the Twiss parameters are computed with the RF on. That
is, the reference momentum at the beginning of the sliced lattice branch, which is computed from the
closed orbit phase space $p_z$ of the unsliced lattice, may be non-zero. This will affect the Twiss
calculation. If this is not what is wanted, the RF can be turned off before the \vn{slice_lattice}
command which will ensure the reference momentum is zero at the beginning of the lattice
branch. Example:
\begin{example}
  expand_lattice
  rfcavity::*[is_on] = False    ! Turn RF off
  slice_lattice q1##2:357       ! Slice the lattice
  ! This shows how to reset the RF and geometry if needed.
  rfcavity::*[is_on] = True     ! Turn RF back on
  parameter[geometry] = closed  ! Change the geometry.
  ... etc ...
\end{example}

To create a lattice slice that wraps around the lattice ends, that is, joins a section at the end of
the lattice followed by a section at the beginning of the lattice, use a \vn{start_branch_at}
statement before a slice_lattice statement. Example:
\begin{example}
  expand_lattice
  start_branch_at Q9
  slice_lattice Q9:Q1   ! With Q1 being before Q9 in the original lattice.
\end{example}

%--------------------------------------------------------------------------
\section{Start_Branch_At Statement}
\label{s:start.branch}

The \vn{start_branch_at} statement is used to shift the starting point of a lattice branch while
keeping the relative order of the elements the same. The syntax of the \vn{start_branch_at}
statement is
\begin{example}
  start_branch_at <lattice-element>                    ! or
  start_branch_at, move_end_marker <lattice-element>
\end{example}
where \vn{<lattice-element>} is the name or index of a lattice element to be moved to the start of
the branch the element is in. The \vn{start_branch_at} statement must come after lattice expansion
(\sref{s:expand}).

The shifting only applies to elements in the tracking part of the lattice (\sref{s:lord.slave}). The
\vn{BEGINNING} element of the branch (\sref{s:use}) and any lord elements are unaffected. The
\vn{end} marker element (\sref{s:branch.construct}), if it is present (that is, if no
\vn{parameter[no_end_marker]} is used), will also remain at the end of the branch except if the
\vn{move_end_marker} option is used with \vn{start_branch_at}. For example:
\begin{example}
  expand_lattice           ! Lattice expansion must happen first
  start_branch_at Q3
\end{example}
In this example, the elements in the lattice branch containing \vn{Q3} would be shifted so that the
\vn{Q3} element is the first element in the branch. Thus, if the lattice branch originally consisted
of the elements
\begin{example}
  Beginning, Q1,  Q2,  Q3,  Q4,  Q5,  Q6, End
\end{example}
then the shifted lattice would be
\begin{example}
  Beginning, Q3,  Q4,  Q5,  Q6, Q1,  Q2, End
\end{example}
Elements that originally come before the new starting point are always wrapped around to the end of
the branch. If the move_end_marker is present:
\begin{example}
  start_branch_at, move_end_marker Q3
\end{example}
then the shifted lattice would be
\begin{example}
  Beginning, Q3,  Q4,  Q5,  Q6, End, Q1,  Q2
\end{example}

Also see the \vn{slice_lattice} statement (\sref{s:slice}).

%--------------------------------------------------------------------------
%\section{Rename Statement}
%\label{s:rename}
%
%The \vn{rename} statement is used to rename elements in the lattice to produce elements with unique names.
%For example, suppose a lattice has a number of elements named ``

%--------------------------------------------------------------------------
\section{Debugging Statements}
\label{s:debug}
\index{write_digested statement}
\index{no_digested statement}
\index{parser_debug statement}
\index{debug_marker statement}
\index{no_digested statement}
\index{lattice files!parser debugging}

There are a few statements which can help in debugging the \bmad lattice parser itself. That is,
these statements are generally only used by programmers.  These statements are:
\begin{example}
  debug_marker
  no_digested
  parser_debug
  write_digested
\end{example}

The \vn{debug_marker} statement is used for marking a place in the lattice file
where program execution is to be halted. This only works when running
a program in conjunction with a program debugging tool. 

The \vn{no_digested} statement if present, will prevent \bmad from 
creating a digested file (\sref{s:digested}). That is, the lattice file will always
be parsed when a program is run. The \vn{write_digested} statement will cancel a
\vn{no_digested} statement.

The \vn{parser_debug} statement will cause information about the lattice to be printed out at the
terminal. The syntax is
\begin{example}
  parser_debug <switches>
\end{example}
Valid \vn{<switches>} are
\begin{example}
  particle_start      ! Print the particle_start information.
  const               ! Print table of constants defined in the lattice file.
  ele <n1> <n2> ...   ! Print full info on selected elements.
  lattice             ! Print a list of lattice element information.
  lord                ! Print full information on all lord elements.
  seq                 ! Print sequence information.
  slave               ! Print full information on all slave elements.
  time                ! Print timing information.
\end{example}
Here $<n1>$, $<n2>$, etc. are the index of the selected elements in
the lattice.  Example
\begin{example}
  parser_debug var lat ele 34 78
\end{example}


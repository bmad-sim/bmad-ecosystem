\chapter{The lat_struct}
\label{c:lat.struct}
\index{lat_struct|hyperbf}

The \vn{lat_struct} is the structure that holds of all the information
about a lattice (\sref{s:lattice.def}). The components of a
\vn{lat_struct} are listed in \fig{f:lat.struct}.
\begin{figure}[htb]
\centering
\begin{verbatim}
type lat_struct
  character(40) use_name                   ! Name in USE statement
  character(40) lattice                    ! Lattice name
  character(80) input_file_name            ! Lattice input file name
  character(80) title                      ! From TITLE statement
  character(100), allocatable :: print_str(:) ! Saved print statements
  type (expression_atom_struct), allocatable :: constant(:)  ! Constants defined in the lattice
  type (mode_info_struct) a, b, z             ! Tunes, etc.                       
  type (lat_param_struct) param               ! Parameters                        
  type (bookkeeping_state_struct) lord_state  ! lord bookkeeping status.          
  type (ele_struct) ele_init                  ! For use by any program            
  type (ele_struct), pointer ::  ele(:) => null()  ! Array of elements [=> branch(0)].
  type (branch_struct), allocatable :: branch(:)   ! Branch(0:) array
  type (control_struct), allocatable :: control(:) ! Control list
  type (photon_reflect_surface_struct), pointer :: surface(:) => null()
  type (coord_struct) particle_start      ! Starting coords
  type (pre_tracker_struct) pre_tracker   ! For OPAL/IMPACT-T
  type (nametable_struct) nametable       ! For quick searching by element name.
  real(rp), allocatable :: custom(:)      ! Custom attributes.
  integer version                         ! Version number
  integer n_ele_track                     ! Number of lat elements to track through.
  integer n_ele_max                       ! Index of last valid element in %ele(:) array
  integer n_control_max                   ! Last index used in control_array
  integer n_ic_max                        ! Last index used in ic_array
  integer input_taylor_order              ! As set in the input file
  integer, allocatable :: ic(:)           ! Index to %control(:)
  integer :: photon_type = incoherent$    ! Or coherent$. For X-ray simulations.
  integer :: creation_hash                ! Number to determine if lattice is different.
end type
\end{verbatim}
\caption{Definition of the \vn{lat_struct}.}
\label{f:lat.struct}
\end{figure}

\index{lat_struct!\%ele_init}
The \vn{%ele_init} component within the \vn{lat_struct} is not used
by \bmad and is available for general program use.

%----------------------------------------------------------------------------
\section{Initializing}
\label{s:lat:init}
\index{lat_struct!initializing}

Normally initialization of a \vn{lat_struct} lattice is done by
\Hyperref{r:bmad.parser}{bmad_parser} when a lattice file is parsed and does not have to be done by
the programmer. When a programmer needs to initialize a lattice, however,
\Hyperref{r:init.lat}{init_lat} is used to initialize the lattice with a single branch. After this
initial setup, the routines \Hyperref{r:allocate.branch.array}{allocate_branch_array} and
\Hyperref{r:allocate.lat.ele.array}{allocate_lat_ele_array} can be used to set up additional
branches. Example:
\begin{example}
  type (lat_struct) lat
  ...
  call init_lat (lat, 1000)           ! Branch(0) has 1001 elements.
  call allocate_branch_array (lat, 2) ! Allocate Branch(1) and Branch(2).
  call allocate_lat_ele_array (lat, 20, 1)  ! Branch(1) has 21 elements
  call allocate_lat_ele_array (lat, 30, 2)  ! Branch(2) has 31 elements.
\end{example}


%----------------------------------------------------------------------------
\section{Pointers}
\label{s:lat:point}
\index{lat_struct!pointers}

Since the \vn{lat_struct} has pointers within it, there is an extra burden on the programmer to make
sure that allocation and deallocation is done properly. To this end, the equal sign has been
overloaded by the routine \vn{lat_equal_lat} so that when one writes
\begin{example}
    type (lat_struct) lattice1, lattice2
    ! ... some calculations ...
    lattice1 = lattice2
\end{example}
the pointers in the \vn{lat_struct} structures will be handled properly. The result will be that
lattice1 will hold the same information as \vn{lattice2} with all the lattice elements in the same
place but the pointers in \vn{lattice1} will point to different locations in physical memory so that
changes to one lattice will not affect the other.

\Hyperref{r:deallocate.lat.pointers}{deallocate_lat_pointers} Initial allocation of the pointers in
a \vn{lat_struct} variable is generally handled by the \Hyperref{r:bmad.parser}{bmad_parser} and
\Hyperref{r:lat.equal.lat}{lat_equal_lat} routines.  Once allocated, local \vn{lat_struct} variables
must have the save attribute or the pointers within must be appropriately deallocated before leaving
the routine.
\begin{example}
  type (lat_struct), save :: lattice     ! Either do this at the start or ...
  ...
  call deallocate_lat_pointers (lattice) ! ... Do this at the end.
\end{example}
Using the save attribute will generally be faster but will use more
memory. Typically using the save attribute will be the best choice.

%----------------------------------------------------------------------------
\section{Branches in the lat_struct}
\label {s:lat.struct}

\index{lat_struct!\%branch(:)}
\index{fork}\index{photon_fork}\index{branch!root}
The lattice is divided up into the ``root branch'' (\sref{s:use}) and,
if there are \vn{fork} or \vn{photon_fork} elements, a number ``forked''
\vn{branches}.

\index{branch_struct}
The branches of a lattice is contained in the \vn{lat%branch(0:)}
array. The \vn{%branch(0:)} array is always indexed from 0 with the
$0$ branch being a root branch. The definition of the
\vn{branch_struct} structure is
\begin{example}
  type branch_struct
    character(40) name
    integer ix_branch                    ! Index in lat%branch(:) array.
    integer ix_from_branch               ! -1 => No forking element to this branch.
    integer ix_from_ele                  ! Index of forking element
    integer, pointer :: n_ele_track      ! Number of tracking elements
    integer, pointer :: n_ele_max
    type (mode_info_struct), pointer :: a, b, z
    type (ele_struct), pointer :: ele(:)
    type (lat_param_struct), pointer :: param
    type (wall3d_struct), pointer :: wall3d(:)
    type (ptc_branch1_info_struct) ptc
    type (normal_form_struct) normal_form_with_rf, normal_form_no_rf
  end type
\end{example}
The value of the \vn{%branch(i)%ix_branch} conponent is the branch
index and will thus have the value \vn{i}. This can be useful when
passing a branch to a subroutine.  The \vn{%branch(i)%ix_from_branch}
component gives the branch index of the branch that the $i$\Th branch
branched off from.  \vn{%branch(i)%ix_from_ele} gives the index in the
\vn{%branch(j)%ele(:)} array of the \vn{fork} or \vn{photon_fork}
element that marks the beginning of the $i$\Th branch. Example:
\begin{example}
  type (lat_struct), target :: lat
  type (ele_struct), pointer :: ele
  ...
  ib = lat%branch(3)%ix_from_branch
  ie = lat%branch(3)%ix_from_ele
  ! ele is the fork or photon_fork element for lat%branch(3)
  ele => lat%branch(ib)%ele(ie)
  ! This is the same as the above.
  ele => pointer_to_ele(lat%branch(3)%ix_from_branch, lat%branch(3)%ix_from_ele)
\end{example}

\index{lat_struct!\%ele(:)}\index{lat_struct!\%n_ele_track}
\index{lat_struct!\%n_ele_max}\index{ele_struct!in lat_struct} 
\index{girder}\index{overlay}\index{group}\index{ramper}
The \vn{%branch%ele(:)} array holds the array of elements in the
branch. Historically, the \vn{lat_struct} was developed at the start
of the \bmad project and branches were implemented well after that. To
maintain compatibility with older code, the following components point
to the same memory blocks
\begin{example}
  lat%ele(:)              <--->  lat%branch(0)%ele(:)
  lat%n_ele_track         <--->  lat%branch(0)%n_ele_track
  lat%n_ele_max           <--->  lat%branch(0)%n_ele_max
  lat%param               <--->  lat%branch(0)%param        
\end{example}

\begin{table}[tb]
\begin{center}
\begin{tabular}{lll}
\toprule
              & \multicolumn{2}{c} {\em Element index}   \\ \cmidrule{2-3}
{\em section} & {\em min}          & {\em max}            \\ \midrule
tracking      & 0                  & \vn{%n_ele_track}    \\ 
control       & \vn{%n_ele_track}+1 & \vn{%n_ele_max}     \\ \bottomrule
\end{tabular} 
\caption[Bounds of the root branch array.]
{Bounds of the tracking and control parts 
of the root branch (\vn{lat\%branch(0)\%ele(:)}) array.}
\label{t:part.extent}
\end{center}
\end{table}

All \vn{%branch%ele(:)} arrays are allocated with zero as the lower
bound. The \vn{%ele(0)} element of all branches is an \vn{beginning_ele}
element with its \vn{%name} component set to ``\vn{BEGINNING}''.
\vn{%ele(0)%mat6} is always the unit matrix. For the root branch, the
\vn{%branch(0)%ele(0:)} array is divided up into two parts: The
``tracking'' part and a ``control'' part (also called the ``lord''
part). The tracking part of this array holds the elements that are
tracked through. The control part holds elements that control
attributes of other elements (\sref{s:lat.control}). The bounds of
these two parts is given in Table~\ref{t:part.extent}.
Only the root branch has a lord section so 
\vn{%branch%n_ele_track} and \vn{%branch%n_ele_max} are the same for 
all other branches. Since the root branch can also be accessed via the \vn{lat%ele(:)}
array, code that deals with the lord section of the lattice may use \vn{lat%ele(:)}
in place of \vn{lat%branch(0)%ele(:)}.

for a given \vn{fork} or \vn{photon_fork} element, the index of
the branch that is being forked to and the index of the element that
is being forked to is stored in:
\begin{example}
  ix_branch  = nint(branch_ele%value(ix_branch_to$))  ! branch index
  ix_element = nint(branch_ele%value(ix_element_to$)) ! element index
  direction  = nint(branch_ele%value(direction$))
\end{example}
The direction will be +1 for forward forking and -1 for backward forking.

%----------------------------------------------------------------------------
\section{Param_struct Component}
\label{s:lat.param.struct}
\index{lat_param_struct}

The \vn{%param} component within each \vn{lat%branch(:)} is a
\vn{lat_param_struct} structure whose definition is shown in
\fig{f:lat.param.struct}
\begin{figure}[tb]
\centering
\begin{verbatim}
  type lat_param_struct
    real(rp) n_part             ! Particles/bunch.
    real(rp) total_length       ! total_length of lattice
    real(rp) unstable_factor    ! closed branch: growth rate/turn. 
                                !   all branches: |orbit/limit|
    real(rp) t1_with_RF(6,6)    ! Full 1-turn 6x6 matrix
    real(rp) t1_no_RF(6,6)      ! Transverse 1-turn 4x4 matrix (RF off).
    integer particle            ! +1 = positrons, -1 = electrons, etc.
    integer geometry            ! open$, etc...
    integer ixx                 ! Integer for general use
    logical stable              ! For closed branch. Is lat stable?
    type (bookkeeper_status_struct) bookkeeping_state
                                          ! Overall status for the branch.
  end type
\end{verbatim}
\caption{Definition of the \vn{param_struct}.}
\label{f:lat.param.struct}
\end{figure}
This structure would be more aptly named \vn{branch_param_struct} but
is named otherwise for historical reasons.

\index{lat_param_struct!\%total_length}
\vn{%param%total_length} is the length of the branch that a beam
tracks through defined by
\begin{example}
  %param%total_length = %ele(n_ele_track)%s - %ele(0)%s
\end{example}
Normally \vn{%ele(0)%s} = 0 so \vn{%param%total_length} =
\vn{%ele(n_ele_track)%s} but this is not always the case.

\index{lat_param_struct!\%n_part}
\index{beambeam}
\index{lcavity!and param\%n_part}
\vn{%param%n_part} is the number of particles in a bunch and is used in various calculations.
Historically, this parameter has been used to set the number of strong beam particle with
\vn{BeamBeam} elements but it is strongly recommended to use the \vn{beambeam} element's
\vn{n_particle} parameter instead.

\index{lat_param_struct!\%t1_with_RF}
\index{lat_param_struct!\%t1_no_RF}
\index{lat_param_struct!\%stable}
For closed branches, \vn{%param%t1_with_RF} and \vn{%param%t1_no_RF} are the 1--turn transfer
matrices from the start of the branch to the end. \vn{%param%t1_with_RF} is the full transfer matrix
with RF on. \vn{%param%t1_no_RF} is the transverse transfer matrix with RF off. \vn{%param%t1_no_RF}
is used to compute the Twiss parameters. When computing the Twiss parameters \vn{%param%stable} is
set according to whether the matrix is stable or not. If the matrix is not stable the Twiss
parameters cannot be computed. If unstable, \vn{%param%unstable_factor} will be set to the growth
rate per turn of the unstable mode.

Besides being set when the 1--turn transfer matrix is calculated, \vn{%param%unstable_factor} will
be set if a particle is lost in tracking to:
\begin{example}
  orbit_amplitude / limit - 1
\end{example}

The particle type for a branch is stored in the integer variable \vn{%param%particle}. The value of
this variable will encode for a fundamental particle, atom, or molecule. See the file
\vn{particle_species_mod.f90} for more details. If the particle corresponds to a fundamental
particle, \vn{%param%particle} will correspond to one of the following constants:
\index{positron\$}\index{electron\$}\index{proton\$}\index{antiproton\$}
\index{muon\$}\index{antimuon\$}\index{pion_0\$}\index{pion_minus\$}
\index{pion_plus\$}
\begin{example}
  electron\$,     positron\$,   
  muon\$,         antimuon\$,
  proton\$,       antiproton\$,
  photon\$,       pion_0\$,
  pion_minus\$,   pion_plus\$
  deuteron\$      deuteron_0\$
\end{example}
To print the name of the particle use the function
\vn{species_name}. A particles mass and charge can be obtained from
the functions \vn{mass_of} and \vn{charge_of} respectively. 
\vn{charge_of} returns the particle's charge in units of $e$. Example:
\begin{example}
  type (lat_struct) lat
  ...
  print *, "Beam Particles are: ", species_name(lat%param%particle)
  if (lat%param%particle == proton\$) print *, "I do not like protons!"
  print *, "Particle mass (eV):    ", mass_of(lat%param%particle)
  print *, "Particle charge:       ", charge_of(lat%param%particle)
\end{example}

%----------------------------------------------------------------------------
\section{Elements Controlling Other Elements}
\label{s:lat.control}

\index{lord|hyperbf}\index{slave|hyperbf}
In the \vn{lat_struct} structure, certain elements in the \vn{%ele(:)} array (equivalent to the
\vn{%branch(0)%ele(:)} array), called \vn{lord} elements, can control the attributes (component
values) of other \vn{%branch(:)%ele(:)} elements.  Elements so controlled are called \vn{slave}
elements.  The situation is complicated by the fact that a given element may simultaneously be a
\vn{lord} and a \vn{slave}. For example, an \vn{overlay} element (\sref{s:overlay}) is a lord since
it controls attributes of other elements but an \vn{overlay} can itself be controlled by other
\vn{overlay} and \vn{group} elements. In all cases, circular lord/slave chains are not permitted.

\index{ele_struct!\%lord_status}\index{ele_struct!\%slave_status}
The lord and slave elements can be divided up into classes. What type of lord an element is, is set
by the value of the element's \vn{ele%lord_status} component. Similarly, what type of slave an
element is is set by the value of the element's \vn{ele%slave_status} component.  Nomenclature note:
An element may be referred to by it's \vn{%lord_status} or \vn{%slave_status} value. For example, an
element with \vn{ele%lord_status} set to \vn{super_lord\$} can be referred to as a
``\vn{super_lord}'' element.

\index{super_lord\$}
\index{group_lord}\index{girder_lord\$}
\index{multipass_lord\$}
\index{girder_lord\$}\index{not_a_lord\$}
The value of the \vn{ele%lord_status} component can be one of:
  \begin{description}
  %
  \item[super_lord\$]\Newline 
A \vn{super_lord} element is created when elements are superimposed on top of other elements
(\sref{c:super}). \vn{super_lords} (along with \vn{multipass_lords}), are called \vn{major} lords
since the attribute values of a \vn{super_slave} are entirely determined by the attribute values of
the \vn{super_lord}(s) of the slave. 
  %
  \item[girder_lord\$]\Newline 
A \vn{girder_lord} element is a \vn{girder} element (\sref{s:girder}). That is, the element will
have \vn{ele%key} = \vn{girder\$}.
  %
  \item[multipass_lord\$]\Newline
\vn{multipass_lord} elements are created when multipass lines are present (\sref{c:multipass}).
\vn{multipass_lords} (along with \vn{super_lords}) are called \vn{major} lords since most of the
attribute values of a \vn{multipass_slave} are entirely determined by the attribute values of the
\vn{multipass_lord} of the slave. The few exceptions are parameters like \vn{phi0_multipass} which can
be set for individual slave elements.
  %
  \item[overlay_lord\$]\Newline 
An \vn{overlay_lord} is an \vn{overlay} element (\sref{s:overlay}). That is, such an element will
have \vn{ele%key} = \vn{overlay\$}.
  %
  \item[ramper_lord\$]\Newline 
A \vn{ramper_lord} is a \vn{ramper} element (\sref{s:ramper}). That is, such an element will have
\vn{ele%key} = \vn{ramper\$}. Note that a \vn{ramper_lord} will not have pointers to its slaves.
That is, \vn{ele%n_slave} will be zero. 
  %
  \item[group_lord\$]\Newline 
A \vn{group_lord} is a \vn{group} element (\sref{s:group}). That is, such an element will have
\vn{ele%key} = \vn{group\$}.
  %
  \item[not_a_lord\$]\Newline
This element does not control anything.
  \end{description}
Any element whose \vn{%lord_status} is something other than \vn{not_a_lord\$} is called a \vn{lord}
element. In the \vn{tracking part} of the branch (\sref{s:lat.struct}), \vn{%lord_status} will
always be \vn{not_a_lord\$}. In the \vn{lord section} of the branch, under normal circumstances,
there will never be any \vn{not_a_lord} elements.  However, it is permissible, and sometimes
convenient, for programs to set the \vn{%lord_status} of a lord element to \vn{not_a_lord\$}.

\index{super_slave\$}\index{multipass_slave\$}
\index{free\$}\index{minor_slave\$}\index{slice_slave\$}
The possible values for the \vn{ele%slave_status} component are:
  \begin{description}
  \item[multipass_slave\$]\Newline
A \vn{multipass_slave} element is the slave of a \vn{multipass_lord}
(\sref{c:multipass}).
  \item[slice_slave\$]\Newline
A \vn{slice_slave} element represents a longitudinal slice of another element.
Slice elements are not part of the lattice but rather are created on-the-fly
when, for example, a program needs to track part way through an element.
  \item[super_slave\$]\Newline 
A \vn{super_slave} element is an element in the tracking part of the branch that 
has one or more \vn{super_lord} lords (\sref{c:super}).
  \item[minor_slave\$]\Newline
A \vn{minor_slave} element is an element that is not a \vn{slice_slave} and does not have a
\vn{major} lord. Major lords are \vn{super_lord}s and \vn{multipass_lord}s. A \vn{minor_slave}
element will some have attributes that are controlled by \vn{overlay_lords}, \vn{group_lords},
\vn{girder_lords}, or \vn{ramper_lords}.
  \item[free\$]\Newline
A \vn{free} element is one that has no lords except for perhaps \vn{ramper_lords}. Additionally,
there still might be field overlap from other elements.
  \end{description}

\vn{super_slave} elements always appear in the tracking part of the branch. The other types can be
in either the tracking or control parts of the branch.

\begin{table}[tb]
  \footnotesize

  \centering
  \begin{subfigure}[b]{0.45\textwidth}
    \begin{tabular}{lcccccc} 
      \toprule
      & \multicolumn{6}{c}{\vn{ele\%lord_status}} \\ 
      \cmidrule(lr){2-7}
      {\vn{ele{\%}slave_status}} &
      \begin{sideways}\vn{not_a_lord\$}\end{sideways} &
      \begin{sideways}\vn{group_lord\$}\end{sideways} &
      \begin{sideways}\vn{girder_lord\$}\end{sideways} &
      \begin{sideways}\vn{overlay_lord\$}\end{sideways} &
      \begin{sideways}\vn{multipass_lord\$}\end{sideways} &
      \begin{sideways}\vn{super_lord\$}\end{sideways}
      \\ \midrule
      %                         NL  GR  GI  O   M   S
      \vn{free\$}             & X & X & X & X & X & X \\ 
      \vn{minor_slave\$}      & X & X & X & X & X & X \\ 
      \vn{multipass_slave\$}  & X &   &   &   &   & X \\ 
      \vn{slice_slave\$}      & X &   &   &   &   &   \\
      \vn{super_slave\$}      & X &   &   &   &   &   \\ \bottomrule
    \end{tabular}
    \caption{Possible {ele\%lord_status} and {ele\%slave_status} combinations within an individual element.}
    \label{f:lord.slave.a}
  \end{subfigure}
  \hfill
  \begin{subfigure}[b]{0.45\textwidth}
    \begin{tabular}{lcccccc} 
      \toprule
      & \multicolumn{6}{c}{\vn{lord\%lord_status}} \\ 
      \cmidrule(lr){2-7}
      {\vn{slave{\%}slave_status}} &
      \begin{sideways}\vn{not_a_lord\$}\end{sideways} &
      \begin{sideways}\vn{group_lord\$}\end{sideways} &
      \begin{sideways}\vn{girder_lord\$}\end{sideways} &
      \begin{sideways}\vn{overlay_lord\$}\end{sideways} &
      \begin{sideways}\vn{multipass_lord\$}\end{sideways} &
      \begin{sideways}\vn{super_lord\$}\end{sideways}
      \\ \midrule
      %                         NL  GR  GI  O   M   S
      \vn{free\$}             &   &   &   &   &   &   \\ 
      \vn{minor_slave\$}      &   & X & X & X &   &   \\ 
      \vn{multipass_slave\$}  &   & X &   & X & 1 &   \\ 
      \vn{super_slave\$}      &   & X &   & X &   & X \\ \bottomrule
    \end{tabular}
    \caption{Possible {\%lord_status} and {\%slave_status} combinations for any lord/slave pair.}
    \label{f:lord.slave.b}
  \end{subfigure}
  \caption[Possible element \%lord_status/\%slave_status combinations.]{Possible 
    \%lord_status/\%slave_status combinations. 
    ``X'' marks a possible combination.
    ``1'' indicates that the slave will have exactly one lord of the type given in the column.}
  \label{f:lord.slave}
\end{table}

\index{ele_struct!\%lord_status}\index{ele_struct!\%slave_status}
Only some combinations of \vn{%lord_status} values and \vn{%slave_status} values are permissible for
a given element.  Table~\ref{f:lord.slave.a} lists the valid combinations. Thus, for example, it is
{\em not} possible for an element to be simultaneously a \vn{super_lord} and a \vn{super_slave}.

\begin{figure}[tb]
\centering \includegraphics[width=5.5in]{superimpose-and-multipass.pdf} \caption[Example of
multipass combined with superposition] {Example of multipass combined with superposition. A
\vn{multipass_lord} element named \vn{A} controls a set of \vn{multipass_slaves} (only one shown).
The \vn{multipass_slave} elements are also \vn{super_lord} elements and they will control
\vn{super_slave} elements in the tracking part of the branch.}  \label{f:super.mul}
\end{figure}

For lord/slave pairs, Table~\ref{f:lord.slave.b} lists the valid combinations of \vn{%lord_status}
values in the lord element and \vn{%slave_status} values in the slave element. Thus, for example, a
\vn{super_slave} may only be controlled by a \vn{super_lord}. In the example in
Section~\sref{c:multipass}, element \vn{A} would be a \vn{multipass_lord} and \vn{A\B1} and
\vn{A\B2} would be \vn{multipass_slave}s. When superposition is combined with multipass, the
elements in the tracking part of the branch will be \vn{super_slave}s.  These elements will be
controlled by \vn{super_lord}s which will also be \vn{multipass_slave}s and these
\vn{super_lord}/\vn{multipass_slave} elements will be controlled by \vn{multipass_lord}s. This is
illustrated in \fig{f:super.mul}.

\index{ele_struct!\%n_lord}\index{ele_struct!\%n_slave}
The number of slave elements that a lord controls is given by the value of the lord's \vn{%n_slave}
component. Additionally, the number of lord elements that the slave has is given by the value of the
slave's.  \vn{%n_lord} component. To find the slaves and lords of a given element, use the routines
\Hyperref{r:pointer.to.slave}{pointer_to_slave} and
\Hyperref{r:pointer.to.lord}{pointer_to_lord}. Example:
\begin{example}
  type (lat_struct), target :: lat
  type (ele_struct), pointer :: this_ele, lord_ele, slave_ele
  ...
  this_ele => lat%ele(321)    ! this_ele points to a given element in the lattice

  do i = 1, this_ele%n_lord   ! Loop over all lords of this_ele
    ! lord_ele points to the i^th lord element of this_ele
    lord_ele => pointer_to_lord (this_ele, i)  
    ...
  enddo

  do i = 1, this_ele%n_slave  ! Loop over all slaves of this_ele
    ! slave_ele points to the i^th slave element of this_ele
    slave_ele => pointer_to_slave (this_ele, i) 
    ...
  enddo
\end{example}
For non-ramper elements, the lord/slave bookkeeping is bidirectional. That is, for any given
element, call it \vn{this_ele}, consider the i\Th lord:
\begin{example}
  lord_ele_i => pointer_to_lord (this_ele, i)
\end{example}
then there will always be some index j such that the 
element pointed to by
\begin{example}
  pointer_to_slave(lord_ele_i, j)
\end{example}
is the original element \vn{this_ele}. The same is true for the slaves of any given element. That
is, for the i\Th slave
\begin{example}
  slave_ele_i => pointer_to_slave (this_ele, i)
\end{example}
there will always be some index j such that the 
element pointed to by
\begin{example}
  pointer_to_lord(slave_ele_i, j)
\end{example}

\index{slave!ordering}\index{lords!ordering}
\index{super_lord}\index{super_slave}
\index{multipass_lord}\index{multipass_slave}
\index{patch}
The following ordering of slaves and lords is observed:
  \begin{description}
  \item[Slaves of a super_lord:] \Newline
The associated \vn{super_slave} elements of a given \vn{super_lord} element are ordered from the
entrance end of the \vn{super_lord} to the exit end. That is, in the code snippet above,
\vn{pointer_to_slave (this_ele, 1)} will point to the slave at the start of the \vn{super_lord} and
\vn{ pointer_to_slave (this_ele, this_ele%n_lord)} will point to the slave at the exit end of the
\vn{super_lord}.
  \item[Slaves of a multipass_lord:] \Newline
The associated \vn{multipass_slave} elements of a \vn{multipass_lord} element are ordered by pass
number. That is, in the code snippet above, \vn{pointer_to_slave (this_ele, i)} will point to the
slave of the $i$\Th pass.
  \item[Lord of a multipass_slave:] \Newline
A \vn{multipass_slave} will have exactly one associated \vn{multipass_lord} and this lord will be
the first one. That is, \vn{pointer_to_lord (this_ele, 1)}.
  \end{description}

\index{control_struct}
The element control information is stored in the \vn{lat%control(:)} array.  Each element of this
array is a \vn{control_struct} structure
\begin{example}
  type control_struct
    type (expression_atom_struct), allocatable :: stack(:) ! Evaluation stack
    type (lat_ele_loc_struct) slave ! Slave location
    type (lat_ele_loc_struct) lord  ! Lord location
    integer ix_attrib               ! index of controlled attribute 
  end type
\end{example}
\index{girder}\index{overlay}\index{group}
\index{lat_struct!\%control}
Each element in the \vn{lat%control(:)} array holds the information on one lord/slave pair. The
\vn{%lord} component gives the location of the lord element which is always in the root branch ---
branch 0. The \vn{%slave} component give the element location of the slave element.  The \vn{%stack}
and \vn{%ix_attrib} components are used to store the arithmetic expression and attribute index for
\vn{overlay} and \vn{group} control. The appropriate control_struct for a given lord/slave pair can
be obtained from the optional fourth argument of the \Hyperref{r:pointer.to.lord}{pointer_to_lord}
and \Hyperref{r:pointer.to.slave}{pointer_to_slave} functions.  Example: The following prints a list
of the slaves, along with the attributes controlled and coefficients, on all group elements in a
lattice.
\begin{example}
  type (lat_struct), target :: lat
  type (ele_struct), pointer :: lord, slave
  type (control_struct), pointer :: con
  ...
  do i = lat%n_ele_track+1, lat%n_ele_max  ! loop over all lords
    lord => lat%ele(i) 
    if (lord%lord_status = group_lord$) then 
      print *, "Slaves for group lord: ", lord%name
      do j = 1, lord%n_slave
        slave => pointer_to_slave (lord, j, con)
        attrib_name = attribute_name (slave, con%ix_attrib)
        print *, i, slave%name, attrib_name, con%coef
      enddo
    endif
  enddo
\end{example}

\index{lat_struct!\%control}\index{lat_struct!\%n_slave_field}
\index{lat_struct!\%ix1_slave}\index{lat_struct!\%n_slave}
The elements in the \vn{lat%control(:)} array associated with the slaves of a given lord are in the
same order as the slaves and the index of the associated \vn{lat%control(:)} element of the first
slave is given by the \vn{%ix1_slave} component of the lord Example:
\begin{example}
  type (lat_struct), target :: lat
  type (ele_struct), pointer :: lord, slave
  type (control_struct), pointer :: con1, con2
  ...
  lord => lat%ele(i)                    ! Point to some lord
  do j = 1, lord%n_slave
    slave => pointer_to_slave (lord, j, con1)
    con2 => lat%control(lord%ix1_slave+j-1) ! con1 and con2 are the same.
  enddo
\end{example}

\index{ele_struct!\%ic1_lord}\index{ele_struct!\%n_lord}
\index{lat_struct!\%ic}\index{ele_struct!\%n_lord_field}
Except for a \vn{slice_slave}, the \vn{%ic1_lord}, \vn{%n_lord}, and \vn{%n_lord_field} components
of a given slave element, along with the \vn{lat%ic(:)} array, can be used to find the lords of the
slave.  Simplified, the code for the \Hyperref{r:pointer.to.lord}{pointer_to_lord} function is:
\begin{example}
  function pointer_to_lord (slave, ix_lord, con, ...) result (lord_ptr)
    implicit none
    type (lat_struct), target :: lat
    type (ele_struct) slave
    type (ele_struct), pointer :: lord_ptr
    type (control_struct), pointer, optional :: control
    integer ix_lord, icon
    !
    icon = lat%ic(slave%ic1_lord + ix_lord - 1)
    lord_ptr => lat%ele(lat%control(icon)%lord%ix_ele)
    if (present(con)) con => lat%control(icon)
  end function
\end{example}
This method for finding the lords of an element is considered
``private''. That is, no code outside of the official \bmad library
should rely on this.

\vn{slice_slave} element bookkeeping has is different depending upon whether the element being
sliced is a \vn{super_slave} or not. If the element being sliced is a \vn{super_slave}, a
\vn{slice_slave} element that is created is, for bookkeeping purposes, considered to be a slave of
the \vn{super_slave}'s lords. In this case, the bookkeeping is exactly the same as that of any
\vn{super_slave}, and \vn{pointer_to_lord} will return a pointer to one of the \vn{super_slave}'s
lords.

On the other hand, if a non \vn{super_slave} element is being sliced, the \vn{%lord} pointer
component of the \vn{slice_slave} will be set to point to the element being sliced.

%----------------------------------------------------------------------------
\section{Lattice Bookkeeping}
\label{s:lat.book}

\index{reference energy}
The term ``lattice bookkeeping'' refers to the updating of the appropriate parameter values when a
given parameter in the lattice is changed. For example, if the accelerating gradient of an
\vn{lcavity} element is modified, the reference energy parameter of all elements downstream of the
\vn{lcavity} will need to be changed and this can also alter the transfer maps of the \vn{lcavity}
and downstream elements. \bmad divides the lattice bookkeeping into a ``core'' part and everything
else. The core part itself is divided into five parts:
  \begin{description}
  \item[Attribute bookkeeping] \Newline
This refers to intra-element dependent attribute bookkeeping (\sref{s:depend}).
  \item[Control bookkeeping] \Newline
This refers to Lord/Slave bookkeeping for \vn{overlay}
(\sref{s:overlay}) and \vn{group} (\sref{s:group})elements, and for
\vn{superposition} (\sref{c:super}) and \vn{multipass}
(\sref{c:multipass}) lords.
  \item[Floor Position bookkeeping]
This refers to bookkeeping to keep track of an elements global ``floor'' position
stored in the \vn{ele%floor} structure.
  \item[Length bookkeeping]
This refers to bookkeeping to keep track of the longitudinal s-position of an element
stored in the \vn{ele%s} component.
  \item[Reference Energy bookkeeping]
This refers to the reference energy assigned to each element (\sref{s:ref.energy.prog}).
\vn{ele%value(E_tot\$)} and \vn{ele%value(p0c\$)}
  \end{description}

Lattice elements have a \vn{bookkeeper_status} component which is of type \vn{bookkeeper_status_struct}
which looks like
\begin{example}
  type bookkeeper_status_struct
    integer attributes      ! Intra element dependent attribute status
    integer control         ! Lord/slave bookkeeping status
    integer floor_position  ! Global (floor) geometry status
    integer length          ! Longitudinal position status
    integer ref_energy      ! Reference energy status
    integer mat6            ! Linear transfer map status
    integer rad_int         ! Radiation integrals cache status
  end type
\end{example}
All components of this structure give the status of some lattice bookkeeping aspect. The first five
components of this structure correspond to the five core bookkeeping parts discussed above. The
other two components are discussed below.

Possible values for the status components are
\begin{example}
  super_ok\$
  ok\$
  stale\$
\end{example}
The \vn{set_flags_for_changed_attribute} routine sets the appropriate status components of an
element to \vn{stale\$} which marks that element for the appropriate bookkeeping. When the
bookkeeping is done by \vn{lattice_bookkeeper}, the \vn{stale\$} status components are set to
\vn{ok\$}. The \vn{super_ok\$} value is reserved for use by any program that needs to do its own
custom bookkeeping. How this works is as follows: The \bmad bookkeeping routines will never convert
a status component with value \vn{super_ok\$} to \vn{ok\$} without first doing some needed
bookkeeping. Thus if a program sets a status component to \vn{super_ok\$} and then later on finds
that the status component is set to \vn{ok\$}, the program knows that bookkeeping has been done. An
example will make this clear. Suppose a program needs to keep track of a collection of high order
transfer maps between various points in a lattice. Suppose that the constant calculation of these
maps would slow the program done so it is desired to recalculate a given map only when necessary. To
implement this, the program could set the \vn{ele%status%mat6} attribute of all the element to
\vn{super_ok\$} when the maps are calculated. If the program subsequently finds a
\vn{ele%status%mat6} attribute of an element set to \vn{ok\$} it knows that it should recalculate
any transfer maps that span that element.

It is guaranteed that when \vn{lattice_bookkeeper} is run, all five core status components will not
be \vn{stale\$}.  The routines used by \vn{lattice_bookkeeper} are:
\begin{example}
  \Hyperref{r:attribute.bookkeeper}{attribute_bookkeeper}      ! Intra-element attributes
  \Hyperref{r:control.bookkeeper}{control_bookkeeper}        ! Lord/slave control
  \Hyperref{r:s.calc}{s_calc}                    ! Longitudinal element s-position
  \Hyperref{r:lat.geometry}{lat_geometry}              ! Global (floor) positions.
  \Hyperref{r:lat.compute.ref.energy.and.time}{lat_compute_ref_energy_and_time}  ! Reference energy 
\end{example}
In general, these routines should not be called directly since the correct way to do things is not
always straight forward. See the code for \vn{lattice_bookkeeper} for more details.

After the core bookkeeping is done, a program can call \Hyperref{r:lat.make.mat6}{lat_make_mat6} to
remake the transfer matrices.  \vn{lat_make_mat6} will remake the transfer matrices if either the
\vn{ele%status%mat6} flag is \vn{stale\$} or the reference orbit around which the existing transfer
matrix was computed has shifted.  \vn{lat_make_mat6} will set the \vn{ele%status%mat6} flag to
\vn{ok\$} for all elements whose transfer matrices are recomputed.

%----------------------------------------------------------------------------
\section{Intelligent Bookkeeping}
\label{s:intelligent.book}
\index{bookkeeping!automatic}
\index{bookkeeping!intelligent}
Historically, as the code for lattice bookkeeping (\sref{s:lat.book}) was being developed
calls to bookkeeping routines were added to calculational
routines such as the tracking routine \Hyperref{r:track1}{track1} and the routine for calculating
the linear transfer map \Hyperref{r:make.mat6}{make_mat6}.  This ``automatic'' bookkeeping system is
inefficient since there is no good way to keep track of what element attributes have been modified
which leads to redundant bookkeeping calculations. Eventually, as \bmad developed and became more
complicated, it was found that the unnecessary bookkeeping load was generally causing a significant
slowdown in program execution time --- even in programs where no element attributes were changed. To
avoid this, an ``intelligent'' bookkeeping system was developed which could be switched on by setting
the parameter:
\begin{example}
  bmad_com%auto_bookkeeper = .false.
\end{example}
To keep things back compatible with existing programs, the automatic bookkeeping system was set as
the default. However, given the fact that the automatic bookkeeping system has known deficiencies,
and given the overhead with maintaining two bookkeeping systems, the automatic bookkeeping system
has been retired and old programs needed to be upgraded if needed. Rule: A program that does not
``directly modify'' element attributes does not have to modified. Modification of element attributes
via \bmad routines (for example, using the \vn{set_on_off} routine) is ``indirect''. A direct modification
is something like the following appearing in the program:
\begin{example}
  lat%ele(ie)%value(hkick$) = ...
\end{example}

\index{bmad_common_struct!auto_bookkeeper}
To use intelligent bookkeeping, a program must set the global \vn{bmad_com%auto_bookkeepper} to
false. This only needs to be done once at the start of the program before \vn{bmad_parser} is called. 
If lattice parameters are not modified in the program, this is the only thing that needs to
be done.

When a set of attributes needs to be modified, the
\Hyperref{r:set.flags.for.changed.attribute}{set_flags_for_changed_attribute} routine must be called
for each element attribute that is set. After all the attributes have been set,
\Hyperref{r:lattice.bookkeeper}{lattice_bookkeeper} is called to do the core bookkeeping. Example
\begin{example}
  type (lat_struct) lat
  ...
  bmad_com%auto_bookkeeper = .false.    ! Done once. Put this before the call to bmad_parser.
  ...
  lat%ele(i)%value(gradient$) = 1.05e6  ! Change, say, the gradient of an RFCavity
  call set_flags_for_changed_attribute (lat%ele(i), lat%ele(i)%value(gradient$))

  ... Set attributes of other elements ...

  call lattice_bookkeeper (lat)         ! Do once after all attribute sets done.
\end{example}
The argument list for \vn{set_flags_for_changed_attribute} is 
\begin{example}
  set_flags_for_changed_attribute (ele, attribute)
\end{example}
The \vn{attribute} argument may be either real, integer, or logical.

\index{ele!\%status}
\index{bookkeeper_status_struct}
The \vn{set_flags_for_changed_attribute} routine sets flags in the \vn{ele%status} structure
\sref{s:lat.book}.

%----------------------------------------------------------------------------
\section{particle_start Component}
\label{s:lat.beam.start}

\index{particle_start}
\index{lat_struct!\%particle_start}
The \vn{lat%particle_start} component is a \vn{coord_struct} structure for holding the information
obtained from \vn{particle_start} statements (\sref{s:beam.start}) in a \bmad lattice file.

This component is not used in any standard \bmad calculation. It is up to an individual program to
use as desired. Use \Hyperref{r:init.coord}{init_coord} to initalize a particle position. Example:
\begin{example}
  type (lat_struct) lat
  type (coord_struct) orbit_start
  ...
  call bmad_parser("lat.bmad", lat)  ! Read in a lattice.
  ...
  ! orbit_start is initalized for tracking from the beginning of the lattice
  call init_coord(orbit_start, lat%particle_start, lat%ele(0), downstream_end$)
\end{example}

%----------------------------------------------------------------------------
\section{Custom Parameters}
\label{s:lat.custom}

Custom parameters defined for the lattice as a whole (\sref{s:cust.att} are stored in
\vn{lat%custom}. The following shows how to print a table of the custom parameters
\begin{example}
  type (lat_struct) lat
  character(80) aname
  ...
  if (allocated(lat%custom)) then
    do i = 1, size(lat%custom)
      aname = attribute_name(def_parameter$, i+custom_attribute0$)
      if (aname(1:1) == "!") cycle    ! Ignore non-existant parameters
      print "(a, es12.4)", "  parameter[" // trim(aname) // "] = ", lat%custom(i)
    enddo
  endif
\end{example}

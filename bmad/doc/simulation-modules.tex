\chapter{Simulation Modules}

In the \bmad ``ecosystem'', various modules have been developed to
simulate machine hardware. This chapter provides documentation.

%-----------------------------------------------------------------
\section{Instrumental Measurements}
\label{s:meas.calc}
\index{measurement}

\bmad has the ability to simulate instrumental measurement errors
for orbit, dispersion, betatron phase, and coupling measurements.
The appropriate attributes are listed in \sref{s:meas.attrib} and
the conversion formulas are outlined below.

%-----------------------------------------------------------------
\subsection{Orbit Measurement}
\index{orbit!measurement}

For orbits, the relationship between measured position $(x, y)_{\text{meas}}$ and true position $(x,
y)_{true}$ is
\begin{equation}
  \begin{pmatrix}
    x \\ y
  \end{pmatrix}_{\! \text{meas}}
  =
  n_f \, 
  \begin{pmatrix}
    r_1 \\ r_2
  \end{pmatrix}
  +
  \bfM_m \, 
  \left[
  \begin{pmatrix}
    x \\ y
  \end{pmatrix}_{\! true}
  -
  \begin{pmatrix}
    x \\ y
  \end{pmatrix}_{\! 0}
  \right]
  \label{xynrr}
\end{equation}
where the Gaussian random numbers $r_1$ and $r_2$ are centered at zero and have unit width and the
factor $n_f$ represents the inherent noise in the measurement. In the above equation, $(x, y)_0$ 
represents a measurement offset and $\bfM_g$ is a ``gain'' matrix written in the form
\begin{equation}
  \bfM_m
  =
  \begin{pmatrix}
     (1 + dg_x) \, \cos (\theta + \psi) & (1 + dg_x) \, \sin (\theta + \psi) \\
    -(1 + dg_y) \, \sin (\theta - \psi) & (1 + dg_y) \, \cos (\theta - \psi) 
  \end{pmatrix}
  \label{m1dg}
\end{equation}
Here $dg_x$ and $dg_y$ represent gain errors and the angles $\theta$ and $\psi$ are tilt and 
``crunch'' errors.

In the above equations, various quantities are written as a difference between an ``error'' quantity
and a ``calibration'' quantity:
\begin{alignat}{1}
  x_0     &= x_{\text{off}} - x_{\text{calib}} \CRNO
  y_0     &= y_{\text{off}} - y_{\text{calib}} \CRNO
  \psi    &= \psi_{\text{err}}   - \psi_{\text{calib}} \CRNO
  \theta  &= \theta_{\text{err}} - \theta_{\text{calib}} \\
  dg_x    &= dg_{x,\text{err}} - dg_{x,\text{calib}} \CRNO
  dg_y    &= dg_{y,\text{err}} - dg_{y,\text{calib}} \nonumber
\end{alignat}
See \sref{s:meas.attrib} for the element attribute names that correspond to these quantities.

The calibration component is useful in a simulation where initally the error quantities are set to
represent the errors in the monitors. After this, analysis of orbit data with the machine in various
states can be used to calculate a best guess as to what the errors are. The calculated error values
can then be put in the calibration quantities. This represents a correction in software of the
errors in the monitors. Further simulations of orbit measurements will show how well the actual orbit
can be deduced from the measured orbit.

%-----------------------------------------------------------------
\subsection{Dispersion Measurement}
\label{Dispersion!measurement}

A dispersion measurement is considered to be the result of measuring the orbit at two different
energies. The measured values are then
\begin{equation}
  \begin{pmatrix}
    \eta_x \\ \eta_y
  \end{pmatrix}_{\! \text{meas}}
  =
  \frac{\sqrt{2} \, n_f}{dE/E} \, 
  \begin{pmatrix}
    r_1 \\ r_2
  \end{pmatrix}
  +
  \bfM_m \, \left[
  \begin{pmatrix}
    \eta_x \\ \eta_y
  \end{pmatrix}_{\! true}
  -
  \left(
  \begin{pmatrix}
    \eta_x \\ \eta_y
  \end{pmatrix}_{\! err}
  -
  \begin{pmatrix}
    \eta_x \\ \eta_y
  \end{pmatrix}_{\! calib}
  \right)
  \right]
\end{equation}
The factor of $\sqrt{2}$ comes from the fact that there are two measurements. $\bfM_m$ is given in \Eq{m1dg}.

%-----------------------------------------------------------------
\subsection{Coupling Measurement}
\label{Coupling!measurement}

The coupling measurement is considered to be the result of measuring
the beam at a detector over $N_s$ turns while the beam oscillates at a
normal mode frequency with some amplitude $A_{\text{osc}}$.  The
measured coupling is computed as follows. First, consider excitation
of the $a$-mode which can be written in the form:
\begin{equation}
  \begin{pmatrix}
    x_i \\
    y_i
  \end{pmatrix}_{\! \text{true}}
  =
  A_{\text{osc}} \,
  \begin{pmatrix}
    \cos \phi_i \\
    K_{22a} \, \cos \phi_i + K_{12a} \sin \phi_i
  \end{pmatrix}_{\! \text{true}}
  \label{xyapk}
\end{equation}
$i$ is the turn number and $\phi_i$ is the oscillation phase on the $i$\Th turn.
The coefficients $K_{22a}$ and $K_{12a}$ are related to the coupling $\bfCbar$ via
Sagan and Rubin\cite{b:coupling} Eq.~54:
\begin{alignat}{1}
  K_{22a} &= \frac{-\sqrt{\beta_b}}{\gamma \, \sqrt{\beta_a}} \, \bfCbar_{22} \CRNO
  K_{12a} &= \frac{-\sqrt{\beta_b}}{\gamma \, \sqrt{\beta_a}} \, \bfCbar_{12}
  \label{kabgbc}
\end{alignat}
To apply the measurement errors, consider the general case where the
beam's oscillations are split into two components: One component being
in-phase with some reference oscillator (which is oscillating with the
same frequency as the beam) and a component oscillating out-of-phase:
\begin{equation}
  \begin{pmatrix}
    x_i \\
    y_i
  \end{pmatrix}_{\! \text{true}}
  =
  \begin{pmatrix}
    q_{a1x} \\
    q_{a1y}
  \end{pmatrix}_{\! \text{true}}
  \, A_{\text{osc}} \, \cos (\phi_i + d\phi) +
  \begin{pmatrix}
    q_{a2x} \\
    q_{a2y}
  \end{pmatrix}_{\! \text{true}}
  \, A_{\text{osc}} \, \sin (\phi_i + d\phi)
  \label{xykkap}
\end{equation}
where $d\phi$ is the phase of the reference oscillator with respect to
the beam.  Comparing \Eq{xyapk} with \Eq{xykkap} gives the relation
\begin{alignat}{1}
  K_{22a} &= \frac{q_{a1x} \, q_{a1y} + q_{a2x} \, q_{a2y}}{q_{a1x}^2 + q_{a2x}^2} \CRNO
  K_{12a} &= \frac{q_{a1x} \, q_{a2y} - q_{a2x} \, q_{a1y}}{q_{a1x}^2 + q_{a2x}^2} 
  \label{kaqqqq}
\end{alignat}
This equation is general and can be applied in either the true or
measurement frame of reference.  \Eq{xynrr} can be used to transform
$(x_i, y_i)_{\text{true}}$ in \Eq{xyapk} to the measurement frame of
reference. Only the oscillating part is of interest.  Averaging over
many turns gives
\begin{equation}
  \begin{pmatrix}
    q_{a1x} \\
    q_{a1y}
  \end{pmatrix}_{\! \text{meas}}
  =  
  \bfM_m \, 
  \begin{pmatrix}
    q_{a1x} \\
    q_{a1y}
  \end{pmatrix}_{\! \text{true}}
  \comma \qquad
  \begin{pmatrix}
    q_{a2x} \\
    q_{a2y}
  \end{pmatrix}_{\! \text{meas}}
  =  
  \bfM_m \, 
  \begin{pmatrix}
    q_{a2x} \\
    q_{a2y}
  \end{pmatrix}_{\! \text{true}}
  \label{kkmkk}
\end{equation}
This neglects the measurement noise. A calculation shows that the noise gives a 
contribution to the measured $K_{22a}$ and $K_{12a}$ of
\begin{equation}
  K_{22a} \rightarrow K_{22a} + r_1 \, \frac{n_f}{N_s \, A_{\text{osc}}} 
  \comma \qquad
  K_{12a} \rightarrow K_{12a} + r_2 \, \frac{n_f}{N_s \, A_{\text{osc}}} 
  \label{kkrnn}
\end{equation}
Using the above equations, the transformation from the true
coupling to measured coupling is as follows: From a knowledge of the
true $\bfCbar$ and Twiss values, the true $K_{22a}$ and
$K_{12a}$ can be calculated via \Eq{kabgbc}. Since the value of $d\phi$
does not affect the final answer, $d\phi$ in \Eq{xykkap} is chosen to
be zero.  Comparing this to \Eq{xyapk} gives
\begin{equation}
  \begin{pmatrix}
    q_{a1x} \\
    q_{a1y}
  \end{pmatrix}_{\text{true}}
  =
  \begin{pmatrix}
    1 \\
    K_{22a}
  \end{pmatrix}_{\text{true}}
  \comma \qquad
  \begin{pmatrix}
    q_{a2x} \\
    q_{a2y}
  \end{pmatrix}_{\text{true}}
  =
  \begin{pmatrix}
    0 \\
    K_{12a}
  \end{pmatrix}_{\text{true}}
\end{equation}
Now \Eq{kkmkk} is used to convert to the measured $q$'s and
\Eq{kaqqqq} then gives the measured $K_{22a}$ and $K_{12a}$. Finally,
Applying \Eq{kkrnn} and then \Eq{kabgbc} gives the measured
$\bfCbar_{22}$ and $\bfCbar_{12}$. 

A similar procedure can be applied to $b$-mode oscillations to
calculate values for the measured $\bfCbar_{11}$ and $\bfCbar_{12}$.
$K_{11b}$ and $K_{12b}$ are defined by
\begin{equation}
  \begin{pmatrix}
    x_i \\
    y_i
  \end{pmatrix}_{\! \text{true}}
  =
  A_{\text{osc}} \,
  \begin{pmatrix}
    K_{11b} \, \cos \phi_i + K_{12b} \sin \phi_i \\
    \cos \phi_i
  \end{pmatrix}_{\! \text{true}}
  \label{xyakp}
\end{equation}
Comparing this to Sagan and Rubin\cite{b:coupling} Eq.~55 gives
\begin{alignat}{1}
  K_{11b} &= \frac{ \sqrt{\beta_a}}{\gamma \, \sqrt{\beta_b}} \, \bfCbar_{11} \CRNO
  K_{12b} &= \frac{-\sqrt{\beta_a}}{\gamma \, \sqrt{\beta_b}} \, \bfCbar_{12}
  \label{kbbgbc}
\end{alignat}
The $q_{x1b}$, $q_{y1b}$, $q_{x2b}$ and $q_{y2b}$ are defined by using
\Eq{xykkap} with the ``a'' subscript replaced by ``b''. The
relationship between $K$ and $q$ is then
\begin{alignat}{1}
  K_{11b} &= \frac{q_{b1y} \, q_{b1x} + q_{b2y} \, q_{b2x}}{q_{b1y}^2 + q_{b2y}^2} \CRNO
  K_{12b} &= \frac{q_{b1y} \, q_{b2x} - q_{b2y} \, q_{b1x}}{q_{b1y}^2 + q_{b2y}^2} 
  \label{kbqqqq}
\end{alignat}


%-----------------------------------------------------------------
\subsection{Phase Measurement}
\label{Phase!measurement}

Like the coupling measurement, the betatron phase measurement is
considered to be the result of measuring the beam at a detector over
$N_s$ turns while the beam oscillates at a normal mode frequency with
some amplitude $A_{\text{osc}}$.  Following the analysis of the
previous subsection, the phase $\phi$ is
\begin{equation}
  \begin{pmatrix}
    \phi_a \\
    \phi_b
  \end{pmatrix}_{\! \text{meas}}
  =
  \begin{pmatrix}
    \phi_a \\
    \phi_b
  \end{pmatrix}_{\! true}
  +
  \frac{n_f}{N_s \, A_{\text{osc}}} \, 
  \begin{pmatrix}
    r_1 \\ 
    r_2
  \end{pmatrix}
  -
  \begin{pmatrix}
    \tan^{-1} \left( \frac{q_{a2x}}{q_{a1x}} \right) \\
    \tan^{-1} \left( \frac{q_{b2y}}{q_{b1y}} \right)
  \end{pmatrix}_{\! \text{meas}}
\end{equation}

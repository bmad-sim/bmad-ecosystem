\documentclass[12pt]{article}

\usepackage[margin=1in]{geometry}
\usepackage{amsmath,amsthm,amssymb,graphicx,gensymb,multicol,mathtools}
\usepackage{hyperref}
\hypersetup{
  colorlinks=true,
  linkcolor=blue,
  filecolor=red,
  urlcolor=cyan,
}
\usepackage{color}
\usepackage[english]{babel}
\usepackage[autostyle, english = american]{csquotes}
\MakeOuterQuote{"}
\graphicspath{{figures/}}
\usepackage{tikz, pgfplots}
\pgfplotsset{compat=newest}
\allowdisplaybreaks

\usepackage{biblatex}
\addbibresource{bibfile.bib}

\newcommand{\pp}{\mathbf{p}}
\newcommand{\uu}{\mathbf{u}}
\newcommand{\vv}{\mathbf{v}}
\newcommand{\ww}{\mathbf{w}}
\newcommand{\xx}{\mathbf{\hat{x}}}
\newcommand{\yy}{\mathbf{\hat{y}}}
\newcommand{\zz}{\mathbf{\hat{z}}}
\newcommand{\rr}{\mathbf{\hat{r}}}
\newcommand{\rrr}{\mathbf{r}}
\newcommand{\ppp}{\mathbf{p}}
\newcommand{\xxx}{\mathbf{x}}
\newcommand{\nnot}{\sim \!}
\let\oldemptyset\emptyset
\let\emptyset\varnothing

%for QM:
\newcommand{\intii}{\int_{-\infty}^\infty}
\newcommand{\intoi}{\int_0^\infty}
\newcommand{\HH}{\mathbb{H}}
\newcommand{\ang}[3]{\,^{#1} {#2}_{#3}}
\usepackage{braket}
\newcommand{\tr}{\mathrm{Tr}}

%units:
\newcommand{\s}{\, \mathrm{s}}
\newcommand{\m}{\, \mathrm{m}}
\newcommand{\eV}{\, \mathrm{eV}}
\newcommand{\MeV}{\, \mathrm{MeV}}
\newcommand{\ly}{\, \mathrm{ly}}

\newenvironment{problem}[2][Problem]{\begin{trivlist}
\item[\hskip \labelsep {\bfseries #1}\hskip \labelsep {\bfseries #2.}]}{\end{trivlist}}

\begin{document}

\title{Bmad Converter Model}
\author{John Mastroberti}

\maketitle

\newcommand{\dxds}{\frac{dx}{ds}}
\newcommand{\dyds}{\frac{dy}{ds}}
\newcommand{\dxdsmin}{\dxds_{\mathrm{min}}}
\newcommand{\dxdsmax}{\dxds_{\mathrm{max}}}
\newcommand{\dydsmax}{\left| \dyds \right|_{\mathrm{max}}}
\newcommand{\exes}{\texttt{converter\_simulation}}
\newcommand{\exef}{\texttt{converter\_fitter}}
\newcommand{\configfile}{\texttt{config.txt}}
\newcommand{\bmad}{\textit{Bmad}}

\tableofcontents

\newpage

\section{Introduction}

This monograph discusses the converter model used in converter lattice elements in \bmad. In a
converter, incoming particles generate particles of a different type. For example, a converter may
be used to simulate the production of positrons produced when a tungsten target plate is bombarded
by electrons.

The converter model currently in \bmad discussed here replaces an older model that was developed by
Daniel Fromowitz\cite{b:fromowitz}. This older model used equations to model the output
distribution.  The problems with this approach were the approximations that were used in developing
the equations coupled with uncertainty as to how to calculate the various coefficients that were
needed if parameters like the target material or the species of particles simulated were varied.

The present model replaces the equations of the older model with probability distribution tables for
the energy and radial distribution of the outgoing particle along with a generalized
parameterization of the probability distribution of the outgoing particles's direction of
propagation.  Probability distribution table values and coefficients needed to characterize the
velocity distribution are obtained through a Geant\cite{geant} simulation. These tables and
coefficients can then be stored in a \bmad lattice file and used for efficient generation of
outgoing particles. The present model is not only more accurate but can also simulate a wider range
of parameters in terms of converter thickness, converter material, incoming particle energy, and
different particle species.

While it would be technically feasible to use Geant directly to simulate the converter process, the
production of outgoing particles in the converter is a stochastic process, the details of which are
computationally expensive. The use of probability distribution tables, which only have to be
computed once, while not as accurate, speeds up the computation time by orders of magnitude.

The imputus for developing the new model was to better simulate the converter in the Cornell CESR
Linac which generated positrons due to bombardment of a tungsten plate by electrons. The electrons
have an energy of order $\sim 100 \MeV$.  As the incident electrons pass through the
converter, they emit photons via Bremsstrahlung, which in turn decay to $e^+ e^-$ pairs:
\begin{align}
  e^- + Z \rightarrow e^- + Z + \gamma \rightarrow e^- + Z + e^+ + e^-
\end{align}

\section{The converter model}

For now, assume that the incoming particle's momentum is perpendicular to the surface of the
converter. The coordinate system is shown in Figure \ref{fig:coords}.
\begin{figure}
\centering
\includegraphics[width=0.6\textwidth]{coords1.pdf}
\includegraphics[width=0.8\textwidth]{coords2.pdf}
\caption{Coordinates used to describe the outgoing particles exiting the converter. The incoming particle is 
labeled $e^-$ and the outging particle is labeled $e^+$.}
\label{fig:coords}
\end{figure}
The outgoing particle's position on the downstream face of the converter is then described by $r$,
its distance from the $z$ axis, and the angle $\theta$ shown in the figure. By symmetry, $\theta$
must be distributed uniformly between $0$ and $2\pi$ and so the probability distributions will be
independent of $\theta$. The $x$-axis is defined to be in the same direction as $\rrr$. We then
choose our $y$ axis such that $(x,y,z)$ is a right handed orthogonal coordinate system.

\subsection{Probability Distributions}

The incoming particle is labeled by a plus symbol subscript and the outgoing particle is labeled by
a minus symbol subscript. The outgoing particle as it leaves the surface of the converter is
characterized by $\theta$, its momentum $p_+$, the offset from the origin $r$, and the $dx/ds$ and
$dy/ds$ slopes with
\begin{align}
p_+ c & = \left| \mathbf{p}_+ \right| c \\
\dxds & = \frac{p_x}{p_s} \\
\dyds & = \frac{p_y}{p_s}
\end{align}
Ignoring $\theta$, we seek the distribution
\begin{align}
P \left( p_+ c, r, \dxds, \dyds \right)
\end{align}
which describes the probability that an outgoing particle will attain particular values of $p_+ c$,
$r$, $\dxds$, and $\dyds$. This probability distribution is dependent upon the incoming particle's
energy, as well as the thickness and material type of the converter. This will be discussed later.

$P$ is normalized to the number of outgoing particles produced per incoming particle:
\begin{align}
\int P \left( p_+ c, r, \dxds, \dyds \right) \, d(p_+ c) \, dr \, d \! \left( \dxds \right) d \! \left( \dyds \right) & = \frac{N_+}{N_-}
\end{align}
This normalization lets us easily account for the fact that number of outgoing particles produced
varies with the incoming particle energy and converter thickness. $P$ can be decomposed into two distributions:
\begin{align}
P \left( p_+ c, r, \dxds, \dyds \right) & = P_1 \left( p_+ c, r \right) P_2 \left( \dxds, \dyds ; p_+ c , r \right),
\end{align}
where we choose $P_1$ to be normalized to $N_+/N_-$ and $P_2$ to be normalized to 1.

\subsection{$P_1$ and $P_2$ Parameterization}

Using Geant\cite{geant}, A number of incoming particles of a given energy incident upon a converter
of a given thickness is simulated.  The values of $p_+ c$, $r$, $\dxds$, and $\dyds$ for each
outgoing particle at the downstream face of the converter is recorded. This data is then binned into
a two-dimensional histogram by $p_+ c$ and $r$. The sizes of the bins are chosen non-uniformly so
that each bin holds approximately the same number of outgoing particles. The binned data produces a
probability distribution table that characterizes $P_1(p_+c, r)$.

To model $P_2$, for each $(p_+ c, r)$ bin, the distribution of particles in $dx/ds$ and $dy/ds$ is
fit to the functional form
\begin{align}
P_2 \left( \dxds, \dyds; p_+ c, r \right) & = A \frac{1 + \beta \dxds}{1 + \alpha_x^2 \left( \dxds - c_x \right)^2 + \alpha_y^2 \left( \dyds \right)^2}. \label{eq:cauchy}
\end{align}
Since $P_2$ is normalized to $1$, $A$ is not a true fit parameter, but is fixed by the
normalization.  The fit gives values of $c_x$, $\alpha_x$, $\alpha_y$, and $\beta$ in each $(p_+ c,
r)$ bin.  The distribution of $\dxds$ and $\dyds$ is also characterized by $\dxds_{min}$,
$\dxds_{max}$ and $\left| \dyds \right|_{max}$, which define the rectangle in $\dxds \times \dyds$
space where $P_2 \left( \dxds, \dyds \right)$ is significantly nonzero.  Fits to each of the
parameters $c_x, \alpha_x, \alpha_y, \beta, \dxds_{min}, \dxds_{max}, \left| \dyds_{max} \right|$ as
functions of $p_+ c$ and $r$ are made as follows:
\begin{itemize}
\item
At each value of $p_+ c$ below a user-defined cutoff, we perform a 1D fit of the form
\begin{align}
\pi(r) = a_0 + a_1 r+ a_2 r^2 + a_3 r^3 + a_4 r^4 \label{eq:fit1}
\end{align}
for each parameter $\pi =c_x, \alpha_x, \alpha_y, \beta, \dxdsmin, \dxdsmax, \dydsmax$.  For $c_x$
and $\beta$, $a_0$ is fixed to be 0, as $c_x$ and $\beta$ must be zero at $r=0$ by symmetry.  For
all other parameters, $a_4$ is fixed to be zero, so that a third degree polynomial is fit instead of
a fourth degree polynomial.

\item
Above the user-defined $p_+ c$ cutoff, we perform a 2D fit of the form
\begin{align}
\pi(r) = (1 + a_1 (p_+c) + a_2 (p_+c)^2 + a_3 (p_+c)^3)
         (b_0 + b_1 r + b_2 r^2 + b_3 r^3) e^{-(k_p (p_+ c) + k_r r)} + C \label{eq:fit2}
\end{align}
for each parameter $\pi =c_x, \alpha_x, \alpha_y, \beta, \dxdsmin, \dxdsmax, \dydsmax$.  The
parameter $C$ is only used for $\dxdsmax$.  Note that we set the constant term for the $p_+ c$
polynomial to $1$ in each case.  This must be done so that the fitting problem is full rank.

\end{itemize}
With these fits in hand, we have an approximation of $P_2 \left( \dxds, \dyds; p_+ c, r \right)$.

%--------------------------------------------------------

\section{Setup for Generating the Probability Parameters}

Generating the probability parameters has two main stages. In the first stage, the particle creation
events are simulated with Geant, and the resulting outgoing particles are binned into a
histogram. In the second stage, fits are performed for $\dxds$ and $\dyds$.  As such, we have
developed two executables: one responsible for the first stage, \exes, and one responsible for the
second, \exef.

After the probability parameters are generated, a \bmad lattice file can be created that contains
these parameters and this lattice file is used for simulating the converter independent of Geant.

\subsection{Dependencies} 
\label{s:deps} 

To build and run the simulation proper, \exes, you will need an up to date installation of Geant4 on
your system.  See the Geant4 installation guide below for details on how to get Geant4 up and
running on Linux.  You will also need \texttt{cmake} installed, although this is already a
requirement for a standard \bmad \, distribution.

To build and run the fitting program, \exef, you will need the GNU Scientific Library (GSL)
installed on your system.  This is distributed with \bmad, so you should already have it on your
system.

Both executables require a C++ compiler with support for C++17; GCC 8 or higher should be fine.

\subsection{Geant4 Installation Guide}

This guide is an abbreviated version of the instructions found on
\href{http://geant4-userdoc.web.cern.ch/geant4-userdoc/UsersGuides/InstallationGuide/html/}{the
Geant4 website}.
\begin{enumerate}
\item
Download the source files from \url{https://geant4.web.cern.ch/support/download}.

\item
Create a directory where Geant will be installed.  I'll be using \texttt{\$HOME/geant}.

\item
\texttt{cd} to this directory and unpack the downloaded files with
\begin{verbatim}
$ tar xzvf ~/Downloads/geant4.10.06.tar.gz
\end{verbatim}
(change the version number as appropriate).

\item
Make another directory where you will build Geant with
\begin{verbatim}
$ mkdir geant4.10.06-build
\end{verbatim}

\item
\texttt{cd} to this new directory, and use \texttt{cmake} to configure the Geant4 build with
\begin{verbatim}
cmake -DGEANT4_INSTALL_DATA=ON \
      -DCMAKE_INSTALL_PREFIX=$HOME/geant/geant4.10.06-install \
      ../geant4.10.06
\end{verbatim}
The first \texttt{-D} flag will cause the necessary data sets to be downloaded when we build Geant, and the second \texttt{-D} flag sets the install directory.
If you chose a different location for installing Geant, edit this flag as needed.

Note: if you encounter the the error
\begin{verbatim}
Could NOT find EXPAT (missing: EXPAT_LIBRARY EXPAT_INCLUDE_DIR)
\end{verbatim}
at this step, try editting the file \\
\texttt{../geant4.10.06/cmake/Modules/Geant4OptionalComponentents.cmake}, \\
replacing the line
\begin{verbatim}
option(GEANT4_USE_SYSTEM_EXPAT "Use system Expat library" ON)
\end{verbatim}
with
\begin{verbatim}
option(GEANT4_USE_SYSTEM_EXPAT "Use system Expat library" OFF)
\end{verbatim}
and then re-run the above \texttt{cmake} command.

\item
After \texttt{cmake} finished running, start building Geant with
\begin{verbatim}
$ make -jN
\end{verbatim}
where \texttt{N} is the number of threads you want to use for the compilation.

\item
Once the compilation has finished, install to the install directory you specified in step 5 with
\begin{verbatim}
$ make install
\end{verbatim}

\item
The file \texttt{\$HOME/geant/geant4.10.06-build/geant4make.sh} must be sourced to add Geant4 to your path.
To do so, add the following to your \texttt{.bashrc}:
\begin{verbatim}
cd $HOME/geant/geant4.10.06-build && source geant4make.sh && cd -
\end{verbatim}
Again, if you chose a different location for installing Geant, modify this as necessary.

\end{enumerate}



\subsection{Compiling the Executables}
The converter simulation and fitting programs are distributed as part of the \texttt{util\_programs} project, which is included in a standard \bmad \, distribution.
If you are working with a release instead of a distribution, you can pull down the latest version of the \texttt{util\_programs} project with
\begin{verbatim}
$ svn co https://accserv.lepp.cornell.edu/svn/trunk/src/util_programs
\end{verbatim}
Once you have \texttt{util\_programs} on your system, you will need to add
\begin{verbatim}
export ACC_BUILD_TEST_EXES="Y"
\end{verbatim}
to your \texttt{.bashrc}, and then start a new shell.
\texttt{cd} to the \texttt{util\_programs} folder, and then simply run \texttt{mk} to build \exes \, and \exef \, (or \texttt{mkd} for debug builds).



\section{How to run the programs}

\subsection{Configuration}
Both \exes \, and \exef \, are configured by editing the file \configfile, which should be in the working directory where you run both executables.
Each line in this file should have the form
\begin{verbatim}
setting = value
\end{verbatim}
Comments can be inserted with an exclamation mark \texttt{!} and last until the end of the line.
An example config file, with all available settings listed, is shown below.
\begin{verbatim}
! Example configuration file
! The ! introduces a comment that lasts until the end of the line
target_material = tungsten ! Defines the converter material
target_thickness = 6.35 mm, 1.0 cm ! Defines the target thicknesses to be simulated
pc_in = 300 MeV, 500 MeV, 1 GeV ! Defines the incoming particle energies to be simulated
out_pc_min = 0 ! Minimum pc cutoff for outgoing particles, defaults to 0
out_pc_max = 100000000 ! Maximum pc cutoff for outgoing particles (in eV here)
dxy_ds_max = 10 ! Maximum cutoff for the magnitude of dx/ds
                ! and dy/ds allowed for outgoing particles
output_directory = sim_data ! Name of the directory where data will be output,
                            ! should be specified relative to the working directory
                            ! Defaults to sim_data
num_bins = 15 ! Number of bins to use for both pc and r histogram binning
              ! with just this line, you would have a 15x15 histogram
              ! Defaults to 15
num_pc_bins = 12 ! Number of pc bins to use for histogram binning
num_r_bins = 20 ! Number of r bins to use for histogram binning
fit_crossover = 10 MeV ! For alpha and beta fits, this defines the
                       ! point where the fitter transitions from 1D
                       ! to 2D fits, defaults to 10 MeV
\end{verbatim}
\newcommand{\targetm}{\texttt{target\_material}}
\newcommand{\targett}{\texttt{target\_thickness}}
\newcommand{\pcin}{\texttt{pc\_in}}
\newcommand{\outpcmin}{\texttt{out\_pc\_min}}
\newcommand{\outpcmax}{\texttt{out\_pc\_max}}
\newcommand{\dxydsmax}{\texttt{dxy\_ds\_max}}
\newcommand{\outdir}{\texttt{output\_directory}}
\newcommand{\numbins}{\texttt{num\_bins}}
\newcommand{\numrbins}{\texttt{num\_r\_bins}}
\newcommand{\numpcbins}{\texttt{num\_pc\_bins}}
\newcommand{\fitxpt}{\texttt{fit\_crossover}}

All settings accept a single value, except for \pcin \, and \targett, which accept a comma separated list of values.
The settings \outpcmin, \outdir, \numbins, and \fitxpt \, have default values, while the settings \targetm, \targett, \pcin, \outpcmax, and \dxydsmax \, must be specified in the file.

The settings \pcin, \outpcmin, and \outpcmax \, take values with dimensions of energy.
These default to eV if no unit is specified, although \texttt{MeV} and \texttt{GeV} can be added as suffixes to use MeV and GeV instead as shown in the sample file.
The \targett \, setting takes values with dimensions of length.
The default unit is meters, although \texttt{cm} and \texttt{mm} are supported as well.

The settings \numbins, \numpcbins, and \numrbins \, control the number of bins used in the histogram.
If you only provide a value for \numbins, it will be used for both the number of $p_+c$ bins and then number of $r$ bins.
If you provide \numpcbins \, or \numrbins \, in your config file, this value will supersede \numbins \, for the number of $p_+ c$ or $r$ bins respectively.
If you do not set \numbins \, in your config file, you must set both \numpcbins \, and \numrbins.

\subsection{The Simulation Program}

To run \exes \, and perform the converter simulation, first create and edit the configuration file \configfile, and place it in your working directory.
Then, just run
\begin{verbatim}
$ converter_simulation
\end{verbatim}
at your command prompt.  The program will parse your config file and report the settings it read,
and report if there are any problems reading your config file.  It will then verify that the
directory you set for \outdir \, does not exist or is empty, and will ask you if you want to
overwrite it if it already exists.  Then, for each value of \pcin \, and \targett \, specified in
the config file, the program will simulate many particle creation events for those settings.  For example,
with the above config file, six simulations will be run with the following settings:
\begin{itemize}
\item
300 MeV \pcin \, and 6.35 mm \targett

\item
500 MeV \pcin \, and 6.35 mm \targett

\item
1 GeV \pcin \, and 6.35 mm \targett

\item
300 MeV \pcin \, and 1 cm \targett

\item
500 MeV \pcin \, and 1 cm \targett

\item
1 GeV \pcin \, and 1 cm \targett
\end{itemize}

Depending on your computer and the number of different simulations that need to be run, this step may take several hours.
%Fortunately, this only has to be done once to set up the \bmad \, converter element.


\subsection{The Fitting Program}

Once the simulations are complete, just run
\begin{verbatim}
$ converter_fitter
\end{verbatim}
in the same directory where you ran \exes.
\exef will re-parse your config file for the settings it needs, and will again report on any errors it encounters.
It then performs the fit from Equation \ref{eq:cauchy} in each of the $(p_+c, r)$ bins for each simulation.
At this stage, the program may report that the fitting iteration limit has been reached a few times; this is not cause for concern.
Once this step is complete, and the program has obtained values of $c_x$, $\alpha_x$, $\alpha_y$, and $\beta$ in each $(p_+c, r)$ bin,
it performs the fits from Equations \ref{eq:fit1}-\ref{eq:fit2} on these fit parameters.
Finally, the results of the simulation, as well as the results of the fits, are output to the file \texttt{converter.bmad}, located in the \outdir \, specified in the config file.

\section{Output from the Programs}

\subsection{Simulation Output}

After running the \exes \, program, the directory specified by the \texttt{output\_directory} setting in the configuration file will exist in your working directory.
Inside it, there will be one file of the format '\texttt{E\{pc\_in\}\_T\{thickness\}\_er.dat}', for each incoming $p_- c$ and target thickness specified in the configuration file, where \texttt{pc\_in} and \texttt{thickness} are the $p_- c$ and thickness in MeV and cm respectively.
These files contained the binned data which approximate $P_1(p_+ c, r)$.
The output directory will also contain a directory \texttt{dir\_dat}, with subdirectories \texttt{E\{pc\_in\}\_T\{thickness\}\_er.dat} for each $p_- c$ and target thickness combination.
Each of these directories will contain files named \texttt{E\{pc\_out\}\_r\{r\_out\}\_bin.dat}, which contain the binned $\dxds$ and $\dyds$ data used by \exef.

\subsection{Fitting Output}

After running the \exef \, program, the output directory will also contain a file called \texttt{converter.bmad}.
This file aggregates all the information about $P_1$ and $P_2$ at each $p_+ c$ and target thickness tested, and is designed for use with a \bmad \, converter element.

\subsubsection{Gnuplot Files}

\exef \, also generates several gnuplot scripts for inspecting the quality of the obtained fits.
These are all written to the individual \texttt{E\{\}\_r\{\}} directories under \texttt{dir\_dat}.

Each of the $(p_+ c, r)$ bins gets two gnuplot scripts: \texttt{cauchy\_E\{pc\_out\}\_r\{r\_out\}.gp} and \texttt{meta\_E\{pc\_out\}\_r\{r\_out\}.gp}.
The scripts with the \texttt{cauchy} prefix display the distribution $P_2$ obtained by directly fitting Equation \ref{eq:cauchy} to the data in each bin.
The scripts with the \texttt{meta} prefix display the distribution $P_2$ obtained from evaluating the fits from Equations \ref{eq:fit1}-\ref{eq:fit2} to the Cauchy fit parameters.

\exef \, also outputs scripts for viewing the fits from Equations \ref{eq:fit1}-\ref{eq:fit2} across all $(p_+ c, r)$ bins.
These are named \texttt{c\_x\_master.gp},
\texttt{a\_x\_master.gp},
\texttt{a\_y\_master.gp},
\texttt{beta\_master.gp},
\texttt{dxds\_min\_master.gp},
\texttt{dxds\_max\_master.gp},
and \texttt{dyds\_max\_master.gp}.

To view any of these plots, simply open Gnuplot in the \texttt{E\{\}\_T\{\}} directory of interest, and call the script.
For example:
\begin{verbatim}
$ pwd
/home/user/sim_data/dir_dat/E300_T0.635
$ gnuplot

        G N U P L O T
        Version 5.2 patchlevel 8    last modified 2019-12-01

        Copyright (C) 1986-1993, 1998, 2004, 2007-2019
        Thomas Williams, Colin Kelley and many others

        gnuplot home:     http://www.gnuplot.info
        faq, bugs, etc:   type "help FAQ"
        immediate help:   type "help"  (plot window: hit 'h')

Terminal type is now 'qt'
gnuplot> call 'c_x_master.gp'
\end{verbatim}
This will open the plot for $c_x$ across all $(p_+ c, r)$ bins for $p_+ c = 300$ MeV, $T = 0.635$ cm.

\section{The \bmad \, Converter Element}

As mention in the previous section, \exef outputs a file, \texttt{converter.bmad}, which encodes all of the simulation and fitting output, and can be used to specify the properties of a \bmad \, converter element.
See the \bmad \, manual for the full details regarding the converter element in \bmad.

The user should be aware that \bmad's method of generating outgoing particles may not be completely faithful to the simulation results.
In particular, \bmad uses linear interpolation for the $P_1(p_+c, r)$ distribution.
This can cause noticeable discrepancies on the edges of the distribution, especially in the bins with the lowest values of $p_+ c$.
Since $P_1(p_+c, r)$ changes rapidly at low $p_+ c$, and \bmad \, additionally does not generate outgoing particles with $p_+ c$ lower than the lowest value of $p_+ c$ for the bins, \bmad \, does not generate as many outgoing particles at low $p_+ c$ as it should.


\appendix

\section{Notes for ACC Computer Users}

As detailed in Section \ref{s:deps}, building \exes and \exef requires GSL, Geant, and a C++ compiler with support for C++17.
GSL is already available on the lab machines, and a build of Geant is provided at \texttt{/nfs/acc/temp/jmm699/geant}.
To get access to this Geant build, you can simply add the following to your \texttt{.bashrc}:
\begin{verbatim}
cd /nfs/acc/temp/jmm699/geant/geant4.10.06.p01-build && source geant4make.sh && cd -
\end{verbatim}
As for the C++ compiler, you can get access to GCC 8.3 by adding
\begin{verbatim}
source /opt/rh/devtoolset-8/enable
\end{verbatim}
to your \texttt{.bashrc}.








\printbibliography


\end{document}

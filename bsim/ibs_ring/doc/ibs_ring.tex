\documentclass[11pt]{article}
\usepackage{tocloft}
\usepackage{geometry}            % See geometry.pdf to learn the layout options. There are lots.
\usepackage{xspace}
\geometry{letterpaper}           % ... or a4paper or a5paper or ... 
%\usepackage[parfill]{parskip}   % To begin paragraphs with an empty line rather than an indent
\usepackage{graphicx}
\usepackage{amssymb}
\usepackage{amsmath}
\usepackage{alltt}
\usepackage[T1]{fontenc}   % so _, <, and > print correctly in text.
\usepackage[strings]{underscore}    % to use "_" in text
\usepackage[pdftex,colorlinks=true,bookmarksnumbered=true]{hyperref}

%---------------------------------------------------------------------------------

\newcommand{\sref}[1]{$\S$\ref{#1}}
\newcommand{\srthree}{\texttt{Synrad3D}\xspace}
\newcommand\ttcmd{\begingroup\catcode`\_=11 \catcode`\%=11 \dottcmd}
\newcommand\dottcmd[1]{\texttt{#1}\endgroup}
\newcommand{\Begineq}{\begin{equation}}
\newcommand{\Endeq}{\end{equation}}
\newcommand{\fig}[1]{Figure~\ref{#1}}
\newcommand{\vn}{\ttcmd}           
\newcommand{\Th}{$^{th}$\xspace}
\newcommand{\Newline}{\hfil \\}


\newcommand{\bearray}{\begin{eqnarray}}
\newcommand{\eearray}{\end{eqnarray}}
\newcommand{\be}{\begin{equation}}
\newcommand{\ee}{\end{equation}}
\newcommand{\bearraynn}{\begin{eqnarray*}}
\newcommand{\eearraynn}{\end{eqnarray*}}
\newcommand{\benn}{\begin{displaymath}}
\newcommand{\eenn}{\end{displaymath}}
\newcommand{\eq}[1]{{Eq.~(\ref{#1})}}
\newcommand{\eqs}[2]{{Eqs.~(\ref{#1}--\ref{#2})}}

\newlength{\dPar}
\newlength{\ExBeg}
\newlength{\ExEnd}
\setlength{\dPar}{1.5ex}
\setlength{\ExBeg}{-\dPar}
\addtolength{\ExBeg}{-0.5ex}
\setlength{\ExEnd}{-\dPar}
\addtolength{\ExEnd}{-0.0ex}

\newenvironment{example}
  {\vspace{\ExBeg} \begin{alltt}}
  {\end{alltt} \vspace{\ExEnd}}

%---------------------------------------------------------------------------------

\setlength{\textwidth}{6.25in}
\setlength{\hoffset}{0.0in}
\setlength{\oddsidemargin}{0.0in}
\setlength{\evensidemargin}{0.0in}
\setlength{\textheight}{8.5in}
\setlength{\topmargin}{0in}

\setlength{\parskip}{\dPar}
\setlength{\parindent}{0ex}

\setlength\cftparskip{0pt}
\setlength\cftbeforesecskip{3pt}
\setlength\cftaftertoctitleskip{15pt}

%---------------------------------------------------------------------------------

\title{ibs_ring Simulation Program}
\author{Michael Ehrlichman, D. Sagan}
\date{December 5, 2023}

%---------------------------------------------------------------------------------

\begin{document}
\maketitle

\pdfbookmark[1]{Contents}{Contents}
\tableofcontents

%------------------------------------------------------------------
\section{Introduction} 
\label{s:intro}

\vn{ibs_ring} is a program for simulating intra-beam scattering (IBS) in a storage ring.
The source code for this program lives in the \vn{bsim} directory in
the standard Bmad\cite{b:bmad} distribution.

Many formulas for calculating IBS rates are implemented.  They are:
\begin{enumerate}
\item Modified Piwinski with Zotter's integral.
\item Modified Piwinski with constant Coulomb Log.
\item Completely Integrated Modified Piwinski (CIMP).
\item Bjorken \& Mtingwa's formula.
\item Bane's approximation of Bjorken \& Mtigwa's formula.
\item Kubo \& Oide's generalization of Bjorken \& Mtingwa's formula.
\end{enumerate}
These formula's are described in the following section.

%---------------------------------------------------------------------------------
\section{Methods for calculating IBS growth rates}

%---------------------------------------------------------------------------------
\subsection{Modified Piwinski with Zotter's integral}
This formula is selected by setting {\tt ibs_formula=`mpzt'}.

Piwinski's original formula for calculating IBS rates is Ref.~\cite{b:pw}.  The
original formula contained a numerically difficult triple integral.  In
Ref.~\cite{b:zotter}, the triple integral is reduced to a single integral that is 
much easier to evaluate.  No 
approximations are applied to obtain the single integral, it is exact.

The original formula included Twiss $\beta$ and dispersion $\eta$, but neglicted
the derivatives of the lattice functions Twiss $\alpha$ and $\eta'$.
In Ref.~\cite{b:martini}, the derivatives of the lattice functions are included.
This form of Piwinski's original formula with derivatives of the lattice
functions is usually called ``Modified Piwinski''.

%---------------------------------------------------------------------------------
\subsection{Modified Piwinski with constant Coulomb Log}
This formula is selected by setting {\tt ibs_formula=`mpxx'}.

Piwinski's origional IBS formula typically gives a considerablly different growth rate than
Bjorken \& Mtingwa's, Kubo \& Oide's, and approximations of Piwinski.  This is because
Piwinski's formula is unique in how it treats the Coulomb Logarithm.

In the {\tt `mpxx'} IBS formula, Piwinski's formula has been rederived and the
log term pulled out of the integral.  This allows the Coulomb Logarithm to be treated the
same as it is in other IBS formulas.  This derivation is available in Ref.~\cite{b:ehr-thesis}.

%---------------------------------------------------------------------------------
\subsection{Completely Integrated Modified Piwinski (CIMP)}
This formula is selected by setting {\tt ibs_formula=`cimp'}.

In Ref.~\cite{b:wolski}, a high energy approximation of the Modified Piwinski formula is 
obtained.  This formula contains one integral which can be easily and quickly tabulated.
This formula is very fast and, for ILC and CesrTA, returns growth rates similar to those obtained
from the more general IBS formulas.

%---------------------------------------------------------------------------------
\subsection{Bjorken \& Mtingwa's formula}
This formula is selected by setting {\tt ibs_formula=`cimp'}.

In Ref.~\cite{b:bjmt}, the authors take a distinct approach to calculating IBS growth rates.
In Ref.~\cite{b:bane-comp}, Bjorken \& Mtingwa's formula is compared to
the Modified Piwinski formula.  It is found that, for high energy beams, after some
modifications to Piwinski's formula, the two formulas are algebraically similar
and give similar results.

%---------------------------------------------------------------------------------
\subsection{Bane's approximation of Bjorken \& Mtingwa's formula}
This formula is selected by setting {\tt ibs_formula=`bane'}.

In Ref.~\cite{b:bane-approx}, a high energy approximation of Bjorken \& Mtingwa's formula
is obtained.  The formula is simpler and numerically easier to evaluate, but does
not give sensible results when vertical dispersion is zero.

%---------------------------------------------------------------------------------
\subsection{Kubo \& Oide's generalization of Bjorken \& Mtingwa's formula}
This formula is selected by setting {\tt ibs_formula=`kubo'}.

In Ref.~\cite{b:kubo}, a generalization of Bjorken \& Mtingwa's formula
is derived.  This formula is unique in that it is based on the $6\times6$
matrix of the second order moments of the beam distribution (the beam sigma matrix),
rather than on Twiss parameters.

This formula should be able to handle arbitrary coupling conditions, though
that has not been tested in experiment.

When {\tt ibs_formula=`kubo'}, it may be helpful to set {\tt use_t6_cache=.true.}.
This causes {\tt ibs_ring} to pre-compute the one-turn maps.  This greatly speeds up
{\tt ibs_ring} on larger accelerators.  With {\tt use_t6_cache=.true.}, the simulation
is ${\cal O}\left(n\right)$.  With {\tt use_t6_cache=.false.}, the simulation
is ${\cal O}\left(n^2\right)$.  This setting has no effect for other IBS calculation methods.

\section{Setting vertical dispersion}
Vertical dispersion is zero in an ideal flat storage ring.  Realistically, storage rings are not
ideal and have misalignments which result in vertical dispersion.  One method to simulate the effect
of misalignments is to use a lattice with a realistic distribution of element misalignments.
Since it is often the case that obtaining such a lattice is not possible, {\tt ibs_ring}
allows the vertical dispersion $\eta_b$ and derivative of vertical dispersion $\eta_b'$ to be
specified in the parameters file.

The following setting can be used to set the vertical dispersion.
\begin{verbatim}
   set_dispersion = <logical>
          eta_set = <float>
         etap_set = <float>
\end{verbatim}

The interpretation of these parameters depends on the IBS formula being used.  These differences 
are described below.  Note that the two interpretations are not equivalent.  Results obtained
from the Twiss-based IBS formulas using a particular {\tt eta_set} and {\tt etap_set} are
not necessarily comparable to results obtained from the {\tt kubo} formula using the
same {\tt eta_set} and {\tt etap_set}.

\subsection{For all IBS calculation formulas except {\tt kubo}}
If {\tt set_dispersion=.true.}, then {\tt ele\%b\%eta} and {\tt ele\%b\%etap} will be overwritten
with {\tt eta_set} and {\tt etap_set} for every element.

\subsection{For {\tt kubo} IBS formula}
If {\tt set_dispersion=.true.}, then the one-turn matrix $\mathbf{T}_6$ will be replaced
with $\mathbf{\tilde{T}}_6$, where $\mathbf{\tilde{T}}_6=\mathbf{T}_6\mathbf{W}$ and
\begin{equation}
\mathbf{W}=\begin{pmatrix}
1&0&0&0&0&0\\
0&1&0&0&0&0\\
0&0&1&0&0&-\eta_b\\
0&0&0&1&0&-\eta_b'\\
0&0&\eta_b'&-\eta_b&1&0\\
0&0&0&0&0&1\\
\end{pmatrix},
\end{equation}
where $\eta_b$ and $\eta_b'$ are set by {\tt eta_set} and {\tt etap_set}.

\section{Potential well distortion (pwd)}
Intrabeam scattering does not directly change the bunch length.  It changes the energy spread,
which in turn results in a change in bunch length.

If {\tt do_pwd=.false.}, then the simulation maintains the ratio between energy spread and bunch length.
i.e.,
\begin{equation}
\frac{\sigma_{p,IBS}}{\sigma_{z,IBS}}=\frac{\sigma_{p0}}{\sigma_{z0}},
\end{equation}
where $\sigma_{p0}$ and $\sigma_{z0}$ are the zero-current energy spread and bunch length,
and $\sigma_{p,IBS}$ and $\sigma_{z,IBS}$ are the energy spread and bunch length after
including IBS effects.

If {\tt do_pwd=.true.}, then the the effect of potential well distortion is simulated as a 
current-dependent defocusing rf voltage $V_{pwd}$,

\begin{equation}
V_{pwd}=\frac{-L N e c^2}{\sqrt{2\pi}\sigma_z^3 E_0},
\end{equation}
where $e$ is electric charge, $c$ is the speed of light,
$\sigma_z$ is the bunch length, and $E_0$ is the beam energy.
$L$ is inductance and is a parameter set using {\tt inductance = <float>}.
Typical values of $L$ are tens of nH.

%---------------------------------------------------------------------------------
\section{Methods for determining equilibrium emittances}

%---------------------------------------------------------------------------------
\subsection{Derivatives method}
This method is selected by setting {\tt eqb_method=`der'}.

This method finds the equilibrium beam size using differential equations to 
evolve the emittance through time.  The time step is hard coded to $\frac{\tau}{10}$,
where $\tau$ is the horizontal damping time.

The differential equations are,
\begin{align}
\frac{d\epsilon_a}{dt}&= -\left(\epsilon_a-\epsilon_{a0}\right)\frac{2}{\tau_a}+\epsilon_a\frac{2}{T_a}\label{e:da}\\
\frac{d\epsilon_b}{dt}&= -\left(\epsilon_b-\epsilon_{b0}\right)\frac{2}{\tau_b}+\epsilon_b\frac{2}{T_b}\label{e:db}\\
\frac{d \sigma_p}{dt}&= -\left(\sigma_p-\sigma_{p0}\right)\frac{1}{\tau_z}+\sigma_p\frac{1}{T_z}\label{e:dp},
\end{align}
where $\epsilon_{a0}$, $\epsilon_{b0}$, and $\sigma_{p0}$ are the zero-current emittances and energy spread,
$\tau_a$, $\tau_b$, and $\tau_z$ are the damping times, and $T_a$, $T_b$, and $T_z$ are the IBS
growth rates given the IBS formulas.  The factors of $2$ come about because $\tau_a$, $\tau_b$,
$T_a$, and $T_b$ are betatron growth rates.  i.e.  they are the rates of change the beam sizes,
rather than the emittances.

%---------------------------------------------------------------------------------
\subsection{Relaxation of equilibrium equations}
\label{ss:rlx}

This method is selected by setting {\tt eqb_method=`rlx'}.

The solutions to differential equations (\ref{e:da}), (\ref{e:db}), and (\ref{e:dp}) are,
\begin{align}
\epsilon_a&=\frac{1}{1-\frac{\tau_a}{T_a}}\epsilon_{a0}\label{e:eea}\\
\epsilon_b&=\frac{1}{1-\frac{\tau_b}{T_b}}\epsilon_{b0}\label{e:eeb}\\
\sigma_p&=\frac{1}{1-\frac{\tau_z}{T_z}}\sigma_{p0}\label{e:eep}.
\end{align}
Note that $T_a$, $T_b$, and $T_z$ are functions of $\epsilon_a$, $\epsilon_b$, and $\sigma_p$.

In Ref.~\cite{b:wolski}, a method for approximating the effect of 
transverse coupling replaces Eqn.~\ref{e:eeb} with,
\begin{equation}
\epsilon_b=\left(\left(1-r_\epsilon\right)\frac{1}{1-\frac{\tau_b}{T_b}}+
r_\epsilon\frac{1}{1-\frac{\tau_a}{T_a}}\right)\epsilon_{b0},
\end{equation}
where $r_\epsilon$ describes the amount of vertical emittance that is due to transverse mode coupling.
$0<r_\epsilon<1$.
$r_\epsilon=0$ describes a situation where there is no mode coupling and $\epsilon_{b0}$ is
determined entirely by physics in the vertical plane.  $r_\epsilon=1$ describes a situation
where $\epsilon_{b0}$ is determined entirely by coupling from the horizontal plane.

In the {\tt ibs\_ring} simulation $r_\epsilon$ is set by {\tt ratio}.

The starting point for the relaxation method must be greater than the equilibrium emittances,
otherwise the solver will encounter non-physical negative emittances.  The starting points
for the three emittances is obtained by multiplying the radiation integrals emittances by
the {\tt initial_blow_up} parameter.  {\tt initial_blow_up} is a 3-element array.  Typically,
these can be set between $3$ and $8$.  Larger values result in a slower simulation, while lower 
values may fail.

\section{Coulomb Logarithm}
The IBS growth rates are directly proportional to a quantity that has come to be called
the Coulomb Logarithm
The Coulomb Logarithm is,
\begin{equation}
\log\frac{b_{max}}{b_{min}},
\end{equation}
where $b_{max}$ is the largest impact parameter for particle-on-particle collisions
within a bunch, and $b_{min}$ is the smallest impact parameter.  The IBS growth rates
blow up as $b_{min}$ goes to zero or $b_{max}$ gets large.

$b_{max}$ is set to the smallest of either the beam height or mean interparticle distance. 

In machines with little damping, such as proton rings and linacs, $b_{min}$ is set to the
impact parameter associated with a scattering angle of $\frac{\pi}{2}$.

It was proposed in \cite{b:raubenheimer}, that
in machines with strong damping, such as light sources, damping rings, and circular
lepton colliders, $b_{min}$ should be set according to the damping rate.  This 
is called the {\it Tail Cut} because the idea is to exclude from the calculation of the
rise time rare, large angle scattering events that populate non-gaussian tails
of the beam distribution.

The Tail Cut was also applied in Ref.~\cite{b:kubo}, albeit using a different formulation.
The formulation there is,
\begin{align}
b_{min1}&=\frac{r_e}{\left(p_\perp \gamma\right)^2}\\
b_{min2}&=\sqrt{\frac{vol}{N\pi p_\perp c \tau_a}}\\
b_{min}&=max\left(b_{min1},b_{min2}\right),
\label{e:tailcut}
\end{align}
where $r_e$ is the classical electron radius, $p_\perp$ is the average 
transverse momentum of particles in the bunch, $\gamma$ is the relativistic factor
of the bunch centroid, $vol$ is the volume of the bunch envelop, $N$ is the number
of particles in the bunch, $c$ is the speed of light, and $\tau_a$ is the 
horizontal damping rate.

In {\tt BMAD}, the method used for calculating the Coulomb Log is set 
using {\tt clog_to_use = <integer>}.  

For all IBS formulas except for {\tt kubo}, the following options are available,
\begin{description}
\item[1] Classic Coulomb Log (no Tail Cut, $\frac{\pi}{2}$ scattering angle).
\item[2] Integral-based Tail Cut given in Ref.~\cite{b:raubenheimer}.
\item[3] Tail Cut Eqn.~\ref{e:tailcut}, as prescribed in Ref.~\cite{b:bane-tc}.
\item[4] Tail Cut Eqn.~\ref{e:tailcut}, similar to that in Ref.~\cite{b:kubo}.
\end{description}

For the {\tt kubo} IBS formulas, the following options are available,
\begin{description}
\item[1]    Tail Cut disabled (no Tail Cut, $\frac{\pi}{2}$ scattering angle).
\item[else] Tail Cut applied as described in Ref.~\cite{b:kubo}.
\end{description}


\section{Beam size calculations}
{\tt ibs_ring} calculates beam size by computing the sigma matrix $\mathbf{\Sigma}$ 
of a beam matched to the machine optics.  
The beam envelop projected into the horizontal, vertical, and longitudinal planes
are the $11$, $33$, and $55$ elements of the sigma matrix.  This method of calculating
beam sizes naturally takes into account arbitrary coupling conditions.

$\mathbf{\Sigma}$ is obtained
from,
\begin{equation}
\mathbf{\Sigma S}=\mathbf{NDN}^\dagger,
\end{equation}
where $\mathbf{S}$ is the symplectic matrix, $\mathbf{N}$ is formed from the
eigenvectors of the one-turn transfer matrix at a particular element,
and $\mathbf{D}$ is,
\begin{equation}
\mathbf{D}=\begin{pmatrix}
0&\epsilon_a&0&0&0&0\\
-\epsilon_a&0&0&0&0&0\\
0&0&0&\epsilon_b&0&0\\
0&0&-\epsilon_b&0&0&0\\
0&0&0&0&0&\epsilon_c\\
0&0&0&0&-\epsilon_c&0\\
\end{pmatrix}.
\end{equation}

This calculation and $\mathbf{N}$ are discussed in detail in Ref.~\cite{b:prst-crab}
and Ref.~\cite{b:wolski-N}.

The {\tt ibs_ring} parameters {\tt x_view}, {\tt y_view}, and {\tt z_view}
are the element indexes
where the horizontal, vertical, and longitudinal beam sizes are calculated.

\section{Current range}
The {\tt ibs_ring} simulation is designed to generate beam size versus current graphs.
The current range, in mA, is set by {\tt low_current}, {\tt high_current}, and {\tt delta_current},

{\tt high_current} is the highest current and {\tt low_current} is the lowest current.
{\tt delta_current} is the approximate step size.  The actual step size is adjusted by the simulation
such that an current range is spanned by equally sized steps.

\section{{\tt ibs_ring} output files}

\subsection{{\tt emittance.dat}}
Properties of beam envelop at equilibrium.  By column:
\begin{description}
\item[{\tt current}] Current in Amps.
\item[{\tt emit_a}] a-mode emittance.
\item[{\tt emit_b}] b-mode emittance.
\item[{\tt sigE/E}] Energy spread.
\item[{\tt sigma_x}] Size of horizontal projection of beam envelop at element {\tt x_view}.
\item[{\tt sigma_y}] Size of vertical projection of beam envelop at element {\tt y_view}.
\item[{\tt sigma_z}] Size of longitudinal projection of beam envelop at element {\tt z_view}.
\end{description}

\subsection{{\tt ibs_rates.out}}
IBS growth rate at each element at equilibrium.  Useful in for answering question
``Where in the lattice is the IBS growth coming from?''.  By column:
\begin{description}
\item[{\tt ele ix}] Element index
\item[{\tt s}] Element location in meters.
\item[{\tt inv_Ta}] $\frac{1}{T_a}$, where $T_a$ is the a-mode IBS growth time.
\item[{\tt inv_Tb}] $\frac{1}{T_b}$, where $T_b$ is the b-mode IBS growth time.
\item[{\tt inv_Tz}] $\frac{1}{T_z}$, where $T_z$ is the longitudinal IBS growth time.
\end{description}

\subsection{{\tt rad_int.out}}
Beam properties at zero current, determined either by radiation integrals or set
in parameters file.  Also contains beam properties at high current ({\tt high_current}).

\section{Parallelization}
The {\tt ibs_ring} code has been parallelized with OpenMP.  Enabling OpenMP
requires setting the environment variable {\tt ACC_ENABLE_OPENMP=1} before
compillation.  The entire
BMAD distribution, including packages such as LAPACK and forest, must be
compiled with {\tt ACC_ENABLE_OPENMP=1}, otherwise the resulting code
will not be thread safe.

If {\tt ibs_ring} is compiled with OpenMP enabled, then the environment
variable {\tt OMP_NUM_THREADS=<integer>} will set the number of cores used for 
computation.

Signs of the code not being thread safe include unstable lattice errors.
Thread safety can be diagnosed by setting {\tt OMP_NUM_THREADS=1}.

%------------------------------------------------------------------
\section{Parameter file summary}
\label{s:run}

The input for the \vn{ibs_ring} program uses Fortran90 namelist
syntax: The data begins with the string \vn{\&parameters} and ends
with a slash \vn{/}. Everything outside this is ignored.
The input parameters are:
\begin{example}
\&parameters
  lat_file = <lattice-file-name>
  granularity = <real>   ! -1 for element-by-element.
  ptc_calc = <logical>  ! If true, use PTC for rad. int. calculations.
  b_emit = <real>  ! (m*rad) Zero current vertical emittance.  Set to -1 for rad int.
  a_emit = <real>  ! (m*rad) Zero current horizontal emittance.  Set to -1 for rad int.
  energy_spread = <real>  ! (rel.) Zero current energy spread.  Set to -1 for rad int.
  fake_3HC = <real> ! If greater than zero, reduce rates by this factor.  
                    ! IBS rates scale with 1/sigma_z
  high_current = <real>  ! (mA) highest current
  delta_current = <real> ! (mA) step size
  low_current = <real>   ! (mA) lowest current
  ibs_formula = <type>  ! 'cimp', 'bjmt', 'bane', 'mpzt', 'mpxx', or 'kubo'
  clog_to_use = <int>   ! 1=no tail cut, 2=raubenheimer, 3=kubo, 4=kubo w/vertical
  eqb_method = <type>   ! 'der' derivatives or 'rlx' for Wolski's relaxation.  
                        ! 'der' is default.
  initial_blow_up = <real>, <real>, <real> ! Multiplier for initial emittances for 
                  ! 'rlx' method.  For the relaxation method, the starting 
                  ! emittances must be greater than the equilibrium emittances.
  ratio = <real>  ! "Coupling parameter r" for relaxation eqb_method
  x_view = <int>  ! ix of element where projection is taken 
                  ! for horizontal beam size calculation.
  y_view = <int>  ! ix of element where projection is taken 
                  ! for vertical beam size calculation.
  z_view = <int>  ! ix of element where projection is taken 
                  ! for longitudinal beam size calculation.
  do_pwd = <logical>  ! Apply potential well distortion to bunch length
  inductance = <real>  ! (arb.) A PWD parameter akin to inductance
  set_dispersion = <logical> ! Assign a vertical dispersion (eta_set and etap_set).
  eta_set = <real> ! (m) If set_dispersion, then at every element, 
                   ! set vertical dispersion to eta_set.
  etap_set = <real> ! (1) If set_dispersion, then at every element, 
                    ! set vertical dispersion' to etap_set.
/
\end{example}

  \begin{description}
  \item[\vn{lat_file}] \Newline
Bmad lattice file describing the ring.

  %------------------------------
  \item[\vn{ibs_formula}] \Newline
Name of the algorithm to use for the calculation. Possibilities are:
\begin{example}
  'cimp'
  'bjmt'
  'bane'
  'mpzt'
  'mpxx'
  'kubo'
\end{example}

  %------------------------------
  \item[\vn{eqb_method}] \Newline
Method used for finding the equilibrium solution. Possibilities are:
\begin{example}
  'der'
  'rlx'
\end{example}
'der' finds the equilibrium emittances using differential equations.
The differential is with respect to time.

'rlx finds the equilibrium emittances by iterating to find the
solution to an analytic form for the equilibrium emittance.  The 'rlx'
method includes the controversial 'coupling parameter' which says that
the vertical emittance has contributions from both the vertical and
horizontal IBS rates and damping rates.

In the limit that the coupling parameter is zero, 'der' and 'rlx' are
equivalent.  In fact, the equations used for 'rlx' are the equilibrium
solution to the 'der' method's differential equations.

'der' is much faster and robust.  However, it does not allow for the
'coupling parameter', which has its uses.  Also, the two different
methods can be useful in diagnosing the code.

  %------------------------------
  \item[\vn{clog_to_use}] \Newline
Logarithmic cutoff to use. Possibilities are:
\begin{example}
  1  ! Classic, no tail cut.  
  2  ! Raubenheimer.  
  3  ! Bane.
  4  ! Oide.  
\end{example} 

  %------------------------------
  \item[\vn{inductance}] \Newline
Longitudinal inductance for PWD calc.
Effects bunch length vs. current.

  %------------------------------
  \item[\vn{set_dispersion}] \Newline
If true, then apply eta_set and etap_set.
If false, then do not.

  %------------------------------
  \item[\vn{eta_set}] \Newline
Used only if ibs_formula set to 'kubo'.
Applies x-pz coupling to each
element of lattice when calculating IBS rates.

  %------------------------------
  \item[\vn{etap_set}] \Newline
Used only if ibs_formula set to 'kubo'. Applies px-pz coupling to
each element of lattice when calculating IBS rates.

  %------------------------------
  \item[\vn{a_emit}] \Newline
Zero current horizontal emittance. If set to -1 then value is 
obtained from an evaluation of the radiation integrals.

  %------------------------------
  \item[\vn{b_emit}] \Newline
Zero current vertical emittance. If set to -1 then value is 
obtained from an evaluation of the radiation integrals.

  %------------------------------
  \item[\vn{energy_spread}] \Newline
Zero current energy spread. If set to -1 then value is 
obtained from an evaluation of the radiation integrals.

  %------------------------------
  \item[\vn{ratio}] \Newline
"Coupling parameter r" hack (\sref{ss:rlx}) for including coupling.

  %------------------------------
  \item[\vn{granularity}] \Newline
Step size along lattice in meters to evaluate the various 
integrals.  Set to -1 for one step per element.

  %------------------------------
  \item[\vn{x_view}] \Newline
Index of element where projection is taken for horizontal beam size
calculation.

  %------------------------------
  \item[\vn{y_view}] \Newline
Index of element where projection is taken for vertical beam size
calculation.

  %------------------------------
  \item[\vn{z_view}] \Newline
Index of element where projection is taken for longitudinal beam size
calculation.

  %------------------------------
  \item[\vn{high_current}] \Newline
Largest current per bunch in mA.

  %------------------------------
  \item[\vn{low_current}] \Newline
Smallest current per bunch in mA.

  %------------------------------
  \item[\vn{delta_current}] \Newline
mA step size.

  %------------------------------
  \item[\vn{initial_blow_up}] \Newline
Array of 3 numbers used to obtain the starting point for the 'rlx' method.


  \end{description}


%------------------------------------------------------------------
\begin{thebibliography}{9}

\bibitem{b:bmad}
D. Sagan,
"Bmad: A Relativistic Charged Particle Simulation Library"
Nuc.\ Instrum.\ \& Methods Phys.\ Res.\ A, {\bf 558}, pp 356-59 (2006).
The Bmad web site:
\hfill\break
\hspace*{0.3in} \url{http://www.lepp.cornell.edu/~dcs/bmad}

\bibitem{b:pw}
Piwinski, A., ``Intra-Beam Scattering''.
Proceedings of the 9th International Conference on High Energy Accelerators,
Stanford, CA, 1974. p. 405.

\bibitem{b:martini}
Martini, M., CERN PS/84-9 (AA) (1984).

\bibitem{b:zotter}
Evans, L. and Zotter, B., ``Intrabeam Scattering in the SPS''.
CERN-SPS-80-15, CERN, 1980.

\bibitem{b:wolski}
Kubo, K., Mtingwa, S. K. and Wolski, A., ``Intrabeam Scattering Formulas for 
High Energy Beams,'' Phys. Rev. ST Accel. Beams, 8, 2005.

\bibitem{b:ehr-thesis}
Ehrlichman, M., ``Thesis: Normal Mode Analysis of Single Bunch, Charge Density
Dependent Behavior in Electron/Positron Beams''.  Cornell University (2013).

\bibitem{b:bjmt}
Bjorken, J., Mtingwa, S., ``Intrabeam Scattering'', Particle Accelerators. 13. 
pp. 115-143. (1983).

\bibitem{b:bane-comp}
Bane, K., ``An Accurate, Simplified Model of Intrabeam Scattering'', SLAC-AP-141,
arXiv:physics/0205058. (2002).

\bibitem{b:bane-approx}
Bane, K. in {\it Proceedings of the 8th European Particle Accelerator Conference, Paris,
France, 2002}, p. 1443, (2002).

\bibitem{b:kubo}
Kubo, K., Katsunobu, O., ``Intrabeam Scattering in Electron Storage Rings'',
Phys. Rev. ST Accel. Beams 4, 124401 (2001).

\bibitem{b:raubenheimer}
Raubenheimer, T. O., ``The Core Emittance with Intrabeam Scattering in e$^+/$e$^-$ Rings'',
Part. Accel. 45, pp. 111-118, (1994).

\bibitem{b:bane-tc}
Bane, K. L. F, Hayano, H., Kubo, K., Naito, T., Okugi, T., Urakawa, J.,
``Intrabeam scattering analysis of measurements at KEK's ATF damping ring'',
SLAC-PUB-9227 (also KEK-Preprint 2002-26), (2002).

\bibitem{b:prst-crab}
Ehrlichman, M., et. al., ``Measurement and Compensation of Horizontal Crabbing at the
Cornell Electron Storage Ring Test Accelerator'',
Submitted to Phys. Rev. ST Accel. Beams (2013).

\bibitem{b:wolski-N}
Wolski, A., ``Alternative approach to general coupled linear optics'', 
Phys. Rev. ST Accel. Beams 9, 024001 (2006).



\end{thebibliography}
\end{document}  

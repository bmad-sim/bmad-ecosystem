\documentclass{hitec}

\usepackage{index}
\usepackage{xr}
\usepackage{textgreek}
\usepackage{setspace}
\usepackage{graphicx}
\usepackage{moreverb}    % Defines {listing} environment.
\usepackage{amsmath, amsthm, amssymb, amsbsy, mathtools}
\usepackage{alltt}
\usepackage{rotating}
\usepackage{enumitem}
\usepackage{subcaption}
\usepackage{xspace}
%%\usepackage{makeidx}
\usepackage[section]{placeins}   % For preventing floats from floating to end of chapter.
\usepackage{longtable}  % For splitting long vertical tables into pieces
\usepackage{multirow}
\usepackage{booktabs}   % For table layouts
\usepackage{yhmath}     % For widehat
\usepackage{xcolor}      % Needed for listings package.
\usepackage{listings}
\usepackage[T1]{fontenc}   % so _, <, and > print correctly in text.
\usepackage[strings]{underscore}    % to use "_" in text
\usepackage[nottoc,numbib]{tocbibind}   % Makes "References" section show in table of contents
\usepackage[pdftex,colorlinks=true,bookmarksnumbered=true]{hyperref}   % Must be last package!

%---------------------------------------------------------------------------------

\newcommand{\Bf}[1]{{\bf #1}}
\newcommand{\bfn}{{\Bf n}}
\newcommand{\bfr}{{\Bf r}}
\newcommand{\bfs}{{\Bf s}}
\newcommand{\bfu}{{\Bf u}}
\newcommand{\bfz}{{\Bf z}}
\newcommand{\bfJ}{{\Bf J}}
\newcommand{\bfM}{{\Bf M}}
\newcommand{\bfQ}{{\Bf Q}}
\newcommand{\bfR}{{\Bf R}}
\newcommand{\bfS}{{\Bf S}}
\newcommand{\bfPhi}{{\Bf \Phi}}
\newcommand{\Plim}{P_{lim}}
\newcommand{\Begineq}{\begin{equation}}
\newcommand{\Endeq}{\end{equation}}
\newcommand{\Th}{$^{th}$\xspace}
\newcommand{\sss}{\vn{Spin_Stroboscope}\xspace}
\newcommand{\eq}[1]{{(\protect\ref{#1})}}
\newcommand{\Eq}[1]{{Eq.~(\protect\ref{#1})}}
\newcommand{\Eqs}[1]{{Eqs.~(\protect\ref{#1})}}
\newcommand{\sref}[1]{$\S$\ref{#1}}

\newcommand{\vn}{\begingroup\catcode`\_=11 \catcode`\%=11 \dottcmd}
\newcommand\dottcmd[1]{{\usefont{T1}{lmss}{bx}{n} #1}\endgroup}

\makeatletter   % So can use @ sign in \renewcommand
\renewcommand{\l@subsection}{\@dottedtocline{2}{1.5em}{3.5em}}
\makeatother    % Reset 

\definecolor{lightcyan}{rgb}{0.88, 1.0, 1.0}
\newcounter{main}
\setcounter{main}{1}
\lstnewenvironment{code}[1][firstnumber=\themain,name=main]
  {\lstset{ %language=haskell,
           %columns=fullflexible,
           columns=fixed,
           basicstyle=\small\ttfamily,
           %numbers=left,
           numberstyle=\tiny\color{gray},
           backgroundcolor=\color{lightcyan},
           #1
          }
}
{\setcounter{main}{\value{lstnumber}}}

%---------------------------------------------------------------------------------

\renewcommand{\ttdefault}{txtt}
%\lstset{basicstyle = \small\asciifamily,columns=flexible}  % Enable cut and paste
\definecolor{backcolor}{rgb}{0.8824,1.0,1.0}   % To match code environment
\lstset{basicstyle = \small, backgroundcolor=\color{backcolor}, escapeinside = {@!}{!@}}

\renewcommand{\textfraction}{0.1}
\renewcommand{\topfraction}{1.0}
\renewcommand{\bottomfraction}{1.0}

\settextfraction{0.9}  % Width of text
\setlength{\parindent}{0pt}
\setlength{\parskip}{1ex}
%\setlength{\textwidth}{6in}
\newcommand{\Section}[1]{\section{#1}\vspace*{-1ex}}

\newenvironment{display}
  {\vspace*{-1.5ex} \begin{alltt}}
  {\end{alltt} \vspace*{-1.0ex}}

\title{Spin Stroboscope Program}
\author{}
\date{David Sagan \\ September 11, 2024}

\begin{document}

\phantomsection
\pdfbookmark[1]{Cover Page}{Cover Page}
\maketitle

\cleardoublepage
\phantomsection
\pdfbookmark[1]{Contents}{Contents}
\tableofcontents

\newpage

%------------------------------------------------------------------
\Section{Introduction} 
\label{s:intro}

\sss is a program which for calculating the invarient spin field using spin stroboscopic
averaging\cite{b:sagan.spin}. Alternative techniques for calculating the invarient spin field
include adiabatic anti-damping and the SODOM-2 algorithm.

Adiabatic anti-damping is a technique that exploits the fact that, away from resonances, the
quantity $\bfS \cdot \bfn$, where $\bfS$ is a particle's spin, is an adiabatic
invariant\cite{b:hoff.spin}. Thus by starting a particle on the closed orbit with its spin parallel
to $\bfn_0$, the closed orbit invariant spin (which is straightforward to compute), and then
tracking the particle while slowly increasing the orbital amplitude to the point where the phase
space ampltitude of interest is reached, the particle's spin will remain, more or less, pointing
along $\bfn$. Issues with this method include the fact that it is not suitable if the particle
crosses any resonances as its amplitude is increased and the computation time needed for a given
accuracy.

Another method for computation of the invariant spin field is the SODOM-2 algorithm\cite{b:sodom2}.
This method computes $\bfn$ by inverting a matrix equation involving the Fourier components of
$\bfn$ and the Fourier components of the transport map.  The number $N$ of components used needs to
be choisen small enough so that the problem can be solved in a reasonable time yet large enough for
the solution to be accurate. Computation time problems arrise for orbital motion in more than one
degree of freedom or for motion near spin-orbit resonances\cite{b:hoff.spin}.

%------------------------------------------------------------------
\Section{Running \sss} 
\label{s:run}

Syntax for invoking \sss:
\begin{code}
  spin_stroboscope \{<master_input_file_name>\}
\end{code}
Example:
\begin{code}
  spin_stroboscope my_input_file.init
\end{code}
The \vn{<master_input_file_name>} optional argument is used to set the master input file name. The
default is ``\vn{spin_stroboscope.init}''.

\sss is built atop the the Bmad software library \cite{b:bmad}. The Bmad library, developed at
Cornell, has been developed for modeling relativistic charged particles in storage rings and Linacs,
as well as modeling photons in x-ray beam lines.

%------------------------------------------------------------------------
\Section{Simulation Technique and Physics Models}
\section{Overview} 

Consider a particle circulating in a storage ring. Let $\bfz(s)$ be the six dimensional
phase space vector of the particle at a given longitudinal position $s$ and let $\bfs(s)$ be the the
spin vector of the particle. It is assumed that any Stern-Gerlach forces are negligible so that the
one-turn orbital map $\bfM(s;\bfz)$, which maps $\bfz(s)$ from $s$ back to $s$, is independent of
the spin $\bfs$ and the one-turn spin transport map $\bfR(s;\bfz)$ is also independent of the spin.
In fact, since the magnitude of the spin is an invarient, $\bfR$ is just a $3 \times 3$ rotation
matrix.

The motion of a particle can be described in action-angle coordinates. The three normal modes of the
orbital motion are denoted $a$, $b$ and $c$. The action vector $\bfJ = (J_a, J_b, J_c)$ is a
constant of the motion and each turn the phase vector $\bfPhi = (\Phi_a, \Phi_b, \Phi_c)$ changes by
the tune $2\pi\bfQ$ where $\bfQ = (Q_a, Q_b, Q_c)$.

Particles with constant action $\bfJ$ travel on a three-torus in six dimensional space. It can be
shown that if there are no resonances, there is an {\em invariant spin field} $\bfn (s,
\bfJ, \bfPhi)$ which is a continuous function with unit amplitude and has the property that if a particle
initially has it's spin parallel to the invariant spin field it will always maintain its spin
parallel to the invariant spin field. In general, the spin of a particle can be written as
\Begineq
  \bfs(\bfPhi) = s_1 \, \bfu_1(\bfPhi) + s_2 \, \bfu_2(\bfPhi) + J_S \, \bfn(\bfPhi)
\Endeq
where $(\bfu_1, \bfu_2, \bfn)$ form a right handed coordinate system. Here, for the sake of
compactness, the dependence on $\bfJ$ and $s$ is implicit. The coefficient $J_S$ is called the spin
action and is an invariant of the motion. The transverse component of the spin $(s_1, s_2)$ will
rotate and average to zero. The quantity $\Plim(\bfJ)$, which is the average of $\bfn$ over all
$\bfPhi$, is the polarization that a set of particles, at a given $\bfJ$, and uniformly distributed
in $\bfPhi$, with spins pointing along $\bfn$, has. Any other distribution of spins will have less
long-term polarization since any spin components perpendicular to $\bfn$ will average to zero. Since 
$\Plim$ is the maximum achievable polarizaiton, it is important that machine lattices are designed
to maximize $\Plim$ over the range of actions $\bfJ$ where the density of particles is significant.

If there are no resonances, the invarient spin field is unique up to a sign flip ($\bfn
\longrightarrow -\bfn$).

The two common methods for computing the inveriant spin field are using a Fourier decomposition
(Hoff 4.1.1) and by averaging the spin motion over many turns. The Fourier decomposition method,
which is embodied by the SODOM and SODOM-2 algorithms, decomposes $\bfn$ into its Fourier
components. Approximating $\bfn$ by its first $N$ components (where $N$ needs to be choisen small
enough so that the problem can be solved in a reasonable time yet large enough for the solution to
be accurate), A matrix equation involving the Fourier components of $\bfn$ and the Fourier
components of the transport map $bfR$ can be constructed. Inverting this matrix equation gives a
solution for the Fourier components of $\bfn$. Computation time problems arrise for orbital motion
in more than one degree of freedom or for motion near spin-orbit resonances.

The alternative to Fourier decomposition is spin-orbit tracking over many turns. An advantage of
spin-orbit tracking is that the computation does not involve resonance denominators which can make
the computation difficult near spin-orbit resonances (Hoff, Sec 4.2 pg 159). However, like Fourier
decomposition, spin-orbit tracking can have computation time issues. Esspecially with large
rings. The calculation of $\bfn$ via spin-orbit tracking uses what is called stroboscopic averaging.
There are several varients. The one used in this program is as follows. The idea is to track a
particle and average the spin over many turns. The trick is that the averaging must be done at a
selected phase space point since $\bfn(\bfPhi)$ is dependent $\bfPhi$. Once $\bfn$ has been
calculated for one $\bfPhi$, $\bfn$ may be computed at other phase space points by simply setting
the particle's spin to $\bfn$ and propagating the particle for a number of turns. Interpolation can
then be used to evaluate $\bfn$ at any given $\bfPhi$.

The averaging of the spin at a given phase space point $\bfPhi_0$ starts with a particle at
$\bfPhi_0$. At this phase space point a guess, $\bfn_g(\bfPhi_0)$, is made as to the invarient spin
direction and this guess is taken as the $N = 0$ average where $N$ is the number of turns
tracked. Also the spin transfer map (matrix) $\bfR_N(\bfPhi_0)$ for $N = 0$ is
saved. $\bfR_N(\bfPhi_0)$ is the spin transfer map for a particle starting at $\bfPhi_0$ tracked
over $N$ turns. $\bfR_0(\bfPhi_0)$ is just the $3 \times 3$ unit matrix. The invarient spin
direction averge $\bfs_N$ over $N$ turns is then
\Begineq
  \bfn_N = \frac{ \bfs_N }{ | \bfs_N | }
\Endeq
where
\Begineq
  \bfs_N = \sum_{n = 0}^N \bfn_g(\bfPhi_n) \, \bfR_n^{-1}(\bfPhi_0)
  \label{snng}
\Endeq
where $\bfPhi_n$ is the phase of the particle after the $n$\Th turn. $\bfR_n$ can be computed
from saving $\bfR_{n-1}$ and then simply multiplying by the spin map for the $n$\Th turn, the
computation time for $\bfn_N$ is linear in $N$. The guess $\bfn_g(\bfPhi_n)$ used in the computation
of $\bfn_N$ is typically chosen to be the invarient spin direction on the closed orbit. The
invarient spin direction on the closed orbit is simply the axis of rotation for the one turn map
on the closed orbit. 

The convergence of \bfn_N to the actual invarient spin direction as $N \rightarrow \infty$ is linear
in $N$. To motivate this (See Hoffstaetter for a more regorous discussion), take as an example the
case where the one-turn spin map is a rotation around the $z$ axis with rotation angle $\nu$
independent of the phase space position of the particle. In this case the invarient spin field is
along the $z$-axis for all points in phase space. Assume that the guess $\bfn_g$ is, for all phase
space points
\Begineq
  \bfn_g = (\alpha, 0, \beta)
\Endeq
where $\alpha^2 + \beta^2 = 1$. Assuming $N$ is large, the averaged invarient spin direction is
\Begineq
  \bfn_N = \frac{1}{f} \left[ \frac{\alpha}{N+1} \, 
    \Re \left( \frac{e^{i (N+1) \nu} - 1}{e^{i \nu} - 1} \right),
    0, \beta \right]
\Endeq
where $\Re()$ stands for real part and $f$ is a normalizing factor to make the magnitude of $\bfn_N$
unity. In the limit of large $N$ $f$ converges to $\beta$. The factor $\Re(\ldots)$ in the above
equation is of order unity so at large $N$ the error $\epsilon$ will vary as
\Begineq
  \epsilon \sim \frac{|\alpha|}{|\beta| \, (N+1)}
\Endeq
That is, the error in the calculated invarient direction decreases as $1/N$. Since $\alpha$ and
$\beta$ are expected to be of order unit, it is expected that the error will scale as $1/N$ with a
coefficient of order unity. As an example of this, Hoffstaetter reports (Section 4.2 pg 159) that
for an accuracy on the level of $10^{-3}$, simulations using the code SPRINT needed around 3000
turns of tracking. And a simulation of HERA-p neede 10,000 turns of tracking to get an accruacy of
$10^{-3}$.


is important since it can be used to
calculate the maximum achievable time averaged polarization by averaging over the invariant spin
field. Expressing the orbital phase space $\bfr$ in terms of the three actions $\bfJ$ and three normal
mode phases $\bfPhi$, the limiting polarization $\Plim(\bfJ)$ at a given orbital amplitude is
\Begineq
  \Plim(\bfJ) = \int d\bfPhi \, \bfn(\bfJ, \bfPhi)
\Endeq
And the maximum polarization of a beam is just
\Begineq
   \Plim(beam) = \int d\bfJ \, \Plim(\bfJ) \, \widehat\rho(\bfJ) 
\Endeq
where $\widehat\rho(\bfJ)$ is the density of the beam, normalized to 1, as a function of orbital
amplitude.

\section{Superconvergent Calculation}

The convergence of the spin-orbit tracking calculation may be speeded up using an iterative method
where the result of the computation, $\bfn_N$ is used to produce a refined guess for $\bfn_g$ which
is then used in \Eq{snng}. This makes the calculation superconvergent. That is, convergance is
faster than $1/N$ which is what is had with the standard calculation of the previous section. The
refined guess starts with $\bfn_N(\bfPhi_0)$ and propagates this fowrard using the transfer maps
$\bfR_n(\bfPhi_0)$. These maps can be saved from the initial tracking run so there is negligible
computation time in calculating the refined guess. If one would now naively use this propagated
guess in \Eq{snng}, no improvement in the accuracy would be obtained (in fact the computed $\bfn_N$
would be unchanged). The reason why the accuracy would not be imporved, despite using a more
accurate $\bfn_g$, is that the errors in $\bfn_g(\bfPhi_N)$ would add coherently. 



\begin{itemize}
%
\item For simple case where spin invariant is independent of phase the superconvergent calculation will produce the
correct solution after tracking one turn!
%
\item For Fourier method suseptabal to problems with nonlinearities in the phase space transport since it is assumed
that the phase space advance is constant (check that this statement is true).
%
\item Superconvergent algorithms: Self-Consistant Stroboscopic Averaging and Scatter Minimization.
%
\item Merit min algorithim is the real space analogue of the Fourier algorithim.
%
\item Scatter Minimization can give accurate results with just a few turns of tracking if there are two phase
space points that are close together.
%
\end{itemize}

%------------------------------------------------------------------
\Section{Master Input File} 
\label{s:master}

%------------------------------------------------------------------
\section{Fortran Namelist}
\label{s:namelist}

Fortran namelist syntax is used for parameter input by \sss. The
general form of a namelist is
\begin{code}
  &<namelist_name>
    <var1> = ...
    <var2> = ...
    ...
  /
\end{code}
The tag \vn{"\&<namelist_name>"} starts the namelist where
\vn{<namelist_name>} is the name of the namelist. The namelist ends
with the slash \vn{"/"} tag. Anything outside of this is
ignored. Within the namelist, anything after an exclamation mark
\vn{"!"} is ignored including the exclamation mark. \vn{<var1>},
\vn{<var2>}, etc. are variable names. Example:
\begin{code}
  &place section =   0.0, "arc_std", "elliptical", 0.045, 0.025 /
\end{code}
here \vn{place} is the namelist name and \vn{section} is a
variable name.  Notice that here \vn{section} is a ``structure'' which
has five components -- a real number, followed by two strings,
followed by two real numbers.

Everything is case insensitive except for quoted strings.

Logical values are specified by \vn{True} or \vn{False} or can be
abbreviated \vn{T} or \vn{F}. Avoid using the dots (periods) that one
needs in Fortran code.

%------------------------------------------------------------------
\section{Example Master Input File} 
\label{s:master.example}

The master input file can be specified on the command line invoking \sss.
If not given, the default name for the master input file is ``\vn{spin_stroboscope.init}''.

Fortran namelist syntax is used (\sref{s:namelist}). Example master input file:
\begin{code}
  &strobe_parameters
    lattice_file = "bmad.lat"                ! Bmad lattice file.
    rf_on = F
    ix_lat_branch = 0
    dat_file     = "spin_stroboscope.dat"    ! Output data file.

    orbit_start = 0, 0, 0, 0, 0, 0           ! Start orbit x, px, y, py, z, pz
    orbit_stop = 0, 0.004, 0                 ! Stop orbit
    n_points = 1, 0, 10, 0, 0, 1
    linear_in_action = T
    ix_spin_branch

    n_turns = 400                            ! Number of turns to track.
    calc_every = 100

    evaluation_fraction
    bin_fraction
    n_relaxation_cycles
    spin_tune_branch_correction
    invar_axis_tolerance
    keep_invar_axis_parallel
    

  /
\end{code}

  \begin{description}
  \item
  \end{description}

%------------------------------------------------------------------
\Section{Output Files} 

%------------------------------------------------------------------
\section{Main Output File}
\label{s:main.out}

%------------------------------------------------------------------
\Section{Custom Spin-Orbit Tracking}

Custom spin-orbit tracking is useful for doing strobascopic averaging with simplified models for
spin-orbit tracking. An example model is the single resonance spin-orbit tracking model used in the
paper by Sagan. 

Custom spin-orbit tracking is implemented by modifying the \vn{track1_custom.f90} routine that is in
the \vn{bsim/spin_stroboscope} source directory and then recompiling the \sss executable. [If you do
not know where this source directory is or how to recompile ask your local Bmad Guru.] The custom
code must be matched with an appropriate Bmad lattice file. An example of such a lattice file along
with an appropriate \sss master input file is in the directory:
\begin{code}
  bsim/spin_stroboscope/examples/
\end{code}


%------------------------------------------------------------------
\begin{thebibliography}{9}

\bibitem{b:sagan.spin}
David Sagan, 
``A Superconvergent Algorithm for Invariant Spin Field Stroboscopic Calculations'',
9th International Particle Accelerator Conference (2018).

\bibitem{b:bmad}
D. Sagan,
``Bmad: A Relativistic Charged Particle Simulation Library''
Nuc.\ Instrum.\ \& Methods Phys.\ Res.\ A, {\bf 558}, pp 356-59 (2006).

\bibitem{b:hoff.spin}
G.~Hoffstaetter, 
{\it High-Energy Polarized Proton Beams, A Modern View}, 
Springer. Springer Tracks in Modern Physics Vol~218, (2006).

\end{thebibliography}
\end{document}  

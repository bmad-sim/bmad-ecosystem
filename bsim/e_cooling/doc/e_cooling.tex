\documentclass[letterpaper,11pt]{article}
%\usepackage{parskip}
\usepackage[left=1in,top=.75in,bottom=1in,right=1in,nohead]{geometry}
\usepackage{url}
%\usepackage{units}
\usepackage{amsmath}
\usepackage{graphicx}
\usepackage{bm}
\usepackage{amssymb}
%\usepackage{multirow}
\usepackage{color}
\usepackage{textcomp}
\newcommand\txt[1]{{\text{#1}}}
\newcommand{\vn}{\begingroup\catcode`\_=11 \catcode`\%=11 \dottcmd}
\newcommand\dottcmd[1]{\texttt{#1}\endgroup}
%\usepackage{setspace}
%\usepackage[super]{natbib}
\renewcommand{\figurename}{FIG.}
\renewcommand*{\thefootnote}{\fnsymbol{footnote}}
%\usepackage[labelsep=period]{caption}
%\bibliographystyle{plain}
\renewcommand{\thesection}{\Roman{section}.} 
\title{Coherent Electron Cooling Element in BMAD}
\author{William Bergan}
\date{\today}
\begin{document}
\maketitle
%\section{Motivation for the Theory}
%\doublespacing



\section{Physics Background}

In coherent electron cooling (CeC), each turn, each hadron receives a longitudinal momentum kick based on its longitudinal delay in traveling between two straight sections, called the modulator and kicker, as well as on its transverse coordinates in each. The source of this feedback is an electron beam having the same relativistic gamma as the hadrons which copropagates with them in both the modulator and kicker sections. In the modulator, the electrons receive energy kicks from the hadrons. In between the modulator and kicker, the electrons traverse an amplification section where their initial energy modulations are amplified and turned into density modulations. In the kicker, the density-modulated electron beam provides energy kicks to the hadrons. More details can be obtained in \cite{cite:stupakovorig}, \cite{cite:stupakovamp}, and \cite{cite:stupakov3d}.

Although the details of the electron-hadron interaction and the electron amplification can be complex, the relevant physics can be reduced to a knowledge of a wake function, which gives the momentum kick provided by the electrons to a test hadron at a location $(x_k, y_k, z_0+\Delta z)$ in the kicker due to a single hadron in the modulator with coordinates $(x_m, y_m, z_0)$. Each turn, a given hadron will receive a coherent kick due to its own wake, as well as an incoherent, diffusive kick due to the wakes of the other nearby hadrons and initial noise in the electron beam.


\subsection{Coherent Kick}
Based on \cite{cite:nagaitsev}, the coherent hadron self-wake is approximated as a function of the form 

\begin{align}\label{eqtn:cec_wake_form}
	w(\Delta z) = A\sin(\kappa \Delta z)e^{-\Delta z^2/2\lambda^2}
\end{align}
\noindent
with $A$, $\kappa$, and $\lambda$ dependent on the hadron's transverse coordinates in the modulator and kicker. For simplicity, we assume that these fit parameters are separable in the relevant coordinates, so that $A = A_{xm}(x_m) A_{ym}(y_m) A_{xk}(x_k) A_{yk}(y_k)$, and similar for $\kappa$ and $\lambda$, where $A_{xm}$, $A_{ym}$, etc are all independent, 1-dimensional functions of the specified coordinate. We note that, in general, the hadron will undergo transverse motion within the modulator and kicker, so that the fit parameters are written

\begin{align}\label{eqtn:cec_wake_avg}
	A &= \langle A_{xm}(x_m) A_{ym}(y_m) \rangle \langle A_{xk}(x_k) A_{yk}(y_k)\rangle \\\nonumber
	\kappa &= \langle \kappa_{xm}(x_m) \kappa_{ym}(y_m)\rangle  \langle \kappa_{xk}(x_k) \kappa_{yk}(y_k)\rangle \\\nonumber
	\lambda &= \langle \lambda_{xm}(x_m) \lambda_{ym}(y_m)\rangle  \langle \lambda_{xk}(x_k) \lambda_{yk}(y_k)\rangle 
\end{align}
\noindent
where the averages are done over longitudinal coordinates in the modulator and in the kicker, as appropriate \cite{cite:napac2022}. In all cases, the kick is applied to the hadron at the center of the kicker element.

The electron current will vary along the length of the bunch. Based on the current-dependence of Eqtn. 26 and 29 of \cite{cite:stupakovamp} for the case of two amplifier straights, we reduce the amplitude of the wake function by a factor $\frac{I^2(z)}{I^2_0}\frac{\sin^2\big(\phi_0\sqrt{I(z)/I_0}\big)}{\sin^2(\phi_0)}$, where $I(z)$ is the electron current at a given hadron's longitudinal coordinate, $I_0$ is the peak electron current, and $\phi_0$ is the phase advance of plasma oscillations in each amplifier straight at the mean wake wavenumber.

A hadron which does not have the design momentum will slip relative to the electrons in the modulator and kicker, smearing out the wake. This effect is approximated by reducing the wake amplitude by a factor $\frac{\sin(R\delta L_m/2\gamma^2)}{R\delta L_m/2\gamma^2} \frac{\sin(R\delta L_k/2\gamma^2)}{R\delta L_k/2\gamma^2}$, where $\delta$ is the hadron's fractional momentum offset, $L_m$ and $L_k$ are the modulator and kicker lengths, respectively, $\gamma$ is the shared relativistic gamma of the hadron and electron beams, and $R$ is an empirically-determined reduction-scale parameter.

\subsection{Diffusive Kick}
The diffusive kick is separated into two parts: a kick due to noise in the hadron beam and a kick due to noise in the electron beam, with Poisson noise assumed in both cases. The electron diffusion is separated into three terms in order to account for the fact that electron noise enters in as both a raw density perturbation as well as an energy kick to nearby electrons, plus a cross-term \cite{cite:ipac2021} \cite{cite:e_noise_arxiv}.

Labeling these diffusion coefficients by $D_h$, $D_{e11}$, $D_{e12}$, and $D_{e22}$, we give each hadron a momentum kick each turn in the kicker drawn from a Gaussian distribution with mean $0$ and mean squared value 
\begin{align}\label{eqtn:cec_diff}
	\sigma_{\Delta\delta}^2 = D_h\lambda_h + (D_{e11} + D_{e12} + D_{e22})\lambda_e
\end{align}
\noindent
where $\lambda_h$ and $\lambda_e$ are the longitudinal hadron and electron densities at the location of the kick. As in the case with the fit parameters, the diffusion coefficients are assumed separable and written as functions of the hadron's transverse coordinates averaged over the kicker, so that 

\begin{align}\label{eqtn:cec_diff_avg}
	D_h &= \langle D_{h,x}(x_k) D_{h,y}(y_k)\rangle\\\nonumber
	D_{e11} &= \langle D_{e11,x}(x_k) D_{e11,y}(y_k)\rangle\\\nonumber
	D_{e12} &= \langle D_{e12,x}(x_k) D_{e12,y}(y_k)\rangle\\\nonumber
	D_{e22} &= \langle D_{e22,x}(x_k) D_{e22,y}(y_k)\rangle
\end{align}

Similar to above, we also apply current rescalings as 

\begin{align}
	D_h &\rightarrow D_h \frac{I^4(z)}{I_0^4}\frac{\sin^4\big(\phi_0\sqrt{I(z)/I_0}\big)}{\sin^4(\phi_0)}\\\nonumber
	D_{e11} &\rightarrow D_{e11}\frac{I^4(z)}{I_0^4}\frac{\sin^4\big(\phi_0\sqrt{I(z)/I_0}\big)}{\sin^4(\phi_0)}\\\nonumber
	D_{e12} &\rightarrow D_{e12}\frac{I^3(z)}{I_0^3}\frac{\sin^4\big(\phi_0\sqrt{I(z)/I_0}\big)}{\sin^4(\phi_0)}\\\nonumber
	D_{e22} &\rightarrow D_{e22}\frac{I^2(z)}{I_0^2}\frac{\sin^4\big(\phi_0\sqrt{I(z)/I_0}\big)}{\sin^4(\phi_0)}
\end{align}

The forms here are due to the fact that the mean squared kicks are typically proportional to the square of the wake function, and so the correction factors should be the squares of the wake correction factor. The exceptions come from the fact that one of the electron wakes is missing a factor of electron current, as seen in Eqtn. 4 of \cite{cite:ipac2021} and Eqtn. 15 of \cite{cite:e_noise_arxiv}.


\section{Programming Necessities}

The \vn{e_cooling} module tracks a given initial distribution of particles through a lattice a specified number of times and gives the instantaneous cooling rates turn-by-turn.

The master input file (\vn{e_cooling.init}) specifies the \vn{lat_file}, the \vn{position_file}, and the \vn{wake_and_diffusion_file}. It also specifies \vn{n_turns}, which tells the program how many times to simulate the beam's passage through the cooler. As discussed below, this may not correspond to the number of turns in the physical machine.

The \vn{lat_file} is the usual lattice file, with some feedback element included. If desiring to simulate more than one turn, the particle distribution from the end will be transported back to the beginning each turn. A \vn{feedback} element is specified with two salve elements, labeled as the \vn{input} and \vn{output}. The kick to the particles occurs in the \vn{output} element, using measurements obtained at both \vn{input} and \vn{output} elements. These two elements represent the physical modulator and kicker elements. Each of these takes as arguments \vn{l} (a real number) and \vn{num_steps} (a positive integer). \vn{l} is the length of the element (in meters). Note that altering this will NOT affect the wake and diffusion coefficients, since those are specified in the \vn{wake_and_diffusion_file}, discussed below. \vn{num_steps} tells us how many discrete steps to take when performing the s-averages described in Eqtn. \ref{eqtn:cec_wake_avg} and \ref{eqtn:cec_diff_avg}.

The \vn{position_file} specifies the initial properties of the hadron beam, and in particular the phase-space coordinates of all simulated particles.

The \vn{wake_and_diffusion_file} contains details of the wake and diffusion functions. These include \vn{ie_peak}, the peak electron current (in Amps); \vn{sigma_ze}, the equivalent electron Gaussian bunch length; \vn{supergaussian_order}, the order of the supergaussian function with which we model the electron bunch; \vn{off_E_reduction}, which tells us how much to reduce the wake for an off-energy hadron; \vn{phi_avg}, the average plasma phase advance in the CeC amplifiers; \vn{turns_per_step}, the number of physical turns to simulate in each timestep; \vn{p_per_bunch}, the number of hadrons in the bunch; \vn{num_sigmas} the number of standard deviations in the hadron bunch length to use when longitudinally binning the hadrons (necessary for calculating diffusion); and \vn{frac_sigmas}, the number of bins to put in each standard deviation of the hadron bunch for this longitudinal binning. More details on some of these parameters are below.

Most of \vn{wake_and_diffusion_file} is spent giving details of the evolution of the wake and diffusion parameters off-axis. The \vn{xm}, \vn{ym}, \vn{xk}, and \vn{yk} arrays specify the coefficients as functions of the hadron's horizontal (x) or vertical (y) offsets in the modulator (m) or kicker (k). \vn{s} gives the relevant offset in meters. \vn{A}, \vn{k}, and \vn{lambda} give the fit functions for the wake discussed in Eqtn. \ref{eqtn:cec_wake_form} and Eqtn. \ref{eqtn:cec_wake_avg}. \vn{D_h}, \vn{D_e11}, \vn{D_e12}, and \vn{D_e22} give the diffusion coefficients discussed in Eqtn. \ref{eqtn:cec_diff} and \ref{eqtn:cec_diff_avg}. These will be interpolated.

We assume that the electron beam has a supergaussian distribution, so that the current as a function of $z$ position is $I(z) = I_0 e^{-(z^2/2\sigma^2)^N}$, where $I_0$ is the peak current, $N$ is the supergaussian order, and $\sigma$ is some falloff parameter. We specify the equivalent Gaussian bunch length, $\sigma_{ze}$, defined so that $I_0 = \frac{Qc}{\sqrt{2\pi}\sigma_{ze}}$, where $Q$ is the electron bunch charge and $c$ is the speed of light. This is related the the parameter $\sigma$ by $\sigma = \frac{N\sqrt{\pi}}{\Gamma(1/2N)}\sigma_{ze}$. Neither of these is the RMS bunch length.

\vn{turns_per_step} tells us how many real turns to simulate with each step through the cooler. If this is set to 1, each hadron traveling through the system will receive its nominal cooling and diffusive kicks. However, for \vn{turns_per_step}$ = N > 1$, the cooling kick will be increased by a factor of $N$ and the incoherent diffusive kick will be increased by a factor of $\sqrt{N}$, since noise adds in quadrature. This will statistically simulate the effect of many passes through the cooler, allowing for fast simulations.

\begin{thebibliography}{9}   % Use for  10-99  references
        \bibitem{cite:stupakovorig}
		G. Stupakov, ``Cooling rate for microbunched electron cooling without amplification'', Phys. Rev. Accel. Beams 21, 114402 (2018).
        \bibitem{cite:stupakovamp}
		G. Stupakov and P. Baxevanis, ``Microbunched electron cooling with amplification cascades'', Phys. Rev. Accel. Beams 22, 034401 (2019).
        \bibitem{cite:stupakov3d}
		P. Baxevanis and G. Stupakov, ``Transverse dynamics considerations for microbunched electron cooling'', Phys. Rev. Accel. Beams 22, 081003 (2019).
	\bibitem{cite:nagaitsev}
		 S. Nagaitsev, V. Lebedev, G. Stupakov, E. Wang, and W. Bergan, ``Cooling and diffusion rates in coherent electron  cooling concepts'', technote FERMILAB-CONF-21-054-AD, 2021.
	\bibitem{cite:napac2022}
                W. F. Bergan and G. Stupakov, ``Effects of Transverse Dependence of Kicks in Simulations of Microbunched Electron Cooling'', in Proceedings of NAPAC 2021 (Albuquerque, NM, USA, 2022) pp. 780-783, WEPA67.
	\bibitem{cite:ipac2021}
                W. F. Bergan, P. Baxevanis, M. Blaskiewicz, E. Wang, and G. Stupakov, ``Design of an MBEC Cooler for the EIC'', in Proceedings of IPAC'21 (Campinas, Brazil, 2021) pp. 1819-1822, TUPAB179.
	\bibitem{cite:e_noise_arxiv}
		W. F. Bergan, ``Electron diffusion in microbunched electron cooling'', arXiv:2403.17721, 2024.


\end{thebibliography}


\end{document}

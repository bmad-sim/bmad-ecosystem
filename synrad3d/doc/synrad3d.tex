\documentclass[11pt,landscape]{article}
\usepackage{geometry}               % See geometry.pdf to learn the layout options. There are lots.
\geometry{letterpaper}                   % ... or a4paper or a5paper or ... 
%\geometry{landscape}              % Activate for for rotated page geometry
%\usepackage[parfill]{parskip} % Activate to begin paragraphs with an empty line rather than an indent
\usepackage{graphicx}
\usepackage{amssymb}
%\usepackage{epstopdf}
%\DeclareGraphicsRule{.tif}{png}{.png}{`convert #1 `dirname #1`/`basename #1 .tif`.png}
\usepackage{hyperref}

\newcommand{\srthree}{\texttt{SYNRAD3D}}

%---------------------------------------------------------------------------------

\title{ \srthree information-DRAFT}
\author{G. Dugan, D. Sagan}
\date{\today}                                      % Activate to display a given date or no date

\begin{document}
\maketitle

%------------------------------------------------------------------
\section{Introduction} 

\srthree is a program which simulates
the production and scattering of synchrotron radiation generated by an
electron beam in a high energy machine. The program was written by David
Sagan of Cornell University.

The principal motivation for this simulation is to estimate the
distribution of photon absorption sites around the vacuum chamber as a
function of longitudinal position in the machine. This information is
then input to simulation programs such as \texttt{POSINST} and
\texttt{ECLOUD}, which simulate photoelectron production in various
magnetic environments, which seeds the formation of the electron
cloud.

\srthree is not to be confused with an older program called
\texttt{SYNRAD}. The \textt{SYNRAD} program is used for calculating
wall heating from the primary beam.  \texttt{SYNRAD} only tracks
photons in the horizontal plane and does not simulate scattering of
the photons from the wall. The advantage of \texttt{SYNRAD} is that it
is fast and directly gives wall heating numbers. The advantage of
\srthree is that it tracks photons in three dimensions and it does
simulate scattering from the wall.

%------------------------------------------------------------------------
\section{Simulation technique and physics models}
\subsection{Simulation technique} 

The \srthree program is based on the Monte
Carlo method. A section of the machine (which may be the complete machine)
is designated for photon generation.

The total number of photons generated is set by the user. \srthree
calculates how many photons need to be generated within each machine
element. \srthree then slices up each element longituinally and
generates photons from each slice. The number of photons generated in
a slice weighted by the local probability of photon emission, which
depends on the instantaneous orbit curvature. A transverse location
and angle is selected for the photon generation which mirrors the
transverse distribution of the beam.

A photon is generated with a randomly chosen direction (relative to
the direction of the beam particle) and energy. The directions and
energies follow a distribution based on the standard synchrotron
radiation formulae. The photon is tracked from the point of origin to
the point at which it hits the vacuum chamber wall. The angle of
incidence relative to the local normal to the vacuum chamber is
computed. A scattering probability is computed, based on this angle
and the photon's energy. Depending on the value of this probability,
the photon is either absorbed at this location, or scattered. If it is
scattered, the scattering is taken to be specular (angle of reflection
equals angle of incidence) and elastic (photon energy does not
change).

The photon is then tracked to the next encounter with the vacuum
chamber wall, and the probability of scattering is again
computed. This process continues until the photon is absorbed.

%------------------------------------------------------------------
\subsection{Photon generation}

Photon generation is based on the standard synchrotron radiation
formulae, applicable for dipoles quadrupoles, and wigglers. The
radiation is assumed to be incoherent, so this program cannot treat
undulator radiation. Polarization of the photon is ignored.

The finite dimensions of the electron beam (both in position and
angle) are included in the simulation when the photon is
generated. One of the inputs to \srthree is the model lattice
to use and the simulation uses the lattice to compute the
variation of the beam size around the lattice, including dispersive
effects, is included. The ``wiggle" variation in the reference orbit
in a wiggler is included.

Typically the beam will not have a closed orbit distortion relative to
the reference orbit, but if one is included in the lattice, then
radiation due to the orbit distortion, for example, in a quadrupole,
is included.  

%------------------------------------------------------------------
\subsection{Photon scattering} 

Synchrotron radiation
photon scattering is treated as purely specular, with a scattering
probability which depends on the photon energy and polar angle with
respect to the chamber normal at the point of incidence.

The scattering probabilities are taken from an X-ray scattering
database at LBNL~\cite{ref:1}. The scattering probabilities depend on
the material of which the vacuum chamber is composed, and on the
surface condition (oxide coating, roughness, etc.) In priniciple, the
scattering probability also depends on the photon polarization
relative to the incident normal to the surface, but this dependence is
ignored in the simulation.

The scattering probabilities in the table currently used by the
simulation were computed for a photon p-polarization (electric field
in the plane of incidence), for an aluminum substrate with a 8 nm
layer of Al$_2$O$_3$, and a surface roughness of 2 nm. See
Fig.~\ref{f1} for some example reflectivity curves, compared with
radiation spectra for wigglers and dipoles in CesrTA at 2 GeV.

\begin{figure}[h!]
\begin{center}
\includegraphics[width=6in, trim = 1in 1in 1in 0.75in]{fig1}
\caption{Reflectivity vs. energy for several grazing angles. 
Points shown are a measurement from INFN~\cite{ref:2}}
\label{f1}
\end{center}
\end{figure}

%------------------------------------------------------------------
\subsection{Main input file} 

The main input file can be specified on the command line involking synrad3d.
If not given, the default name for the main input file is \texttt{synrad3d.init}.
Example main input file:
\begin{verbatim}
&synrad3d_parameters
  ix_ele_track_start = 1 ! Radiation is produced from the *exit* end of this element...
  ix_ele_track_end   = 912 !  ... to the exit end of this element. If = -1 -> Go to the end of the lattice
  photon_direction = 1     ! 1 = Forward generation, -1 = Backward generation.
  num_photons      = 50000  ! Total number of photons generated. 
  ds_step_min      = 0.01  ! Photons are generated at discrete points. 
                           ! Multipole photons can be generated at each point.
                           ! This is minimum distance between points.
  emit_a           = -1     ! Horiontal emit. If < 0 -> Use value from radiation integral calc.
  emit_b           = 7.52E-11     ! Vertical emit.  If < 0 -> Use value from radiation integral calc.
  sig_e            = -1     ! Sig_E/E. If < 0 -> Use value from radiation integral calc.

  lattice_file     = '../lattice/cesr/bmad/bmad_6wig_8nm_2085.lat                         ' 
  wall_file        = 'synrad3d.wall'                          ! File specifying the vacuum chamber wall
  dat_file         = 'synrad3d.dat'

  synrad3d_params%allow_reflections = .true.  ! For testing purposes.
  synrad3d_params%debug = .false.             ! If true then create a file (fort.2) 
                                              !    of photon before/after reflection coords.
  random_seed = 123456   ! 0 -> Use sys clock.
/
\end{verbatim}
The emittances are geometric. The format of the output files
\texttt{dat\_file} and (optionally) \texttt{fort.2} are discussed
below.  

%------------------------------------------------------------------
\subsection{Vacuum chamber wall definition file: \texttt{wall\_file}} 

Here is a sample of a file defining the vacuum
chamber wall.

%------------------------------------------------------------------
\subsection{Vacuum chamber specification}

The vacuum chamber cross section can be specified to be either
elliptical or rectangular. Regions in the longitudinal direction can
be defined at which the vacuum chamber has one of these cross section
types. In the transition region between types, an interpolation is
used for the vacuum chamber cross section.  

%------------------------------------------------------------------
\section{Input files}

The statements which are not commented out describe a chamber with an
elliptical cross section starting at $z=0$ and ending at the end of
the lattice. (At the last point, $z$ must be 0, and the actual lattice
length will be used.)

The statements which are commented out describe with an elliptical
chamber from $z=0$ to $z=74.307$ and from $z=101.3$ to the end of the
lattice, and a rectangular chamber between $z=82.891$ and
$z=91.176$. In the transition regions from $z=74.307$ to $z=82.971$,
and from $z=91.176$ to $z=101.3$, the chamber cross section is a
linear interpolation between elliptical and rectangular.
\begin{verbatim}
! format is:
!		wall_pt(n) = z, shape, x_half_width, y_half_height
!
! Notes:
!		At the last point, the actual lattice length will be substituted for  z.
!		Possible shapes: 'elliptical', 'rectangular'

&synrad3d_wall
! wall_pt(0) = 0.0, 'elliptical', 0.045, 0.025
! wall_pt(1) = 74.307, 'elliptical', 0.045, 0.025
! wall_pt(2) = 82.971, 'rectangular', 0.045, 0.025
! wall_pt(3) = 91.176, 'rectangular', 0.045, 0.025
! wall_pt(4) = 101.300, 'elliptical', 0.045, 0.025
! wall_pt(5) = 0.0, 'elliptical', 0.045, 0.025
    wall_pt(0) = 0.0, 'elliptical', 0.045, 0.025
    wall_pt(1) = 0.0, 'elliptical', 0.045, 0.025
/
\end{verbatim}


%------------------------------------------------------------------
\subsection{Lattice definition file: \texttt{lattice\_file}} This file
defines the machine optics. Any file format readable by BMAD is
acceptable (e.g, bmad format, mad format, xsif format).

%------------------------------------------------------------------
\section{Output files} 

%------------------------------------------------------------------
\subsection{Main output file:
\texttt{dat\_file}} The first 20 lines of this file echo the
information provided in the file \texttt{synrad3d.init}.
Subsequently, \texttt{dat\_file} contains 4 lines for each photon
generated. An example of the output generated for two photons is shown
below.
\begin{verbatim}
 1       2     71.52  6.614E-03  ! index, n_reflect, eV, intensity
    0.000542    0.000037   -0.000073    0.000080       9.015    1.000000  ! Start position
    0.015165    0.004010    0.023538    0.017410      18.661    0.999840  ! End position
      37   DRIFT             ! Lat ele index and class
       2       1     11.25  6.614E-03  ! index, n_reflect, eV, intensity
   -0.000604   -0.000046    0.000003    0.000826       9.036    1.000000  ! Start position
    0.013187   -0.004171   -0.023902   -0.009377      15.524    0.999947  ! End position
      36   SBEND             ! Lat ele index and class
\end{verbatim}
The first line gives the photon index number, the number of times the
photon struck the vacuum chamber before absorption, the photon energy
(in eV), and the photon intensity.

The photon intensity is always the same for all photons in a given
run. If $N_{\gamma}$ is the total number of photons per beam particle
emitted in one revolution, then the photon intensity in a simulation
run containing $N_{sim}$ photons is defined as $I=N_{\gamma}/N_{sim}.$
Thus, if a simulation run is done using $N_{sim}$ photons, and $N_L$
is the number of photon absorption sites in a longitudinal region of
length $L$, then the average number of photons per beam particle per
unit length absorbed in this region is $(N_L/L)\times I.$ This number
is needed as input for electron cloud simulation programs.

In the second line, the initial position of the generated photon is
given. The six numbers are $x$ (m), $v_x/c$, $y$ (m), $v_y/c$, $s$
(m), and $v_s/c.$ $x$ is the horizontal position (direction along the
local normal to the closed orbit, in the bend plane, zero on the
closed orbit, positive to the outside of the machine, ), $y$ is the
vertical position (direction perpendicular to the bend plane, zero on
the closed orbit, positive up), and $s$ is the longitudinal position
(direction tangent to the closed orbit, zero at the beginning of the
lattice, positive in the direction of motion of the beam).

The third line gives the position of the point at which the photon is absorbed.

The fourth line gives the lattice element index and element type for
the longitudinal position at which the photon is absorbed. This is
provided so that the photon hits can be sorted by magnetic environment
(e.g., drift, sbend, wiggler, etc.)  


%------------------------------------------------------------------
\subsection{Optional output file: \texttt{fort.2}} 


If the switch \texttt{synrad3d\_params\%debug} is set
to \texttt{.true.}, then an additional output file \texttt{fort.2} is
generated, which contains more detailed information. An example of the
output generated for 4 photons is shown below. (This example is not
for the same run as in the previous section).
\begin{verbatim}
*********************************************
       1       0    1.0000    0.0000    0.0000              0.031988    0.135670
    0.045000    0.031988    0.000086    0.000031   70.549655    0.999488
 *********************************************
       2       0    1.0000    0.0000    0.0000              0.031988    0.733201
    0.045000    0.031988    0.001106    0.000393   70.549655    0.999488
               -0.031988                0.000393                0.999488
 *********************************************
       2       1    1.0000    0.0000    0.0000              0.022632    0.803131
    0.045000    0.022632    0.003817    0.000393   77.441042    0.999744
               -0.022632                0.000393                0.999744
 *********************************************
       2       2    1.0000    0.0000    0.0000              0.022632    0.803131
    0.045000    0.022632    0.005382    0.000393   81.419627    0.999744
               -0.022632                0.000393                0.999744
 *********************************************
       2       3    1.0000    0.0000    0.0000              0.026406    0.774138
    0.045000    0.026406    0.007719    0.000393   87.361447    0.999651
 *********************************************
       3       0    1.0000    0.0000    0.0000              0.031988    0.547595
    0.045000    0.031988   -0.000434   -0.000154   70.549655    0.999488
               -0.031988               -0.000154                0.999488
 *********************************************
       3       1    1.0000    0.0000    0.0000              0.022632    0.691365
    0.045000    0.022632   -0.001498   -0.000154   77.441042    0.999744
 *********************************************
       4       0    1.0000    0.0000    0.0000              0.031988    0.135670
    0.045000    0.031988    0.000088    0.000031   70.549655    0.999488
\end{verbatim}
Individual entries are separated by a row of asterisks. Each entry
corresponds to a photon striking the vacuum chamber. There may be one
or more entries for each photon, depending on the number of times it
strikes the vacuum chamber before being absorbed.

The first line in the first entry starts with two integers: the photon
index, followed by the number of reflections. For the first entry for
each photon, the number of reflections will always be zero. In the
example shown, the next three real numbers are always 1.0, 0.0 and
0.0. The following real number is $\cos{\theta_\perp}$, the angle
relative to the surface normal with which the photon strikes the
wall. The grazing angle is $\theta_g=\pi/2-\theta_\perp.$ The last
real number in this line is the probability of reflection, computed
from the X-ray data for the given grazing angle and photon energy.

The following line gives the 6-coordinate position of the photon when
it strikes the chamber (same conventions as given in the previous
section). If it is reflected, there is a third line which gives the
new direction of the scattered photon ($v_x$ changes sign, but $v_y$
and $v_s$ are unchanged, corresponding to specular reflection).

In the example given above, the first photon is absorbed at the first
encounter with the vacuum chamber. The second photon reflects 3 times,
the third photon reflects once, and the fourth photon is absorbed
immediately, like the first one.

\begin{thebibliography}{}

\bibitem[1]{ref:1}
 B.L. Henke, E.M. Gullikson, and J.C. Davis. X-ray interactions: photoabsorption, scattering, transmission, and reflection at E=50-30000 eV,=1-92, Atomic Data and Nuclear Data Tables Vol. 54 (no.2), 181-342 (July 1993) {\tt http://henke.lbl.gov/optical\_constants/}

\bibitem[2]{ref:2}N. Mahne et. al., ``Experimental Determination of ECLOUD Simulation Input
Paramaters for DA$\Phi$NE", EuroTev-Report-2005-013
\end{thebibliography}{}
\end{document}  
